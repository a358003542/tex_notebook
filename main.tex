% !Mode:: "TeX:UTF-8"

\documentclass[12pt,oneside]{book}

\newlength{\textpt}
\setlength{\textpt}{12pt}
    
\newcommand{\flypage}[1]{\begin{titlepage}\begin{center}\vspace*{\stretch{1}}#1\vspace*{\stretch{1}}\end{center}\end{titlepage}}
    
%========基本必备的宏包========%
\RequirePackage{calc,float,moresize}
%\RequirePackage[onehalfspacing]{setspace}
\linespread{1.5}
%1.3 onehalfspacing
%1.6 doublespacing

%===========加入目录 某章或某节=====%
\makeatletter

\newcommand{\addchtoc}[1]{
        \cleardoublepage
        \phantomsection
        \addcontentsline{toc}{chapter}{#1}}

\newcommand{\addsectoc}[1]{
        \phantomsection
        \addcontentsline{toc}{section}{#1}}

%===========全文基本格式==========%
\setlength{\parskip}{1.6ex plus 0.2ex minus 0.2ex}   %段落間距
\setlength{\parindent}{\textpt * \real{2}}

%=========页面设置=========%
\RequirePackage[a4paper, %a4paper size 297:210 mm
  bindingoffset=10mm,%裝訂線
  top=35mm,  %上邊距 包括頁眉
  bottom=30mm,%下邊距 包括頁腳
  inner=10mm,  %左邊距or inner
  outer=10mm,  %右邊距or  outer
  headheight=10mm,%頁眉
  headsep=15mm,%
  footskip=15mm,%
  marginparsep=10pt, %旁註與正文間距
  marginparwidth=6em,includemp=true% 旁註寬度計入width%旁註寬度
  ]{geometry}

%color
\RequirePackage[table,svgnames]{xcolor}

%================字體================%
%设置数学字体
\RequirePackage{amssymb,amsmath}
\RequirePackage{stmaryrd}
\everymath{\displaystyle}

\RequirePackage{fontspec}
%設置英文字體
\setmainfont[Mapping=tex-text]{DejaVu Serif}
\setsansfont[Mapping=tex-text]{DejaVu Sans}
\setmonofont[Mapping=tex-text]{DejaVu Sans Mono}


%中文環境
\RequirePackage[]{xeCJK}
\xeCJKsetup{PunctStyle=plain}
\xeCJKDeclareSubCJKBlock{LIUSHISIGUA}{ "4DC0 -> "4DFF}
\setCJKmainfont[FallBack=DejaVu Serif, ItalicFont=KaiTi,LIUSHISIGUA=DejaVu Sans]{Source Han Serif CN}
\setCJKsansfont[FallBack=DejaVu Sans]{Source Han Sans CN}
\setCJKmonofont[FallBack=DejaVu Sans Mono]{KaiTi}


%%===============中文化=========%
\renewcommand\contentsname{目~录}
\renewcommand\listfigurename{插图目录}
\renewcommand\listtablename{表格目录}
\renewcommand\bibname{参~考~文~献}
\renewcommand\indexname{索~引}
\renewcommand\figurename{图}
\renewcommand\tablename{表}
\renewcommand\partname{部分}
\renewcommand\appendixname{附录}
\renewcommand{\today}{\number\year{}年\number\month{}月\number\day{}日}


%=======页眉页脚格式=========%
\RequirePackage{fancyhdr}   %頁眉頁腳
\RequirePackage{zhnumber}  %计数器中文化
\pagestyle{fancy}
\renewcommand{\sectionmark}[1]
{\markright{第\zhnumber{\arabic{section}}节~~#1}{}}

\fancypagestyle{plain}{%
    \fancyhf{}
    \renewcommand{\headrulewidth}{0pt}
    \renewcommand{\footrulewidth}{0pt}
    \fancyhf[HR]{\ttfamily \footnotesize \rightmark }
    \fancyhf[FR]{\thepage}}
\pagestyle{plain}


%=========章節標題設計=========%
\RequirePackage{titlesec}
%修改part
\titleformat{\part}{\huge\sffamily}{}{0em}{}
%修改chapter
\titleformat{\chapter}{\LARGE\sffamily}{}{0em}{}
%修改section
\titleformat{\section}{\Large\sffamily}{}{0em}{}
%修改subsection
\titleformat{\subsection}{\large\sffamily}{}{0em}{}
%修改subsubsection
\titleformat{\subsubsection}{\normalsize\sffamily}{}{0em}{}


%================目录===============%
%toc label to contents space   dynamic adjust
\RequirePackage{tocloft}%
\renewcommand{\numberline}[1]{%
  \@cftbsnum #1\@cftasnum~\@cftasnumb%
}

%==============超鏈接===============%
\RequirePackage[colorlinks=true,linkcolor=blue,citecolor=blue]{hyperref} %設置書簽和目錄鏈接等
\newcommand{\hlabel}[1]{\phantomsection \label{#1}}%某一小段的引用


%=================文字強調=========%
\RequirePackage{xeCJKfntef}

\let\oldemph\emph % Save emph in oldemph
\renewcommand{\emph}[1]{\textcolor{red}{#1}}  

%==================插入圖片=======%
\RequirePackage{wrapfig}
\RequirePackage{graphicx}
\graphicspath{{figures/}}
%change the caption style a little like 1-1
\renewcommand{\thefigure}{\arabic{chapter}-\arabic{figure}}

%插入代码
\RequirePackage{fancyvrb} 
\fvset{frame=lines,tabsize=4 ,baselinestretch=1.8, fontsize=\footnotesize}


%==========其他宏包===========%
\RequirePackage{tikz} 
\usetikzlibrary{calc}

%========脚注=========%
\newcommand*\circled[1]{
\tikz[baseline=(char.base)]
\node[shape=circle,draw,inner sep=0.4pt,minimum size=4pt] (char) {#1};}

\newcommand*\circledarabic[1]{\circled{\arabic{#1}}}

\RequirePackage{perpage} %the perpage package
\MakePerPage{footnote} %the perpage package command

\renewcommand*{\thefootnote}{\protect\circledarabic{footnote}}


\renewcommand\@makefntext[1]{\vspace{5pt}\noindent
\makebox[20pt][c]{\fontsize{10pt}{12pt}\@thefnmark}
\fontsize{10pt}{12pt}\selectfont #1}

\setlength{\skip\footins}{20pt plus 10pt}
%main body 与脚注之间的距离

\RequirePackage{indentfirst} 

% hr
\newcommand\hr{\par\noindent\hrule}

%重新定义quotation
\AtBeginEnvironment{quotation}{\ttfamily}
\AtBeginEnvironment{quote}{\ttfamily}

\makeatother

\title{冶文房}
\author{Wander}
\hypersetup{
  pdfkeywords={},
  pdfsubject={制作者邮箱:a358003542@outlook.com},
  pdfcreator={Wander}}
  
\newcommand{\bookcover}[1]{\tikz[remember picture,overlay]{\node[inner sep=0] at (current page.center)
{\includegraphics[width=\paperwidth,height=\paperheight]{#1}}}} 
 

  
\begin{document}
\frontmatter 

\thispagestyle{empty}

\bookcover{book_cover.png}

\cleardoublepage



\addchtoc{前言}
\chapter*{前言}
冶炼文字,以成此集。


\addchtoc{目录}
\setcounter{tocdepth}{2}    
\tableofcontents



\mainmatter

\part{十三世纪之前}
\chapter{黄帝内经关于养生之道的讨论}
引用参考资料1:
\hr
[黄帝]乃问于天师曰:余闻上古之人,春秋皆度百岁,而动作不衰;今时之人,年半百而动作皆衰者,时世异耶,人将失之耶?

歧伯对曰:上古之人,其知道者,法于阴阳,和于术数,食饮有节,起居有常,不妄作劳,故能形与神俱,而尽终其天年,度百岁乃去。今时之人不然也,以酒为浆,以妄为常,醉以入房,以欲竭其精,以耗散其真,不知持满,不时御神,务快其心,逆于生乐,起居无节,故半百而衰也。夫上古圣人之教下也,皆谓之虚邪贼风,避之有时,恬惔虚无,真气从之,精神内守,病安从来。是以志闲而少欲,心安而不惧,形劳而不倦,气从以顺,各从其欲,皆得所愿。故美其食,任其服,乐其俗,高下不相慕,其民故曰朴。是以嗜欲不能劳其目,淫邪不能惑其心,愚智贤不肖不惧于物,故合于道。所以能年皆度百岁而动作不衰者,以其德全不危也。
\hr

 
黄帝曾向天师岐伯请教道:我听说上古时代的人们,都能够年过百岁,行动起来却不显衰老;而现在的人,年过半百,行动就已经显得很衰老了。这是由于时代不同了呢?还是由于人们违背了养生之道呢?

岐伯答道:上古时代的人,大都懂得养生之道,效法于天地阴阳变化,和谐于身心养生之术,饮食有节制,起居有规律,劳作适度,所以能够使身体与精神协调一致,从而尽享天年,度过百岁才离开人世。现在的人却不是这样,把酒当做果浆来喝,把任意妄为当作生活的常态,酒醉之后还强行房事,在放纵淫欲中使自己的精气枯竭,真元耗散,不懂得保持精气的盈满,不知适时御凝己神,只求一时快意开心,如此违逆了生活真正的乐趣,起居没有规律,故年半百就衰老了。上古时代圣人的教诲,都是这样说的:要适时回避四季不正的虚邪贼风,思想上恬淡虚无,体内真气随从其中,精神持守其内,疾病又从那里来呢。如此志闲淡而少欲,心安宁而无惧,形劳作而不疲倦,真气随从而调顺,每个人欲望和愿望都很容易得到满足。如此吃什么都很香甜,穿什么都很舒服,快乐地享受他们的习俗,彼此也不羡慕地位高低,民风可称纯朴。如此嗜好欲望颇深之物不能劳累其眼,淫乱邪恶之物不能惑乱其心,不论愚笨还是聪明,贤能或者不才之人都不惧怕物欲之物,而合于养生之道。所以他们能年过百岁而行动不衰,就是因为他们[如上]德行全备而无偏颇的缘故啊。


\part{十三世纪}

\part{十四世纪}

\part{十五世纪}



\part{十六世纪}
\chapter{基督教新教创立}
马丁·路德发动欧洲基督教改革运动,创立基督教新教。核心思想是因信称义,极大地削弱了原基督教教皇权威。教会势力渐渐退出历史主流舞台,欧洲的国家政权势力逐渐成为主流。基督教新教在各个国家有着不同的发展,比如英国的国教,则是试图在新教的基础上来完成政教合一。

\chapter{英国国教的清教徒运动}
英国清教徒运动作为一次宗教改革运动,对于英国国教的批判,同当时英国国内政治,资产阶级革命活动是相辅相成的。清教徒中的长老派代表英国的大资产阶级;而清教徒中的独立派代表当时英国的中小资产阶级。比如独立派的领袖人物克伦威尔,上台对长老派进行了清洗,进行了共和革命。关于那段历史,总的来说都可以归结为当时英国的国内政治斗争史,更多的是政治诉求,而不是宗教信念上的东西。

这场运动的最终结果就是1688年英国光荣革命,1689年《宽容法案》颁布。


\part{十七世纪}
\chapter{五月花号背后的风险投资}
本章节主要参考自参考资料11:

\hr 

1616年,这群作为早期移居美洲的英国清教徒载入史册的人,原本是居住在荷兰的一个流亡的独立派团体。这些信徒中最早的成员于1608年逃离英国,最初落脚在阿姆斯特丹。几年后,他们辗转进入荷兰内地城市莱顿。...莱顿是一个“和善、漂亮的城市”,但是独立派信徒主要从事的是需要“艰苦而持续劳动”的工作,其中很多人从事布匹加工。

荷兰对他们的宗教迫害相对英国来说很少,不过在旅居荷兰十二年之后,一些人看到了宗教自由面临的潜在风险,于是这些分离主义者开始物色另外一个栖身之处,主要目的是改善经济境遇。

对于这些独立派信徒来说,最初促使他们离开英国前往荷兰的动机是过上好日子,并吸引更多基督教徒追随他们。然而,后来,莱顿的这些先驱者从事的苦力营生和“高强度劳作”彻底吓退了那些潜在的追随者。据布雷德福说,当时很明显的是,“一些人宁愿选择英国的监狱,也不愿意选择荷兰这些艰苦条件下的自由”。另外,他们中的年长者开始死亡,因为繁重的劳作往往会让这一宿命提前到来。同时,这些信众的儿女,“要分担父母亲的一部分重担”,不得不在类似其前辈要忍受的条件下劳动。如果这一切还不够糟糕的话,荷兰的“多重诱惑”还会吸引那些刚刚步入成年阶段的人远离教堂,走上“放纵”的道路,做出可能“亵渎神”的堕落之事。这些信众中的年长者清晰地意识到,如果没有会众的壮大,他们将在一代人的时间里被世俗的荷兰社会同化;这场实验最终将悄无声息地失败。总而言之,看起来,对他们宗教原教旨主义威胁最大的不是政治迫害,而是严峻的经济形势。

解决方案是再次迁移。最初的想法是改变方向,前往“地域广阔,人烟稀少的美洲地区”。这一提议引起了激烈的内部争论,主要是因为一些人想象的恶劣天气、野人、疾病、饥荒和“赤身裸体”的土著。此外,还有一个风险需要小心应对:在美洲西班牙领地附近落脚的想法被排除了,因为信仰天主教的“西班牙佬可能和美洲野人一样残忍”。其他美洲土地,大多数地方被英国声索\footnote{声明索取某地区领土主权。},剩下的相对来说面积小很多的地区被荷兰声索,所以,这就只剩下了两个选择。不管选择哪一个国家的领地,都要与对方谈判,获得许可。

当然,选择英国领地的可能性是一个戏剧性的嘲讽:当初逃离英国的信徒现在要考虑同先前迫害他们的那个国家友好谈判。不过,这个过程并非一帆风顺。

早在1606年,英国就向一个名为“伦敦弗吉尼亚公司”的私人风险项目颁发了特许证。虽然新世界的这一风险项目的内部运作由该公司内熟悉商业运作的人负责,但是监管和治理的权力仍然在英王手中。英王通过他设立的弗吉尼亚委员会来行使这些权力。和大多数新发起的风险项目一样——不管这些项目是海外项目还是国内项目——弗吉尼亚公司的开局很不妙。在其成立的头十年里,它数次亏损得一无所有,不得不再次筹资。更为糟糕的是,绝大多数被送往海外的劳工都悲惨地死去。

在最初的十年运作期里,弗吉尼亚公司饱受各种挫折和困难。当莱顿的独立派教徒派两人去伦敦,与他们商谈在其领地上落脚的事宜时,该公司喜出望外,就像是生意惨淡的店主看到了当天唯一的顾客。弗吉尼亚公司作为商业实体,能否盈利取决于殖民地的经营情况。那里急需大批孤注一掷,将生死置之度外的人。荷兰来的那两个教徒的迫切心情正好与弗吉尼亚公司的急迫心情不谋而合。不过,独立派信徒移居弗吉尼亚还有一个障碍:他们需要明确的许可,以便在那里从事宗教活动。看重经济效益的弗吉尼亚公司对眼前的机会非常乐观,向他们保证说,这个问题是小事一桩,只要英王例行公事的批准。然而,事实并非如此。这个过程不断拖延。弗吉尼亚委员会认为,如果他们批准独立派教徒在海外从事宗教活动,就会在某种程度上破坏英王陛下禁止其在英国从事宗教活动这一禁令的权威。后来,通过弗吉尼亚公司的斡旋,双方达成了妥协。英王的弗吉尼亚委员会既不批准也不禁止他们在海外从事宗教活动,前提是这些来自莱顿的信徒必须服从英王的管辖。

这种有意识的含糊成为英国政府代理人和那些信徒之间的中间地带。弗吉尼亚公司为莱顿的信徒授予了一个“许可证”,允许他们落脚新世界。实际上,不同于传统说法,最初移居美洲的清教徒根本不用躲避英王的迫害,他们是自愿向海外拓展英王主权的。

获得许可之后,身在荷兰的教派长老们现在面临着同样复杂的资金问题。虽然弗吉尼亚公司有能力提供许可,但没有能力提供远洋航行所需的资金,教友们需要自己筹资。对于谨慎保守的有钱人来说,船只、水手和给养所需的支出不菲,作为一项投资来说风险太大,尤其是作为海外风险项目,很可能血本无归。后来,解决方案似乎是物色愿意放手一搏的有钱人——想要获得巨额收益,不计较一两个项目失败。

正在这时候,一个荷兰商业机构向这些教徒抛出了橄榄枝,与弗吉尼亚公司的方案形成竞争关系。听说他们与弗吉尼亚公司的谈判后,这个荷兰机构劝说莱顿的独立派教徒前往荷兰在美洲的殖民地。然而,有了弗吉尼亚公司颁发的居留许可证,提供资金这件事就落到了英国的投资者身上。具体地说,一个叫托马斯·韦斯顿(Thomas Weston)的项目发起人,代表伦敦商业风险投资协会,辗转前往莱顿,百般游说独立派牧师约翰·罗宾逊(John Robinson)。韦斯顿劝说这位牧师,称自己“可以让朋友们投资”,于是双方草拟了正式的条款。用现在的话说,那些条款相当于一个意向书或投资条件说明书,一个需要根据最终谈判进行充实的框架。现代创业人士都知道,从协议双方讲明各自目标到最终达成协议,其间充满着越来越多的焦虑和对意志的考验,往往会持续到达成协议的最后一刻。1620年“五月花号”的筹资活动也不乏各种争执和波折。

\begin{center}
* * *
\end{center}

风险投资在17世纪之前的很长时间里就已经成了气候。早在1505年,一个叫作“英格兰商业风险投资协会”的组织就获得了官方的批准。这种协会不是将资本或资源汇集在一起的正式组织,而是松散的行业团体,个体成员可以有选择地参与团体的风险项目。后来(同一个世纪),海外风险项目所需的资金越来越多,恰逢股份公司出现并推动了这些项目的发展——相对于出资人相互熟稔、相对封闭的合伙关系,“股份制”指的是股权可向任何人转让的股东关系。

除了股权可转让,商业法规的不断进步也为减少个体风险投资者的个人责任提供了有利条件——如果项目亏损,投资者可以不用承担超过其初始投资的亏损金额。有限责任概念是法律领域的一个新事物,在自由市场中并不自发或有机地存在。有限责任可以让投资者获得无限的利润潜力,而限制可能的亏损程度,这大大增加了航海探险投机活动的吸引力。有限责任形式不一定和海外殖民有关。主权国家经常向国民授予独占捕鱼权、勘探权,向私营实体授予贸易路线。政府通过向私营实体颁发特许证,鼓励私人资本投向海外风险项目,从而为国内创造经济效益。

有限责任对于鼓励投资,实现上述目标发挥了关键作用。对于远在遥远海域的船只和持续时间长达数月或数年的海外贸易项目,由于其商业性质,身在英国的投资者很少有参与决策的机会。这就强化了对公司制的迫切需求,因为在公司制下,被动型投资者无须对未知债务承担责任。同时,这种商业项目所涉及的航行距离和海外使命持续时间,决定了这些项目需要数量庞大的资本,远远超过了任何单个投资者的风险承受能力,不管他多么富有。股份公司可以让很多投资者出资参股某个项目,获得相关利润。促使英国企业采用股份制的最后一个推动因素是1553年成立的一家叫作“俄罗斯公司”的股份公司。当时,众多风险投资者以每股25英镑的价格,总共投入6000英镑;这标志着法人形式首次被用于海外投资。

从那时起,连英国的私掠者(经政府批准可以抢夺敌国海上货物的海盗船)都开始采用股份公司形式向风险投资人筹集资金。私掠者想分散风险还有另外一个原因。对于从事这一行当的个人来说,如果私掠活动的出资群体中有足够多的声望很高的人物,那么因政治风向发生转变(即使当初被政府批准)而被控犯罪的风险会大大降低。这些私掠者根本不是传说中戴着黑眼罩、肩膀上站着鹦鹉的草莽人物。有关每次私掠活动的会计报表都会详细记录动用船只的吨位、投入的资金、参与的人员和船只数量。在有关弗朗西斯·德雷克爵士(Sir Francis Drake)组织的一次私掠活动的财务报表中,赫然记录着这次冒险总投资额为5.7万英镑,动用了21艘船和1932名水手。W. R. 斯科特(W. R. Scott)深入研究那个时代的众多股份公司之后,认为这种公司结构的灵活性为经营活动提供了两种优势,即投资的多样化和风险分摊,尤其是在私掠方面。私掠活动对巨额亏损的容忍程度催生了现代风险资本的基本理念。

假如,一个出资人打算在私掠项目上投资2000英镑,那么只够为一艘排水量为200吨或两艘更小的船只配备相关设备。这样,如果仅凭一己之力,私掠活动的力量就会过于薄弱,难以获得有价值的斩获。但如果他用这些资金和很多人共同投资多次大规模的私掠行动,即使某次活动完全失手,他仍旧可能从其他活动的利润中获得不菲的收益。

收益不菲,确实如此。德雷克的这次私掠活动获得了4700\%的回报,也就是其投入资本的47倍。在权衡投资机会的风险投资者看来,这一奇高的收益率是极为诱人的,要知道,他们可不是保守的伦敦银行家。

17世纪初,经过一场极为残酷的战争,英国经济遭到重创,陷入萧条。除了英国本土之外,其他地区经济形势都很好。1600年,东印度公司成立。数年后,伦敦的弗吉尼亚公司成立。公司成立没多久,就印制了吸引风险投资者的小册子。弗吉尼亚公司的宣传册刻意淡化了所有不利因素。在有关印第安人可能带来的危险方面,宣传是这样说的:“他们大都善良和气,无微不至地关照我们。”广告中还详细叙述了可以运回英国做木材的各种树木,蕴藏丰富宝藏的“从未探察过的”山脉和辽阔“富饶”的土地。接下来,小册子号召广大英国同胞发扬爱国精神,积极参与,说他们必须“前往世界的每个角落,解决英国的物资匮乏问题,用一个王国的物资供应另一个王国”。然后,小册子讲到国内严重的失业状况,到处是“大批闲人”,可以将他们送往海外从事这一项目。

接着,小册子开始说眼前的项目。对于想留在英国的风险投资人,弗吉尼亚公司一股的价格是12英镑10先令。对于移居者,也就是那些真正愿意去弗吉尼亚的人,不用出任何资金就可以得到公司的一股,前提是在弗吉尼亚殖民地做满七年工。另外,公司负责提供伙食费、材料费和殖民地的维护支出。公司拥有所有生产资料,对所有经营活动享有垄断权。在七年结束之际,包括已开垦土地在内的公司资产将被分配给所有股东。对于很多经济拮据,为生活苦苦挣扎的年轻人来说,只要出力气就可以在新世界获得一块土地,对他们具有相当的诱惑力,尤其是当他们看到很多富人也愿意以超过12英镑的价格购买这些股票时。

轻松赚大钱的梦想往往很容易破灭。弗吉尼亚公司后来变成了一场灾难。头几批到达的劳工饱受疾病、饥荒、严寒和印第安人袭扰的折磨。随后几批给养船到达的劳工被眼前长期食不果腹的同胞吓坏了,早期到来的一批人干脆逃到了印第安人中间。连续经历了几次失败后,老股东不愿意继续投钱,公司不得不物色新股东,并进行多次重组。1614年,公司财务状况极度恶化,急需大批资金注入。再加上,他们建立的殖民地的状况也一团糟,于是,人们群情激奋,要求撤销该公司的特许权,上交王室。弗吉尼亚公司的律师理查德·马丁(Richard Martin)向下议院申请救助,希望英国财政部能提供应急资金援助。

因此,当那两个来自荷兰的访客向弗吉尼亚公司打听如何落脚新世界时,后者愿意招收任何甘冒生命危险,且能忍受恶劣工作条件的人。

\begin{center}
* * *
\end{center}

获得弗吉尼亚公司和英王的许可近两年半后,莱顿的独立派教徒为远航做了最后的准备工作。威廉·布鲁斯特(William Brewster)和约翰·卡弗(John Carver)代表所有信众,动身前往伦敦签订最终条款。

股票定价为每股10英镑。条款类似于弗吉尼亚公司的条款。愿意去新世界耕作的人可以获得一股,愿意投资10英镑的任何人也可获得一股。风险项目拥有殖民地的所有资产和经济权利。七年结束之际,总资产按照所有权比例在所有股东之间进行分配。

然而,问题出在细节中。就在距离原定启程日仅剩几个星期之际,商业风险投资协会更改了协议条款。在最初的条款里,这些信徒每周为该项目工作四天,剩下两天时间用以“个人工作”,一天是安息日休息。后来,投资者坚持要求人们一周给这个项目工作六天。更令人气愤的是,那些劳工给家人建的房子也不属于他们自己,而是属于这一风险项目。

这些信徒的精神领袖约翰·罗宾逊非常不满。他写信给他的代表约翰·卡弗,抱怨久久无法租到船。罗宾逊发现,七年到期后将房屋作为项目资产来分割的条款尤其显得吝啬,因为,该项目的主要利润来源是“捕鱼、贸易等工作”。接着,他请原定牵头进行这次远航的卡弗认真考虑劳作七年且“没有一天不劳动”是一种什么感觉。然而,已经太晚了。代理人已经代表所有信徒在协议上签了字。

在筹集到1200英镑之后,相关各方立刻开始腌制牛肉,储备啤酒、饮用水和海上航行需要的其他物资。在荷兰,随着该项目的一些股份在独立派信徒中发售,他们用这些钱购买了排水量为60吨的小型船只“佳速号”,打算抵达新世界后将这只船用于海岸贸易和捕鱼。在英国,人们租下了排水量为180吨的“五月花号”。

1620年7月的一天,“佳速号”矗立在荷兰港口城市德夫哈芬(现今的鹿特丹),准备载着部分独立派信徒前往英国,与停泊在那里的“五月花号”会合。其中的一个乘客,也就是未来的朝圣者领袖威廉·布雷德福,记述了那个极为伤感的一天:面对从附近的阿姆斯特丹赶来送别的亲友,那天“不知流下了多少泪水”。“德高望重的牧师满脸是泪,跪倒在地”,希望作为先驱者第一批出发的教友一切顺利。不久,“佳速号”徐徐离开了码头。就这样,“他们离开了生活将近十二年的那个优美宜人的城市,不过,他们认为自己是朝圣者”。

没过两天,在一路“顺风”的帮助下,这些朝圣者到达英国南安普敦港。看到“五月花号”的初始兴奋退去之后,手里的协议让人们很快郁闷起来。从荷兰来的信徒立刻对代理人认可的条款提出异议,尤其是工作日由四天变成了六天,以及自建房屋由个人财产变成了共有财产。商业风险投资协会的代表韦斯顿从伦敦赶来,以“确认协议条款”。但是,这些朝圣者不承认新增条款的约束力,虽然他们的代理人签了字。僵局持续了好几天。这时候,事先约定的最后一笔100英镑的资金到了出资日期。这笔钱本应由韦斯顿他们支付,用于让“五月花号”起锚离港,但韦斯顿拒绝出钱。信徒们只好卖掉了3000磅黄油,低价处理掉一些其他物品,迅速筹集了60英镑。

另外,朝圣者的虔诚信仰促使他们一致向其他股东表达了他们不让步的原因,其中包括和商业风险投资协会一起投入资金但仍然留在莱顿的很多信徒。他们提出,在新世界拥有自己的房子是促使他们前往那里的“重要因素”。不过,他们又安慰投资人说:“如果七年内不能产生大笔利润,我们大家还会继续合作。”虽然在其他方面对投资人百般安抚,但是这些朝圣者在房子所有权上丝毫不让步。韦斯顿也不让步。

卖掉一些给养获得资金后,1620年8月5日,“五月花号”和“佳速号”离开码头,驶向新世界。因为筹集资金和租船耽误了一些时间,所以启程时间比原计划稍晚。按照这个时间,他们可能在初秋抵达新世界,勉强有足够的时间为冬天做准备。屋漏偏逢连夜雨。“佳速号”名不副实——因为漏水不得不停靠最近的港口修理。漏水问题解决后,又发生了其他导致返航的问题;这一次,“佳速号”被彻底放弃。最初的一些乘客决定放弃这次前景不妙的旅程,剩余的乘客则转而乘坐“五月花号”。

9月5日,“五月花号”载着102名乘客,开始了驶向新世界的旅程。很多记载说这些朝圣者是因为宗教信仰原因前往新世界的;需要指出的是,“五月花号”上足有一半的乘客并非来自莱顿的独立派信徒,他们只不过是被投资者分配到那艘船上的移居者。

原本生活在相对自由的荷兰,现今在英国人的资金支持下,乘坐英国船员驾驶的从英国租来的船只,在英国国旗下驶往新世界——这一切都不像是逃离英国迫害的样子。政治难民一般不会在逃难前的最后日子里商谈未来七年的经济报酬和资产分配。所以,这些朝圣者从开始就根本不是难民,他们是一场投机活动的关键参与者,同时也是向新世界扩展英王主权的关键参与者。宗教自由只是整个计划中的一部分。

然而,这一风险项目,和弗吉尼亚项目一样,开局非常不顺。11月末,“五月花号”才抵达美洲海岸。除了抵达的季节过晚之外,他们还错过了目的地。弗吉尼亚公司当初授予的“特许证”规定的登陆地点在哈得孙河河口附近。然而,“五月花号”停靠地点却在河口以北220英里处的半岛旁。在派出一个探险队上岸查看海岸线,并物色建立定居点的位置后,大多数朝圣者仍然待在“五月花号”船上,等待探险人员回来。在“五月花号”出航期间,英国的官员们将弗吉尼亚北部的一块土地划给了新英格兰委员会。在不知情的情况下,这些朝圣移民正在为他们在新英格兰的第一个冬天做准备。

威廉·布雷德福在他撰写的历史书《普利茅斯开拓史》中将这一部分的标题定为“饥饿年代”。航行途中只死了一个船员,横渡大西洋结束之前,没有乘客死亡。但在人们等待探险队归来的过程中,死神光顾了那艘船。布雷德福回到船上后,发现妻子死了。而这只是等待着他们的严冬的前奏。2月底,这块新建殖民地的死亡人数达到了“一天两到三人”,而且已经持续了一段时间。布雷德福极为痛苦地说,“只有六七个人”身体无大碍,能够照料别人。到了3月,“五月花号”的乘客将近死了一半。英国谈判导致的耽搁,再加上“佳速号”漏水返航,让登陆新世界的时间推迟了整整一个月。而人们本可以用这些时间来准备应对未来几个月的寒冷天气。

在科德角挨过冬天之后,租来的“五月花号”在4月开始返航。看重经济效益的风险投资人原本期望这艘船返航时从新世界带回木材、毛皮和其他商品。但是,因为那些移居者在第一个冬天里的恶劣境遇,“五月花号”返回英国时基本上是一条空船。

很多投资者——尤其是托马斯·韦斯顿——很不满意,认为死亡率不是借口。在写给约翰·卡弗(当时的卡弗已被那些朝圣者选为总督)的信中,韦斯顿不无挖苦和不满地写道:“船上没有送回来什么东西,你们可真是不容易,人们对你们不满意也不冤枉你们。我知道你们的问题所在,与其说是手脚有问题,不如说是头脑有问题。”韦斯顿这封写给卡弗的信交由一艘叫“财富号”的新船送达,其目的地是普利茅斯。韦斯顿担心卡弗不能充分理解他的意图,还特意在信中告诫对方说,如果卡弗不能让这艘船满载而归,他就要切断他们的资金供应。然而,在“五月花号”回到英国和韦斯顿通过“财富号”寄送的信送达的这段时间里,卡弗去世了。

经历了冬季的痛苦之后,“财富号”的到来让人们大感欣慰。“财富号”除了带来急需的给养之外,还带来了35个移居这里的劳工。韦斯顿那封措辞很不客气的信被交到了新任总督威廉·布雷德福手中。不过,到了这个时候,形势已经开始好转。从出现第一丝春意开始,这些移居者花了好几个月来建造他们的居住区。更重要的是,他们第一次与印第安人有了直接接触。那个印第安人自称萨莫塞特(Samoset)。不可思议的是,与科德角附近从事季节性捕鱼的讲英语的渔民熟悉一些之后,萨莫塞特居然能够结结巴巴地说一些英语。他甚至在首次打招呼时主动走向前说“欢迎你们,说英语的人”。这句欢迎语让对方的紧张情绪大为缓解。

在这次接触的几天后,当地部落首领马萨索伊特(Massasoit)前来造访。这次会面意味着之后延续了二十年的个人关系的开始。印第安部落对普利茅斯殖民地的救助至关重要。具体而言,印第安人能够弄到一种可以在欧洲卖到很高价格的内陆商品:河狸皮。欧洲各国目标市场的有钱人对河狸皮制品趋之若鹜,导致河狸皮迅速成为新英格兰和旧世界之间的商业纽带。同时,跨大西洋贸易塑造了美洲土著和早期殖民者之间共生的经济纽带。因为进入内地的殖民者装备很差,所以印第安人没有领地会被侵占的忧惧,而是将英国人的殖民地看作贸易站。多年来,河狸皮是印第安人冬季御寒衣服的一部分,他们猎捕河狸的技术非常娴熟,因此在弄到殖民者渴望的这种高价值商品方面,印第安人具有很强的竞争优势。长途跋涉到人迹罕至的水塘捕捉河狸,不是一件容易的事情,殖民者们将这件辛苦活儿交给这些行家去做。

北美土著将另一部分价值添加到供应链中来:河狸皮需要大量的加工工作。肢解和扒皮落在了本地女人们的身上。将皮毛上的肉和肥油去掉之后,接下来是一个更消耗体力的任务:软化生皮,将粗糙的硬毛去掉,最终加工成适合欧洲上层人士穿用的光滑柔软的毛皮制品。有时候,这甚至需要穿着生皮一年,用汗水做软化剂,达到软化效果。

这一情况促成了普利茅斯早期殖民者的贸易机会。早在“财富号”抵达之前,殖民者就已经开始用一些简单的物品,比如毯子、玻璃珠、刀具和其他器皿,换取印第安人的河狸皮。印第安人没有铸造锃亮刀具或精致金属用具的技术,却能够用他们看来简单的猎捕和加工河狸来换取这些奢侈品。...

这些殖民者没有把韦斯顿的挖苦放在心上,将大量木材(以板材的形式)装上了“财富号”;更重要的是,他们还给船上装了“两大木桶河狸皮和水獭皮”。这批货物价值“将近500英镑”。这是一个利润颇高的套利交易:据朝圣者中的那位领导说,这些即将出口的东西是殖民者用“一些微不足道之物”换来的。那次航行的初始投资,大约是1200~1600英镑,毛皮和木材的价值提供了投资者第一年红利的30\%——这是为一群本质上是宗教原教旨主义者的人提供资金的坚实回报。不过,上天却不保佑“财富号”。那艘船在大西洋海面上航行时,遭到一艘法国私掠船拦截,货物被抢走。结果,和“五月花号”一样,“财富号”也空手回到了商业风险投资协会那里。

这一灾难性事件与出资人意外遇到的其他几个问题同时发生,并引发了之后的那些问题。如果那批货物安全抵达的话,销售利润可以用来投资殖民地来年所需的给养。现在,采购相关工具装备和另派补给船的成本,需要再进行一番筹资。而接下来的筹资活动难度很大,因为这一项目到目前几乎颗粒无收。同时,韦斯顿和另外一个出资人发生了争吵。一时间指控四起,诉讼不断。这场风波大大影响了韦斯顿在出资人中的威信,尤其是在需要进行额外一轮融资的情况下。不久,韦斯顿将自己的股份让渡给其他所有出资人,这场争执才算结束。

恰似海外项目出资人的纷争,北美殖民者也遇到了棘手的问题。按照最初的协议,定居点和公司的所有人都要参加种植劳动。给养船的到来一再推迟,即使到来,又常常带来比食物更多的需要喂养的人口。在这个时候,种植庄稼关系到人们的生存问题。然而,“人人为我,我为人人”的思想造成的粮食短缺,说明这种集体劳动方式是靠不住的\footnote{历史早就证明了在粮食生产上大锅饭注定是失败的。}。经过激烈争论,大家决定所有家庭都分得一份土地。每个家庭可以自由支配自己的劳动成果。殖民地总督说:“这个办法很成功,它让所有人都勤快起来。女人们积极主动地去地里干活,还带上小孩子帮她们往地里撒玉米种子,不再说孩子没力气不会干活的话了。”这种实验性的共产主义生产形式,至少在种庄稼方面,在这些朝圣者中结束了。

同时,国内投资人逐渐意识到,在项目期限内获得丰厚利润的希望很小。更糟糕的是,英国恶化的经济和政治形势让很多投资者资金紧张。“五月花号”出航不到三年,金融机构就开始瓦解。富有同情心的投资者詹姆斯·谢林(James Sherley)在给布雷德福的信中沮丧地说:“之前,不管哪次出海和交易,你和我们都是参与者和合作伙伴,那种时候再也不会有了。”另外,他更关心的是确保“我们的钱不要亏掉”。谢林估计,出资者对“不少于1400英镑”的公司资产有优先要求权(first claims,在现代风险资产术语中叫“优先清算权”)。又一段谈判开始了。

就在英国投资者内部,以及他们与朝圣者进行激烈谈判的这段时间,该殖民地仍然需要外部资金注入以维持生存。为了弥补资金缺口,这些朝圣者向贸易合作伙伴和经纪人张口借钱,利率超过了50\%。

众多利益群体经过将近两年的谈判,逐渐达成了一个解决方案。普利茅斯的殖民者不愿根据协议将房屋和田地作为公共财产进行分割;同时,身在伦敦的风险投资人也无意直接接手位于数千英里之外偏僻村庄的房产。经过复杂的讨价还价,双方达成一份协议。普利茅斯殖民地负债1800英镑,买入该风险项目的所有股权——这样,出资人就不再是该项目的股东。这使得殖民者可以自由地在内部分配房屋和田产。同时,一些朝圣者,以威廉·布雷德福牵头,承诺承担全部集体债务;从1628年开始,每年支付200英镑,九年共计支付1800英镑。为了缓解普利茅斯公民的偿债压力,布雷德福和他的团队申请到了该殖民地从事毛皮生意的垄断权。

实际上,这一垄断权含金量极高,但前提是没有竞争。到1628年,一船又一船的移居者抵达新英格兰。有的移居者,比如前来定居的清教徒,有宗教倾向,而其他人则是流动商贩。殖民地位于哈得孙河的荷兰人在北至康涅狄格河的地方建立了贸易站。法国人也不甘落后。对于以上各路人来说,毛皮都是至关重要的商品。北美土著居民继续他们作为猎人和毛皮加工者的角色,是跨大西洋贸易的关键人物。对于当地河狸来说,这可不是什么好消息。繁殖率低,再加上迁徙距离有限,这场毛皮争夺战几乎让新英格兰的河狸迅速灭绝。随着河狸的消失,当地印第安人基本丧失了价值,甚至成了危险因素。

市场形势的压力迫使布雷德福和信徒们重新与出资人谈判,使他们将皮毛生意深入进行下去以获得出路。1645年,当所有矛盾尘埃落定时,仍旧持股的出资人已屈指可数。“五月花号”的资金筹集和每次耗时数周的跨洋远航,形成了一个持续二十五年的风险项目。当那份协议有效期满时,当初船上的朝圣者们绝大多数已不在人世。

\hr 

\section{编者按}
实际上五月花号项目不是弗吉尼亚公司的第一个风险项目,美国现在的弗吉尼亚州的名字不确定是否来自弗吉尼亚公司,但确实弗吉尼亚公司的第一个移民项目就落在了弗吉尼亚州的东南部,1607年,105名英国人来到了美国的弗吉尼亚州,建立詹姆斯敦殖民地。

而詹姆斯敦殖民项目也不是弗吉尼亚公司的第一个风险项目,此前的18个都定居失败了。至于詹姆斯敦殖民项目很快也失败了。1622年印第安人对詹姆斯敦定居点进行了大屠杀,300多名英国移居者丧生。

五月花号作为弗吉尼亚公司的风险项目,是偏离了原定计划的落脚点的,落在了现在美国的马萨诸塞州。那么五月花号是不是弗吉尼亚公司的最后一个风险项目呢,也可能不是,不过也不远了。上面提到1622年詹姆斯敦定居点惨剧就是一个例子,鉴于弗吉尼亚风险投资项目超高的死亡率和黯淡的前景,1624年詹姆斯一世收回了弗吉尼亚公司的许可权,之后就没有弗吉尼亚风险投资项目了。

显然移民美洲的其他项目仍然在前仆后继的继续着,所以五月花号项目肯定也不是最后一个移民项目,那么美国的历史编写者们为什么要着重强调五月花号呢?也许仅仅是因为这艘船的出发和目的带有一点圣经上的浪漫气息吧。

五月花号上的移居者之后的命运怎样了似乎历史已经选择性遗忘了,因为真实的现在的美国历史的开端,是以东海岸陆陆续续一系列的移民定居点开始的,有的成功了,有的失败了,而五月花号,只是似乎多了一点点浪漫气质,多的不多,甚至有可能是失败的那一批,即使五月花号定居项目成功了,也只是这整个移民定居项目中不起眼的那个罢了。读者在推演美国历史的时候,最好以这样的视角来看待五月花号,当个典型略微了解下即可,别太看重,如果认为美国历史是以此定居点推演一生二慢慢发展起来的,那就错的离谱了。




1637郁金香狂热



\chapter{自然哲学的数学原理}
1687年

引用自参考资料8:
\hr 
牛顿是一个清教徒,这一点在他的整个学术生涯中影响着他。当时英国的大学是奉英国国教为圭臬的,即使王室复辟后,昭示容忍宗教的多元性——英国社会的主流思想一贯主张如此,但大学生还是被迫至少要在形式上服膺于英国国教,所以牛顿不得不把他的清教徒信仰藏在心中。

牛顿的宗教倾向无形中带给他不知多少的潜在问题,这让他与同学(那些正统派)之间树下多重藩篱,但清教徒的道德激励了他学习的决心,令他心无旁骛地全力学习。幼年时代被母亲遗弃,使他深受伤害以致情感无能,清教徒主义的节欲世界赐予他一个极好的借口,作为他对自己无心恋爱的解释。因为在清教徒的世界里,只有上帝和知识这两根精神支柱,而追求知识又是上帝赋予他的神圣使命。这两根支柱可以取代其他一切需要,他以清教徒主义和渴求知识的天性为引导,至少可以躲避性欲的需要。没有结婚或成家的心理压力,更抑制了他对物质的欲望。

...


牛顿的《原理》不仅整合了伽利略和开普勒的理论,成为单一的、内聚的,用数学表达、用实验支持的整套理论,同时也打开了工业革命的大门。牛顿除了解决潮汐如何产生、彗星如何划越天空等种种困惑人类的古老问题,还引入了较新的观念,譬如解释地球自转时的“摇晃”或岁差是因为地球上各点有不同的重力强度。《原理》为力学和动力学的研究奠定了基础,在其后的一个世纪内,引发了持久且真实的人类文明演变。若没有这样的了解,大自然的各种力就不会获得利用,而这正是工业革命的成就:它将人类从黑暗中,从对大自然的奇想中,拉往科技时代,拉向操控宇宙之力的时代。


\hr




\chapter{牛顿的造币厂岁月}
过去的历史记载和牛顿传记对牛顿的那段造币厂岁月是不愿提及或语焉不详的,在历史上更多的被人们谈起的是物理学家牛顿和他的巨著《原理》,而现代社会已经越来越重视从经济金融的角度来看待历史变迁,于是人们也开始越来越重视和讨论牛顿作为造币厂督办或厂长的那段岁月对历史的影响。

引用自瓦伦丁·博斯的《牛顿与俄罗斯》:
\begin{quotation}
1698 年前往英格兰游历的俄国沙皇彼得渴望见识该国的几样非凡之物:造船术、格林尼治天文台、铸币厂和艾萨克·牛顿。
\end{quotation}

实际上在十七世纪晚期,牛顿在学术圈的盛名也仅仅局限在学术圈,让牛顿在英国政坛崛起,获得自己梦寐以求的权力——当选皇家学会主席,被册封爵士,变得小小的富有,都是因为他在造币厂的出色工作。

引用自参考资料6:
\hr
1693年约50岁的牛顿有过一段时间的精神崩溃期,后来牛顿的思想恢复了平静、生活恢复了正常之后,又作了最后一次努力,想矫正他在《原理》中论述月球理论时遇到的一些顽固问题。可以说,这将是牛顿从事的最后一项主要的持续性科学活动。从1694年夏开始,他再次尝试解决月球理论问题。...但到头来,牛顿还是没能解决棘手的“三体”问题。而不解决这个问题,他就无法在月球理论上取得进展。
\hr

认为牛顿因为物理学上棘手的三体问题得不到突破,而突然转型变得对政治和世俗事务感兴趣了这是不对的。实际上牛顿一直就对政治权力很感兴趣,牛顿从天性上就是一个官僚,参考资料8说的好,对于牛顿而言,他在科学上的动力就是求知欲,而知识至于他而言就是权力。

当然就算牛顿政绩不错,也必须站对位置,总的来说牛顿的政治觉悟还是不错的。这里不讨论牛顿真的政治主张,牛顿贪图的是政治权力,政治主张对他不是最重要的,他并不是一个政治主义者,所以后面我们会看到牛顿发现风头不对之后,很自觉的就退出政治舞台了。

当时的英国政治局面是这样的,1688年末詹姆斯二世逃离英格兰,威廉三世来到英国,新的政权新的班子,牛顿作为剑桥大学当选为了国会议员,此时的牛顿对于自己的当选都是有点吃惊的。而后他的投票确认詹姆士二世已经放弃了王位,于此时起牛顿也算是走进英国的政治圈子了,不过也就止步于此了,并没有收到额外的重用。牛顿在一篇法律废除权的分析陈述中认为国王的权力在人民之下,只有人民才有权力废除法律,这表明牛顿是持有辉格党人的政治观点的,其后牛顿便走进了辉格党的圈子。

1702年威廉三世去世,继承王位的安妮女王不喜欢辉格党,托利党受到重用,牛顿敏锐地察觉到了这种变化,其后在政治上保持低调。这种低调自保的态度是明智的,1703年当选皇家学会主席,1705年受封爵士,直到1724年因身体状况不佳主动放弃造币厂总监和皇家学会主席职位,这些政治权力的获得和保持除了基于他造币厂上出色的工作政绩,就是得益于明智的政治站队和之后明智的低调态度。【牛顿获得政治权力之后干的那些破事这里就不细说了】

查尔斯·蒙纳古,当时的财政大臣,同时还是皇家学会的主席,还是牛顿的学生,是不是牛顿的朋友不知道,但显然蒙纳古对于牛顿的学术成就是知晓的。他是这样说的:

\begin{quotation}
我非常高兴,因为我终于能向您证明我的友谊以及国王对您的功绩的赏识。国王已应允我任命牛顿先生为造币厂监督。这个职位最合适不过了,年俸约为五六百英镑,而事情不多,花销不大......
\end{quotation}

虽然当时英国财政上一堆乱事,但说蒙纳古是因为觉得牛顿能力出众,渴望他出山力挽狂澜,那是扯淡。更多的一个闲职“事情不多”,薪资颇丰“一年五六百英镑”。

当时的英国从货币市场上来讲有以下几个问题:

\begin{itemize}
\item 银币被剪钱问题:银币被人偷偷的减去边边角角来获利。提出的解决方案就是回收旧币铸造新币,在牛顿进入造币厂之前,铸造新币的工作已经着手推行了,但因为各种困难而进展缓慢,具体细节后面细讲。

\item 伪造货币问题:新币的铣边可以防止剪钱,不过不法分子仍然可以将银币完全熔化,再掺杂廉价材料重新铸造伪币来获利。一开始牛顿对打击伪造货币犯罪问题不太热心,觉得这不应该归造币厂管,推托无果之后,牛顿花了大力气做起了侦探来打击伪造货币犯罪。

\item 银子外流问题:新币走的是等值银子路线,银含量比较高,欧洲大陆政府权力较大,铸币话语权更大,所以铸币的时候一般会放点水。各个国家货币银含量不同,人们跨国贸易的时候必然采用的是等值货物交易逻辑,这个时候用金子来作为一般等价物是最合适不过的,还有从货物贸易的角度出发,因为各个国家货币银含量情况很不一样,人们在乎的也仅仅是该货币的银含量。我们现在假设有1克金子,在英国这边有旧币,银含量比较低,假设1g金子18s,这里单位是假定的,而新币假设1g金子21s;在法国这边是1g金子17s,还有一种新注水的货币1g金子15s。假设你是一个手头上有1g金子的商人,那么这个时候你首先想到的是跑到英国用1g金子收购新币,把新币熔成银子再跑到法国,那么你就可以兑换更多的金子。上面的讨论如果非得代入现在的货币面值,你可以认为1g金子在英国就是1英元,1g金子在法国就是1法元,也就是从等价物的角度来说,1g金子等于1英元等于1法元。继而有1英元等于21s银子或者18s银子,1法元有17s银子或者15s银子。读者这个时候肯定已经发现问题所在了,长此以往英国含银量较高的新币就好像走入一个黑洞一样必然不断地被民间熔掉。
\end{itemize}





\section{回收旧币铸造新币}
引用自参考资料8:

\hr 
我们可以想象,牛顿的到来给造币厂的领导带来了多大的震撼,因为他是所有人当中最勤快的一个。...在头几个星期中,每天凌晨4点,当压制机开始启动时,牛顿就在厂房里了,夜班开工前他又赶回来监督。有一阵子,他甚至住在工厂旁边为他准备的带有一个小院子的宿舍里。在历任的厂长中,没有一人做过这些事。

这间宿舍狭窄而拥挤,院子也不过是一片延伸到城墙边的永远被阴影笼罩的草坪。宿舍内吵闹不堪,旁边的工厂日夜两班替换交接,只有每晚从午夜至凌晨4点这4个小时停工。每周工作6天,空气潮湿发霉,300个工人和20多匹马散发出的难以忍受的臭味,令人无法久留。...

...

于是造币厂成为旧币再生的机房,每件事都讲求效率——吃进旧币、吐出新币的效率。古老的方法是用人力锤击来制造硬币,这是一种劳动力密集的生产方式,工作进度缓慢且效率低;而新式的压制机原理是由法国的布隆多引进的,自查理二世时代开始即以小规模操作。在由50匹马供应动力的10个机房中,巨大的滚筒轧出厚度精确的金属薄片,自薄片剪下空白硬币送进压制机中,那台压制机是由几个工人推动的一根大转轴,轴两端各有一把重锤,一个无助的少年硬币工将空白硬币一次一个地送进压制机下的机槽内,然后那把重锤随即击下,把皇族的肖像压印在硬币面上。可怜的少年工做不了几天就会损失至少一根手指头。最后,再由铣工将硬币的边缘铣出不易剪开的花纹。

牛顿最关心的事是提高工作效率。他仔细观察制作过程中的每一个步骤,制定工作系统的时间动作分析表,找出在何处以何种方法可以加以改善。他发现如果压制机的重锤运动与少年硬币工的动作相互配合适当,一个硬币工可以每分钟做50~55次入料和取出。他有几本记载造币厂工艺的记事簿,里面详细地记录分析了制造钱币的过程。

据在造币厂工作的职员海恩斯所写的钱币铸造经过,他认为牛顿所下的功夫是使那项作业成功的根本原因。牛顿的数学技巧使生产的过程流畅,这大大地提升了工作效率,并且“他能评定工人的勤惰”,这点可以想象得到。

...

牛顿之前的科学大师是不会这么做的。如果说牛顿对下属真的严格,那么他对自己比对下属要求更苛刻,他并不满足于把这个闲差职位变成全职的工作。就在到达弥漫着狂狷之气的造币厂和财政部之后,短短几个星期内,牛顿已成为这些机构中举足轻重的人物。尽管在生产高峰时期每天工作16个小时,牛顿仍然有时间和精力去取回本来属于他的职权——被他视为当仁不让的权力。

造币厂的总监尼尔并不是真正的阻碍,他极满意于他那终身任职的舒适工作,根本不在意牛顿逐渐加强自己的权责。在牛顿看来,多位前任厂长对工作的疏怠导致这个职位一度拥有的权力在逐渐失去。牛顿一生总是喜好权力与地位,他会去争夺或谋取,如今他既然握有了那一点儿权力,就会尽力再往上构筑。造币厂厂长的头衔只是踏脚石,是他攀登社会阶梯计划中的第一步。

...

为了达到这个目的,牛顿同时开始向两个方向行动。他先把旧的法令规章和政府文案找出来,以研究造币厂内部的职权划分。然后,他扩大了对职权的解释,只要曾经属于厂长职责的,他都不会放过。这里隐约可看见阿里乌的阴影作祟,他向“三位一体”的说法挑战时,即是先阅览一遍《启示录》,从《但以理书》中参悟出预言。此刻,他再度为需求寻找支持的证据,但这次不是为了圣言之战,而是为了权力之争。造币厂的文件中没有上帝,但确确实实有提高牛顿自我的东西。

他一找到事实真相之后,就立即将其呈送给财政部里受到蒙蔽的长官,向他们详细地解释造币厂的功能和组织,指出厂长应有的权力几十年来因无知而被剥夺。他甚至还研读了能拿到手的每一本经济学论著,吸取最先进的财政思想家的见解,包括布鲁斯特、朗兹、布瓦扎尔,以及他的朋友洛克。他特别珍惜由法国政府出版、装订成一套的《货币制作》,里面收录了180份官方文件。当后人整理牛顿的图书室、进行编目的时候,依工作人员所描述的书本状况,我们觉得它像是牛顿最常翻阅的书籍之一(在他去世后,人们归纳出他最常用的书籍中有31册属于经济方面的论著)。

就如早年追寻点金石和追究“三位一体”论为谎言的时候一样,牛顿一旦收集到足够有用的资料就开始写作,用他有限的空闲时间写满一页页的经济史、商业理论、各国的钱币制度及原理等。他将造币厂里的组织系统和已被混淆的权责关系用图表来说明,并仿效他的哲学记事簿和实验室记录的做法,在每一页上都列出了不同门类,例如在“关于造币厂的观察”的大标题下,又有“成分分析”“熔融技术”“制造钱币”等副标题。他雇用了几个抄写员整理他的草稿,并将他写好的每一篇文章都做了副本。据康杜伊特说,18世纪20年代,有一箱箱的文件被烧毁,其中有许多箱是牛顿与财政部的上司进行复杂争论时,所写的各种文件的副本。

可想而知,财政部当局一定被他的大量宣传品弄糊涂了,但当局并没有完全否认他所争取的权力。他们对尼尔的信任基本上可以确定已被新任厂长表现的高效率动摇了,因为很明显地,造币厂最近的成就和总监无关。牛顿也很可能得到了蒙塔古的勉励,蒙塔古则尽力为牛顿的许多请求铺平道路。

\hr 


\section{打击伪币}
在参考资料7中关于牛顿如何打击伪币和那段鲜为人知的侦探岁月有着详细的描述,这里就略过了。


\section{提出金本位制度}
牛顿爵士主要是因为出色地解决了前两个问题的工作而被封的爵士,但新币铸造计划却是一个注定失败的计划。后面牛顿就发现了不断地铸造新币,好像市面上仍然是旧币占绝大多数,而铸币厂后面却常常是连银子都找不到了。这在经济学上常称之为劣币驱逐良币现象。市场的力量是不容违背的,任何对抗市场规律的尝试,不管是谁,都将遭受惨败,这在历史上一次又一次的重演。

1717年,牛顿先英国议会提交了一份报告,分析了各国金银价格的情况,并提议将每盎司黄金的价格固定在3英镑17先令10.5便士,来解决前面提到的白银外流问题。这个提议被英国议会通过了,直到1816年英国议会通过《金本位制法案》正式在法律上宣告大英帝国的金本位制度。而在这段过渡时期,银子外流问题仍然存在,市面上流通的银币越来越少,所以牛顿的那段造币厂岁月让牛顿功成名就,但实际上从某种意义上来说这是一段失败的岁月。不过幸好晚年的牛顿没有骄傲自大,接受了这个失败,分析失败的原因,从而为金本位制的发展打下了地基。

货币是现代经济运行的基石,货币基础不打牢,之后的经济稳定运行和良好发展怕也是空中楼阁。此时的英国和其他国家,经济运行上一个基本情况就是货币供应严重不足制约了经济发展,确立金本位的反面意思就是放弃银本位,放弃历史传统的银币的银价值属性,这样为大众熟知的货币从价值含量载体更多地变成了一个价值符号,这就进一步让大众接受纸币打下了基础。只有纸币大行其道,才真正解决了现代经济的货币供应不足问题。并且随着纸币为大众接受,其他价值符号如债券股票等等才能被大众接受,如此现代金融才能继续发展下去。



\chapter{美洲早期殖民状况}
本章节主要参考自参考资料11:



\part{十八世纪}


\chapter{密西西比泡沫和南海泡沫}
\section{约翰·劳鼓吹国家发行纸币无果}
1705年约翰·劳出版了《论货币和贸易:兼向国家供应货币的建议》小册子,其中鼓吹了国家要繁荣就要发行纸币的观点。

引用自参考资料4:
\hr
当时苏格兰的经济正处于不景气时期,劳相信自己看到了问题的症结所在:经济不景气与货币有关。这本册子还提出了一种从未有过的说法——“货币需求”。劳试图向读者说明,由于货币供给量太少,所以货币的利率就太高。解决的办法就是增加货币供给量。他声称,扩大货币供给量能够降低利率,而且,只要国家以全部生产能力运行,就不会导致通货膨胀。

他还提出了另外一项建议:在苏格兰建立一家“土地银行”。该银行可以发行银行券,但发行银行券的价值绝不超过国家所拥有土地的价值。持有银行券的人可以获得利息,并且有权选择在特定时间将银行券兑换成土地。这个新的方案有两方面的优点:

\begin{itemize}
\item 它将减轻国家的负担,即避免为了适应经济增长而购买越来越多的贵金属来铸造钱币。
\item 它将使国家更容易管理流通中的货币量,以便适应国家需求的变化。
\end{itemize}


这个建议非常好,产生了很大的轰动效应,同时也引发了争议。批评者嘲弄这是一个“沙滩银行”,将会破坏国家的命脉。但是,另外有一些人则支持劳的想法,最后,议会也对这个问题进行了严肃的辩论。然而,事情也就到此为止,大部分议员最终还是拒绝了这个方案。约翰·劳对此感到非常失望,再加上他又得不到英格兰法庭对他过失杀人罪的赦免(当时的英格兰与苏格兰是两个不同的国家),于是,他又回到欧洲大陆。...

但在劳的心里,还一直挂念着有关纸币的想法。他确信欧洲的繁荣需要纸币信用。大约1708年,他在法国的法庭上向一位检察官提出了建立土地银行的计划。然而,这个建议再一次被拒绝。之后,他又在意大利作过尝试,结果同样被拒之门外。
\hr 

\section{法国政府正陷入严重的债务危机}
1715年法国的路易十四去世,此时路易十五年仅7岁,真正执政的是摄政王奥尔良大公。由于路易十四对珠宝与宫殿的兴趣,挥霍无度,法国的财政已经是摇摇欲坠。

此时法国国家债务20亿里弗尔,年财政收入1.45亿里弗尔,而此时的法国每年财政支出1.42亿里弗尔,这些债务每年需要支付利息9000万里弗尔,相当于贷款利率4.5\%。

为了解决债务危机,奥尔良大公决定让硬币缩水,他下令所有硬币都要回收到造币厂,并禁止使用旧硬币,然后替换为新的硬币,新币含金量只有旧币的80\%。这种做法不得民心,对国家财政也仅仅贡献了7000万里弗尔。

奥尔良大公又发布政令,如果有人举报腐败的政府官员,其定罪后,举报人可以获得罚没财产的20\%。这个政策是令人高兴的,政府查抄了1.8亿里弗尔,将1亿里弗尔拨给了新官员,这样还剩8000万里弗尔。

上面这两个办法政府获得的总收入为1.5亿里弗尔,也仅仅填补了国家债务的7.5\%,此时的奥尔良大公绞尽脑汁,也无计可施了。

\section{劳氏公司}
1716年,奥尔良大公接见了约翰·劳,和他讨论了施政的政策。约翰·劳再次重复了以前说过无数次的话:要繁荣,就需要纸币,而且这种纸币还应该是硬通货,不贬值,不缩水。他提议设立一家银行来管理王室的收入,这家银行所发行的银行券要完全由贵金属或者土地作为支撑,换句话说,这是改良的“土地银行”。结果,大公高兴地同意了。

\hr 
1716年5月5日,一家名为劳氏公司(Law \& Company)的银行创立了。银行从做担保业务开始,宣布所有税收都要用劳氏银行所发行的银行券缴纳,法国由此采用了纸币。

劳氏公司的资本为600万里弗尔,如果要购买其股份,需要用硬币支付其中25\%,其余75\%用行政债券支付。这是非常聪明的一步棋。行政债券是路易十四为给其巨额花销进行融资而发行的债券,这些债券在刚发行的时候售价为100里弗尔,现在则只值21.5里弗尔,因此被认为是垃圾债券。

债券的市场价值如此之低的原因当然是人们担心国家破产。然而,有一个可能的解决办法就是,政府按照当前很低的市场价格回购债券。如果这样做,政府实际上就可以将债务从20亿里弗尔(按政府出售债券时的价格算)减少到4.3亿里弗尔(按当前的市价算)。这样做实际上没有伤害到任何人!而且,此举有助于恢复信心。从原则上讲,政府还可以通过发行新的债券来支付债务利息——现在是按照4.5\%的利率来计算4.3亿里弗尔债务的利息,可见利息负担已大为减轻,新的利息负担大约仅为每年1900万里弗尔。

现在的问题是,政府如何设法把20亿里弗尔的垃圾债券收回,并且不至于抬高其市场价格。如果人们真认为菲利普·奥尔良会从价格挤压中摆脱出来,他们肯定就会提高债券的报价,这样一来,他的图谋就会失败。约翰·劳劝说人们专门用行政债券购买劳氏公司股票的办法,也只能解决一小部分问题。

在这个阶段,劳的“债务-股权互换”部分仅占债务总额很小的比例,剩余的政府债务还有18.5亿里弗尔。发售劳氏公司的股票所购回的行政债券仅为450万里弗尔,即600万里弗尔的75\%——相比20亿里弗尔的债务总额,几乎可以忽略不计。但是,劳已经有了下一步的计划,他又做了下面的三件事情:
规定劳氏公司银行券可以“见票即付”。这就是说,无论何时,只要你愿意来到劳氏公司,出示持有的劳氏公司银行券,都可以足额兑换硬币。

规定其银行券可以兑换旧币。如果政府采用缩水硬币(之前常常如此),那么约翰·劳仍然会支付原始含金量的硬币。

他公开宣称,任何银行家在发行银行券时如果没有足够的储备作为支持,就应该“受死”。

他这样做的结果是:新的纸币作为硬通货被接受,而且一开始的交易价格为101里弗尔,也就是说,与相同名义价值的硬币相比,还有1\%的溢价。这种可靠交易手段的出现,很快就刺激了贸易的发展:商业出现好转,对纸币的需求也与日俱增。不久,劳氏公司就在里昂、罗谢尔、图尔、亚眠和奥尔良开设了分支机构。

1717年,劳氏公司的纸币价格用硬币计价已经上升到了115里弗尔。

奥尔良大公开始对劳氏公司银行很感兴趣,他决定采用若干特许权——包括冶炼金银的唯一特权——进一步支持银行。他甚至还同意了一件从一开始就不愿意做的事情:将银行命名为“皇家银行”。很显然,这时他已经掌控了这家银行,而且高兴怎么干就可以怎么干。他之所以这样胆大妄为,是因为看到了以下四个方面的问题:

\begin{itemize}
\item 人们对纸币已经树立信心
\item 纸币是政府借款的“无痛方法”
\item 由于纸币处于溢价交易状态,很显然供给不足
\item 纸币似乎带来了繁荣兴旺
\end{itemize}

既然如此,为什么不多印发纸币呢?如果人们购买银行券来兑换硬币,大公就可以花费那些硬币!于是,他下令该银行印制10亿里弗尔的纸币——这超出了之前所印制纸币的16倍之多。这个命令遭到了大臣达古梭的反对,大公于是立即用更加听话的人取代了达古梭。约翰·劳对此感到了恐惧。
\hr 


\section{密西西比公司}
为了把剩余的行政债券消化吸收掉,约翰·劳进一步提出实施一项新的债务-股权互换计划。他建议奥尔良大公应该同意设立一家公司,这家公司获得与两个殖民地进行贸易的垄断权——这两个殖民地是1684年法国政府占领的密西西比河与路易斯安那州。在公司公开出售股权时,人们应该用行政债券购买,如此一来,国家的债务就消失了。大公对此提议非常兴奋,于是开始着手准备这个新的“密西西比计划”。

\hr
1719年年初,约翰·劳启动他的密西西比计划。新的密西西比公司的特许权得以扩大,包括:

\begin{itemize}
\item 在密西西比河、路易斯安那州、中国、东印度和南美享有贸易专权
\item 为期9年的皇家造币权
\item 为期9年的国家税负征收权
\item 烟草专卖权
\end{itemize}


除了这些,密西西比公司还获得了塞内加尔公司(The Senegal Company)、中国公司(The China Company)的全部财产,以及部分法国东印度公司(The French East India Company)的财产。随着法国东印度公司被控制,人们期望这个新的巨人能够挑战全能的英国东印度公司。

由于拥有了这些特权,不难想象,这家公司会创造出巨额的利润。公司被命名为“印度公司”后,又宣布增发价值2500万里弗尔的公众股票,从而使公司的总股本增加到了1.25亿里弗尔。约翰·劳对外宣称公司的预期红利能够达到5000万里弗尔,这就意味着投资年收益率达到40\%。然而,由于投资者是用“太阳王”的垃圾债券来购买股票的,所以实际上获得的投资收益率比40\%要高得多。

我们可以举一个例子来算一下,假如你购买价值100万里弗尔的股票:
\begin{Verbatim}
股票名义价格:100万里弗尔

预期年红利:40万里弗尔

用名义价值100万里弗尔的债券(折现率为0.2)购买的价格:20万里弗尔
\end{Verbatim}

因此,实际投资收益率竟然高达200\%!!!顷刻之间,申购股票的投资者蜂拥而至,股票被超额认购。公司职员要花几个星期的时间来整理认购人的名单。...越来越多的人挤满了坎康普瓦街,没过多久,人数就增加到好几千。这可不是一群普通的人,其中就有诸位公爵、伯爵和侯爵夫人,所有人都急切地想狠赚一笔。最终名单出来的时候,这次股票发行已经至少被超额认购6倍。而在自由市场上,股票价格急剧飙升到了每股5000里弗尔,是发行认购价格的10倍。劳和奥尔良大公决定好好利用人们的这种激情,于是又增发价值15亿里弗尔的股票,这次的发行规模达到了前两次的12倍之多。

\hr

\section{密西西比泡沫}
\hr
这次股票发行确实应该引起投资者的担忧。我们这样想一想:投资者投入的是垃圾债券,并没有新的资本——仅仅是利息——注入公司,但是,随着资本份额的扩大,相应的每股收益已经被大大稀释,仅为原来的1/13。

然而,公众对此并不担忧,导致如此巨额的股票发行仍然有3倍的超额认购。于是,极为离奇的事情发生了。尽管4年前法国还陷在深深的绝望之中,但仅仅过了4年,整个国家又开始沸腾起来,到处充满了喜悦与幸福。所有的奢侈品价格都开始上涨,花边缎带、丝绸、宽布和天鹅绒的产量翻了几番。工匠的工资涨了4倍,失业率也下降了,到处都在忙着建造新的房屋。每个人都看到价格在不停地上涨,谁都想赶在价格进一步大幅上涨之前去抢购物品,抢着投资,抢着囤积。

在巴黎,经济比其他任何地方都要热。据估计,在此期间,巴黎的人口增加了30.5万。街道上常常塞满了新的马车,挤得谁也动弹不了。巴黎从世界各地进口了大量的工艺品、家具和装饰品,这种情形还从未有过,消费者也不再仅仅由贵族构成,还包括新兴的中产阶层。那些购买了股票的人突然发现,区区几千里弗尔居然可以增长到100多万里弗尔。很快,法语增加了一个新的词汇——“百万富翁”。不过,最大的受益者还是贵族阶层。其中当然包括约翰·劳,此外还有他的朋友理查德·坎蒂隆,坎蒂隆当时是法国巴黎最成功的一位银行家。

约翰·劳、坎蒂隆以及他的兄弟伯纳德一起在密西西比购买了16平方里格的土地,并且招募了大约100位想淘金的移民到那里种植烟草。不久之后,伯纳德就和他的移民一起乘坐贩奴船起程了。当他到达那里的时候,才发现实际情况并不像先前在巴黎的沙龙里所描绘的那样,而是布满了荆棘与敌意——在接下来的4年里,他带来的人有3/4死于疾病或印第安人的袭击。

然而,这类故事需要经过一些时日才能传回故乡,所以巴黎的投机狂热并没有丝毫减轻的迹象。先前在萧条中受到残酷压榨的许多中产阶层人士,如今依靠对印度公司的股票投机得救了。波旁公爵就是其中之一,他在股票交易上赚了大笔的钱,这足够让他在尚蒂伊重建一座无比奢华的宅第。他的投机还让他能够从英格兰进口150匹精心挑选的赛马以及购买一大片土地。其他许多中产阶层人士也都发了大财,但最大的玩家之一还是理查德·坎蒂隆,他是劳的朋友与投资合伙人,在股票价格还很高的时候,手里便累积了大量的股票。

...

随着牛市的继续,在劳位于坎康普瓦街的房子外发生了一些稀奇的事情:整条街都变成了股票交易场所,挤满了针对印度公司股价变化进行投机买卖的投机商。股票经纪人与中间商在这条街上到处租房子,租金比通常的价格高出12~16倍,而且连酒吧与餐馆也改成了股票交易场所。随着投机商和金钱而来的,还有小偷与骗子。所以,派一群士兵到坎康普瓦街来维持夜里的治安,已经是见怪不怪的事情了。

最后,劳实在受够了外面喧嚣的噪音与拥挤,于是在宽敞的凡登广场旁找了一个新的住处。但是,他不能从这些人当中搬走,因为在这些人的眼中,他是所有活动的中心。对他们来说,他比历史上任何国王都要伟大,是最伟大的金融天才,他独自创造了一个国家的繁荣新景象。贵族们用大笔金钱贿赂劳的仆人,就是为了能成为劳的听众。无论何时他驾车外出,皇家骑兵都要在前开道,为他挡开那些崇拜者。那些投机商与股票经纪人必须清楚地了解他的一举一动,就像圣·西蒙在他的回忆录中所写的那样:

\begin{quote}
劳被那些信徒与野心家紧紧围绕着,有的人把他的屋门挤坏了,有的人从他的花园翻窗而入,还有一些人从他办公室的烟囱上爬了下来。
\end{quote}

...

因此,就像工蜂追随着蜂王一样,人们也紧紧跟随着约翰·劳。不久,他家门前的广场上又搭满了摊位与帐篷,凡登广场也变成了一个兴旺繁忙的集市,人们不仅在这里从事股票与债券的买卖,还做起了各种各样的生意。广场上一片喧嚣,这比先前在坎康普瓦街时还要糟糕。奥尔良大公听到了一些对这种乱七八糟的状况的抱怨,尤其是首席法官的抱怨。因为他主持的法庭正好也在凡登广场,外面噪音已经让他听不清律师的讲话。约翰·劳决定再找个新的地方,于是买下了苏瓦松酒店,这个酒店的后面有一个大花园。就在同时,法庭发出了明文规定,除了这个花园之外,禁止在其他任何地方进行股票交易。于是人群再一次蜂拥着跟了过来,酒店的后面立即搭起了500多个大大小小的帐篷。这一次,巴黎的每一个人,不论男女老幼,几乎都在投机买卖印度公司的股票,而这只股票正处于加速的牛市行情。故事还在继续,当时清醒的阿贝·特诺松与和他同样清醒睿智的朋友拉莫特相互庆贺彼此都没有卷入这场全民的疯狂。然而,几天过后,拉莫特禁不住诱惑,跑去买了一些印度公司的股票。但是,当他走进苏瓦松酒店的时候,迎面碰见从酒店里面出来的人是谁呢?当然是阿贝,他刚刚在市场上买进了股票。在这个插曲之后的很长时间里,他们在经常进行的哲学讨论中都避免谈及投机的话题。

与此同时,奥尔良大公还在通过皇家银行印发更多的纸币。为什么不呢?难道不是发行货币的做法使国家重新繁荣起来的吗?既然是这样,为什么不多印发一些货币呢?打个简单的比方,货币对于经济这部机器来说就像油一样,不是吗?油灌得越多,机器就会运转得越好!这对股票市场也同样有好处。印度公司的股票价格已经从初始的每股150里弗尔飙升到超过8000里弗尔。就在这一天,一位生病的投机商听到如此令人难以置信的价格后,就打发他的佣人去卖掉250股的股票。当佣人来到市场,他看到价格实际上更高,卖出的价格不低于每股10000里弗尔,这已是原始发行价的67倍了——股票价格已经令人惊异地飙升了6700\%。他回来的时候,交给了主人400万里弗尔的预期收入。然后,他回到自己的屋子,收拾好东西,卷起剩下的50万里弗尔,迅速离开了这个国家。

1720年年初的一天,非常奇怪的事情发生了。一个人拉着两马车的纸币来到皇家银行门前,他愤怒了,而且非常愤怒……孔王子相信自己有充分的理由愤怒。他想购买一些印度公司的新股票,但是劳没有同意。这个傲慢的苏格兰杂种!踢开他!王子愤愤地骂道。于是,他拉着满满两车的纸币来到银行门前,径直走进了大门。“瞧,先生们!你们的纸币,所谓‘见票即付’的纸币。现在,你们瞧见了吗?那好,给我换成硬币吧!”银行随即把纸币换成了硬币,装了两马车。奥尔良大公听到这件事之后,显然大为震怒,立刻命令孔把2/3的金属硬币退回了银行。事情仅此而已。后来公众便不喜欢孔了,而且谴责他不合情理的做法。但是,这个事件仍然产生了重要的影响:它在民众的心里播下了一点点怀疑的种子。如果有更多人都拿着纸币要兑换,那会是什么样子呢?如果所有人都拿着纸币去银行兑换呢?银行会有那么多的黄金吗?我自己是否要去兑换呢?!

在随后的几个月里,一些机敏的投机商开始从股票市场抽身,卷走收益,而一些股票价格在短暂地摸高到每股10000里弗尔的水平后便开始下滑。有一对兄弟俩,鲍登与拉·理查蒂埃尔,开始悄悄地拿着纸币到皇家银行去兑换,每一次兑换的数额都比较少。他们还开始尽量收购白银与珠宝,并且把白银、珠宝和硬币一起秘密地运到荷兰与英格兰。一位成功的股票交易商沃默雷特也完全卖空了股票,把价值100万里弗尔的金属硬币装进了马车。他在上面覆盖了干草与牛粪,自己假装成农夫,驾着马车跑到了比利时。许多人离开了法国,剩下的人对纸币也越来越没有信心,并秘密储藏金属硬币。人们要么把硬币藏在床垫下,要么就把它们运到国外,这样一来,法国的货币流通速度慢了下来。

在这种情况下,大公采取的措施实在不够高明。首先,他把纸币兑硬币的兑换价调高了5\%。显然,他第一步是想恢复信心,但是这对资本外逃毫无效果,于是他把兑换价又调高了5\%,但还是不见效果。

1720年2月,他干脆禁止使用硬币。在法国,任何人财产中的硬币价值不得超过500里弗尔,否则就有被罚没充公的危险。他还禁止收购白银、珍贵宝石和其他珠宝。任何举报收购这类贵重物品的人,将会得到罚没财产价值的一半,当然这一半的价值是要用纸币支付的。最后,大公在2月1日到5月底的这段时间里,又印发了价值15亿里弗尔的纸币,纸币的总供应量已经达到了26亿里弗尔。很显然,公爵采取所有这些措施的目的,就是要迫使人们继续使用纸币,然而,这些举措此时已经回天乏力,毫无效果。经济已经开始紧缩,人们心里充满了恐慌。法兰西的未来在哪里呢?这又该谴责谁呢?

约翰·劳。正是这个约翰·劳应该受到谴责。不是他最先编造了纸币的故事吗?他的密西西比计划又怎样了呢?人们在那边除了被蚊子咬死或是被印第安人杀死,还能干些什么呢?印度公司的股票真的比皇家银行的纸币还值钱吗?难道不是这样吗?最好还是把这些东西统统卖掉吧!于是,股票价格很快崩溃了,大约超过50万人亏了本,成千上万的投资者破产了。那些在股票投机上亏本的人还不了别人的钱。面对这样一条残酷无情的反应链,需要采取一些补救的措施,以使印度公司的股东们相信公司实际上仍然运转良好。补救的办法很简单:把巴黎最穷的人与罪犯征召起来送到新奥尔良为公司挖黄金。有6000多个“穷鬼”参加了这个计划,这支队伍推推攘攘地在巴黎的街道上游行,准备去码头,然后坐船到美洲去。起初,人们喜欢这个计划,因为6000名工人已经是一支很大的队伍了。如果他们能找到金矿,那么公司当然会顺利运转。如果能用这些黄金铸造新的硬币,那甚至可以让法兰西再次振作和繁荣起来。于是,有一段短暂的时间,印度公司在股票市场上又重整旗鼓了。

但是,古怪的事情发生了:那些在街上游行的人绝大部分根本没有离开这个国家。3个人中就有两个把配发的新衣服与工具卖掉了,根本没上船,而是回到了家里。在巴黎忍受贫穷也比到新奥尔良挖黄金强。很明显,这个密西西比冒险计划已经不可能实现人们曾经寄托的希望。劳和他的朋友理查德·坎蒂隆也就放弃了从他们合伙购买的那块土地上挣钱的希望。

坎蒂隆对此显得很从容,因为他正在做着另外一桩挣钱的大买卖。当银行大量发行货币的时候,他的反应并不像其他人,实际上他早就看到了法国货币迟早是要贬值的,所以他收回了所有钉住法国货币的贷款,而把收益投在了英国货币上。劳听说了这种情况,于是来到坎蒂隆的办公室,威胁他如果不答应在48小时之内离开这个国家,今晚你就会被送进巴士底狱。

坎蒂隆马上卖光了全部资产,大约净赚了2000万里弗尔,这确实是一笔巨大的财富。然后他火速离开了法国。

这时,印度公司的股价还在持续下跌,奥尔良大公变得绝望起来。很显然,他越是采取限制使用硬币的措施,人们越是想要持有硬币。他决定把皇家银行与印度公司合并起来,希望两者能够相互支撑。可是这也没有奏效。

1720年5月初,他召集了一个约翰·劳和所有大臣都参加的紧急委员会会议。在会议的日程中,首要的便是处理正在流通的价值26亿里弗尔的纸币,而这每一张纸币都可以从官方兑换金币和银币。实际的硬币数额还不到一半,而且多数已经被民众藏在了床垫下面(法国人的这个习惯在此后几个世纪里已是臭名昭著)。会议决定将纸币贬值一半,从5月21日起生效。这对法国民众的打击简直太沉重了。由于社会动荡的不断升级和反抗的威胁,仅仅过了一个星期,即5月27日,原来的法令就被取消了。也正是在这一天,皇家银行暂停支付金属硬币,而约翰·劳也被解除了职务。

然而,这天晚上,大公派人去请劳,劳从一个密道进了王宫。大公竭尽所能地安抚劳,说劳这次成为众矢之的,被民众憎恨,是如何不公平。过了两天,他邀请劳去歌剧院看演出,劳还带着家人一起来,好让每个人都看到他们一家和大公在一起。但是,这对劳来讲,几乎是一个致命的错误。他的马车刚到家门口,民众就用石头进行袭击。车夫驾着车迅速躲进了大门,佣人随即把门砰地关上,劳才免遭皮肉之苦。劳受了惊吓之后,大公派了一队瑞士卫兵日夜驻扎在劳的宅子里。即使这样,劳还是感觉不安全。很快,他搬进了王宫,和大公享受同样的保护。

大公现在完全隐退了。为了帮忙收拾混乱局面,他决定重新起用两年前被他解职的大臣达古梭。为了能够劝他回来救场,他派劳坐着邮政马车去面见达古梭。达古梭同意了,而且和劳一起回来了。很快,在6月1日,禁止自由持有硬币的法令被废除。也是在同一时间,价值2500万里弗尔的新纸币得以发行,这些纸币是用巴黎的税收作为支持的。6月10日,皇家银行重新开张,也作好了纸币兑换金属硬币的准备——但它们已经不全是以往的贵金属硬币,现在有一部分被换成了铜币!

历史上,铜的价格经历了好多次牛市,但这一次更是独特。在接下来的几个月里,总有一群人聚集在银行门前,每个人都要把纸币兑换成一堆铜币。有几次聚集的人太多,以至于有人被挤死。为了缓解压力,7月9日,士兵封锁了大门,于是外面的人就开始投掷石块。一个士兵开枪还击,打死了1人,还伤了1人。8天之后,又有15个人因挤压而毙命。人们被激怒了,他们用担架抬着3具尸体游行到了皇宫花园。在这里,他们发现了约翰·劳的马车,于是就把它砸得粉碎。

委员会不得不寻求新的解决之道。下一个紧急措施就是进一步扶持印度公司,公司贸易特权的范围将进一步扩大,以至于垄断法国所有的海上贸易。这样做将使数千名独立的商人丢掉生意,于是议会收到了一封接一封满是怨言的请愿书。议会因此否决了这个方案。大公对此恼羞成怒,就把议会和所有议员驱逐到偏僻的蓬图瓦兹。

8月15日,一道新的法令强加到了可怜的法国人身上。该法令规定,除了购买年金、存入银行账户或者购买分期付款的印度公司股票之外,不允许进行全部纸币价值合计1000~10000里弗尔的交易。10月,印度公司的许多特权被拿掉了,纸币也贬值了。股东们被迫与公司一道持有股票,而且,那些已经同意购买公司新股票的人还被强迫按照几乎是当时市场估价30倍的价格购买。许多人试图离开这个国家,以逃避这恐怖的惩罚。于是,所有的边防哨所都接到了命令,要求扣留任何想出境的人,直到弄清楚他们是否认购了印度公司的股票。那些已经设法出境的人则要因缺席被判处死刑。

1720年法国有效货币供给下降有三个主要原因:

\begin{itemize}
\item 资本外逃。人们携带金币和银币离开法国。
\item 货币流通(速度)下降。人们因不相信纸币而储藏硬币,随后可能由于对每个人持有硬币数额的限制,人们更是竭尽所能地保存硬币。
\item 银行信用降低。法令强制规定,价值合计1000~10000里弗尔的所有纸币,只能用来购买债券、印度公司股票和存入银行账户,这就减少了有效货币供给。
\end{itemize}

约翰·劳现在整天生活在恐惧之中,他成了法国最遭憎恨的人。离开了皇家庇护所,他要么隐姓埋名,要么得找到一个强大的保护队伍。他请求搬到一个乡下庄园去,大公对此求之不得。几天后,他收到了大公的回信,大公在信中展现了仁慈,并且还允许他离开法国——如果他想离开的话。大公还同意送给他一笔钱,想要多少都可以,他恭敬地婉谢了大公的好意。随后,就在开启这场冒险旅程5年之后,他只带了一颗大钻石,离开了法国前往威尼斯,这一年他49岁。
\hr

\section{南海泡沫}
南海泡沫也就比密西西比泡沫稍晚几个月爆发,将它们放在一起将最基本的逻辑就是密西西比泡沫爆发后资金从法国逃离到英国,进一步推动了南海泡沫。而专门研究金融史的书籍比如参考资料5将它们统称为1720年泡沫,认为这场泡沫的本质是跨大西洋贸易股份公司泡沫。除此之外南海泡沫和密西西比泡沫还有其他共性。

英国政府也被不断增加的巨额公共债务紧紧地缠住,其解决问题的措施也与法国类似。“南海公司”接管了偿付政府债务的义务,作为回报,它被授权垄断与南美的贸易。

前面提到的理查德·坎蒂隆,成功从密西西比泡沫破裂之前逃离到英国之后,又将资金投入到了南海公司中。1720年6月,南海公司的股价达到了历史顶峰,再一次,理查德·坎蒂隆又一次成功从顶部逃离。而在接下来的3个月里,股价下跌幅度达到了85\%,它也像法国的印度公司那样崩溃了。

许多投资南海公司的人是靠借钱来购买股票的,由于股票价格的崩溃,他们也失去了偿付债务的能力。于是造成银行倒闭的恐慌,结果拖累了很多金融机构的经营,导致了违约高潮的出现。

英国南海公司的结局又怎么样呢?它最终在1855年解散,其股票转换成了债券。在南海公司存续的140年时间里,它从来没有在南海做过什么辉煌的贸易。


\section{编者按}
参考资料4在讨论中有意将密西西比泡沫和约翰·劳的鼓吹纸币联系起来,这是不对的。南海泡沫之前的英国已经建立了一套运转良好的纸币制度了。1720年泡沫中的法国、英国、荷兰各有各的差异性,这些国别的差异性显然不是1720年泡沫的本质部分,我们应该关注和分析的是1720年泡沫中的那些跨越国别的经济中的共性问题。

【这里后续将会更深入的讨论】


\chapter{国富论}
1776年


\part{十九世纪}
1837经济大恐慌

\part{二十世纪}
1907银行危机

1929-1933股市大崩溃

1987黑色星期一

1994墨西哥金融危机

1997亚洲金融危机


\part{二十一世纪}
2008金融危机

\chapter{聪明的投资者}
本章节主要引用自参考资料10:本杰明·格雷厄姆编写,贾森·兹威格评论。后面将简称为格雷厄姆和兹威格。

\section{巴菲特第四版序}
1950年年初,我阅读了本书的第一版,那年我19岁。当时,我认为它是有史以来投资论著中最杰出的一本。时至今日,我仍然认为如此。

要想在一生中获得投资成功,并不需要顶级的智商、超凡的商业头脑或内幕消息,而是需要一个稳妥的知识体系作为决策基础,并且有能力控制自己的情绪,使其不会对这种体系造成侵蚀。本书能够准确和清晰地提供这种知识体系,但对情绪的约束是你自己必须做到的。

...

巴菲特在怀念恩师格雷厄姆的文章中提到,在即将80岁之际,格雷厄姆向一位朋友表达了他的想法:希望每天都做一些“傻事、有创造性的事和慷慨的事”。

刘建位对这人生三事也是投资三事点评到:

1. 做些傻事——防御型分散投资,定期定额投资指数基金。

巴菲特说道:通过定期投资指数基金,一个什么都不懂的投资人通常能打败大部分专业经理人,很奇怪的是,当傻钱了解到自己的极限之后,它就不再傻了。

2. 做些趣事——进攻型做价值投资,强调安全边际。

格雷厄姆本人的投资策略以买便宜股票为主,巴菲特早期完全模仿格雷厄姆,不过后来他发现,便宜货股票越来越少了,于是转向购买相对不受市场追捧的大公司股票的策略;后来巴菲特受查理·芒格的投资策略影响较深,比如芒格最有名的那句话是:股价公道的伟大企业比股价超低的普通企业好。

巴菲特说道:我们寻找的是一个具有持续竞争优势并且由一群既能干又全心全意为股东服务的人来管理的企业。当发现了具备这些特征的企业,而且我们又能以合理的价格购买时,我们几乎不可能出错。


3. 做些慷概的事——帮助别人,成就自己。利人利己。

巴菲特在怀念恩师是说道:我是以一个学生、一个雇员和一个朋友的身份认识格雷厄姆的。无论从哪一种关系来看(在他所有的学生、雇员和朋友看来),格雷厄姆在自己的想法、时间和精力等方面都表现出了毫无保留的慷慨。如果想寻求一种明晰的思维,那么没有比格雷厄姆更好的人选了。而且,如果需要获得鼓励和忠告,就可以随时去找格雷厄姆。

沃尔特·李普曼曾经说起过那些为后人栽树的人。格雷厄姆就是这样的人。

\section{兹威格写的格雷厄姆生平简介}
本杰明·格雷厄姆何许人也?为什么我们要倾听他的建议?

格雷厄姆不仅是他生活的那个时代最佳投资人之一,而且是有史以来最伟大的实践投资思想家。在格雷厄姆之前,资金管理活动在很大程度上就像中世纪的行会,为迷信、臆测和神秘的巫术所左右。格雷厄姆的《证券分析》作为一部教科书,把这个乌烟瘴气的圈子转变成了一种现代职业。

而格雷厄姆的这部《聪明的投资者》,则是有史以来第一本面向个人投资者并为其提供投资成功所需的情绪框架和分析工具的专业图书。至今,它仍然是面向投资大众最好的一本书。《聪明的投资者》是我在1987年作为一名涉世未深的记者加入《福布斯》杂志后阅读的第一本书。格雷厄姆坚信,任何牛市最终都将遭遇惨败;这一说法给了我极大的冲击。那年10月,美国股票市场遭遇了有史以来最大的单日暴跌,而我也被套牢其中。(如今,在经历了20世纪90年代末期狂猛的牛市和2000年年初开始的熊市之后,《聪明的投资者》读起来显得比以往更具预见性。)

格雷厄姆深刻的见解来之不易:它来自于他本人投资失败的惨痛经历,以及对市场心理学数十年历史的孜孜研究。1894年5月9日,格雷厄姆出生于伦敦,当时名叫本杰明·格罗斯鲍姆(Benjamin Grossbaum);他的父亲是一个瓷器餐具和小塑像经销商。在格雷厄姆一岁时,他们举家迁往纽约。起初,他们的生活很优越:有一个女佣、一名厨师和一个法国女管家,并且住在第五大道北区。但格雷厄姆的父亲于1903年去世了,其瓷器生意亦摇摇欲坠,一家人的生活也逐渐陷入了困顿。格雷厄姆的母亲把自己的家变成了廉价的寄宿公寓,然后借钱进行股票的“保证金”交易。1907年的股灾把她的本金一扫而光。格雷厄姆有时会回想起那段屈辱的日子,当他的母亲兑付支票时,银行出纳员会讥讽地问道:“多萝西·格罗斯鲍的信用能值5美元吗?”

幸运的是,格雷厄姆赢得了哥伦比亚大学的奖学金,他的才华也在这里结出了硕果。1914年,格雷厄姆以全班第二名的成绩毕业。在他的最后一个学期,该校有三个系(英语系、哲学系和数学系)邀请他担任教职。当时,他只有20岁。

格雷厄姆并没有从事教学生涯,而是决定到华尔街闯荡一番。起初,他加入了一家债券交易公司担任文员,旋即成为一名分析师,然后是合伙人。不久以后,他开始经营自己的合伙投资机构。

如今互联网股票的暴涨暴跌并不会令格雷厄姆感到惊讶。1919年4月,他在Savold轮胎公司上市的第一天大赚了250\%。当时,正是汽车企业大受追捧的时代。同年10月,该公司爆出欺诈丑闻,其股票变得一文不值。

格雷厄姆变成了一个对股票进行微观研究(甚至是细致研究)的大师。1925年,他与美国的州际商业委员会合作,对石油管道公司晦涩难懂的年报进行了深入研究。他发现,北部管道公司拥有至少价值每股80美元的高等级债券,而当时该公司的股价仅为65美元。(他买进了该只股票,敦促该公司的管理层提高分红率,然后在3年以后以110美元卖掉了该股。)

尽管在1929~1932年大萧条期间,格雷厄姆的亏损接近70\%,但他闯过了这一关,并在其后的岁月卷土重来,在大牛市的废墟上收获了大量有利的交易。格雷厄姆的早期收益记录如今已经散失,但从1936年起直至1956年退休,他的格雷厄姆-纽曼公司的年收益率不低于14.7\%,高于同期股票市场12.2\%的整体收益率——这一成绩可以跻身于华尔街有史以来最佳的长期收益率之列。

格雷厄姆是怎样做到的呢?凭借非凡的智慧,敏锐的判断力,再加上丰富的经验,格雷厄姆建立了一套自己的核心原则。这些原则至今仍然适用,一如其在格雷厄姆的时代:

- 股票并非仅仅是一个交易代码或电子信号,而是表明拥有一个实实在在的企业的所有权;企业的内在价值并不依赖于其股票价格。

- 市场就像一只钟摆,永远在短命的乐观(它使得股票过于昂贵)和不合理的悲观(它使得股票过于廉价)之间摆动。聪明的投资者则是现实主义者,他们向乐观主义者卖出股票,并从悲观主义者手中买进股票。

- 每一笔投资的未来价值是其现在价格的函数。你付出的价格越高,你的回报就越少。

- 无论如何谨慎,每个投资者都免不了会犯错误。只有坚持格雷厄姆的所谓“安全性”原则——无论一笔投资看起来多么令人神往,永远都不要支付过高的价格——你才能使犯错误的几率最小化。

- 投资成功的秘诀在于你的内心。如果你在思考问题时持批判态度,不相信华尔街的所谓“事实”,并且以持久的信心进行投资,你就会获得稳定的收益,即便是在熊市亦如此。通过培养自己的约束力和勇气,你就不会让他人的情绪波动来左右你的投资目标。说到底,你的投资方式远不如你的行为方式重要。

...

第一次乃至第三次或第四次阅读格雷厄姆的这部杰作,你都会有一种兴奋和愉悦的感觉,这令我非常羡慕。像所有的经典著作一样,本书会改变我们看待世界的方式;而且,通过教育我们,本书也能够获得新生。你读得越多,就理解得越深刻。以格雷厄姆为向导,你必定会成为一个非常聪明的投资者。

\section{导言}
本书的目的,是为普通人在投资策略的选择和执行方面提供相应指导。相比较而言,本书很少谈论证券分析的技巧,而将注意力更多地集中于投资的原理和投资者的态度方面。然而,我们也会对一些特定的证券进行简单的比较(主要以一对一的形式,对纽约股票交易所上市的股票进行比较),从而以具体的形式,让人们理解普通股选择中涉及的一些重要内容。

我们将以很大的篇幅讨论金融市场的历史演变模式,有时还要追溯到几十年以前的陈年往事。要想聪明地进行证券投资,你必须事先对不同的债券和股票在不同条件下的表现有足够的知识,至少其中某些条件会在一个人的经历中反复重演。对于华尔街来说,没有哪一句话比桑塔耶纳的告诫再真切和适用不过了:“忘记过去的人,必将重蹈覆辙。”

本书的内容是面向那些和投机者有区别的投资者的,而我们要做的第一件事,就是阐明并强调这一几乎被人们遗忘了的区别。我们首先要指出的是,这并不是一本教人“如何成为百万富翁”的书。在华尔街,就像在其他任何地方一样,并没有一条可靠和简单的致富之路。最好是以一点金融史的内容,来说明我们刚才提出的观点——尤其是因为,我们可以从这一点历史中得到更多的教训。在股市投资狂热的1929年,一位享誉华尔街乃至全美国的大人物约翰·拉斯科布曾经在为《女士之家杂志》(Ladies'Home Journal)撰写的一篇题为“每个人都应该成为富人”的文章中,为资本主义的美好前景大唱赞歌。他的说法是:如果你每月储蓄15美元,并将其投资于某一只优质的普通股,同时将其红利用于再投资,那么,20年后你累计投入的3 600美元将变成80 000美元。如果通用汽车这样的企业巨头真能一路走好,这确实不失为一条简单易行的致富之路。这一建议有多大的正确性呢?我们对此进行了一番粗略的估计:以道琼斯工业平均数的30只成分股为投资标的,如果按照拉斯科布的办法,在1929~1948年间进行投资,那么,1949年年初你将拥有8 500美元。这笔钱比这个大人物所允诺的80 000美元要少得多,它也说明,任何乐观的预测和保证,都是多么的不可信。但是,同时我们也可算出,这一投资的实际回报折合为年复合收益率后高达8\%以上;考虑到该投资是在道琼斯工业指数为300点时开始买进,而1948年年底其投资截止日的点数仅为177点,这一收益率就显得更不容易了。这一记录表明,不管市场如何,每月定期买入优质股的这种原则是很有说服力的,这种方案被称为“美元成本平均法”(dollar-cost averaging,也叫“定期定额投资法”——译者注)。

既然本书并不是写给投机者的,因而对于短线投资者并无意义。大多数投机者都是根据走势图或其他大致机械的方法,来决定买入或卖出的恰当时机的。几乎所有这些“技术方法”均采用这样的原则:因为股市上涨而买进,同时因为股市下跌而卖出。这种做法是与其他商业领域的合理经营原则背道而驰的,而且很难在华尔街取得长久的成功。根据我们自己长达50余年的市场经验和观察,我们从来没有发现过一个依据这种“追随市场”的方法而长期获利的投资者。我们可以大胆地认为,此种方法无疑是荒谬的,虽然它仍然十分流行。随后,我们将通过简要分析著名的股市交易道氏理论来说明我们的这种观点,当然,这不能被看成是一种证明。

自从1949年第1版问世以来,我们对《聪明的投资者》这本书的修订大约每5年进行一次。在这一修订版中,我们要对1965年版问世以来出现的许多新情况进行分析。其中包括:

1. 高等级债券利率的空前上涨。

2. 截止到1970年5月,一些龙头股的价格下跌了35\%。这是近30年以来的最大跌幅(大批低质股的跌幅更大)。

3. 批发和零售物价的持续上涨,即使在1970年出现经济整体衰退的情况下,物价上涨的势头还在增强。

4.“综合性”企业、特许经营以及其他较为新颖的商业和金融模式的快速发展(其中包括某些带有欺诈性的手段,如存信股、大量出现的股票期权、误导性的名称、利用外国银行,等等)。

5. 美国最大铁路公司的破产,许多以前强大而稳固的大公司的短期和长期债务过多,乃至华尔街的一些机构所面临的令人烦恼的清偿能力问题。

6. 投资基金,包括某些银行经营的信托基金,由于普遍开始追逐“业绩”而导致的一些令人担心的后果。

我们将对这些现象进行仔细考察,其中有些现象还会改变以前版本的一些结论和侧重点。稳健投资的基本原理是不会随着年代的更替而改变的,但这些原理的应用,则必须随着金融机制和金融环境发生重大变化而作出相应调整。

在这一版写作之时,上面的最后一句话得到了检验,本次修订版的初稿完成于1971年1月。在此期间,道琼斯工业指数从1970年的最低点(632点)开始强劲反弹,并于1971年达到了最高点(951点);与此同时,整个市场出现了乐观情绪。1971年11月,在本修订的最后一稿完成之时,市场正在经受着新一轮下跌的阵痛——指数已经跌到了797点,人们再次对市场前景感到惶惶不安。我们一直没有让这种波动影响我们对稳健投资策略的总体看法,自从本书初版于1949年问世以来,这些策略并没有发生重大的改变。

1969~1970年间的市场下跌,有助于驱散在过去20年间逐渐形成的一种幻觉:在任何时间以任何价位买入大盘蓝筹股最后肯定都能够获利,其间发生的任何损失都会随着市场的再创新高而得到弥补。这种说法未免有些夸张。从长期来看,股票市场最终会“回归正常”,这意味着,无论是投机者还是股票投资者,都不得不准备承受其股票市值的大幅缩水乃至长时间的被套;反之亦然。

对于许多二线乃至三线股,特别是那些新上市的股票来说,上一次市场崩盘带来的损失是灾难性的。这并不是什么新鲜事儿,1961~1962年股市下跌造成的损失,在程度上亦与此相当。但这一次也有一些新情况:某些投资基金大量地介入了这种高度投机且价值明显高估的股票。虽然热情在其他行业是一项必不可少的品质,但在华尔街却总会招致灾难;显然,这一警告并非仅仅只适用于那些新手。

高等级债券利率的大幅上升,也是我们必须加以讨论的一个重大问题。自1967年年底以来,投资者从这种债券所获得的收益,相当于一般普通股股息的两倍多。在1972年,最高等级债券的利息高达7.19\%,而工业股的股息仅为2.76\%。(1964年年底,这两种收益率分别为4.4\%和2.92\%)。令人难以置信的是,在本书第一版出版的1949年,这两个数字几乎完全相反:债券的收益率仅为2.66\%,而股息则为6.82\%。在本书的上一版,我们曾多次指出,对于保守的投资者来说,其股票投资比例至少应为25\%;一般来说,这两种证券的投资比例应各占50\%。鉴于目前债券利息远高于股票红利的现实,我们现在必须考虑是否应将债券投资的比例扩大为100\%,直至这两种投资工具的收益回归合理的比率——就像我们预期的那样。显然,持续的通货膨胀对我们的这一决策具有重要影响。我们将专辟一章来讨论这一问题。

以前我们曾把本书所面向的投资者分为两个基本类型:“防御型”和“进取型”。防御型(或被动型)投资者的首要目的是避免重大错误或损失;其次则是不必付出太多的努力、承受太大的烦恼去经常性地作出投资决策。进取型(或积极型或激进型)投资者的主要特点是,他们愿意为挑选合理且更具吸引力的股票而付出时间和精力,以获取超出平均水准的回报。经过几十年的耕耘,这种进取型的投资者可以期望他们的额外努力和技能有一个相应的回报,并且有比被动型投资者更高的平均回报。在现今的形势下,我们对积极的投资者是否能获得相当的超额收益,确实颇感怀疑。但明年或许多年以后,情况会有所不同。因此,我们将继续对进取型投资的可能性予以关注;这些可能性过去曾经存在过,今后也可能会再度出现。

长期以来,有一种流传甚广的观点认为:成功的投资技巧首先在于找出未来最有可能增长的行业,然后再找出其中最有前途的公司。例如,精明的投资者或其精明的顾问很早就会发现整个计算机行业,尤其是IBM公司,巨大的增长潜力。同样的情形也适用于其他诸多成长性行业,以及其中的成长性公司。但事先发现这些行业和企业,并不像事后看到的那样简单。为了一开始就说明这一点,我们不妨回顾本书1949年头一版中的一段话:

这种投资者也许会买进——比如说——航空股,因为他认为,该行业的前景会比其当前的市场估值更为看好。对这种类型的投资者而言,本书的价值与其说在于其介绍的备受推崇的投资技术,毋宁说在于它对这种投资方法潜在危险的警告。

事实证明,这种危险在我们提到的行业表现得尤为突出。当然,我们很容易地算出,航空运输量会在未来数年获得长足的增长。正因为如此,航空股亦成为投资基金的最爱之一。但是,尽管该行业的业务收入不断增长,其速率甚至高于计算机行业,但由于技术问题,再加上产能的过度扩张,该行业的利润十分不稳定,有时会非常糟糕。1970年,尽管该行业的运输量创下了新高,但却为其股东带来了2亿美元的亏损。(1945年和1961年,该行业同样也出现过亏损。)与此相应,这些公司的股价在1969~1970年再次出现超出市场整体水平的跌幅。这一业绩记录表明,即使是拿着高薪的全职投资基金专家,也会把这样一个并不奥妙的重要行业的短期未来完全搞错。

另一方面,虽然投资基金对IBM公司投入了相当多的资金,并取得了不菲的收益,但是,其股价过高,再加上其未来的增速不确定,从而使得基金在这家业绩极佳的公司的投入还不到3\%。因此,他们对这只表现优异的股票的投资,并不能使其整体业绩增加多少。此外,他们对IBM公司以外的许多(如果不是大多数的话)计算机公司的投资,似乎并不赚钱。从这两个实例中,我们的读者可以得出如下两条教训:

1. 某一行业显而易见的业务增长前景,并不一定会为投资者带来显而易见的利润。

2. 即使是专家,也没有什么可靠的方法,能使其挑出前景光明的行业中最有前途的公司,并将大量的资金投入该股票。

在笔者的基金经理生涯中,从未遵循过这种方法,因此,我不能向那些企图尝试此种方法的人士,提供任何具体的建议或鼓励他们这样做。

那么,本书的宗旨究竟何在?本书的目的在于,指导读者避免陷入严重的错误,并建立一套令其感到安全放心的投资策略。我们将以较大的篇幅讨论投资者的心理问题。因为,实际上,投资者的最大问题甚至是最可怕的敌人,很可能就是他们自己。(“亲爱的投资者,问题不在我们的命运,也无关乎我们的股票,而在于我们自己。……”)近几十年来,这一点尤其得到了事实的证明。因为,即使是保守型的投资者,也不得不更多地投资于普通股,因此,必然会受到股市的刺激和诱惑。通过说理、举例和劝告,我们力图使读者在其投资决策方面,形成一种恰当的心智和情绪。我们已经看到,那些情绪适合于投资活动的“普通人”,比那些缺乏恰当情绪的人,更能够赚取钱财,也更能够留住钱财,尽管后者拥有更多的金融、会计和股票市场知识。

此外,我们希望读者能够建立度量或量化的观念。对于99\%的股票而言,我们都可以发现,它们在某些价位相当便宜,值得购买;在另一些价位上则过于昂贵,应当抛出。将所付出的与所得到的进行比较,这种习惯是投资方面的一种宝贵特征。许多年前,我们曾在一本妇女杂志中劝告读者,购买股票要像购买食品杂货一样,而不要像买香水一样。过去的几年(以前也发生了许多类似的情况),我们之所以会在股票投资中遭受惨重的损失,都是因为我们在买股票时忘了问一声:“它价值几何?”

1970年6月,这个“价值几何”问题的答案就是9.4\%这个奇妙的数字,即新发行的高等级公用事业公司债券的收益率。这一收益率目前已降至7.3\%,但即使这一收益率仍然值得我们去问:“为什么要给出其他的答案呢?”但是,还存在其他一些可能的答案,因此我们必须对它们加以认真的考虑。此外,我们要再次指出的是,无论是我们自己还是我们的读者,都必须事先考虑一些与此完全不同的情况,比如,1973~1977年可能出现的情况。

因此,我们将较为详细地提出我们的普通股投资方略,其中部分内容适合上述两种类型的投资者,另一些内容则仅适合进取型的投资者。奇怪的是,我们将建议我们的读者,只买那些价格不高于其有形资产价值太多的股票,并以此作为我们的第一项要求。这种看上去有些过时的建议,是出于实践和心理两方面的考虑。经验告诉我们,尽管有许多成长性突出的企业的价值数倍于其净资产,但这种股票的买家会过分地受制于股票市场的变化和波动。与此相反,那些以大致接近净资产价值买进——比如说公共事业公司——股票的投资者,则总是可以把自己视为稳健和成长企业的权益拥有者,而不管股票市场对此有什么不同的看法。这种保守策略的最终效果,可能会超过极其兴奋地涉足于预期增长十分看好的危险行业所获得的结果。

投资艺术具有一种并不广为人知的性质。普通投资者只需付出很小的努力和具备很小的能力,就可以取得一种可靠(即便并不壮观)的成果;但是,要想提高这一可轻易获得的成果,却需要付出大量的努力和非同小可的智慧。如果你只想为你的投资计划付出一点额外的知识和智慧,却想取得大大超出一般的投资成果,你很可能会发现自己陷入一种更糟糕的境地。

既然任何人均可通过买入并持有一批代表性的股票,取得相当于市场平均水平的成绩,那么,“超越平均水平”似乎就是一件相当容易的事情。但实际上,那些试图这么做的聪明之士的失败比率却相当高。多年以来,大多数投资基金也不能击败市场,尽管它们拥有经验丰富的专家。与此同时,证券经纪公司所公布的股市预测结果也不能够令人满意,因为强有力的证据表明,它们精心预测出的结果,还不如简单的掷硬币方法可靠。

在本书的写作过程中,我们一直试图把这种基本的投资陷阱牢记在心。我们一直在强调简单证券组合策略(购买若干高等级债券的同时,持有一组多样化的龙头股)的优点——只要得到专家的一点帮助,任何投资者都可以这样去做。任何越出这一合理而安全区域的投资冒险,均会遭遇许许多多难以逾越的障碍,尤其是性格方面的障碍。在尝试这种风险投资之前,投资者及其顾问必须考虑清楚,尤其是,是否能够准确区别投资和投机,以及股票的市场价格与内在价值。

深思熟虑的投资方法,是稳固建立在安全边际原则基础上的,这种方法能够为我们带来可观的收益。但是,在缺乏大量自我检验的情况下,不是确保防御型投资的收益而贪图这种可观回报的决策不要做。

最后,我们将以如下回顾来结束这篇导言。当年轻的作者在1914年6月投身于华尔街时,他对未来半个世纪将发生何种变化一无所知。(华尔街甚至没有猜到,第一次世界大战将于两个月后爆发,并且会迫使纽约股票交易所暂时停业。)目前(1972年),我们发现自己已经成为世界上最富有和最强大的国家,但仍然面临一系列重大问题的困扰;而且,对未来多有忧虑,而不是更具信心。然而,如果我们集中关注美国的投资经历,仍然可以从过去的57年中获得一些安慰。尽管经历了与地震一样无法预测的波折和事故,但是这一点是不会改变的:稳健的投资原则一般会带来稳妥的结果。在以后的行动中,我们仍然必须坚持这些原则。

读者须知,本书针对的不是储蓄者和投资者所面临的总体财务策略;它只针对那些准备投入交易(或流通)证券的资金,即投入债券和股票的资金。因此,诸如储蓄和定期存款、储蓄和贷款协会账户、人寿保险、年金以及不动产抵押和股权投资等各种重要的理财方式均不在本书的讨论之列。读者还应记住,本书中的“现在”一词指的是1971年年底或1972年年初。

\section{导言点评}
请注意,格雷厄姆开门见山地指出,本书并不会告诉你如何去战胜市场。任何讲真话的书,都不会这样去做。

相反,本书的真正目的,是教你掌握以下3个重要的投资方法:

- 如何使你的投资亏损的概率最小化。

- 如何使你获得持续收益的机会最大化。

- 如何约束你的自我挫败行为,这种行为使大多数投资者无法完全发挥自己的潜能。

让我们回到牛气冲天的20世纪90年代后期。技术股一路高歌猛进,似乎每天都在翻番;所谓“你可能几乎会赔光自己所有的钱”的说法似乎很荒诞。但到了2002年年底,许多网络股和电信股的市值损失了95\%甚至更多。一旦你的亏损达到95\%,你必须挣得1900\%才能回到原来的起点。承担愚蠢的风险,会使你坠入几乎无法走出的深渊。这就是格雷厄姆(不仅在本书第6章、第14章和第20章,而且贯穿全书的)反复强调避免损失的重要性的缘由。

但是,无论如何小心谨慎,你投资购买的证券的价格仍将会不时走低。虽然无人能够根除此种风险,格雷厄姆却可以告诉你如何管理它,以及如何控制你的恐惧。

\subsection{你是一个聪明的投资者吗}
现在,我们来回答一个至关重要的问题:格雷厄姆所说的“聪明的投资者”,到底是什么意思?在本书的第一版,格雷厄姆曾给出过定义,并且明确地指出,这种品质与智商(IQ)或学术能力评估考试毫不相干。它的确切含义只是,要有耐心,要有约束,并渴望学习;此外,你还必须能够驾驭你的情绪,并进行自我反思。格雷厄姆解释说,这种智慧“与其说是表现在智力方面,不如说是表现在性格方面。”

事实证明,高智商或较高的学历,并不足以使投资者变聪明。1998年,长期资本管理公司(由一群数学家、计算机专家以及两名诺贝尔经济学奖获得者管理的对冲基金)进行了一场豪赌,赌债券市场将回归“正常”状态,结果在几个星期内就损失了20多亿美元。但是,债券市场却仍然朝着不正常的方向发展。由于长期资本管理公司以前曾经借入过巨额的资金,因此,它的倒闭几乎使全球的金融体系倾覆。

让我们回到1720年的春天。当时,牛顿爵士拥有一些英国最炙手可热的南海公司股票。看到股票市场正在失去理智,这位伟大的物理学家声称,他“可以计算出天体的运动,却无法揣摩人类的疯狂”。牛顿清空了所有南海公司的股票,获利7 000英镑,回报率达100\%。但仅过了一个月,在市场狂热情绪的感染下,牛顿又以高得多的价格买回了这只股票,结果赔了20 000英镑(换算成现在的货币价值,大约相当于300万美元)。此后终其一生,他都不许任何人再在他面前提及“南海”二字。

按照绝大多数人对于“聪明”一词的定义,牛顿肯定是有史以来最聪明的人士之一。但按照格雷厄姆的说法,牛顿与聪明的投资者还有很大的差距。由于被群体的狂热蒙蔽了自己的理智,这位世界上最伟大的科学家的行为就像是一个傻瓜。

简言之,如果你未能在投资领域取得成功,这并不是因为你愚笨,而是因为你像牛顿那样,没有为自己建立投资成功所需的心理约束。在第8章,格雷厄姆介绍了如何通过驾驭自己的情绪、避免陷入市场的非理性狂热来扩展自己的智慧。那时你将真正体会到,要想成为一个聪明的投资者,重要的是性格而不是智力。

\subsection{一系列的灾难}
本小节罗列了很多股市崩盘或公司暴雷或者政治黑天鹅事件,这些都略去了,因为这样的事件不计其数,此起彼伏,没有特殊性,根本不需要单独列出来,每个进行股票投资的人只要想去了解一些,随便往上回溯几年,这样的事件一抓一大把。

金融活动和工厂生产活动一样,安全问题永远是最重要的问题:

安全问题至关重要,安全问题至关重要!


\subsection{未曾兑现的确定之事}
本小节罗列了很多基金经理的大放厥词的言论,同样这些话语不具有特殊性,不需要单独列出来,这里就略去了。读者只要上网,充斥着各种这样的言论,很多还带上了权威或者专业的光环,你再细心观察,就会发现从超级乐观到超级悲观,从超级多头到超级空头,各种矛盾的言论层出不穷,你感到困惑,这是很正常的,因为正是所有这些声音构成了市场,但最好将所有的这些声音都称之为市场的噪音。

在投资上你不能成为一个人云亦云的人,否则你将遭受惨败!



\subsection{困难时刻要看到光明}
如果说人们在20世纪90年代似乎觉得所有的股票价格都不高,到2003年时,则觉得所有的股票价格都不低了。正如格雷厄姆一向指出的那样,钟摆又摆回来了——从非理性的兴奋,摆向了毫无根据的悲观。在2002年,投资者从股票型基金抽回了270亿美元的资金;另据证券业协会的调查,已有十分之一的投资者,将其股票投资减少了25\%以上。那些在20世纪90年代末疯抢股票——当时其价格正一路走高,因而越趋昂贵——的人,开始随着股价的一路走低而抛售股票,而在我们看来,此时的股价已变得较为低廉了。

正如格雷厄姆在本书第8章中生动描述的那样,这种做法完全是南辕北辙。随着股价的走高,聪明的投资者会认识到风险在加大,而不是减少;反之,股价的走低,也会使风险随之减少,而不是增加。聪明的投资者在牛市中反而会忧心忡忡,因为它会使股票变得昂贵;相反(只要你手上持有足够应付日常生活所需的资金),你应该欢迎熊市,因为它会把股价拉回到低位。

所以,不要沮丧:牛市的结束并不像人们认为的那样是一个坏消息。由于股价下跌,现在已经进入一个相当安全和理性的财富积累时期。继续往下读,让格雷厄姆告诉你如何去做。


\section{投资与投机:聪明投资者的预期收益}
本章将简要介绍其余各章要论述的主要观点。尤其是我们希望在本书的一开始,就针对个人及非专业投资者,确立恰当的证券组合策略这一概念。

\subsection{投资与投机}
何谓“投资者”?在本书中,这一说法始终是与“投机者”相对应的。早在1934年我们撰写的那部《证券分析》教科书中,就试图准确地定义两者之间的差别:“投资操作是以深入分析为基础,确保本金的安全,并获得适当的回报;不满足这些要求的操作就是投机。”

尽管在随后的38年中,我们一直坚守这一定义,但是值得注意的是,在这一时期使用“投资者”这一术语时,情况发生了重大改变。在1929~1932年的市场大崩溃之后,所有的普通股都被看成是投机性的。(一位知名的权威人士直言不讳地宣称,只有购买债券才能称之为投资。)因此,那时我们不得不为自己的定义进行辩护,因为人们认为,我们的投资定义过于宽泛。

现在,我们却不得不做相反的事情。我们必须告诫读者不要受流行语的影响,把股票市场上的每一个人都视为“投资者”。

...

[现在]在华尔街,所有的证券交易者均被简单地冠以[投资者]这一称呼,而无论他们买的是何种证券,出于何种目的,以何种价格,也不论他们是在做现金交易,还是在做保证金交易。与此相对照,在1948年时,公众对普通股的态度则与此大相径庭。当时,有90\%以上被访者不赞成购买普通股。大约一半人给出的理由是,买股票“像赌博一样,不安全”;也有一半左右的人的理由是,“对此不熟悉”。具有讽刺意味(但并不令人意外)的是,就在公众普遍认为所有的股票投资均具有高度投机性和巨大风险之时,股票价格其实相当有吸引力,而且,很快就会出现有史以来的最大涨幅。相反,当按过去的经历来看,股票价格已被推升到危险的高度时,买进股票反而被称为投资,所有购入股票的大众反而被称之为“投资者”了。

区分普通股买卖中的投资与投机总是具有重要意义的。因此,这种区别的消失会引起人们的担心。我们经常指出,整个华尔街系统应当不断重申这一区别,并在与公众的交往中对此反复加以强调。否则,人们早晚会把惨重的投机损失归咎于股票买卖,那些承担此种损失的人,未曾得到过适当的警示。另外,具有讽刺意味的是,许多证券公司最近的窘境,似乎正是由于它们再次把一些投机性很强的股票纳入到自己的资本金之中。我们相信,本书的读者将对普通股买卖所隐含的风险获得相当清晰的了解,这种风险与股票的高盈利机会是密不可分的,因此,在投资核算中必须同时考虑这两个方面。

以上论述表明,那种买入一组代表性普通股的纯正投资策略已经不复存在了。这种策略认为,人们总可以在没有令人担心的市场或“报价”风险的情况下购买股票。大多数时期,投资者必须认识到,所持有的普通股经常会包含一些投机成分。自己的任务则是将投机成分控制在较小的范围,并在财务和心理上作好面对短期或长期不利后果的准备。

就股票投机本身而言,需要补充两段话——股票投机有别于包含在大部分代表性普通股中的投机成分。直接的投机并不违法,也与道德无关,而且(对大多数人而言)也不会充实你的腰包。此外,某些投机是必然的和不可避免的,因为就大多数普通股而言,其赚钱和亏损的机会同在,所以必须有人去承担这些风险。就像投资一样,投机也可以是明智的。但在很多时候,投机并非明智之举,尤其是在下列情况下:(1)自以为在投资,实则投机;(2)在缺乏足够的知识和技能的情况下,把投机当成一种严肃的事情,而不是当成一种消遣;(3)投机投入的资金过多,超出了自己承担亏损的能力。

根据我们较为保守的观点,任何从事保证金交易的非专业人士都应认识到,事实上他是在进行投机,而且其经纪人有义务对此加以提示。任何抢购所谓“热门”股票或有类似行为的人也是在投机,或者说是在赌博。投机总是令人兴奋的,如果你能成为赌场上的赢家,其乐趣尤其妙不可言。如果你想试试运气,不妨拿出一部分资金——越少越好——并为此单开一个账户。千万不要因为市场的上涨或利润的激增而加大对该账户的投入。(此时,应考虑把资金撤出投机账户。)不要把你的投机操作与投资操作放在一个账户中进行,也不能在思想上将二者混为一谈。

\subsection{防御型投资者的预期投资成果}
我们已经将防御型投资者定义为关心资金安全同时又不想多花时间和精力的人。那么,总体上他们应遵循何种路线,并且在“一般正常条件”下(如果这种条件确实存在的话),可以期望获得什么样的投资成果呢?要回答这一问题,我们首先来看7年前对这一话题的讨论,然后我们将分析影响投资者预期收益率的基本因素在后来发生了哪些重大变化,最后,我们介绍在当今(1972年早期)条件下,投资者应当做些什么并抱有怎样的预期。

格雷厄姆建议防御型投资者将资金分配于高等级债券和蓝筹股上;其中,债券所占的比例不低于25\%,且不高过75\%,而股票的比例则与之相适应。最简单的选择则是,两者各占一半,并根据市场情况的变化进行小幅(比如5\%左右)调整。另一种策略是,当“感觉市场已处于危险的高位时”,将股票持有比例减少到25\%;并在他“感到股价的下跌已使其吸引力与日俱增时”,将持股量提升到最大限度,即75\%的比例。

其次第二个调整的思考点就是高等级债券的利率,格雷厄姆从历史数据出发说明了高等级债券利率波动幅度也是很大的,在某个时候债券利率相当诱人,这个时候就应该适当提高资金中债券占比。但即使在投资债券显得比投资股票更有利可图的时候,格雷厄姆仍然建议防御型投资者保留一定的股票资金占比,因为通货膨胀加速、美国企业利润大增、股市出现巨大投机风潮等还可能其他想不到的原因,都可能导致投资者后悔把100\%的资金投入债券,尽管其当前的利率较股票更具吸引力。

...

我们将毫不犹豫地重申(因为对这一告诫如何重申也不过分):投资者不能期望通过买任何新股或“热门”股(那些被人们认为可以迅速致富的股票),而获得优于平均水平的收益。从长远来看,这几乎无疑会产生相反的结果。防御型投资者只应购买那些长期具有盈利记录和强有力的财务状况的重要公司的股票。(任何称职的证券分析师都可以为他开列这样一份清单。)激进型投资者可以购买其他类型的普通股,但它们的吸引力一定要建立在理智分析的基础上。

在本节的结尾,我们将向防御型投资者简要介绍三个相互补充的概念或做法。首先,他可以买入一只地位稳固的投资基金的股份,以代替自己亲自构建股票组合的做法。他也可以利用许多州的信托公司或银行经营的“共同信托基金”(common trust funds)或“混合基金”(commingled funds)。如果资金规模较大,他也可以聘用一家知名的投资咨询公司,从而使其投资得到按标准程序进行的专业化管理。第三种做法是采取“美元成本平均法”,即每月或每季度投入同等数额的美元资金来购买股票。采用这种方法,投资者可以在市场低迷时买到比市场高位时更多的股票;而且,他还有可能为自己所持股份最终获得满意的总体价格。严格地说,这是所谓的“程式投资”(formula investing)这种更一般方法的应用。我们建议投资者将其资金投资于股票的比例限制在25\%~75\%之间,并根据股市的动向进行反向操作,这种策略正是“程式投资”原理的体现。这些想法对防御型投资者很有益处,在以后的章节中,我们将对此进行更详细的讨论。


\subsection{激进型投资者的预期投资结果}
我们的激进型投资者当然会期望自己取得比防御型或被动型投资者更好的收益。但是,他必须首先确定,自己不会收获更糟的结果。我们经常会看到,投入更多的精力、进行大量的研究同时具备很好天赋的人,在华尔街不仅没赚到钱,反而亏损了。如果用力的方向是错误的,这种力量就好比是一种障碍。因此,激进型投资者必须首先搞清楚,什么样的行动方针能提供合理的成功机会,什么样的行动方针是无法成功的。

首先让我们来看一看,投资者乃至投机者为取得优于平均水平的成果而经常采取的几种方法,其中包括:

1. 择时交易。这通常是指在股价上升时买进股票,而在其掉头向下时抛出。他们选择的大多数是那些“表现”优于市场平均水平的股票。少数专业投机者经常会进行卖空交易。他们会卖出他们并不拥有而是从股票交易机构借入的股票,其目的是在其随后下跌时以更低价格把它们买回来,并从中获利。

2. 短线择股。买进那些已经报告或预计将报告业绩增长,或有其他利好消息的公司股票。

3. 长线择股。此种方法看重公司过去拥有的良好成长纪录,而且这种成长很可能会延续到未来。有时,“投资者”也会选择那些尚未取得优良业绩,但预计以后会形成高盈利能力的公司。(这些公司往往属于高技术行业,如计算机、医药、电子等,而且它们通常正在开发一些被认为是大有前途的新工艺和新产品。)

从总体上来看,投资者能否通过以上方法获得较大的成功机会呢?我们已经对此表示过否定。无论从理论抑或实践角度来看,第一种方法都算不上投资。股票交易并不是“经过彻底分析,就能够保证本金的安全并获得满意回报”的一种操作。在以后章节中,我们将对此给予更多的论述。

至于那些力求发现最有前景的股票的投资者,无论其着眼于短期还是长期,他们都会面临两重障碍:首先,人总是会犯错误;其次,人的竞争能力有限。他也许会错误地预测未来;即使其判断无误,当前的市场价格也许已经充分体现了他的这种预测。就短线择股而言,公司当年的业绩已经成为华尔街众所周知之事了;公司次年的业绩——如果能够预测的话——也已经被人们仔细考虑过了。因此,那些主要根据公司当年的优秀业绩或被告知的来年预期增长率进行股票选择的投资者,会发现其他人也在基于同样的理由做同样的事。

基于长期前景选择股票的投资者,也会面临同样的障碍。其预测完全错误的可能性(我们已在前文以航空业的例子作过说明),无疑会比那些只根据短期业绩行事的投资者更大。由于专家经常会在此种预测上误入歧途,因此,从理论上来说,如果投资者能在华尔街整体出现错误时作出正确预测,那么肯定能大赚一笔。但是,这只是一种理论上的可能。估算未来的长期利润是专业分析师最喜欢的游戏。那么,有多少激进投资者可以指望自己在聪明才智和预测能力上超过专业分析师呢?
由此,我们将得出以下合乎逻辑但令人不安的结论:要想能够持续并合理地获得优于平均业绩的机会,投资者必须遵循以下两种策略:(1)具有内在稳健性和成功希望的策略;(2)在华尔街并不流行的策略。

对于进取型投资者而言,这样的投资策略是否存在?从理论上来说,其答案仍然是肯定的;从实践上来看,我们也拥有很多理由对此给出肯定的答案。众所周知,投机性股票的价格往往会走过头——就市场整体而言,经常是这样;对某些个股而言,则任何时候都是这样。此外,某些股票会因为无人关注或毫无根据的普遍偏见而受到低估。我们还可进一步断言,就相当多的股票交易而言,其交易者似乎并不知道(说得文雅一点)最简单的区别。本书中,我们将通过许多(以往的)实例来说明,股票的市场价格与其价值之间是有差价的。由此看来,似乎任何具有良好计算能力的聪明之士都能在华尔街通过别人的愚蠢行为而稳定地获得一笔收益。但这只是表面上的东西,实际上并没有这么容易。要想通过买入一只受到忽略因而被低估的股票赚钱,通常需要长期的等待和忍耐;而卖空一只过去热门因而被高估的股票,则不仅是对卖出者胆略和毅力的考验,而且也是对其财力的考验。这种投资原则是稳健的,其成功的运用也并非是不可能的,但它绝不是一种可以轻易掌握的技术。

由于存在很多的“特殊情况”,因此,驾轻就熟的老手只需承担最低的风险,即可以期望在许多年内获得20\%以上的年收益率。这包括不同证券的交叉套利、资产清算、某些种类的保护性对冲等。其中最典型的是即将发生的并购,并购方往往会提出比其公告日股价高出相当幅度的报价。近年来,这种类型的交易急剧增加,为这一领域的行家带来了丰厚的利润。但随着并购交易的成倍增加,其面临的障碍也成倍增长,并导致许多交易流产。不少人因而在这种曾一度稳赚不赔的操作中赔钱。或许,过多的竞争也已使这种交易的整体利润下降了。

这些特殊情况下盈利能力的下降,似乎说明存在着某种(类似于收益递减法则的)自我衰减过程,自本书出版以来,这一过程一直在不断发展。1949年,我们可以对此前75年内的股市波动进行研究。该研究证实,根据利润和当期利率得出的一个公式,可以用来确定买进和卖出道琼斯成分股的价格水平(在价格低于其“中心”价值或“内在”价值时买入,高出时卖出)。这正是罗思柴尔德行为法则的应用:

“贱买贵卖”,该法则与华尔街一贯奉行且遗祸无穷的法则——追涨杀跌——是完全相反的。令人惊奇的是,1949年以后,这一公式已不再适用了。第二个例证是关于股票市场运行的“道氏理论”。该理论在1897~1933年间确实取得了辉煌的成果,但在1934年以后的表现则颇受质疑。

关于这种绝佳机会近年已不再存在的第三个也是最后一个例证,来自我们自己在华尔街的证券投资:其中相当一部分投资集中于购买廉价证券(bargain issues),这些证券的特征是,其售价低于净流动资产(营运资本)本身的价值(不包括厂房等其他资产,但要扣除求偿权优先于该股票的所有债务)。显然,这种证券的卖价远远低于其作为一家非上市企业的价值。任何私人业主或大股东,都不会在如此荒谬的低价位出售其拥有的股份。奇怪的是,这种异常的股票并不难找。从1957年的一份上市股份清单中可以看到,当时市场上有近200只这样的股票。通过不同的实际操作,这些廉价证券最终都是有利可图的,而且其平均年收益率会远远高于其他方式的投资。但在接下来的10年内,这些股票也消失了,同时,进取型投资者因而也丧失了一块可以进行明智和成功操作的领地。不过,在1970年的股市低点,这种“低于营运资本”的股票再次大批涌现了,虽然后来出现了强劲的反弹,但年底股市上仍有不少此类股票,足以构成一个完整的投资组合。

在当今形势下,进取型投资者仍然可以取得优于平均水平的投资成果。从逻辑标准和合理的可靠性标准来看,大量上市交易的证券中一定会包含相当多被低估的证券。从整体上来看,这些股票的收益率将比道琼斯指数或其他一组代表性的股票更令人满意。我们认为,除非投资者能从中获得比平均水平高出5\%以上的税前收益,否则是不值得费心费力去发掘此类证券的。我们将力求为积极投资者设计出更多类似的选股方法。





\section{第1章点评}
为什么你会认为,无论当天的市场表现如何,每当交易结束的钟声响起时,纽约股票交易所的场内经纪人总会心情愉悦?因为只要你进行了交易,他们就会赚钱——不论你是否赚到了钱。一味投机而不是投资,你将减少自己的致富机会,同时增大别人的机会。

关于投资的定义,格雷厄姆说得再明白不过了:“投资操作是建立在透彻分析的基础之上的,目的是要保证本金的安全并获得适当的回报。”请注意,根据格雷厄姆的定义,投资包括以下三个同等重要的因素:

\begin{itemize}
\item 在你买进一只股票之前,要对该公司及其基础业务的稳妥性进行彻底分析。
\item 你必须细心地保护自己,以免遭受重大损失。
\item 你只能期望获得“适当的”业绩,不要期望过高。
\end{itemize}


投资者会根据公司的业务状况计算一只股票的价值;投机者则会打赌股票价格的上涨,因为他们认为,其他人会出更高的价格来购买这只股票。正如格雷厄姆曾经说过的,投资者会“根据公认的价值标准”来判断“股票的市场价格”,而投机者则是“根据市场价格来确定价值标准”。对于投机者来说,连续不断的报价好比是氧气,“切断它就会出人命”;而投资者对股票的价格走势则不那么看重。格雷厄姆敦促你,只有在下列情况下才去进行投资:即使不知道股票每日的价格,你对所持有的股票也是放心的。

就像在赌场下注或赌马一样,股票市场的投机也是令人兴奋甚至有利可图的(如果你运气够好的话)。但这是一条最糟糕的致富之路,因为华尔街就像拉斯维加斯或跑马场一样,已经把输赢的概率确定在有利于庄家的刻度上,因此,它最终能够在自己的投机游戏中战胜任何试图取胜的人。

另一方面,投资是一种独特的赌博游戏,只要你按照有利于自己的规则去参与,最终就不会赔钱。投资者是为自己赚钱,而投机者则是为其经纪人赚钱。这就是为什么华尔街总是一贯贬低脚踏实地的投资,而吹捧华而不实的投机的原因所在。

\subsection{高速行驶的危险}
格雷厄姆警示我们,任何时候把投机与投资混为一谈都是错误的。在20世纪90年代,这种混淆造成了巨大的破坏。所有人似乎都失去了耐心,美国则变成了一个投机王国。急不可耐的交易者就像是夏日田野里呼啸的蝗虫,从一只股票跳到另一只股票。

人们开始认为,投资技术的试金石仅仅在于其是否“管用”。如果某种技术能够在某一时期战胜市场,它就是“正确的”,而不论其危险有多大或是多么愚蠢。但聪明的投资者并不计较一时的对错,要达到你的长期财务目标,必须一直坚持正确的做法。20世纪90年代盛行一时的策略——短线交易、忽视投资的分散化、追逐热门基金、按选股“系统”操作——似乎是正确的,但这些东西不可能获得长期的成功,因为它们不完全符合格雷厄姆为投资制定的三项标准。

为了说明这种短暂的高收益不能证明什么,让我们想象两个相隔130英里的地方。如果我遵守时速65英里的限速规定,我会用2个小时行驶完这段路程。但如果我以130英里的时速行驶,一个小时即可。假如我按照后一种方法做了,并且没有出事故,我是否“正确”呢?因为听到我夸耀它“管用”,难道你也忍不住想试一试吗?那些关于战胜市场的浮华噱头大多也是如此:就短途旅行而言,如果你够幸运,它是管用;但从长期来看,它会要了你的命。

1973年,当格雷厄姆最后一次修订《聪明的投资者》一书时,纽约股票交易所的年换手率为20\%,即股东的平均持股时间为5年。到了2002年,年换手率达到了105\%,平均持股时间缩短到11.4个月。早在1973年,共同基金的平均持股期接近3年;到了2002年,其持股期减至10.9个月。这就好比基金经理琢磨了好长时间,最后发现他们原来不该购进这些股票,然后立刻将其倾销一空,并从头开始买进新的股票。

...



\subsection{模式化操作的失败}
但是,像热锅上的蚂蚁一样急匆匆地进行交易,并非投机的惟一形式。在过去的10来年,一个又一个投机模式被相继推出,流行开来,然后被扔进垃圾堆。所有这些模式均具有一些共同的特点——这样赚钱快!这样赚钱轻松!这样不会赔一分钱!它们至少违背了格雷厄姆在区分投资与投机时所提出的某一项标准。以下是曾经流行一时的投机模式:

\begin{itemize}
\item 赚特定日子的钱...

\item 《华尔街的有效策略》...

\item 按照“傻瓜四部曲”进行操作...

\end{itemize}


这些事例表明,在华尔街只有一种东西从来不会陷入熊市,那就是愚蠢的想法。这些所谓的投资方法,都将输给格雷厄姆的法则。这些旨在赢得超额收益的机械方法,都将沦为“一种自我毁灭的过程,就像收益递减规律那样。”收益的递减是由于以下两个原因:首先,如果此种方法纯粹出于统计上的侥幸(如“傻瓜四部曲”那样),单凭时间的推移,就可以证明它原本就是一种没有意义的方法;其次,如果此种方法过去确实有效(如一月效应),随着广为人知,市场精英们总是会削弱其未来的效力,而且通常会使其完全失效。


所有这些都进一步验证了格雷厄姆的警告,即在进行投机时,你必须像有经验的赌徒走进赌场时那样:

\begin{itemize}
\item 投机就是投机,千万不要自以为是在投资。
\item 如果把投机看得太认真,它就会变得十分危险。
\item 你必须严格限定你的赌注。
\end{itemize}


就像精明的赌徒只带(比方说)100美元下赌场,而把其余资金都锁在旅店的保险箱里一样,聪明的投资者只会把其很少一部分资金拨入“赌资”账户。对于我们大多数人来说,总资产的10\%已经是我们甘愿冒投机的风险上限了。永远不要把你的投机账户与投资账户混为一体;永远不要在思想上把投机与投资混为一谈;无论如何,永远不要使投入“赌资”账户的资产超出10\%。

不管怎样,赌博总是人类天性的一部分。因此,对大多数人来说,甚至对其稍加抑制都是一种徒劳之举。但是,你必须限制和约束它。为了确保自己永远不会把投机与投资相混淆,这种约束是惟一最好的办法。



\section{编者按}
如果读者想要在本书中找到一种策略,这种策略可以确保投资者能获得超过市场平均水平的收益率,读者可能要失望了。格雷厄姆首先肯定了激进的投资者是可能获得高于市场平均水平的收益率的,但这绝非易事,每高出一个百分点就多一个百分点的不确定性或者说风险,投资者必须正视清楚这里面的投机成分,如果投资者从这份冒险中不能获得高于市场平均水平5\%以上的收益率,那么没有必要去冒这个险。

继而格雷厄姆否定了一系列的投资策略,甚至包括自己曾经使用的但已经基本失效的策略【烟蒂股策略】。这其中一些策略实际上现在还有很多投资者在乐此不疲地使用着。这其中追涨杀跌不能称之为策略,可以说有某种人的本能本性在里面,因此有很多投资者会陷入而不是使用这种策略。

至于业绩预测策略,如果目的只是确保本金安全并获得适当的回报,那是没有问题的。但非常遗憾,大部分投资者进行业绩预测都在期盼着某个更高的收益回报。格雷厄姆否定业绩预测策略不是说分析公司业务等等各方面细节不好,而是否定靠着业绩预测策略能够获得高于市场平均水平的收益回报。因为在股市里,额外认知外的风险就代表着额外的收益或亏损,已经确定的事情都已经反映在定价里了。

同样对股票标的进行估值,比如市盈率等,是证券分析中很重要的一个工作,但如果期望简单通过股票估值来建立一套投资策略,期望获得高于市场平均水平的收益率,投资者可能要失望了。

兹威格继而对历史中曾经出现的一些模式化操作策略进行了批判,我对这些文字进行了精简,因为这些策略显得就更加的愚蠢了。

\section{第2章点评}
通货膨胀?谁还理会这个?

毕竟,从1997到2002年,商品和服务价格的年上涨率只有2.2\%;而且,经济学家认为,实际通货膨胀率也许还要低一些。近几年来,美国的实际年通货膨胀率也许只有1\%左右,涨幅如此之小,因此许多权威人士甚至开始宣称,“通货膨胀已经死亡了”。

\subsection{货币幻觉}
投资者之所以忽视通货膨胀的重要性,还有另一个原因,即心理学家所说的“货币错觉”。如果你的年收入增加了2\%,而同年通货膨胀率为4\%,你肯定会觉得,这比收入减少2\%而通货膨胀率为零的情形要好。实际上,这两种变化导致的结果是一样的,扣除通货膨胀因素,你的生活水平下降了2\%。只要名义(或绝对数)的变化为正数,我们就会觉得是一件好事,即使实际(或扣除通货膨胀因素后)的结果是负数。较之一般物价水平的变化,你对自己薪资收入变化的感觉要更加明显和具体一些。同样,投资者在1980年拿到利率达11\%的银行定期存单(CD)时,会感到由衷的喜悦;而在2003年看到利率仅为2\%左右时,则会心生沮丧,尽管扣除通货膨胀因素后,前者的实际利率是负数,而后者的利率则与通货膨胀相当。我们所获得的名义利率,是银行在各种场合公布出来的利率,这一利率较高时,会令我们感觉良好。但通货膨胀会悄悄蚕食我们的高额利息;通货膨胀没有大张旗鼓地显示出来,但却拿走了我们的财富。这就是人们为何会轻易忽视通货膨胀的缘由。因此,衡量你是否投资成功的尺度,并不是你挣了多少,而是在扣除通货膨胀的影响后,你还剩下多少。

更要紧的是,聪明的投资者必须始终提防那些意外的和估计不足的事情。以下三个理由足以令我们相信,通货膨胀并没有死亡:

\begin{itemize}
\item 就在不久之前的1973~1982年间,美国经历了有史以来最严重的通货膨胀。以消费价格指数来衡量,该时期物价水平上涨了一倍,年均上涨近9\%。仅1979年的通货膨胀率就高达13.3\%,从而使美国经济陷入了所谓“滞胀”,而且,许多权威评论家开始质疑,美国是否还拥有全球市场竞争力。1973年年初,价值100美元的产品或服务,到1982年年末变成了230美元,就是说,一美元的购买力变成了不到45美分。凡是经历过这段时期的人,无不对自己的财产损失感到痛心;所有谨慎之士,都不能不提防此种情况的再次发生。
\item 自1960年以来,69\%的市场经济国家都至少经历过一年通货膨胀率高达25\%及以上的时光。从总体上来看,这些通货膨胀使投资者的购买力损失了53\%。我们当然希望美国不会经历此种灾难;但如果认为此种灾难已一去不复返,那可就大错特错了。
\item 价格的上涨,可以让山姆大叔(意指美国政府——译者注)以贬值的美元来偿付其债务。彻底根除通货膨胀,与任何经常举债的政府的经济自利倾向是背道而驰的。
\end{itemize}


\subsection{部分保值}
那么,聪明的投资者应该怎样做,才能抵御通货膨胀呢?通常的回答是:“购买股票”;但是,经常如同普通的答案一样,这种说法并不完全正确。


\begin{figure}[H]
\centering
\includegraphics[width=\linewidth ,totalheight=0.95\textheight , keepaspectratio]{股票总回报率与通货膨胀率.jpg}
\caption{股票总回报率与通货膨胀率}
\end{figure}

上图显示了1926~2002年间,各年的通货膨胀与股票价格之间的关系,它不是按日历顺序,而是按通货膨胀率从低到高的顺序逐年排列的。当通货膨胀率为负数时(见图最左端),股票的表现很差;当通货膨胀较为温和时(大部分时段均如此),股票通常表现良好;但是当通货膨胀率达到非常高的水平时,股票的走势则起伏较大,经常会出现10\%以上的亏损。从图的左边可以看到,在消费品和服务价格下跌的各年,股票收益相当糟糕——股票总市值的下降幅度,最多可达43\%。如果通货膨胀率超过6\%,股票走势亦欠佳,如图右边部分所示。通货膨胀率超过6\%的情况出现过14年,其中8年股票市场的收益是负数;这14年的平均收益仅为2.6\%。

虽然温和的通货膨胀可以使公司把原材料的新增成本转移给消费者,但恶性通货膨胀则会造成灾难。它迫使消费者节衣缩食,并使经济各个环节的活动受到抑制。

历史给出的结果是明白无误的:自准确的股票市场数据在1926年出现以来,我们一共获得了64个五年期的数据(1926~1930年,1927~1931年,1928~1932年,等等,直到1998~2002年),其中有50个五年期(占总数的78\%)的股票收益,超过了同期通货膨胀率。这确实不错,但并不完美,因为在这一时期大约五分之一的时段,股票的收益没有赶上通货膨胀的速度。

\subsection{两种补救方式}
幸运的是,你可以在股票之外寻找防御通货膨胀的工具。在格雷厄姆最后一次修订本书之后,出现了两种为投资者普遍采用的通货膨胀保值工具:

REITs。这是“不动产投资信托”(Real Estate Investment Trusts)的缩写(其读音等同于“reets”),指那些拥有商业和住宅房产,并收取租金的公司。通过与房地产共同基金结合,REITs在通货膨胀保值方面干得相当不错。其中最佳的选择是先锋REIT指数基金,其他低成本的选择还有Cohen \& Steers Realty Shares、哥伦比亚房地产产权基金和富达不动产投资基金等。虽然REITs基金并不是一种十全十美的通货膨胀保值工具,但从长远来看,它多少能保护你免受购买力下降之苦,同时又不影响你的总体收益。

TIPS。这是“通货膨胀保值国债”(Treasury Inflation-Protected Securities)的缩写,它是美国政府于1997年首次发行的一种债券,其利率会随着通货膨胀率的上升而自动增加。由于有国家担保,所有的国债都不存在逾期不还(或不能支付利息)的风险。但TIPS还能保证你的投资价值不会受到通货膨胀的侵蚀。通过这样一种简便易行的方式,你就可以确保自己不会遭受购买力下降带来的财务损失了。

...


\section{第4章点评}
你的投资组合应当承担多大的风险?

格雷厄姆的见解是,这首先取决于你是何种类型的投资者,而不是取决于你拥有怎样的投资品种。要成为一个聪明的投资者,有两种做法:

\begin{itemize}
\item 对一组由股票、债券和共同基金构成的动态投资组合,进行不断的研究、筛选和监控。
\item 或者,以某种自动的方式,创建一个恒久的投资组合,不再付出更多的努力。
\end{itemize}

格雷厄姆把第一种做法,叫做“积极的”或“进取的”的方法,它需要投入大量的时间和精力;而“被动的”或“防御型的”投资策略,无须花费多少时间,但要求投资者始终不为市场喧嚣所动;正如投资思想家查尔斯·埃利(Charles Ellis)表明的那样,积极的方式是劳心费力的,而防御型的方式则要求控制好自己的情绪。

如果你时间充裕,具有高度的竞争性,像一个球迷一样乐此不疲,而且对智力挑战颇有兴趣,那你不妨采用积极的路线。如果你总是觉得太过匆忙,渴望简单的生活,且不愿为金钱操心,则较适合被动的投资方式。(有些人也许更愿意把这两种方式结合起来,从而创建一个以积极为主被动为辅的投资组合;反之亦然。)

这两种方式同样明智,无论采取何种方式均可取得成功。但前提是,你必须对自己有深入的了解,从而采用适合自己的方式,并在自己的整个投资生涯中坚持下去,而且善于控制自己的投资成本和情绪。格雷厄姆对主动投资和被动投资的区分再次提醒我们,财务风险并非只存在于大多数人所关注的地方(经济形势和投资品种),而且也存在于我们的内心。

\subsection{是勇猛出击,还是防守}
那么,防御型投资者应当怎样入手呢?首先,而且是最基本的决策是,确定股票投资与债券和现金的分配比例。(请注意,格雷厄姆将这部分论述放在通货膨胀的章节之后,是为了让你事先了解,通货膨胀是你面对的其中一个最危险的敌人。)

最突出的一点是,格雷厄姆关于股票和债券资产分配的讨论中,根本没有提到“年龄”这一字眼。这使他与时下流行的庸俗看法区别开来。后者认为,你所承担的投资风险,主要取决于你的年龄。有一种传统的经验公式认为,你的股票投资所占的百分比,应当是100减去你的实际年龄,其余部分则应该以债券和现金的形式持有。(假如你28岁,应当将72\%的资金投资于股票;如果是81岁,则只应把19\%的资产投在股市。)像其他所有时髦说法一样,在20世纪90年代末期,这种观点曾风靡一时。1999年,一部通俗著作甚至宣称,如果你不到30岁,你可以把90\%的资金投入股市——即便你的风险承受力很“薄弱”!

除非你把自己的智商数减去100,否则,你一定会发现此类建议有什么地方不对。为什么你的年龄应当决定你可以承受多大的风险?一个拥有300万美元、丰厚的退休金和一群子孙的老太太,把她的大部分资产投资于债券的做法,无疑是愚蠢的。她已经拥有不菲的收入,而她的孙辈(他们最终将继承她的遗产)未来还有几十年的投资生涯。另一方面,一位年仅25岁,但正攒钱准备结婚买房的年轻人,也决不会打算将其全部资金投入股票。一旦股票市场上演高台跳水,他既没有债券的收益来弥补其损失,也没有钱以备不时之需。

此外,无论你多么年轻,你都可能会突然需要一大笔钱——不是在40年以后,而是在40分钟以后。在毫无先兆的情况下,你可能会失业、离婚、身受伤残,或遭受什么天晓得的意外。这些意外会突袭任何人,不管其年龄几何。每个人都应当将其资产的一部分,以现金的形式存放在无风险的安全地方。

最后,有些人正是因为股票市场的下跌,而终止其投资的。心理学家指出,大多数人都不善于预测自己将来遭遇令人沮丧之事时会感觉如何。当股票每年上涨15\%或20\%时(就像其20世纪80年代和90年代那样),不难想象,你会认为,你将与你的股票厮守终生。但是,当你看到你的每一美元投资,都缩水成了一毛钱时,你就很难抗拒将其变成“安全的”债券或现金的诱惑。因此,许多人不是买进并持有其股票,而是以贵买贱卖的痛心结果告终。正因为能够在熊市中有胆量坚守股票的投资者少之又少,格雷厄姆才坚决要求,每个投资者都至少应保持将资产的25\%投资于债券。他认为,投资于债券的这部分缓冲资产,将使你有勇气在股市不景气时,继续持有其余的股票。

为了更好地理解你可能承担的风险,不妨审视一下自己生活的基本环境:什么时候会出现新的情况,什么时候情况会发生变化,这些情况将如何影响到你的现金需求:

\begin{itemize}
\item 你是单身还是已婚?你的配偶或同居者以何为生?
\item 你已有或将会有子女吗?他们的学费什么时候会成为家庭的必要开支?
\item 你会继承一些财产吗?抑或你还要赡养年迈或有病的父母?
\item 哪些因素会对你的工作带来负面影响?(如果你供职于一家银行或建筑公司,利率的突然跳升可能会令你失去工作;如果你供职于一家化工企业,油价的飙升可能是一个坏消息。)
\item 如果你自己从事经营,与你的生意类似的企业,能够存续多长时间?
\item 你需要你的投资所得,来补贴你的日常开支吗?(一般说来,债券可以补贴你,而股票则不能。)
\item 考虑到你的薪水和开支情况,你可以承受多大的投资损失?
\end{itemize}

如果在审视了所有这些因素之后,你觉得自己可以承担拥有较多股票的较大风险,那么你就可以按照格雷厄姆给出的最低比率(25\%),来持有债券和现金。如果不是这样,你最好还是卖掉你的大部分股票,按照格雷厄姆给出的最高比率(75\%),来持有债券和现金。(要知道自己是否应百分之百地持有债券,请参见本书随后的专栏内容。)

一旦你确定了资产配置的最终比例,就不要轻易改动,除非你的生活状态出现了重大变化。既不要因为股市的上涨而加大其投资比例,也不要因为其下跌而更多地卖出。用约束取代猜想,这正是格雷厄姆投资法的精髓。...


\subsection{为什么不能把全部资产投资于股票}
格雷厄姆奉劝你投入股市的资金,永远不要超过自己总资产的75\%。但是,是不是每个人,都不适合把全部资金投入股市呢?对极少数投资者来说,全额投入也许是可行的。如果你属于下列群体,你也许可以这样做:

\begin{itemize}
\item 已经为自己的家庭准备好至少一年生活所需的全部资金
\item 准备在未来20年一直坚持投资
\item 已经成功度过2000年开始的熊市
\item 没有在2000年开始的熊市期间卖出股票
\item 在2000年开始的熊市期间,买入了更多的股票
\item 已经阅读了本书的第8章,并且已开始执行约束自身投资行为的正式计划

\end{itemize}

除非你确实通过了以上诸项测试,你决不能把自己的钱全部投入股票。在上一轮熊市中曾经陷入恐慌的人,在下一轮熊市仍将再次恐慌,并会因为没有债券和现金作为缓冲而深感懊悔。

\subsection{债券投资的细节}
在格雷厄姆的时代,债券投资者面临的基本选择是:购买免税的还是应税的债券?购买短期的还是长期债券?如今还需增加一项:购买债券还是债券基金?

购买免税债券还是应税债券?除非身处最低的纳税等级,否则,你必须将你退休账户以外的资金全部投入免税债券;不然的话,你的许多收益将被国税局拿走。...

购买短期债券还是长期债券?债券与利率的关系,就像一个跷跷板的两端:如果利率上扬,债券的价格就会下降,尽管短期债券的下降幅度远低于长期债券。另一方面,如果利率下跌,债券价格就会上涨,而且长期债券的上涨幅度会大于短期债券。你也可以通过购买5~10年到期的中期债券来缩小这种差别。这种债券既不会在利率飙升时大幅上涨,也不会在利率暴跌时一蹶不振。对大多数投资者来说,中期债券是一种最简单的选择,因为它可以使你不再为猜测未来利率的走势而烦恼。

购买债券还是债券基金?债券通常是以10 000美元为单位出售的,而你需要购买至少10种债券,才能将某种债券的违约风险分散掉。除非你至少有10万美元的投资,否则,购买单个债券的做法是不可取的。(惟一的例外是美国长期国债,因为它是由美国政府担保的,没有违约的问题。)

债券基金可以方便、廉价地提供分散化的好处;而且可以每月拿到利息收入,然后按照现行利率将其再投入该基金,且不收手续费。对于一般投资者来说,债券基金显然要优于直接购买单个债券(国库券和某些市政债券是一个主要的例外)。...


\subsection{其他债券投资}
怎样才能从你的现金中挤出更多的收益?聪明的投资者应当考虑,跳出银行定期存单和货币市场账户之类的传统工具(它们近年来的收益太低了),转向以下现金投资品种:

各种国债 作为美国政府的债务,这些债券实际上没有违约风险,因为山姆大叔用不着赖帐——它随时可以通过增税或多印钞票来还债。...

储蓄债券 与国债不同,储蓄债券是不可交易的;你无法把它卖给其他投资者。如果提前支取,你还会损失三个月的利息。...

抵押证券 它是由全美数千种抵押贷款打包形成的,由联邦国民抵押贷款协会(“房利美”)或政府国民抵押贷款协会(“吉利美”)这类机构发行。然而,它们没有美国财政部的担保,因而其收益率定得较高,以体现其较高的风险性。当利率走低时,抵押债券通常会跌得更凶,但利率上升时也涨得更猛。...

年金 这种类似于保险的投资,可以帮助你把当前需缴纳的税金递延到将来,并在你退休后为你提供收入流。固定年金的收益率是固定的,而可变年金的收益率是浮动的。...

优先股 优先股是身兼两种缺点的投资。一是安全性不如债券,如果公司破产,其偿还权排在债权人之后。二是获利潜力低于普通股,因为如果利率下降或公司信用级别改善,发行公司通常会“赎回”或强行回购这些优先股。而且发行公司支付的股息,不能像其支付的债券利息那样,从应税利润中扣除,冲抵一部分所得税。...


\section{第5章点评}
\subsection{最好的防御就是有力的进攻}
经历了前几年股市的血雨腥风之后,防御型投资者为什么还要投资股市呢?

首先,要记住格雷厄姆所坚持的观点:你应该有多大的防御性,这并不取决于你对风险的容忍程度,而是取决于你愿意在自己的投资组合方面花多少时间和精力。如果你的方法恰当,投资股票就会像持有债券和现金一样轻松容易(在第9章我们将看到,你可以非常轻松地购买股票型指数基金)。

很容易理解的是,身处2000年开始的此轮熊市,你会为此感到焦灼,因此决定从此再也不买股票;这种反应是可以理解的。就像一句古老的土耳其谚语所说的那样——“被热牛奶烫过嘴之后,再喝酸奶也会用嘴去吹。”由于2000~2002年的股市崩溃是如此恐怖,许多投资者现在都觉得股票会烫伤自己。但是,殊不知,正是此种下跌已使股票市场的大部分风险得到了释放。在此之前,它确实是一杯烫牛奶,但现在已降到室温了。

由此可知,如今你是否应继续拥有股票,与几年前拥有的股票给你带来了多大的损失并无关系。如果股票的价格已变得相当合理,足以使你的财富今后得以增值,那你就应当购买它们,而不论其是否曾令你在不久前亏损过。尤其是在当今债券的收益率很低,从而减少你未来的投资回报的时候。

我们在第3章已经指出,与历史平均水平比较,2003年年初的股价只处于略高一点的价位。与此同时,从目前的价格来看,债券的收益率很低;因此,那些出于安全性购买债券的人,就好比一些烟民认为吸焦油含量较低的香烟可以避免肺癌一样。无论你是具有多大防御性的投资者——按照格雷厄姆的观点,即无论将持有的股票降低到多少;按照现在的观点,即无论将风险控制在多低的水平——为了保持今天的价值,你都必须至少拿出一部分钱来购买股票。

幸运的是,如今防御型投资者购买股票比以往任何时候都更容易了。目前有一种持久的自动投资组合系统,它可以使你毫不费力地把自己的钱,按月投入预先设定的投资品种,从此无须为选股浪费大量的时间。

\subsection{应该“买自己熟悉的股票”吗}
但是,首先让我们来看防御型投资者始终反对的一种观点:你无须为选股做任何功课。在20世纪80年代和90年代初期,“购买自己熟悉的股票”是当时最流行的投资口号。曾在1977~1990年间执掌富达麦哲伦基金,并获得共同基金最佳投资战绩的彼德·林奇就是这一信条最有力的鼓吹者。林奇认为,业余投资者拥有业内投资者已然忘却的一项优势,即知道如何去利用“常识的力量”。如果你发现一个很棒的新餐馆、一种新款汽车、牙膏或牛仔裤,或者你看到你家附近的一家店铺的停车场总是车水马龙,或一家公司的总部直到午夜电视剧播完后仍有人在加班,那么,你就会对某只股票形成一种专业分析师和基金经理人都不具备的亲身感受。正如林奇所指出的:“有了多次买汽车和照相机的经验,你会对商品的好坏以及是否好卖,形成自己的感觉……最重要的是,你比华尔街更早获知这一信息。”

林奇的规则——“如果利用自己的优势,投资于所熟悉的公司或产业,你就可以比专业人士做得更好”——并非毫无道理,而且多年来确实有成千上万的投资者从中受益。但林奇的这一法则,只有在你遵循以下结论时才有效:“找到一家看似有前途的公司只是第一步。下一步是对它进行研究。”他的真正意思是,在对一家公司的财务报表进行研究,并对其商业价值进行估量之前,决不能购买股票,无论其产品看起来有多棒,或他家的停车场停了多少辆顾客的汽车。

不幸的是,大多数股票投资者都忽略这一方面。

...


\subsection{填平补齐}
虽然股票市场总是日复一日地上下波动,防御型投资者却有办法控制这种无序的状态。你拒绝采取主动,而且从不假装具有预测未来的能力,这恰恰会成为你最有力的武器。你的每一个投资决策都是按既定程序自动做出的,因此你可以排除那种自以为能够预知市场走势的幻觉,不为市场力量所左右,无论其走势如何异乎寻常。

正如格雷厄姆指出的,“美元成本平均法”使你可以定期将一定数额的资金用于投资。无论股市已经(或者即将)上涨、下跌还是横盘,你都将按周、按月或者按季度买进股票。所有大型共同基金公司和经纪公司,都会为你提供安全的自动电子转账服务,这样你就不必动手填写支票,并为金钱的支出感到心痛了,正所谓“眼不见,心不烦。”

最为理想的美元成本平均法是投资于一组指数基金,从而将所有具有投资价值的股票和债券都一网打尽。这样的话,你就可以摆脱诸如预测股市的走向、了解哪些板块以及其中的哪些股票会表现最好之类的猜谜游戏。

假定你每个月可结余500美元,你可以借助美元成本平均法拥有三只指数基金。其中300美元投资于美国股票市场指数基金,100美元投资于外国股票指数基金,还有100美元投资于美国债券市场指数基金,这样你就可以确信囊括这个星球上几乎所有值得拥有的投资了。精确得像时钟一样,每个月你都会买进更多的股票;如果市场下跌,你预定的投资金额就会比前一月买入更多的股份;如果市场上涨,同样的金额所能买到的股份就会少于前一个月。以这种近乎自动的方式处置你的投资组合并持之以恒,就可以避免以下两种情形的出现:在市场似乎最具吸引力(实际上最危险)的时候,将手中的货币随意投向市场,抑或是在市场崩溃,股价确实便宜(但似乎更具“风险”)的时候,拒绝买进更多的股票。

根据一家颇有影响力的金融研究公司(Ibbotson Associates)的研究,如果你在1929年9月初,以12 000美元买进标准普尔500指数基金,十年之后你的手中将只剩下7 223美元。但是如果你以区区100美元起步,以后每月追加投入100美元,到1939年8月,你的资金就会增至15 571美元!这就是按照规则买进的威力——即便是面临大萧条这样有史以来最糟糕的熊市。

从1999年年底到2002年年底,标准普尔500指数出现了持续的下跌,但如果你以3 000美元的最低限额开立了一个指数基金账户,并每月追加投入100美元,那么你总计6 600美元的总投资,将亏损30.2\%——明显低于大盘41.3\%的跌幅。此外,你在较低价位的持续补仓,将使你在市场出现反弹时获得丰厚的利润。

最有利的是,一旦你以指数基金为核心,建立起一个具有永久性且自动导航式的投资组合,你就能够在面对所有相关市场的问题时,给出一个防御型投资者可能给出的最有力的回答:“我不知道,也不在乎。”如果有人问,债券的收益是否会好于股票,你只需回答:“我不知道,也不在乎。”——毕竟,你已经自动拥有这两个投资品种了。医疗保健股是否会令技术股黯然失色?“我不知道,也不在乎。”——你已经是这两类股票的长期拥有者了。谁会是下一个微软?“我不知道,也不在乎。”——只要它规模够大,你的指数基金就会买进它,而你将搭上这班车。明年外国股票是否会强于美国股票?“我不知道,也不在乎。”——如果真是这样,你将从中获利;如果不是,你将以较低的价格买进更多。

让你能够理直气壮地宣称“我不知道,也不在乎”这种永久性且自动导航式的投资组合会令你获得解放,再也无须为预测市场走势而殚精竭虑,尽管其他人仍沉溺于这种不切实际的追求中。承认自己对未来所知甚少,以及对这种无知的心安理得,正是防御型投资者最强大的武器。


\section{第6章点评}
无论是进取型(积极型)投资者,还是防御型投资者,知道哪些该做以及哪些不该做,这对于你的成功都是很重要的。在本章,格雷厄姆列举了进取型投资者不应做的事情。以下是针对如今的情况列出的。

\subsection{垃圾场的野狗?}
格雷厄姆把高收益债券称之为“二级债券”或“低等级债券”,今天,我们则称之为“垃圾债券”。在格雷厄姆的时代,散户投资者要想通过多元化的投资来分散其违约风险是非常麻烦的,且成本高昂。然而,如今超过130多家专门投资此种债券的基金;他们大量地吃进这些垃圾债券,并且持有10多种不同的债券。这种情况缓解了格雷厄姆关于其无法分散化的担心。(但他对高收益优先股的此种担心依然有效,因为在分散其风险方面,至今仍缺乏成本低廉且普遍可行的手段。)

自1978年以来,年均利率达4.4\%的垃圾债券市场曾多次出现过违约。即使如此,其年回报率仍可达到10.5\%,而当时10年期国债的年回报率为8.6\%。不幸的是,这些垃圾债券基金大都要收取高额的佣金,而且往往无法保证你的本金不受损失。垃圾债券基金也许适合那些已经退休,且试图增加自己的月收入以弥补退休金之不足,同时能够承受其价值波动的投资者。如果你在一家银行或金融机构工作,利率的大幅上升会使你的职位升迁受到影响,甚至威胁到你的工作。因此,由于垃圾债券在利率上升时的表现会好于大多数债券,也许可以将其作为一种对冲手段纳入你的401(k)账户。但对于聪明的投资者来说,垃圾债券只是一种可供选择的权利,并不是一种非买不可的义务。


\subsection{内外兼顾的组合}
格雷厄姆认为,外国债券与垃圾债券差不多,均非优质投资目标。但是,对那些具有较高风险承受能力的投资者而言,如今外国债券已变得较具吸引力了。目前大约有十来只共同基金,专门从事新兴市场国家(或第三世界国家),如巴西、墨西哥、俄罗斯、尼日利亚和委内瑞拉等国的债券投资。明智的投资者购买此类烫手资产的数量,通常不会高于其总资产的10\%。但新兴市场国家的债券,一般不会与美国股票市场同步运行,因此,它们属于那种不会随道指下跌而下跌的少数品种之一。如果你确实需要此种资产,可以考虑将其少量纳入你的投资组合。

\subsection{致命的短线交易}
正如我们在第1章指出的,短线交易,即持有股票的时间在几小时左右,是有史以来人类发明的最佳自杀武器。你的某些交易会赚钱,你的大多数交易会赔钱,但你的经纪人却永远会从中获利。
而且你急于买进或卖出某只股票的行为,也会降低你的收益。如果你一定要买入某只股票,你的出价就会比大多数卖家乐意接受的价格高出10美分。这一额外成本,被称作“市场冲击”(market impact),它虽然不会出现在经纪商的账面上,但确实要耗费你的真金白银。如果你过于急切地想购进1 000股股票,并因此驱使该股的价格上涨了5美分,你就会为此多支出一笔无形但却实实在在的50美元。另一方面,如果一个恐慌的投资者急于卖出其持股,并且以低于其最近的市场价抛售,这种市场影响成本便会再次令你蒙受损失。

就像受到多层砂纸的打磨一样,交易成本会使你的收益层层流失。买进或卖出一只小盘热门股的交易成本达2\%到4\%(“全程成本”,即一买一卖的成本,高达4\%至8\%)。如果你在某只股票上投入1 000美元,它就会使你在起步之前就付出40美元的代价;卖出这只股票,你还要支付另外4\%的费用。

哦,此外还有这么一件事:如果你只是交易而非投资,你就会把长期所得(应交纳资本利得税,最高税率为20\%)变成普通收入(其最高税率为38.6\%)。

把所有这些考虑进来,短线交易者至少需要获得10\%的收益,才能在一买一卖的过程中打个平手。单凭运气,每个人都能够碰上一次;但要经常性的获利,以补偿为此付出的高度紧张(以及由此带来的噩梦般的压力),这是不可能的。

成千上万的投资者对此进行了尝试,但结果是毋庸置疑的:交易越频繁,自己得到的就越少。

...


\subsection{早起的鸟儿被虫咬}
20世纪90年代,对投资大众毒害最烈的一种观点,莫过于那种声称申购IPO产品可快速致富的说法了。所谓“IPO”是指“首次公开发行”,即把公司股票首次出售给公众。...

遗憾的是,每出现一次像微软那样让你赚得盆满钵满的IPO,就会出现数千次令你亏损的IPO。两位心理学家(Daniel Kahnerman和Amos Tversky)的研究表明,人们在估计某种事件发生的概率或频率时,往往做判断依据的不是其实际发生的频率,而是自己对过去事例的印象程度。我们都想买到“下一个微软”,恰恰是因为我们错失了购买第一个微软。但我们却很容易忽略这样一个事实,即大多数IPO都是一些很烂的股票。要想获得上述天文数字的财富,你必须逮住IPO市场上每一只大牛股;由于这些股票十分稀少,这种情况是不可能发生的。最后,那些高收益的IPO股票,大多被专门的小团体拿走了,这个由投资银行和基金公司组成的小圈子,在这些股票出售给公众之前,就以所谓的“包销”价将其吃进了。此外,涨得最猛的股票,通常是一些小盘股,即使大机构也拿不到其新股;市场上根本没有足够的股票可供你去参与。

如果几乎像每一个投资者那样,你只能在这些IPO暴涨之后才能够买进的话,那么,你的结局会相当悲惨。假如你在1980至2001年间以新股上市头一天的收盘价,买进一只典型的新股,并持有3年,你的年收益率将低于市场23个百分点。

...


\section{第7章点评}
\subsection{时机的选择并不重要}
在理想条件下,聪明的投资者只会在价格便宜时购买股票,在价格涨高时将其出售;然后以债券和现金的形式持有这些资金,直到股价再一次变得便宜时再去购买。一项研究表明,从1966年到2001年年底,持续持有的1美元股票最终将上涨到11.71美元。可是,如果你能恰好在每年中5个最坏的日子到来之前平仓的话,那么你最初的1美元将上升到987.12美元。

与市场上大多数的魔幻想法一样,这种想法也是基于一种戏法。你(或其他任何人)怎样才能确切地知道哪几天是最糟糕的——在这些日子到来之前?1973年1月7日,《纽约时报》对美国的一位高级金融预言家进行了专访。这位预言家督促投资者赶快购买股票:“现在绝对处在牛市时期,以往很少见到这样的情况。”这位预言家就是艾伦·格林斯潘,而且非常罕见的是,没有哪一个人像这位日后的美联储主席那样,在当天做出了完全错误的判断。后来的事实证明,自大萧条时期以来,1973年和1974年是经济增长和股市表现最不好的年份。

专业人士对入市时机的判断,会比格林斯潘更准确吗?“我认为,大部分下降压力都已经过去了,”2001年12月3日,R.M.Leary公司择时交易机构的负责人凯特·利里·李说道,“这正是你应该入市的时候。”她补充道——她预测2002年第一季度的股票“会表现得不错”。随后的3个月内,股票的回报仅有微小的0.28\%,比现金的回报还低1.5个百分点。

持此观点的并非利里一人。杜克大学的金融学教授进行的一项研究表明,如果你采纳最好的择时交易刊物(占总刊物的10\%)所提出的建议,那么,你在1991~1995年赚取的年回报率将是12.6\%。可是,如果你不听从它们的建议,而将资金投入到股票指数基金中去,那么,你将获得16.4\%的回报。

正如丹麦哲学家齐克果所指出的,只有回过头来才能理解生活,但是,生活必须往前走。回过头去看,你总能准确地了解应该何时购买股票以及何时出售股票。然而,你不能愚蠢地认为,在实际中自己可以随时判断应该在何时进入,在何时退出。在金融市场上,事后观察永远是完全清楚的,但是,事先预测必定是盲目的。因此,对大多数投资者而言,择时交易从实际上和心理上看都是不可能的。

\subsection{哪些股票会上涨}
正如宇宙飞船进入地球的同温层后会加速一样,成长股似乎也经常会脱离重心引力。我们来看20世纪90年代3种增长最快的股票的变化轨迹:通用电气、家得宝和太阳计算机系统公司。

从1995年到1999年,每家公司每年的规模和利润都在增长。从营业收入上看,太阳公司增长了一倍,家得宝公司增长了一倍多。根据Value Line公司提供的信息,通用电气的营业收入增长了29\%,利润增长了65\%。家得宝和太阳公司的每股收益几乎增加到了原来的3倍。

然而,其他的情况也在发生,这丝毫不会使格雷厄姆感到惊讶。这些公司增长得越快,其股价也越来越昂贵。当股价的上涨超过了公司的成长速度时,投资者最终总是会吃亏的。

如果其股票价格太高的话,一家优秀的企业并不是一个非常棒的投资对象。

股价上涨得越高,似乎就更有可能继续上涨。但是,这种本能的看法完全与金融物理现象的根本法则相抵触:公司的规模越大,增长速度就越慢。销售收入为10亿美元的公司,能够轻易地使自己的销售收入增加一倍;但是,业务量为500亿美元的公司到哪里去寻找另外500亿美元的业务量呢?

在股价合理的情况下,成长股值得去购买。但是,当其市盈率大大高于20或30倍时,再去大量购买就会有些不妙了:...

认为高增长能够永远持续下去的幻觉,并非只是发生在投资者身上。2000年2月,有人问北电网络(Nortel Networks)的首席执行官约翰·罗思,这家光纤行业巨头的规模会达到多大。“该产业正在以每年14\%~15\%的速度增长,”罗思回答说,“而且,我们的增长速度还会加快6\%。就我们这种规模的企业来说,这是相当令人兴奋的事。”前6年,北电的股票几乎每年上涨51\%。当时,该公司的股价是华尔街预测的2000年利润的87倍。股价被高估了吗?“已经上涨到了这么高的水平,”罗思轻松地说,“但是,随着我们无线战略的实施,将还有很大的升值空间。”(他补充说,毕竟思科的股价达到了其预测利润的121倍!)

至于思科公司,2000年11月,其首席执行官约翰·钱伯斯坚持认为,自己公司的年增长率至少会达到50\%。“从逻辑上讲,”他说,“这将是一种特殊情况。”此前,思科的股价一直在下跌——当时的股价仅为上一年利润的98倍——因此,钱伯斯督促投资者去购买。“这时还要等着去买谁的股票呢?”他说,“现在正是机会。”

与此相反,这些成长型企业都萎缩了,而且它们被高估的股价也缩水了。2001年,北电的营业收入下降了37\%,而且当年该公司的亏损超过了260亿美元。2001年,思科的营业收入上升了18\%,但是,该公司最终出现了10亿美元以上的净亏损。当罗斯讲话时,北电的股价为113.5美元,2002年最终降到了1.65美元。当钱伯斯称自己的公司是一种“特殊情况”时,思科的股价为52美元,后来的股价跌到了13美元。

从此以后,两家公司对未来的预测更为谨慎了。

甚至许多公司领导人也无法理解这些差异(参见后文专栏的内容)。然而,聪明的投资者对快速成长股感兴趣,并不是发生在其最受欢迎之时,而是在出现某种问题的时候。2002年7月,强生公司宣布,联邦监管当局正在调查其一家下属药厂会计记录不当的问题,这样,该公司的股价在一天内就下跌了16\%。这使强生公司的股价与前12个月的利润之比,从24倍下降到了仅为20倍。在如此低的水平下,强生有可能再次成为有一定成长空间的成长股——从而变成格雷厄姆所称的“不太受欢迎的大公司”。如果能按较好的价格购买到某家大公司的股票,那么这种暂时性的“不受欢迎”可以给你带来持久的财富。

\subsection{应该将所有的鸡蛋都放进一只篮子里吗}
“将所有的鸡蛋都放进一只篮子里,然后看好这只篮子,”一个世纪前安德鲁·卡内基告诉人们,“不要撒胡椒面。……人生的巨大成功在于目标集中。”正如格雷厄姆所说的,“从普通股中真正获得巨大财富”的来自下列一些人:他们将自己所有的资金投入到了最熟悉的一种投资活动中。

几乎所有的美国富人所获取的财富,都来自于对某个行业甚至是某个公司的集中投资(想一想比尔·盖茨与微软,山姆·沃尔顿与沃尔玛,洛克菲勒家族与标准石油)。比如,自从1982年首次出版以来,《福布斯》列出的400名美国富翁中,大多数人的财富都是集中获取到的。

然而,这样去做的话,几乎连小额财富都无法获取到。而且,许多巨额财富并非一直是以这种方式持有的。卡内基没有想到的是,人生的大多数重大失败也是来自于这种集中投资。我们再来看《福布斯》的富人榜。早在1982年,《福布斯》前400名富豪的平均财富为2.3亿美元。想进入2002年《福布斯》前400名的富豪榜,从平均水平来看,一个1982年的富豪平均每年只需从其财富中获得4.5\%的回报即可。然而,这一时期银行账户的收益甚至都高于4.5\%,而股市的年均回报率达13.2\%。

那么,经过20年之后,《福布斯》1982年前400名富豪中还有多少位留在榜单上呢?最初的400人中,只有64人仍然留在2002年的榜单上——只占可怜的16\%。由于将所有的鸡蛋放到一只篮子中(这只篮子——曾经繁荣过的行业,比如石油和天然气、计算机硬件和基础制造业——使他们首次进入富豪榜),最初的大多数富豪都倒下了。当困难降临时,他们所有人(尽管都拥有巨额财富可以带来巨大的优势)都没有做好应有的准备。当不断变化的经济将他们惟一的一只篮子和所有的鸡蛋压得粉碎时,他们束手无策,只能在可怕的危机前缩成一团。

\subsection{廉价类证券}
你可能认为,在我们所处的无穷的网络世界中,必然可以发现并购买到能满足于格雷厄姆廉价交易标准的一组股票。...

比如,2002年10月31日,Comverse技术公司拥有24亿美元的流动资产和10亿美元的债务总额,因此,其净营运资本为14亿美元。由于公司股份还不到1.9亿股,每股股价低于8美元,因此,该公司的总市值正好在14亿美元以内。因为股价还未达到Comverse公司的现金和存货的价值,所以,公司今后业务的价值实际上没有包含在股价之中。正如格雷厄姆所了解的,购买像Comverse公司这样的股票仍然有可能亏损——因此,只有当你在某时间找到几十种这样的股票时,才能去购买并耐心持有它们。然而,在非常罕见的情况下,当市场上出现众多真正的廉价交易证券时,你肯定可以从中获利。

\subsection{外国证券交易策略}
投资外国股票并不是聪明的投资者必须要做的,但我们必定会建议他们考虑这一投资。为什么呢?让我们来做一个简单的思考。假设现在是1989年年底,假设你是一位日本人。实际情况如下:

\begin{itemize}
\item 过去10年中,你的股市年均增长率为21.2\%,大大高于美国17.5\%的年均增长率。
\item 日本的公司正在收购美国的公司——从Pebble Beach高尔夫球场到洛克菲勒中心;与此同时,美国的一些企业(比如美国金融公司Drexel Burnham Lambert和德士古)正在申请破产。
\item 美国的高科技产业正在消亡,而日本的正处于兴旺时期。
\end{itemize}

1989年,作为一个日本人,你只能得出这样的结论:到日本以外去投资将是自寿司自动贩卖机诞生以来最愚蠢的想法。自然,你会将所有的钱用于购买日本的股票。

结果如何?在随后的10年内,你投入的资金将亏损大约三分之二。

从中可以得出什么教训呢?这并不是说,你永远不应该到日本这样的外国市场去投资;而是说日本人永远不要将自己所有的资金投放在国内。而且你们也不应该像日本人这样去做。如果你生活在美国,工作在美国,以美元获取工资收入,那么你已经对美国的经济投下了多次赌注。为了谨慎,你应该在别的市场拥有一些投资组合。原因很简单:毕竟任何人都不可能知道,本国或外国未来的结果是怎样的。请将三分之一的股票资金用于购买外国(包括新兴市场)股票的共同基金,这一定有助于防范风险,因为本国市场不一定是全球最好的投资市场。

\section{第8章点评}
\subsection{化身博士和市场先生}
在大多数时间里,市场对许多股票的估价都是很准确的。数百万买主和卖主的讨价还价,的确能够从总体上对公司做出较好的估价。然而,有时候价格并不正确。偶然情况下,价格的确会出现严重错误。在这种时候,你需要理解格雷厄姆所描述的市场先生——或许,这是用来解释股票错误定价的一个最好的比喻。情绪变化无常的市场先生,并非总是像分析者或个人买主那样对股票进行估价。相反,当股价上涨时,他会欣然支付比股票客观价值更高的价格;当股价下跌时,他会按低于股票实际价值的价格拼命出售股票。

市场先生还存在吗?他仍然处在两种极端之中吗?答案是肯定的。

2000年3月17日,Inktomi公司的股价创下了231.625美元的新高。自从1998年6月首次上市以来,这家互联网搜索软件公司的股票上涨了约19倍。就在1999年12月之后的几周内,该股票就几乎上涨了两倍。

Inktomi的业务发生了怎样的变化,使得其股票如此值钱?答案似乎很明显:企业的快速增长。在截止于1999年12月的三个月之内,Inktomi产品和服务的销售达3 600万美元,比1998年全年的还要多。如果Inktomi能够将前12个月的增长率仅仅再维持5年,其销售收入将从每季度的3 600万美元,上涨到每月50亿美元。考虑到公司的这种增长速度,股票会上涨得更快,其价格将会越来越高。

但是,在青睐Inktomi的股票的同时,市场先生忽视了其业务的某些东西。公司正在赔钱,而且赔得很多。最近的一个季度亏损了600万美元;此前的12个月亏损了2 400万美元;再往前一年亏损额也是2 400万美元。在公司的整个历史中,Inktomi从未赚取过任何利润。然而,2000年3月17日,市场先生对这家小企业的总估值却高达250亿美元。(是的,数字后的单位是亿美元。)

随后,市场先生突然陷入了噩梦。2002年9月30日,就在每股价格达到231.625美元刚过去两年半之后,Inktomi的股价以每股25美分收盘了——总市值从250亿美元下跌为不到4 000万美元。Inktomi的业务枯竭了吗?根本不是。前12个月,公司获得了1.13亿美元的销售收入。那么,是什么发生了改变呢?只是市场先生的情绪变化了:2000年年初,投资者如此热衷于互联网企业,因此,他们使Inktomi的股价达到了其销售收入的250倍。然而,现在他们愿意支付的股价,只是其销售收入的0.35倍。市场先生像化身博士一样,从兴奋转向了悲观,并且拼命地打压曾经愚弄过自己的所有股票。

但是,市场先生半夜里的愤怒,与他以前的欣喜若狂一样是没有根据的。2002年12月23日,雅虎宣布将按每股1.65美元的价格收购Inktomi,这几乎是该公司9月30日股价的7倍。历史可能会证明,雅虎获得了一笔廉价交易。当市场先生使得股价如此便宜时,人们并不奇怪整个公司将从他的手中被直接收购。

\subsection{进行独立的思考}
如果一个疯子每周至少有5次告诉你,你应该与他想的完全一样,你会允许他这样做吗?你会仅仅因为他的乐观而乐观,或者因为他的悲观而悲观吗?当然不会。你要坚持自己的权利,根据自己的经验和信念来掌控自己的情感生活。然而,每当涉及金融生活,许多人就会让市场先生告诉自己感觉如何,以及应该怎么去做——尽管事实一次又一次明确地表明,他愚蠢至极。

1999年,当市场先生欢欣鼓舞的时候,美国的劳动者总体上将其8.6\%的工薪收入,投入到他们的401(K)退休计划中。到了2002年,当市场先生已花3年时间将一些股票打入冷宫时,总体的缴费比率下降到了仅为7\%——下降近四分之一。股票越便宜,人们越不想购买,因为他们在仿效市场先生,而没有进行独立的思考。

聪明的投资者不应该完全忽视市场先生。相反,他应该与市场先生打交道,但只是为了使其服务于自己的利益。市场先生的任务是向你提供价格,而你的任务是决定这些价格对你是否有利。你不应该仅仅因为他不断乞求就与他打交道。

不让市场先生成为你的主人,你就将他转变成了你的仆人。毕竟,即便他似乎是在消灭价值,但是他也在别的方面创造价值。1999年,由于技术和电信股的推动,Wilshire 500指数(最全面地反映了美国股市的业绩)上涨了23.8\%。但是,尽管整体指数在上升,指数所包含的7 234种投票中,有3 743种出现了价值下降。虽然那些高科技股和电信股炙手可热,可是几千种“旧经济”股却备受冷落,价格越来越便宜。

1999年,CMGI(新的互联网企业的“孵化器”或控股公司)的股价,令人惊讶地上升了939.9\%。与此同时,伯克希尔-哈撒韦(一家控股公司,格雷厄姆最优秀的弟子沃伦·巴菲特通过它而拥有代表旧经济中坚力量的企业,比如可口可乐、吉列和华盛顿邮报等)的股价下跌了24.9\%。

可是,经常发生的情况是,市场会突然转变态度。...

CMGI在2000年继续下跌了96\%,2001年又下跌了70.9\%,2002年再次下跌了39.8\%——累计损失达99.3\%。伯克希尔-哈撒韦2000年上涨了26.6\%,2001年上涨了6.5\%,2002年小幅下降了3.8\%——累计盈利达30\%。

\subsection{你能在职业经理的游戏中取胜吗}
格雷厄姆最强有力的一个见解是:“如果投资者自己因为所持证券市场价格不合理的下跌而盲目跟风或过度担忧的话,那么,他就是不当地把自己的基本优势转变成了基本劣势。”

格雷厄姆所说的“基本优势”指的是什么呢?他指的是:聪明的个人投资者完全可以自由选择是否去追随市场先生。你享有独立思考的权利。

然而,一般的货币经理不得不完全模仿市场先生的行为——高价购买股票,低价出售股票,就像一群没有脑子的跛脚鸭那样,蜂拥着跌跌撞撞地行走。下面是共同基金经理和其他职业投资者所面对的一些障碍:
\begin{itemize}
\item 由于管理着数十亿美元的资金,因此他们必须倾向于购买规模最大的股票——他们购买的每种股票必须达到几百万美元,才能进入自己的证券组合。这样,许多基金最终都同样拥有少数几只被高估的大盘股。
\item 随着市场的上升,投资者会把更多的钱投入到基金中去。基金经理会利用这些新的资金来购买已经拥有的股票,从而使得价格上升到更危险的水平。
\item 市场下跌时,如果基金投资者要求收回投资,基金经理就有可能需要出售股票以获取现金。正如市场上升时基金被迫购买高价股一样,股票便宜时,它们又被迫充当出售者。
\item 业绩高于市场水平时,许多基金经理会得到奖金。因此,他们极其关注自己的回报与参照物(比如标准普尔500指数)的比较。如果某家公司的股票被纳入指数,数百家基金就得被迫购买它。(如果没有购买,而这种股票却表现得很好,人们就会认为基金经理太愚蠢;另一方面,如果购买后结果不理想,也没有任何人去责怪。)
\item 基金经理的业务将越来越专业化。正如医学领域的普通从业者被划分为过敏儿科医生和老年耳鼻喉科医生一样,基金经理也必须专注于只购买“小型成长股”、“中型价值股”或“大型混合股”。如果某家企业规模太大或太小,股价太便宜或太贵,基金就必须将其出售,即使基金经理喜欢这种股票。
\end{itemize}

因此,你有理由像这些职业经理做得一样好。你无法做到(尽管所谓的权威人士认为你可以做到)的是,“在职业经理们自己的游戏中取胜。”即使是职业经理,也无法在自己的游戏中取胜!你为什么要去参与这种游戏呢?如果你按他们的规则行事,你就会输掉——因为你将会像职业经理那样,成为市场先生的奴隶。

相反,人们要认识到,聪明的投资行为在于对能够控制的因素进行控制。你无法控制自己购买的股票或基金的业绩,是否会在今天、下个星期、这个月或这一年胜过市场。在短期内,你的回报将始终受制于市场先生及其古怪的念头。然而,你能够做到的是:

\begin{itemize}
\item 你的经纪成本——避免频繁的交易,耐心等待以及从事费用廉价的交易。
\item 你的所有权成本——拒绝购买年费过于昂贵的基金。
\item 你的预期——根据现实而不是幻想来预测你的回报。
\item 你的风险——决定将自己的多少资产投入到股市中;进行分散化投资;对投资结构进行重新调整。
\item 你的税款——持股期至少长达1年;如果有可能的话,应至少长达5年,以降低你的资本利得税。
\item 最重要的是,你自己的行为。
\end{itemize}

如果你看电视上的金融节目,或者是阅读大多数的股市专栏文章,就会感到投资活动有些类似于体育运动,或者是一场战争,或者是在荒野中的一场生存较量。然而,投资活动并非要在别人的游戏中打败他们,而是要在自己的游戏中控制好自己。聪明的投资者面对的挑战,不是寻找涨幅最大和跌幅最小的股票,而是防止本人成为自身最大的敌人——不要仅仅因为市场先生说“买入!”而高价购买,不要仅仅因为市场先生说“卖出!”而低价出售。

如果你的投资期很长(至少达25年或30年),就只有一个理智的办法:只要一有闲置资金,就在每个月自动购买。这种终身持股的惟一最优选择,就是投资于整个股市指数基金。只有当你需要现金时,才去出售自己的投资。

作为一位聪明的投资者,你还不能够以其他一部分人的业绩,来判断自己的投资是否取得了成功。如果住在杜比克、达拉斯或丹佛的某个人胜过了标准普尔500指数,而你却没做到,你也丝毫不比别人更差一些。任何人的墓碑上都不会写下“他战胜了市场”这样的话。

我曾经在Boca Raton(佛罗里达的一个富人退休社区)访问过一群退休人员。我问这些人(大多数都是七十多岁的人),他们在自己一生的投资活动中是否战胜过市场。有些肯定的回答,也有些否定的回答,但大多数人并不能确定自己是否做到过。后来,有一个人说:“管它呢,我所知道的是,我的投资所得,足以让我在此安享晚年。”

难道还有比这更好的答案吗?毕竟,投资的全部意义并不在于所赚取的钱比一般人要多,而在于所赚取的钱足以满足自己的需要。衡量自己的投资是否成功的最好办法,不是看你是否胜过了市场,而是看你是否拥有一个有可能使自己达到目标的财务计划和行为规范。最终,重要的不在于你比他人提前到达终点,而在于确保自己能够达到终点。

\subsection{你的资金和你的大脑}
那么,为什么投资者认为市场先生如此有诱惑力呢?事实证明,我们的大脑与投资问题密切相关:人类是喜欢遵循某种模式的动物。心理学家已经证明,假设你给人们一个随机结果,并告诉他们结果是不可预测的;然而,他们仍然会试图猜测下一个结果是什么。同样地,人们认为,他们“知道”:下一次掷骰子时将会出现7;一位棒球运动员将要进行安全打;Powerball博彩中的下一个中奖号码一定是4-27-9-16-42-10;而且还知道,某一只热门小盘股将成为下一个微软。

神经科学领域新的突破性研究表明,我们的大脑天生会去感知趋势,即使趋势并不存在。只要一件事连续发生两三次,人类大脑部位的前扣带和阿肯伯氏神经核,就会自动地预感它会再次发生。如果的确再次发生,一种名叫多巴胺的天然化学成分就会释放出来,从而使你的大脑充斥着一定程度的快感。因此,如果某只股票连续上涨几次,那么你将会条件反射式地预期它会继续上涨——随着股价的上涨,你大脑中的化学成分会发生改变,从而给你带来一种“天然的快感”。这样,你实际上就对自己的预测上瘾了。

然而,当股价下跌时,资金上的亏损会激发你的扁桃核——大脑中处理恐惧和忧虑的部位,它带来的最显著的反应就是,“要么战斗,要么逃跑”(这是所有困兽共有的反应)。正如火警响起时,你的心律必然会加快一样;正如旅途中遇到响尾蛇时,你必然会退缩一样;股价大幅下跌时,你必然会感到害怕。

事实上,杰出的心理学家丹尼尔·卡尼曼(Daniel Kahneman,2002年获得了诺贝尔经济学奖——译者注)和阿莫斯·特沃斯基(Amos Tversky)已经证明,资金亏损所带来的痛苦程度,是等额盈利所带来的快感程度的两倍。股市上赚1 000美元会感觉很快乐,但是,1 000美元亏损所带来的心理折磨将是快乐的两倍。赔钱是如此痛苦,因此,许多人由于害怕进一步亏损而在价格接近谷底时卖出,或者是拒绝购买更多。

这可以解释,我们为什么紧盯住市场下降的绝对数,而忘记了以相对比例来表示亏损。因此,如果电视主持人高喊:“市场正在迅速下跌——道指下降了100点!”大多数人都会本能地感到震撼。然而,由于道指近期达到了8 000点的水平,这种下降幅度仅有1.2\%。现在,想一想下面的事情听起来是多么的可笑:有一天,室外温度为华氏81度,电视主持人高声说:“温度正迅速下降——从华氏81度降到了华氏80度!”这种下降幅度也是1.2\%。如果你忘记了以百分比来观察市场价格的变化,那么就很容易因为小幅变动而惊慌。


\subsection{有用的新闻}
20世纪90年代末,许多人在一天内如果不查看几次股价,就会感到自己一无所知了。然而,正如格雷厄姆所说的,“如果没有股市行情,”一般的投资者的情况“可能会更好一些,因为这样的话,他就不会因为其他人的错误判断而遭受精神折磨了。”如果你在下午1:24查看了股票组合的价格,又在下午1:37感到必须再去查看的话,那么,就请问自己这样几个问题:

\begin{itemize}
\item 我在下午1:24向房地产代理人打电话核实过我的房子的市场价了吗?下午1:37我又打过一次吗?
\item 如果我这样做了,房价变化了吗?如果变化了,我是不是应该赶紧出售我的房子?
\item 如果不是每时每刻去核实或了解自己房屋的市场价格,房屋的价值是不是就不会随着时间的变化而上升了呢?
\end{itemize}

这些问题的答案当然是否定的。你应该以同样的方式来看待自己的证券组合。对10年、20年或30年的投资期而言,市场先生每日捉摸不定的波动根本就不重要。无论如何,对想要做长期投资的人来说,股价的不断下跌是好消息,并不是坏消息,因为这使得他们可以花较少的钱,买到更多的股票。股价下降的时间越长、幅度越大,而且你在它们下降时不断地买入,那么最终你赚的钱就会更多——如果你能够一直坚持到最后。不要害怕熊市,而应该欢迎熊市。即使股市在今后10年内不提供每日的价格信息,聪明的投资者也会安心地拥有股票或基金。

神经科学家Antonio Damasio解释说,有些矛盾的是,“你越认识到自己还有许多没有掌握,那么你掌握得就越好。”在认识到自己高买低卖的生理倾向后,你就会认为自己需要采用美元成本平均法、再平衡法和签订投资合约等方法。通过使自己证券组合中的大多数处于永久自我运行状态,你就能克服自己喜欢预测的倾向,使自己关注于长期财务目标,并免受市场先生情绪波动所带来的影响。

\subsection{利用好市场提供的机会}
尽管格雷厄姆教导人们应该在市场先生高喊“出售”时买入,但聪明的投资者需要理解一个例外。熊市抛售的合理性在于,它能够带来税收上的好处。美国国税法允许人们使用已发生的亏损(出售股票带来的任何价值损失),来冲抵普通收入(最大限额为3 000美元)。假设你按每股60美元的价格,于2000年1月购买了可口可乐的200股(总投资额为12 000美元)。2002年年底,股价跌为每股44美元,即你的全部股票价值为8 800美元——亏损了3 200美元。

你可以像大多数人那样去做,对自己的亏损感到痛心,或者把它放到一边,而假装什么事也没有发生。你还可以对损失进行控制。2002年年底前,你可以将可口可乐所有的股票卖掉,从而把亏损锁定在3 200美元。这样,等31天后申报纳税时,你再购买200股可口可乐的股票。结果是:你能够把2002年的应税收入降低3 000美元,而且,你还可以用剩余的200美元亏损额度抵扣2003年的收入。然而,更为有利的是,你仍然拥有一家前景被你看好的公司的股票,但是,现在你拥有的股票所支付的价格,比第一次支付的价格几乎少了三分之一。

在山姆大叔向你的亏损提供补贴的情况下,你当然可以出售股票并锁定亏损。如果山姆大叔想使得市场先生看上去较为合理,我们要责怪谁呢?



\section{第9章点评}
\subsection{近乎完美}
共同基金纯粹是美国的一项发明。1924年,一个名叫爱德华·莱弗勒(Edward G.Leffler)的铝制炊具推销员将共同基金引入了美国市场。共同基金非常便宜,非常方便,种类繁多,由专业人士管理,而且还要受到联邦证券法部分严厉条款的密切监管。通过使投资变得容易,并使几乎每个人都负担得起,共同基金把大约5 400万美国家庭(以及全球更多的家庭)引入了投资大潮中。这或许是有史以来金融民主化的最大进步。

然而,共同基金并非完美,它们只是近乎完美,而且“近乎”这个词道出了所有的差别。由于其并不完美,因此大多数基金都面临着下列问题:业绩低于市场平均水平,向投资者的收费过高,带来了令人头疼的税收问题,并且其业绩会随机大幅波动。聪明的投资者必须极其审慎地选择基金,以免到头来给自己找了一个大麻烦。

\subsection{最畅销的基金}
大多数投资者多会以继续上涨为假设直接购买上涨最快的基金。这是理所当然的。心理学家已经证明,人类有一种与生俱来的倾向:认为可以通过短期内的一系列结果对长期趋势做出预测。此外,从我们自身的经历可以看到,有一些水暖工要大大优于其他水暖工,有些棒球选手更有可能击中本垒打,我们所喜欢的餐馆一直能够提供优质的饭菜,而且聪明的孩子总是能得到高分。在我们身边,技能、智慧和勤劳能获得认可,得到回报,而且这样的事情一直在重复发生。因此,如果某只基金胜过了市场,直觉就会告诉我们:它将继续有优异的表现。

遗憾的是,在金融市场上,运气比技能重要得多。如果某一基金经理在恰当的时间从事市场上的恰当业务,那么他看上去就是一个很棒的人。但是,在很多情况下,热门的东西突然会受到冷落,经理的智商似乎缩水了50分。...

这一情况再次提醒我们,市场上最热门的行业(1999年为技术行业)经常会在毫无预兆的情况下顷刻之间变得冰冷。它还提醒我们,完全凭过去的业绩来购买基金的行为,是投资者所做的最愚蠢的事情之一。金融专家对共同基金的研究持续了至少半个世纪,而且他们几乎一致赞同如下几点:

\begin{itemize}
\item 一般的基金,不可能通过承担研究和交易成本来挑选好的股票。
\item 基金的费用越高,其回报越低。
\item 基金股份交易越频繁,其赚钱的机会越小。
\item 高度不稳定的基金(比平均水平上升和下降幅度更大),有可能长期处于不稳定状态。
\item 过去回报很高的基金,今后不可能长时间成为赢家。
\end{itemize}

这是1999年最热门的10家基金——实际上,是有史以来年收益率最高的基金。但是,随后的3年使得1999年的巨大增值都被抹杀了,而且此后还有下降。

根据以往的回报挑选出未来的最优秀基金的机会非常之小——类似于北美野人和可怕的雪人将穿着粉红的芭蕾舞鞋同时出现在你下一次的鸡尾酒会上。换句话讲,你的机会不是为零,但却非常接近于零。

但是,这也是好事。首先,理解了为什么很难发现一种好的基金,这将有助于你成为一位更明智的投资者。其次,尽管以往的业绩并不能很好地反映未来的回报,但你可以利用其他一些因素来增加你挑选优秀基金的机会。最后,一种基金即使不能够从市场中胜出,也能够提供很大的价值——提供一种廉价的资产组合分散化方法,并且使你不必自己花时间去挑选股票,而把这方面节省下来的时间用于其他方面。

\subsection{前锋将会被落下}
为什么许多获胜的基金不能继续保持下去?

基金的业绩越好,其投资者面对的障碍越多。

基金经理跳槽。...

资产过度膨胀。当某种基金获得高额回报时,投资者都会注意到这一点——他们经常会在几周之内注入上亿美元的资金。...

高超的技巧不复存在。...

费用上升。...

羊群行为。最后,一旦某只基金获得成功,基金经理就习惯于变得胆小和模仿他人。随着基金的扩大,其费用收益会更加可观,从而使得经理们安于现状。...

由于费用高昂和行为不端,大多数基金都难以糊口。难怪高回报几乎转瞬即逝,就像未冷冻的鱼一样。更有甚者,随着时间的推移,其过高费用的拖累使得大多数基金逐步被落下...

那么,聪明的投资者应该怎样去做呢?

首先,要认识到,从长远看,指数基金(它始终拥有市场上所有股票,而从不声称自己能够挑选“最好的”股票和避免“最坏的”股票)将胜过大多数基金。指数基金极低的管理费用(每年0.2\%的操作费用,外加每年仅0.1\%的交易成本)使其具有不可比拟的优势。比如说,如果今后20年内股票的年回报率为7\%,那么,像先锋总股市基金这样的低成本基金的年回报率将接近于6.7\%。(这将使得10 000美元的投资变成36 000多美元。)但是,就一般的股票基金而言,扣除1.5\%的操作费用和大约2\%的交易成本之后,其年回报率能够达到3.5\%就不错了。(这样,10 000美元的投资所得还不到20 000美元,几乎比指数基金的结果少一半。)

指数基金只有一个显著的缺陷:比较令人乏味。你将无法在野餐桌上向人们吹嘘,自己如何拥有全国最好的基金。你也不能去吹嘘自己从市场中胜出,因为购买指数基金是为了与市场回报持平,而不是为了超过市场回报。指数基金经理不可能通过“掷骰子”来打赌下一个最好的行业是什么:是量子远距传物,是高科技网络,还是心灵感应术减肥诊所?指数基金始终拥有每种股票,而不是拥有基金经理极力猜测出的又一只新股。但是,随着时间的推移,指数基金的成本优势将会势不可挡。持有20年或更长时间的指数基金,每个月注入一笔新的资金,这样,你肯定能够比绝大多数的专业投资者或个人投资者做得更好。在自己的晚年,格雷厄姆称赞指数基金为个人投资者的最佳选择,而沃伦·巴菲特也持有同样的观点。

\subsection{转变思路}
当你归纳出基金的所有缺陷时,你感到惊讶的不是看到如此少的基金无法胜过指数基金,而是这种情况居然存在。是的,有些基金的确能获胜。它们具有什么样的共同点呢?

它们的经理都是一些最大的股东。...

它们费用低廉。基金行业最常见的错误观念是“一分钱一分货”——高收费的最好理由就是高回报。这种观点面临着两个问题。首先,它是错误的。几十年的研究已经证明,从长远看,收费较高的基金获得的回报较低。其次,高回报只是暂时的,而高收费几乎是不变的。如果你冲着高回报去购买某基金,你将最终感到失望。然而,你拥有基金的成本几乎肯定不会随着其回报的下降而下降。

它们敢于与众不同。当彼得·林奇经营富达麦哲伦基金时,他购买的都是一些廉价的资产,而不管其他基金经理持有什么资产。1982年,他最大的投资是美国长期国债。此后,他持有最多的是克莱斯勒的资产。尽管大多数专业人士预计,这家汽车制造商将要破产。随后的1986年,林奇几乎以20\%的富达麦哲伦基金购买了一些外国公司(比如本田、挪威铝业和沃尔沃等)的股份。因此,你在购买美国的股票基金之前,要把基金公司最新报告中的持股名单,与标准普尔500指数中的名单进行对比。如果两者极为相似,就去挑选另一只基金。

它们不接纳新的投资者。最佳的基金经常会拒绝接纳新的投资者,而只允许已有的股东购买更多的股份。这就阻止了新的基金购买者蜂拥而入(他们想挤入大基金之中),以免基金承受资产膨胀之痛。这也说明,基金经理没有把个人利益放在客户利益之上。...

它们不做广告宣传。正如柏拉图在《理想国》中所说的,人们理想中的统治者,是那些不想统治的人。最佳的基金经理,通常是那些似乎不想赚你钱的人。...

你还应该注意哪些东西呢?大多数基金购买者首先看重的是以往的业绩,其次是基金经理的声誉,再次是基金的风险状况,最后(如果还有的话)关注的是基金的费用。

聪明的投资者也是观察这些东西,但其顺序正好相反。

因为基金的费用比其未来的风险或回报更容易预测,所以你应该将其作为第一个筛选条件。按种类看,基金的年度操作费用没有理由高出下列水平。

\begin{itemize}
\item 外国股票:应税市政债券:0.75\%
\item 美国(大型和中型公司的)股票:1.0\%
\item 高收益(垃圾)债券:1.0\%
\item 美国(小型公司的)股票:1.25\%
\item 外国股票:1.50\%
\end{itemize}

其次是风险评估。在其招股说明书(或买方指南)中,每只基金都必须以柱形图来展示其一个季度内的最大亏损。如果你不能承受在3个月之内有如此大的损失,就请换家基金看看。还有必要核查一下晨星公司(Morningstar)对基金的评级结果。作为领先的投资研究公司,晨星公司会根据基金风险与回报的对比来确定基金的“星级”(一星最差,五星最优)。然而,与过去的业绩一样,这些评级结果只是对以往的观察。它们只告诉你过去哪些基金是最佳的,并不能告诉你将来的情况。实际上令人难堪的是,五星级基金的业绩,往往会持续低于一星级的基金。因此,首先是要寻找具有下列特征的基金:基金经理为主要股东的低收费基金,敢与众不同的基金,不夸大自身回报的基金,在规模过大之前就不愿意接纳新投资者的基金。然后,而且只有在此时,才去参考晨星公司的评级结果。

最后,观察以往的业绩——请记住,它只是对未来回报的一个不太准确的预测。正如我们所看到的,昨天的赢家往往会成为明天的输家。但是,研究者已经证明,有一点是几乎肯定的:昨天的输家几乎从未成为明天的赢家。因此,不要购买以往业绩一直都很差的基金,尤其是当它们的年费高于平均水平时。


\subsection{懂得何时平仓}
当你拥有一只基金时,如何判断合适的出售时机?传统的忠告是,如果基金的业绩在1年内(连续2年内,或连续3年内)低于市场(或同类资产组合)的业绩,那么就应该将其出售。但是,这项忠告是不明智的。从1970年诞生到1999年的29年中,Sequoia基金有12年(41\%以上的时间)的业绩低于标准普尔500指数。然而,这一时期,Sequoia的收益率高达125倍,而同期标准普尔指数的收益率只有49倍。

大多数基金业绩的下滑,仅仅是由于它们所偏爱的股票暂时不受欢迎了。如果你雇用某位经理以特定方式进行投资,那么为什么要因为他按承诺行事而将其解雇呢?在某一投资方式失宠时将股份卖掉,这种行为不仅锁定了自己的亏损,而且也使自己丧失了几乎必然会发生的反弹机会。一项研究表明,从1998年到2001年,仅仅是由于高买低卖,共同基金投资者的业绩每年就要降低4.7个百分点。

那么,什么时候出售呢?在此,有几个确定的信号:

\begin{itemize}
\item 交易策略突然发生急剧改变,比如1999年“价值”基金大量购买技术股,以及2002年“成长型”基金大量买入保险股。
\item 费用上升,这说明基金经理正在肥自己的腰包。
\item ·过度交易导致经常出现大量税单。
\item 突然产生异常回报,比如当以前的稳健基金遭受巨大亏损时(或者甚至出现惊人的回报时)。

\end{itemize}

正如投资顾问查尔斯·埃利斯所说的:“如果你不准备呆在婚姻里,就不应该去结婚。”基金投资也是一样的。如果你不准备经受基金所带来的至少3年的亏损,你首先就不应该去购买基金。耐心是基金投资者惟一重要的伙伴。

\section{第10章点评}
\subsection{你需要帮助吗}
20世纪90年代末市场繁荣时期,许多投资者都选择了独自从事交易的做法。他们自己做研究,自己挑选投票,自己通过网络经纪人下达指令,因而这些投资者绕开了华尔街昂贵的研究、建议和交易等服务。遗憾的是,许多“自力更生者”声称,在大萧条以来最严重的熊市出现之前的独立自主行为,最终使他们明白,这种独自行事的做法是愚蠢的。当然,这一点并不一定正确,因为,将每一项决策都委托给传统的股票经纪人的那些人也赔了钱。

但是,许多投资者的确从优秀的金融顾问提供的经验、判断和补充观点中获得了帮助。有些投资者需要其他人来告诉自己应该获取多高的投资回报,或者需要有多少额外的储蓄才能实现自己的财务目标。另一些投资者能获得的好处只不过是,当投资失败时可以去责怪他人。这样的话,你就不必在自我怀疑中折磨自己了,而是去批评某个人(这个人通常能为自己辩护,同时又能给你鼓励)。这能提高你的心理承受力,从而当其他投资者退缩时,你还能不断地投资。总而言之,由于你无论如何也无法管理自己的资产组合,因此寻求专业人士的帮助并没有什么使人感到羞耻的。

如何知道自己是否需要帮助呢?这里有一些线索:

巨大的亏损。从2000年年初到2002年年底,如果你的资产组合损失了40\%的价值,那么你的业绩比令人失望的股市的业绩更糟。无论你的失败是由于懒惰、粗心,还是运气不好造成的,当出现此类巨额亏损之后,就说明你的资产组合急需得到帮助。

失败的预算。如果你长年累月苦苦寻求收支相抵,不知道自己的钱流向了哪里;发现自己无法定期储蓄,经常不能按时支付各种账单,那么,这就说明你的财务失控了。一位顾问可以帮助你掌管好自己的钱财——通过设计全面的财务计划来安排你如何花费,如何借款,如何储蓄以及如何投资(花费多少,借多少款,储蓄多少,投资多少)。

混乱的资产组合。有太多的投资者认为,20世纪90年代末期,他们已经做到了资产分散化,因为他们拥有39种“不同的”互联网股票,或者是拥有7种“不同的”美国成长股基金。但这就好比下列想法一样:一个全高音乐团,比一个高音独奏者演奏《老人河》的效果更好。如果乐团中不加上一些低音乐器,无论你增加多少高音乐器,都无法使得低音效果表现出来。同样,如果你所持有的资产总体上升或下降,那么你就无法得到真正的资产分散化所带来的投资和谐的好处。此时,一个专业的“资产分配”计划能给你提供帮助。

重大的变化。如果你成为了一名个体经营者,且需要设计一个退休计划,你日益年迈的父母不会理财,或者你孩子的大学费用似乎无力承担,那么,一位顾问不仅可以使你安心,而且可以帮助你真正改善自己的生活质量。此外,一位合格的专业人士,还可以保证你利用好并遵守好令人眼花缭乱的税法和退休规则。

\subsection{信任,并加以确认}
请记住,金融骗子的盛行在于通过劝说让你相信他们,并通过劝说让你不去调查他们。在把你的财务前景交给某一位顾问之前,一定要查明,他不仅使你放心,而且还具有无可挑剔的诚实品格。正如罗纳德·里根曾经说过的:“信任,并加以确认。”首先,考虑你最了解和最信任的几个人。然后,问他们能否向你推荐一位他们最信任,并且是他们认为最有价值的顾问。你所仰慕的人的信任感,是一个好的开端。

一旦你知晓了顾问的姓名、所在公司的名称,以及他的专长(他是股票经纪人,金融规划师,会计师,还是保险经纪人?),你就可以开始做你应该做的事情了。在谷歌这样的互联网搜索引擎中,输入顾问的名字及所属公司的名称,看显示出什么样的内容(请注意“罚款”“投诉”“法律诉讼”“惩罚”或“停职”等词语)。...


\subsection{了解情况}
最近,一本主要的金融规划刊物对几十位金融顾问进行了调查,让他们谈一谈客户与顾问会面时应该怎么办。在挑选顾问时,你的目标应该是:

\begin{itemize}
\item 确定他是想帮助客户,还是只装装样子而已。
\item 确定他是否理解本书中介绍的基本投资原则。
\item 评估其受教育水平、培训及经验是否足够向你提供帮助。
\item 
\end{itemize}

下面是著名的金融规划师建议潜在客户向顾问提出的一些问题:

你为什么要从事这一行业?你们企业的目标是什么?除了闹钟之外,还有什么会使你早起?

你的投资理念是什么?你在使用股票还是共同基金?你使用技术分析吗?你使用择时交易吗?(对后两个问题中任何一个问题的“肯定”回答,都向你发出了应该“否定”的信号。)

你是专门做资产管理咨询,还是同时在做税收、不动产与退休计划、预算和债务管理以及保险等方面的咨询?你的教育、工作经历和各种证书,能够使你满足此类金融咨询的要求吗?

你的客户通常都有哪些共同的需要?你如何帮助我达到目标?你如何跟踪及报告我的进展情况?你能提供一个清单,供我查看金融计划的实施情况吗?

你如何选择投资?你认为哪种投资方式最为成功,并且有哪些证据表明,你为你的客户取得了这样的成功?当某一项投资在一年内的业绩都很差的时候,你将怎么办?(回答“出售”的顾问不值得雇用。)


当你提供投资建议时,会从第三方接受某种形式的报酬吗?为什么接受,或者为什么不接受?什么情况下会这样?你估计第一年我应该向你支付多少服务费?以后这笔费用将会因为哪些因素而上升或下降吗?(如果每年的费用会消耗你资产的1\%,那么,你或许应该去找别的投资顾问了。)

你有多少客户?你与他们多久沟通一次?你为客户做的最值得骄傲的事是什么?你最喜欢的客户有哪些共同点?你与客户之间最不愉快的经历是什么,你是如何解决的?客户是与你还是你的助手沟通,这是由什么来决定的?你的客户一般会与你保持多久的业务联系?

我能看一下会计报表的样本吗?(如果你看不懂,就请顾问解释。如果你不能理解他的解释,那么就说明他不适合你去雇用。)

你认为自己在财务方面做得成功吗?原因是什么?你如何定义财务上的成功?

你认为我的投资的平均年回报率可以达到多少?(在8\%~10\%以上是不现实的。)

你能够向我提供你的简历、ADV表以及至少三个证明人吗?(如果该顾问及其公司被要求出示ADV表而没有出示的话,就请你起身离开——临走时看好自己的钱包。)

曾经有人正式投诉过你吗?最近与你解约的客户是因为什么原因才这么做的?

\subsection{战胜最大的敌人:自己}
最后要记住的是,优秀的金融顾问并非唾手可得。时常,最优秀的顾问已经招满了客户,因此不愿意接纳你,除非你看上去能够与他进行很好的合作。因此,金融顾问也会问一些难以回答的问题,其中包括:

为什么你认为自己需要一名金融顾问?你的长期目标是什么?在与其他顾问(包括与你自己)打交道时,最失望的是什么?你有预算方案吗?你能做到收支相抵吗?每年你要花费百分之几的资产?从过去的一年来看,我需要为你获得多少收益,才能使你感到高兴?你如何处理冲突或纠纷?你怎样看待2000年开始的熊市?你最担心的财务问题是什么?财务上你最大的希望是什么?你认为你的投资合理回报率应该是多少?(参照第3章的内容去回答。)

如果投资顾问不问你这些问题,以及从直觉上看,他对你认为应该问的其他问题不太感兴趣,那他就不是一个理想的人选。

尤其重要的是,你要对自己的顾问有足够的信任,从而使其能够保护你免受最大的敌人(自己)所带来的风险。评论员尼克·默里说:“你雇用投资顾问的目的不是为了管理钱,而是为了管理你自己。”

金融规划分析师Robert Veres说:“如果投资顾问是你和你的不利冲动倾向之间的一条防线的话,那么他就应该有现成的系统规划,以帮助你们双方控制好自己。”这些系统规划包括:

\begin{itemize}
\item 制定一项综合财务计划,以便安排好你的收入、储蓄、支出、借款和投资等事项。
\item 提供一份投资策略报告,以表明你的基本投资方法。
\item 制定一项资产分配计划,以详细说明你在各种不同的投资类别中如何分配资金。
\end{itemize}


这些就是一个好的财务决策的基础,它们应该由你和顾问共同来决定,而不是由某人单方面来决定。只有当你满意地看到这些基础条件已经具备并且符合你的愿望时,你才能掏钱去投资或做出投资决策。

\section{第11章点评}
\subsection{确定未来的价格}
哪些因素决定你购买股票时愿意支付的价格?是什么使得一家公司的价值达到其利润的10倍,而另一家公司的价值达到其利润的20倍?你如何保证自己不会因明显乐观的未来转变为一场噩梦而支付过大的代价?

格雷厄姆认为,有5种因素具有决定性的作用。他将其归纳为:

\begin{itemize}
\item 企业“总体的长期前景”
\item 企业管理层的水平
\item 企业的财务实力和资本结构
\item 企业的股息记录
\item 企业当期的股息支付率
\end{itemize}

让我们根据如今的市场来分析这些因素。

长期前景。如今,聪明的投资者首先要做的是,从公司网站或EDGAR数据库中下载至少5年的年度财务报告(10-K表),然后对这些财务报告进行梳理,收集证据,回答两个决定性的问题。这个企业增长的原因是什么?企业现在(以及将来)的利润来自何方?需要注意的问题包括:

\begin{itemize}
\item 企业是一个“连环并购者”。平均一年内有2~3起以上的并购,则预示着有可能出现麻烦。毕竟,如果某企业自身都认为应该购买其他企业的股份,而不愿从事自己的投资,那么,你为何不根据这一线索,也去观察一下其他企业?核实一下该企业以往的并购记录,注意食欲过盛的企业——它们吞下大企业后,最终只是又将其吐出来。朗讯、美泰、桂格麦片和泰克国际等企业,就曾经因吐出被并购的企业而遭受了惨痛的损失。其他一些企业因为以前并购价过高,而需要长期进行资产冲销或会计减记。对未来的业务而言,这是一个不祥之兆。
\item 企业是一位OPM成瘾者,通过借债或出售股份来抬升“他人资金”(Other People's Money,OPM)的总量。在年度财务报告的现金流量表中,这些慷慨注入的他人资金被称为“来自于融资业务的现金”。这些资金使得一家有问题的企业似乎正在成长,即使其基础业务并不能带来足够的现金——正如电信企业Global Crossing和世界通信不久前发生的情况一样。
\item 企业不太灵活,其大多数的收入都来自于某一个(或某几个)客户。1999年10月,光纤制造商Sycamore网络公司首次向公众发售了股份。招股说明书表明,该公司整个1 100万美元的收入,全部来自于威廉姆斯通信公司。交易商们乐观地估计,Sycamore的股票价值为150亿美元。不幸的是,只过了两年,威廉姆斯通信公司就破产了。尽管Sycamore公司挑选了其他客户,但其股票在2000~2002年期间损失了97\%。
\end{itemize}


当你研究企业的增长和利润的来源时,请同时关注有利和不利的因素。有利的迹象包括如下几项。

\begin{itemize}
\item 企业有宽广的“防御工事”或竞争优势。与城堡一样,有些企业很容易受到竞争对手的攻击,而另外一些企业则几乎是坚不可摧的。几种因素可以增强企业的防御能力:强有力的品牌形象(想一想Harley Davidson,其购买者会将该企业的标识物刺刻在他们的身上);对市场的垄断或近乎垄断;规模经济,即有能力廉价地提供大量商品或服务(想一想吉列公司,它能够廉价生产出几十亿只刀片);特有的无形资产(看一看可口可乐,其秘密的饮料配方没有任何有形价值,但却能吸引住宝贵的顾客);无法被替代(许多企业都不得不使用电力,因此,公用事业公司不可能在短期内被取代)。
\item 企业是一位长跑运动员,而不是一位短跑运动员。通过查看收益表,你会发现在前10年内,企业的收入和净利润是否在持续平稳地增长。《金融分析师杂志》上最近的一篇文章证实了其他的研究结果(以及许多投资者的惨痛经历):增长最快的企业,一般都会因发展过快而突然停止。从长远看,10\%的税前(或6\%~7\%的税后)利润增长是可持续的。然而,许多企业自己所确定的15\%的快速增长只是一个幻想。更高的增长率(或者说1~2年内的突然急速增长)必然会减缓下来,这就好比一个没有经验的马拉松运动员:试图以100米冲刺的方式来跑完全程。
\item 企业勤于播种和收获。无论企业的产品有多好,品牌有多强大,它都必须花一部分资金来拓展新的业务。尽管研发支出并不是当前增长的源泉,但很有可能是今后增长的源泉,尤其是当事实已经证明企业过去的振兴来自于新的思路和设备时。不同行业和不同企业的平均研发预算是不一样的。2002年,宝洁公司花在研发上的资金大约为其净销售额的4\%;而3M公司和强生公司在这方面的支出,分别为6.5\%和10.9\%。从长远来看,没有任何研发支出的企业,至少会与支出过大的企业一样脆弱。
\item 管理层的品质和行为。企业的高管应该言行一致。查阅以往的年度报告,核实管理者做出过哪些预测,以及他们是否达到了目标。管理者应该诚恳承认自己的失误,并承担相应的责任,而不应该拿“总体经济”“不确定因素”和“需求不足”等通用的理由作为替罪羊。查清企业董事会主席讲话的语气和内容是否前后一致,或者说是否随着最近的华尔街潮流而波动。(尤其要关注,在像1999年这样的繁荣年份,生产水泥或内衣的企业高管,是否突然宣布自己“成为软件革命的先行者”?)
\end{itemize}


这些问题有助于你确定企业管理者的行为是否符合企业所有者的利益:

\begin{itemize}
\item 他们是否在为自己谋求最大利益?

向自己的CEO支付1亿美元年薪的企业,最好能说出足够的理由。或许,他发现并独家拥有了青春泉(Fountain of Youth,传说饮此泉水能治百病,恢复青春;早期西班牙探险家曾在美洲和西印度群岛寻觅此泉——译者注)?或者他与外星人谈判达成协议,由地球上的这家公司单独向外星人提供全部供给?否则,这种可恶的巨额薪酬只是表明企业由管理者控制,并且是在为管理者谋利益。

如果企业对内部人员的股票期权进行重新定价(或者是“重新发售”或“进行交易”),就应该远离这样的企业。经过这种突然的逆转,企业会取消现有雇员和高管手中(一般都是没有价值的)股票期权,然后以新的有利价格的股票期权来取代它们。如果股票期权的价值永远不会下降为零,而且其潜在利润始终是无限的,它怎能激励高管管理好企业的资产呢?任何成熟的企业,如果像一些高科技企业那样对期权进行重新定价,这种行为都是不光彩的。购买此类企业的股票的投资者是自投罗网。

通过查看年度报告中关于股票期权的法定注释,你能够看到“期权溢价”有多大。比如AOL时代华纳公司在其年度报告的开头称,截至2002年12月31日,公司拥有45亿份普通股。但是,报告边缘的注释表明,该公司曾经发行的股票期权不低于6.57亿份。因此,公司未来的盈利将被另外15\%以上的股份去分享。当你估算企业未来的价值时,应该考虑股票期权有可能使新的股份大量增加。

通过EDGAR数据库中“表4”的内容,可以看到企业的高管和董事是否在购买和出售股份。企业内部人员有合理的理由(资产分散化,换更大的房子,办理离婚等)去出售股份,但是多次大规模出售则是一个明显的警示信号。当你在不断地买入而管理者却在不断卖出时,他不可能成为你理想的合作伙伴。
\item 企业高管是管理者还是推销员?

企业高管应该将大部分时间用于管理企业的内部事务,而不应该向公众投资者推销自己的企业。许多情况下,CEO们都会抱怨自己的股价被低估了(无论股价有多高)。他们忘记了格雷厄姆一贯坚持的观点:企业的管理者应该力求防止自己的股价被过分低估或高估。与此同时,太多的首席财务官会提供“利润指导”,即对企业的季度盈利做出猜测。一些企业喜欢到处张扬,不断发布新闻,吹嘘一些暂时的、微不足道的或假想的“机会”。

少数企业(包括可口可乐、吉列和USA Interactive)开始向华尔街的短视思维“直接说不”了。企业中这些少数的勇敢者正在提供越来越多关于其当期预算和长期计划的详情,而拒绝对今后90天内的情况做出推测。(想知道某企业如何与其股东进行平等和诚恳的沟通,请进入EDGAR数据库,查看华盛顿国际快递公司提供的8-K表,它会定期公布该机构与股东之间精彩的问答内容。)

最后,搞清楚企业的会计业务是为了使其财务结果透明化,还是模糊化。如果“一次性的”费用不断发生,“异常”项目经常出现而成为普通项目,那么EBITDA(Earnings Before Interest,Taxes,Depreciation and Amortization的缩写,指的是未计入利息、税费、折旧和摊销的净利润——译者注)这样的字首组合词要比净收益更重要,或者说,“预测的”利润被用来掩盖了实际亏损。这就可以说明该企业还没有学会如何把股东的长远利益放在首要地位。

\item 财务实力和资本结构。对优良企业最基本的定义是:所获取的资金要多于所消耗的资金。优秀的管理者不断寻找各种方法,以将这些资金投入生产活动。从长远来看,满足这一定义的企业,几乎必然要出现价值的增长,而无论股市如何表现。

在企业的年度报告中,首先阅读其现金流量表,以查明过去10年内,其营业现金流量是否在稳步增长。然后再进一步往下看。巴菲特的一个流行概念就是所有者收益,即净收益加上摊销和折旧,再减去正常的资本支出。这就正如Davis Selected Advisors公司的资产组合管理者克里斯托弗·戴维斯所说的:“如果你完全拥有这家企业,年底时你的口袋里将会有多少现金?”由于考虑到了摊销和折旧这样一些不影响企业现金余额的会计项目,因此,所有者收益是比所报告的净收益更好的一个计量指标。为了对所有者收益的定义进行改进,你还应该从所报告的净收益中减去下列项目:

\begin{itemize}
\item 分配股票期权的所有成本——这使得一部分收益从现有股东手中转移到新的所有者手中。
\item 任何“异常的”“一次性的”或“特殊的”费用。
\item 任何来自于企业养老基金的“收入”。
\end{itemize}
如果过去10年中,每股所有者收益总体上一直按高于6\%或7\%的速度增长,这就说明该企业有稳定的现金流,而且增长前景很好。

按下来,看企业的资本结构。从资产负债表中查看企业有多少债务(包括优先股在内)。一般而言,长期债务应该低于总资本的50\%。根据财务报告表中的注释,确定长期债务是固定利率(利息支出不变),还是浮动利率(利息支出会变化,如果利率上升,支出会增加)。

从年度报告中,查看所公布的“利润与固定费用之比”。亚马逊网络公司2002年的年度报告公布的结果表明,该公司的利润比其利息成本少1.45亿美元。将来,亚马逊公司要么需要从其业务中赚更多的钱,要么需要以更低的利率借到资金。否则,该企业最终将不能被其股东所有,而是要被其债券持有者所有——如果这些债券持有者不能获得应得的利息,他们就可以对亚马逊的资产行使求偿权。(公平地讲,亚马逊2002年的利润与固定费用之比比两年前的情况好多了,当时偿还债务的利润缺口为11亿美元。)

下面是关于股息和股票政策的几点看法(更多的内容,请参见第19章)。

\begin{itemize}
\item 最主要的是,企业要证明,如果不支付股息,股东的结果会更好。无论市场好与坏,如果企业都始终能够在竞争中获胜,那么这就清楚地表明,管理者最有利地利用了资金。可是,如果业务在下降,或者股票的表现不如其竞争对手,那么就说明企业的管理者和董事是在通过拒付股息而滥用资金。
\item 不断进行股票分割的企业(以及不断通过新闻发布而吹嘘股票分割的企业),是把投资者当成了傻瓜。就像约吉·贝拉一样(他要把比萨饼分成4份,其原因在于“我想,我吃不了8份。”),喜欢股票分割的股东并不理解这一点。价格为50美元的两份股票,并不比价格为100美元的一份股票更值钱。通过股票分割来推销自己股票的管理者,是在鼓励和纵容投资大众最邪恶的原始本能,因此,在把资金托付给此类表面热心的操纵者之前,聪明的投资者要三思而后行。

\item 企业应该在股价便宜时回购其股份,而不应该在股价处于或接近于最高位时回购股份。遗憾的是,近几年常见的现象是,企业在股价被高估时回购其股份。没有比这种浪费企业资金的行为更可笑的了,因为这种操作的真实意图是使企业高管能够以“提升股东价值”为名义出售自己的股票期权,以获取上百万美元的丰厚回报。
\end{itemize}

事实上,众多的实际证据表明,声称“提升股东价值”的管理者极少能这样去做。与普通的生活一样,在投资领域,最终的获胜者通常是实干家,而不是空谈家。

\end{itemize}



\section{第12章点评}
\subsection{数字游戏}
过去几年内,企业及会计师的行为已经显得十分不得体,如果格雷厄姆在世的话,也会大跌眼镜。在以大量的股票期权获取薪酬之后,企业的高管们意识到,仅仅借助于企业利润在几年之内的增长,他们就能够变得异常富有。大量的企业违背了会计精神(即便没有违背其字面含义):将财务报告弄得无法理解;掩饰不利结果;隐藏费用;凭空捏造利润,等等。让我们来看一看其中的一些不光彩行为。

\subsection{预计利润}
或许最流行的一个会计骗局就是“预计”(pro forma)利润的大量使用。在华尔街有一个古老的说法:任何坏主意的出发点都是好的。预计利润的公布也不例外。预计利润的本意是通过对短期偏差和“不经常发生的”事件进行调整,以便更为真实地反映利润的长期增长。比如,所发布的预计利润可能会让人们看到,如果某公司刚刚收购的另一家企业已经成为该公司的成员满12个月的话,那么该公司在过去一年中的利润将是多少?

...

简而言之,预计利润使得企业可以向人们展示:如果没有失误,它们将会做得多么好。作为一个聪明的投资者,在预计利润这一点上,你惟一能做的就是忽略它们。

\subsection{急于确认收入}
2000年,电信行业的巨头奎斯特国际通信公司(Qwest Communications International Inc.)看上去实力很强大。尽管当年的股市下跌了9\%以上,但该公司的股票下降幅度还不到5\%。

但是,在奎斯特的财务报告中有一个奇怪的小发现。1999年年底,当电话簿刚一出版,奎斯特就决定确认来自电话簿的收益。但是,登载过黄页广告的客户都知道,许多企业是采用按月支付的方式来偿付这些广告费用的。真是莫名其妙!私下“改变会计准则”的做法,使其1999年的税后净利润增加了2.4亿美元,相当于奎斯特当年全部收益的五分之一。

正如冰山一角,急于确认利润经常是存在重大危险的迹象,奎斯特的情况正是如此。2003年年初,在查看了以前的财务报表之后,该公司承认:提前确认了出售设备的利润;不恰当地记录了外部服务的费用;不恰当地将成本当成了资本资产,而没有将其当成费用;不恰当地将资产置换当成了直接的销售。所有这一切加起来,使得奎斯特2000和2001年的销售收入被高估了22亿美元,其中包括前面所说的“改变会计原则”(此时已经纠正过来了)带来的8 000万美元的收入。

\subsection{违反资本方面的规定}
20世纪90年代末,Global Crossing公司雄心勃勃。这家设在百慕大的公司,正在建设所谓的“首条一体化的全球光纤网络”。网络长达10万英里以上,主要分布在全球的洋底。连接全球之后,Global Crossing将向其他通信公司出售有线网络的使用权。仅1998年一年,Global Crossing在建设光纤网络方面就花费了6亿多美元。同一年,几乎三分之一的建设预算从收入中扣除了——以所谓的“出售设施的成本”的名义。如果没有这1.78亿美元的费用,该公司应该可以报告近8 200万美元的净利润——但是,所报告的净亏损达9 600万美元。

接下来的一年,Global Crossing在1999年年度报告的一个不起眼的注释中称“开始使用服务合约会计”。该公司不再将大多数的建设成本作为费用,从出售网络设施的收入中直接扣除。相反,现在大部分的建设成本将不再看做营运费用,而是看做资本支出。因此,这将使该公司的资产总额增加,而不是使其净利润下降。

不可思议!魔杖一挥,Global Crossing的“财产和设备”资产就增加了5.75亿美元,而其销售成本只增加了3.5亿美元,尽管该公司在漫无边际地花钱。

资本支出是企业管理者增强企业实力的一个重要工具。但是,灵活的会计原则使得管理者可以通过将正常营运费用转变为资本资产来夸大其报告利润。从Global Crossing这个案例中可以看到,聪明的投资者一定要搞清楚公司资本化的来源及理由。

\subsection{一个关于存货的故事}
与许多半导体芯片制造商一样,美光科技公司(Micron Technology,Inc.)的销售额在2000年后开始下滑。事实上,需求的急剧下降给美光以沉重打击,以至于它不得不开始下调其存货的价值,因为客户显然不会按美光的报价来购买这些产品。在2001年5月结束的那个季度,美光将存货的价值记录砍去了2.61亿美元。大多数投资者并没有把这项减记看成正常的或经常性的营运成本,而是看成了一个特殊事件。

可是,我们来看随后发生的情况。在随后的6个财务季度内,美光公司都做了存货减记。美光公司存货价值的下降是一次性的事件,还是一种常态呢?理性思考的人可以对这件事做出区分,但有一点是明确的:聪明的投资者必须始终小心那些不断发生的“一次性”成本。

\subsection{退休金方面的问题}
2001年,SBC通信公司(它拥有Cingular Wireless、Pac Tel和新英格兰南部电话公司的权益)获得了72亿美元的净利润。在电信业能力过剩的情况下,这是一个耀眼的业绩。但是,这部分利润并非完全来自于SBC的业务,其中的14亿美元(占公司净收益的13\%)来自于SBC的退休金计划。

由于SBC退休计划中的资金多于员工未来估计要获得的退休金,因此,该公司将这一差额当做当期收入。产生盈余的原因很简单:2001年,SBC将退休计划投资的预期回报从8.5\%提高到了9.5\%,从而降低了如今需要划拨的退休金。

关于更加乐观的预期,SBC的解释是:“2001年结束时每个3年期之内,我们实际的10年投资回报率都超过了10\%。”换句话讲,我们过去的回报率很高,因此,我们可以假设未来的回报率也很高。然而,这不仅不符合最基本的逻辑标准,而且还面对着如下的情况:利率即将降到历史最低水平,这将降低退休金组合中债券的未来收益。

实际上,巴菲特的伯克希尔-哈撒韦公司在同一年,将其退休金资产的预期收益率从8.3\%降低到了6.5\%,SBC认为其退休基金经理的业绩,能大大超过全球最优秀的投资者,你认为这现实吗?不太可能成为现实:2001年,伯克希尔-哈撒韦退休基金的收益为9.8\%,而SBC的退休基金则亏损了6.9\%。

在此,聪明的投资者要考虑这样几个重要的问题:“退休金的净收益”会比公司的净利润高出5\%吗?(如果是这样的话,当将来这些退休金收益不存在时,你还会对该公司的其他收益感到满意吗?)这里所假设的“退休金计划资产的长期回报率”符合情理吗?(截至2003年,任何高于6.5\%的收益率都是不合情理的,而更高的收益只不过是幻想而已。)

\subsection{给投资者的几点忠告}
下面的几点建议,可以使你避免购买有会计隐患的股票。

从后往前看。当你研究某企业的财务报告时,从最后一页开始,慢慢地往前阅读。凡是企业不愿意你看到的东西都放在后面,这就是你应该首先查看后面的原因。

查看说明。在未阅读年度财务报告的有关说明之前,决不能去购买股票。通常,在“重要会计政策总结”这一栏中,有一个关键性的说明会告诉你,该公司如何确认收入,如何记录存货,如何对待分期付款或合约销售,如何分摊营销成本,以及如何记录其他一些主要的业务。在其他注释中,注意查看关于债务、股票期权、客户贷款、损失准备以及其他“风险因素”的信息,因为这些因素将吞食掉大量的利润。会使你的敏感神经抽动的其他一些技术术语有“资本化的”“递延的”和“重组的”,等等。像“开始”“改变”和“然而”等简单明了的英文词汇,则预示着公司已经改变了自己的会计行为。这些词语中的任何一个并不意味着你不应该购买该公司的股票,但它们表明,你应该做进一步的调查。一定要把这些注释内容至少与一个密切竞争企业财务报告中的注释进行比较,以查看你所关注的企业的会计师有多么激进。

阅读更多的内容。如果你是一个积极投资者,愿意在证券组合中花大量的时间和精力,那么你就应该更多地了解财务报告的内容。为了尽可能防止被变化多端的收益表所误导,这是惟一的办法。下面的3本书包含了大量适时的相关例子:Martin Fridson和Fernando Alvarez合著的《财务报表分析》、Charles Mulford和Eugene Comiskey合著的《财务数字游戏》,以及Howard Schilit所著的《财务骗术》。



\section{第13章点评}
\subsection{E字打头的企业}
让我们像格雷厄姆那样对四种股票进行对比。我们使用的是1999年12月31日财务报告中的数据——这一时间可以使我们看到一些股市有史以来出现的最严重的极端价格。

爱默生电气公司(股票交易代码:EMR)成立于1890年,是格雷厄姆当初所举出的四家企业中惟一的幸存者。该公司提供种类繁多的产品,其中包括电动工具、空调设备和电动马达。

EMC公司(股票交易代码:EMC)的历史可以追溯到1979年。它的业务是使得一些公司可以在计算机网络上自动储存电子信息。

华盛顿康捷国际物流公司(股票交易代码:EXPD)于1979年在西雅图成立。它帮助发货人组织并追踪货物在全球的运送。

Exodus通信公司(股票交易代码:EXDS)向公司类客户提供和管理网站,同时还有其他一些互联网服务。它于1998年3月首次向公众发售股份。

下表概括了1999年年底这几家公司的股价、业绩和估值情况。

\begin{figure}[H]
\centering
\includegraphics[width=\linewidth ,totalheight=0.95\textheight , keepaspectratio]{对E字打头的四家公司的估值.jpg}
\caption{对E字打头的四家公司的估值}
\end{figure}


说明:所有数据都考虑到了后来的股票分割。负债、销售收入和利润都以财务年度为单位。市值:普通股总价值。NM:无意义。

\subsection{电气企业的表现并不令人振奋}
在格雷厄姆列举的四种股票中,爱默生电气的价格最为昂贵。但在我们新的一组企业中,爱默生的价格是最便宜的。由于属于旧经济产业,20世纪90年代末,爱默生失去了吸引力。(在互联网时代,谁还会去关注爱默生产的重型干湿真空吸尘器呢?)公司的股价陷入了长期低迷。1998年和1999年,爱默生的股票累计滞后于标准普尔500指数49.7个百分点,这是一个令人沮丧的不良业绩。

然而,这是爱默生公司股票的情况。爱默生公司的业务情况如何呢?1999年,爱默生出售了价值达144亿美元的商品和服务,比前一年增加了近10亿美元。爱默生从这些收入中赚取了13亿美元的净利润,比1998年增长了6.9\%。前5年中,每股收益以平均8.3\%的速度强劲增长。爱默生的每股股息增长1倍多,达到了1.3美元;账面值从每股6.69美元增加到了14.27美元。根据价值线公司提供的数据,整个20世纪90年代,爱默生的净利润率和资本回报率这两个反映其业务效率的关键性指标一直都很强劲,分别为9\%和18\%。此外,爱默生的利润连续增长达42年之久,股息连续增长长达43年之久,这是美国企业稳步增长时间最长的一个例子。1999年年底时,爱默生的股价为其每股净收益的17.7倍。与其生产的电动工具一样,爱默生从不耀眼,但却是可以信赖的——而且没有出现过过热的现象。

\subsection{EMC公司能够快速增长吗}
EMC公司是20世纪90年代股票表现最佳的一家公司——股价上升(或飞涨)速度超过了810倍。如果在1990年年初,你有EMC公司1万美元的投资,那么1999年年底,你的投资价值将上涨到810万美元以上。仅1999年一年,EMC的股票回报率就达到了157.1\%——这超过了爱默生从1992年到1999年8年间的总回报。EMC从未支付过股息,相反,它将所有的利润“用于公司的持续发展”。按12月31日54.625美元的股价来看,EMC股票的交易价格为该公司全年报告利润的103倍——几乎是爱默生股票估价水平的6倍。

EMC的业务状况如何呢?1999年业务收入增长了24\%,达到了67亿美元。其每股收益从一年前的61美分,激增到了1999年的92美分,增长达51\%。在截至1999年的5年内,EMC的利润以28.8\%的年增长率急速增长。而且,由于每个人都认为电子商务浪潮会持续下去,因此其未来前景更加光明。整个1999年,EMC的负责人不断地预测,2001年的业务收入将达到100亿美元——比1998年增加54亿美元。这要求年增长率达到23\%。对于如此大规模的一家公司而言,这将是一个惊人的增长速度。然而,华尔街的分析师及大多数投资者都确信,EMC能够做到。毕竟,在过去的5年中,EMC的销售收入增加了一倍多,其净利润超过了两倍。

然而,根据价值线公司提供的数据,从1995年到1999年,EMC的净利润率从19\%下滑到了17.4\%,其资本回报率从26.8\%降到了21\%。虽然还有较高的盈利,但EMC已经在走下坡路。1999年10月,EMC并购了Data General公司,从而使得自己当年的业务收入大约增加了11亿美元。直接将Data General所带来的收入扣除之后,我们可以看到,EMC已有的业务量,从1998年的54亿美元只增加到了1999年的56亿美元,上涨幅度只有3.6\%。换句话讲,EMC的实际增长率几乎为零——即便在计算机“千年虫”恐慌使得许多公司在新技术方面的花费创下历史纪录的情况下。

\subsection{货运公司的简单情况}
与EMC不同,康捷国际物流公司从未出现过飞速增长。尽管20世纪90年代该企业的股价年增长率为30\%,但其中大部分增长来自于最后一年——1999年股票的回报率快速增长了109.1\%。此前的一年,康捷的股票仅上涨了9.5\%,落后于标准普尔500指数19个百分点。

公司的业务状况如何?实际上,康捷的增长非常迅速:从1995年开始,其业务收入年增长率达19.8\%;到1999年结束时,业务量达14亿美元,这一时期几乎增加了两倍。每股收益的年增长率达25.8\%,股息年增长率高达27\%。康捷公司没有长期债务,其营运资本自1995年后几乎增加了一倍。根据价值线公司提供的数据,康捷的每股账面值上升了129\%,其资本回报率达21\%(上升了三分之一以上)。

无论以任何标准来衡量,康捷都是一家优秀的企业。但是,这家小型货运公司(设在西雅图,而许多业务位于亚洲)在华尔街却几乎没有什么名气。其股份只有32\%被机构投资者持有。事实上,康捷公司只有8 500名股东。1999年倍增之后,该公司的股价为当年净利润的39倍。这并不太低,但是却大大低于EMC令人头晕的估价。

\subsection{应许之地?}
截至1999年年底,Exodus通信公司似乎将自己的股东直接带入了福地。1999年的股价激增了1 005.8\%,足以使得1月1日的1万美元投资,在12月31日变成11万美元。华尔街著名的网络股分析师(包括美林证券的重量级分析师亨利·布洛杰特在内),都预测该公司的股票在随后一年中还将上涨25\%~125\%。

在网络股交易者(他们从Exodus的股票中获得了大量的收益)看来,最为有利的是,1999年该公司的股票进行了三次2比1的分割。在2比1的股票分割中,公司的股份数将增加一倍,股价将下降一半。因此股东拥有的股份最终为原来的两倍,而每股价格只有以前的一半。这有什么了不起的?假如你给我一个10美分的硬币,而我返还给你两个5美分的硬币,并问“你感到现在更富有了吗?”你可能会说我是一个傻瓜,或者说,我误认为你是一个傻瓜。可是,在1999年互联网股市的狂热中,网络交易商的行为似乎就是认为,两个5分的硬币要比一个10分的硬币更值钱。事实上,仅仅按2比1进行股票分割的消息,就能使其股价立即上涨20\%或更多。

为什么会这样?因为拥有更多的股份会使人们感到更加富有。某人在1月份买的Exodus公司的100股,经过4月份的分割后就变成了200股;8月份的200股又变成了400股;12月份400股又变成了800股。使这些人感到振奋的是,起初他们只有100股,但是现在却增加了700股。他们感觉似乎是“发现了财路”,而从不考虑每一次分割,都将使股价下降一半。1999年12月,一位购买了Exodus股票的得意洋洋的股民(其网名为“给我一美元”)兴奋地在网上发帖子称:“我将把这些股票保留到我80岁为止,(因为)随后经过上百次的股票分割之后,我将要成为CEO了。”

Exodus 的业务状况如何?格雷厄姆将不会去碰该公司的股票。Exodus 的业务收入呈爆炸式增长——从1998年的5 270万美元,增加到了1999年的2.421亿美元。但是,1999年的这些业务导致了1.303亿美元的亏损,几乎为前一年亏损额的两倍。Exodus 总共有26亿美元的债务,而且急需资金,因此仅在12月它就借了9.71亿美元。根据Exodus 的年度报告,这笔新的借款将使它下一年的利息支出增加5 000多万美元。1999年年初,该公司的现金为1.56亿美元,即使通过新的融资又筹集了13亿美元,但这一年结束时的现金余额只有10亿美元——这意味着,1999年,公司的业务吃掉了4亿多美元现金。这种公司如何去偿还自己的债务呢?

但是,网络股交易者显然只关注股票的上涨幅度及速度,而不会去关注公司是否稳健。网名为“Launch\_{}Pad1999”的一位交易者夸耀道:“这只股票将继续无止境地上涨。”
Launch\_{}Pad荒谬的预测——何谓“无止境”?——正好提醒人们要记住格雷厄姆经典的告诫。格雷厄姆告诉我们:

\begin{quote}
“如今的投资者是如此关注对未来的预期,以至于已经事先付出了巨大的代价。这样,即使他大力进行细心的研究,得出的预测结果成为现实,也有可能仍然无法获利。如果预测的结果没有完全实现,他实际上将面临着严重的短期亏损甚至是永久性的亏损。”
\end{quote}


\subsection{四家公司的结果如何}
这四种股票1999年之后的表现如何?

2000年,爱默生电气又获得了40.7\%的收益。尽管其股票在2001和2002年均出现了亏损,但2002年结束时的价格,比1999年最后的价格下降了不到4\%。

2000年,EMC的股价也上涨了,其盈利达21.7\%。然而,2001年,其股价降了79.4\%,2002年又降了54.3\%。这使得股价比1999年年底的水平低88\%。预测2001年的业务收入为100亿美元,其结果如何?这一年EMC最终的业务收入只有71亿美元(并且有5.08亿美元的净亏损)。

与此同时,对于康捷国际来说,似乎根本没有出现熊市,其股价在2000年增长了22.9\%,2001年增长了6.5\%,2002年又增长了15.1\%——2002年结束时的股价几乎比1999年年底的高出51\%。

Exodus的股价2000年下降了55\%,2001年下降了99.8\%。2001年9月26日,Exodus依照破产保护法第11章提出了破产申请。该公司的大多数资产被英国的电信巨头Cable \& Wireless收购了。未能将自己的股东带入应许之地,相反,Exodus将他们流放到了荒野之中。2003年年初,Exodus股票的最后交易价为每股1美分。




\section{第14章点评}
\subsection{行动起来}
你该如何应对股票选择这种细致的工作?格雷厄姆建议,防御型投资者可以“直截了当地”购买道琼斯工业平均指数中的每一种股票。如今,防御型投资者能够做得更好:购买整个股票市场指数基金,该基金实际上持有每一种值得拥有的股票。低成本的指数基金,是专门针对小额股票投资的最佳工具。任何想获得更优结果的行为,都将要付出更多的劳动(并且会导致更大的风险和更高的成本),而这对真正的防御型投资者来说是不恰当的。

自己通过研究而挑选股票是不必要的。对大多数人而言,这甚至是不明智的。然而,一些防御型投资者的确会通过挑选个股,来获得娱乐和享受智力挑战。而且,如果你成功经历了熊市,并且仍然能从挑选股票中获得乐趣,那么,格雷厄姆和我所说的任何话,都将不能阻止你这么去做。这种情况下,不要把整个股票市场指数基金作为你的全部证券组合,而是将其作为你的证券组合的基础。一旦你拥有了这个基础,就可以围绕它来做一些自己选择股票的试验了。将购买股票的资金的90\%投入指数基金,而将剩余的10\%用于自己挑选股票。只有在建立了这个稳固的核心之后,你才能去做一些探索。


\subsection{不断的检验}
让我们简要地更新格雷厄姆的股票选择标准。

适度的规模 如今,“为了排除小企业”,大多数防御型投资者应该完全避免购买市值总额不到20亿美元的股票。2003年年初,在标准普尔500指数中,有437家这样的公司可供选择。

然而,与格雷厄姆时代不同,如今的防御型投资者,能够购买专门从事小企业股票交易的共同基金,从而方便地拥有小企业的股份。需要再次指出的是,先锋小企业指数基金(Vanguard Small-Cap Index)是人们的首选,当然,像Ariel、T.Rowe Price、Royce和Third Avenue这样的活跃型基金,也可以按合理的成本购买到。

强有力的财务状况 根据市场战略家加布里斯和摩根士丹利的拉瑟斯提供的信息,在2003年年初,标准普尔500指数中,大约有120家企业能够满足格雷厄姆提出的2:1的流动比标准。由于它们的流动资产至少为流动负债的两倍,因此,这些大规模的流动资本缓冲(一般)能帮助它们度过难关。

...

利润的稳定性 根据摩根士丹利提供的数据,从1993年到2002年,标准普尔500指数的所有公司中,有86\%的公司每年的利润都为正。因此,格雷厄姆所坚持的“过去10年内每年的普通股都有一定的利润”是一个合理的标准——这足以排除经常亏损的企业,但却又不会使你的选择严格局限于不太现实的少数样本企业。

股息记录 根据标准普尔提供的数据,2003年年初,该指数中有354家公司(占总数的71\%)支付了股息。多达225家公司连续支付股息至少长达20年。而且,根据标准普尔提供的信息,该指数中,有57家公司至少在连续25年内提高了自己的股息。虽然这并不能保证将来会永远如此,但这却是一个令人欣慰的迹象。

利润的增长 在截止于2002年的10年间,标准普尔500指数中,有多少家公司的每股收益增长达到了格雷厄姆所要求的“至少三分之一”?(我们将计算出每家公司在1991~1993年的平均利润,然后再看2000~2002的平均利润是否至少增加了33\%。)根据摩托根士丹利提供的信息,标准普尔500股中,有264家公司达到了这一标准。在此,格雷厄姆似乎设定了一个非常低的门槛:10年内33\%的累积增长,意味着年均增长率还不到3\%。如果要求每股收益至少累积增长50\%(年均增长4\%),就不是太保守了。2003年年初,标准普尔500指数中满足这一标准的公司不少于245家,因此,防御型投资者有大量的选择余地。(如果将累积增长门槛调高一倍而达到100\%,即相当于7\%的年均增长率,那么,满足标准的公司为198家。)

适度的市盈率 格雷厄姆建议,只购买当期价格不超过过去3年平均利润15倍的股票。难以置信的是,如今华尔街通行的做法是以当期股价除以所谓的“下一年利润”来对股票进行估价。这就是有时候人们所称的“远期市盈率”。但是,以已知的当期价格除以未知的未来利润来获取市盈率的方法是不明智的。货币经理德曼告诉我们,从长远看,华尔街“一致”采用的利润预测方法中,有59\%会出现很大的偏差——比实际的报告利润低估或高估至少15\%。根据短视的预言家对下一年的预测结果来进行投资,这种行为的风险就好比在射箭锦标赛上,自愿为有资格参赛的盲人举起靶子一样。相反,投资者应该利用格雷厄姆的方法(以当期股价除以过去三年的平均利润),亲自计算股票的市盈率。

2003年年初,标准普尔500指数中,有多少种股票的价格没有超过其2000~2002年平均利润的15倍?根据摩托根士丹利提供的信息,一共有185家公司达到了格雷厄姆的标准。

适度的价格与账面值之比 格雷厄姆建议,“股价与资产之比”(价格与账面值之比)不得高于1.5。最近几年,公司价值中有越来越多的部分来自于特许权、品牌、专利权和商标等无形资产。由于这些因素(以及并购所获取的商誉)不符合账面值的标准定义,因此,如今大多数公司的价格账面值要高于格雷厄姆时代的情况。根据摩托根士丹利提供的信息,标准普尔500指数中有123家公司(占四分之一)的价格与账面值之比要低于1.5,总共有273家公司(占该指数的55\%)的价格与账面值之比在2.5以内。

格雷厄姆建议,将市盈率乘以价格与账面值之比后,观察其结果是否小于22.5。那么,这方面的情况如何呢?根据摩托根士丹利提供的数据,2003年年初,标准普尔500股中,至少有142种股票符合这一标准,其中包括Dana 公司、电子数据系统公司、太阳计算机系统公司和华盛顿互助银行。因此,格雷厄姆的“混合乘数”,仍然可以作为合理定价股票的初选工具。


\section{第15章点评}
\subsection{练习,练习,再练习}
Mutual Series基金的创始人马克斯·海涅(Max Heine)喜欢这样一句话:“通往耶路撒冷的路不止一条。”这位股票大师的意思是,他自己以价值为核心的股票挑选方法,并非是成功的投资者惟一可使用的方法。在本章点评中,我们将介绍如今一些优秀的货币经理在挑选股票时所使用的几种方法。

然而,首先要再次指出的是,个股选择对大多数投资者而言是不必要的,尽管可以这样去做。大多数专业人士在股票选择方面都做得很差,这也就意味着业余人士不可能做得更好。绝大多数试图挑选股票的人都发现,他们做的并没有想象的那么好。极为幸运的人早就发现了这一点,而不太幸运的人则要花几年时间才能懂得这一点。少数投资者擅长挑选自己的股票,其他人要通过别人的帮助(最好是通过指数基金)才能做得更好。

格雷厄姆建议投资者首先要练习,正如优秀的运动员和音乐家在每次实际表演之前的练习和排练一样。他建议,投资者首先花一年的时间去跟踪和挑选股票,但并不是真的去投资购买。...

在实际投资之前对你的技巧进行检验,你即使犯错也不会造成任何实际损失,并且还能避免形成频繁交易的习惯,将自己的方法与那些优秀货币经理的方法进行比较,了解哪些方法是有效的。最为有利的是,跟踪你所有股票挑选的结果,将使你不会忘记,你的一些预感最终被证明是错误的。这将迫使你既向成功者学习,也从失败者身上吸取教训。一年之后,将你的结果与标准普尔500指数基金进行对比,看情况如何。如果你不喜欢这项试验,或者你的选择结果很差,这也没有什么损害,这说明个股选择的做法不适合你。如果这样,就去挑选一家指数基金,不要在股票选择上浪费时间了。

如果你很喜欢这项试验,并且获得了足够高的回报,就可以逐步建立股票组合了。但是,这一股票组合在你的整个证券组合中的比重不能超过10\%(将其余资金投入指数基金),而且要记住,如果你对此不再感兴趣,或者你的回报变得太差的话,你可以随时停下来。



\subsection{老板是谁}
最后要讲的是,大多数优秀的专业投资者都希望公司由这样的人来管理:“不只是经理人,而是像所有者那样进行思考。”(Oakmark公司的尼格伦如是说。)这方面有两个简单的标准:公司的财务报告易于理解,还是模棱两可?“一次性的”、“异常的”和“特殊的”费用是事实上如此,还是令人不愉快地经常出现?

Longleaf公司的梅森·霍金斯寻找的是这样的公司经理:他们是“良好的合伙人”——他们开诚布公地讨论问题,他们对当期和未来现金流的分配有明确的计划,而且他们拥有该公司大量的股份(最好是通过现金购买的,而不是通过派送期权获得的)。可是,“如果企业管理层大谈股价而少谈业务,”Torray基金的罗伯特·托雷告诫说,“我们是不感兴趣的。”Davis基金公司的戴维斯希望企业将股票期权的发行额控制在现有股份大约3\%的范围之内。

先锋Primecap基金的霍华德·肖跟踪的是,“公司第一年所说的是什么,第二年发生的情况又是什么。我们不仅要了解管理层是否对股东诚实,而且还要了解他们自己是否说到做到。”(如果某公司的老板在业务混乱时坚持认为一切都令人满意,请保持警觉。)如今,即使只拥有少数的几股,人们也可以定期参加公司的股东大会。想知道股东大会召开的时间,可以电话联系公司总部的公共关系部门,或者访问公司的网站。

FPA资本基金的罗德里格斯会查看公司年度报告的背面——这里列有各个业务部门负责人的名单。如果在新的CEO任期内的头一年或两年中更换了许多名字,这可能是一个好兆头:说明他正在淘汰不称职的人。可是,如果此后还是不断更换人名的话,这种调整就有可能转变成混乱。


\subsection{从每股收益转向投资资本回报}
由于股票期权派送以及会计盈利和扣除等因素的影响,近几年的每股净收益或每股收益(EPS)已经被扭曲了。想了解公司经营活动中所使用的资本带来的真实利润,就需要从每股收益转向投资资本回报(ROIC)。Davis基金的克里斯托弗·戴维斯将ROIC以下列公式来定义:

ROIC=所有者收益÷投资资本

\begin{verbatim}
公式中的所有者收益等于:
营业利润
加上 折旧
加上 商誉的摊销
减去 联邦所得税(按公司的平均所得支付)
减去 股票期权的成本
减去 “维持”(或实际的)资本支出
减去 不可持续的养老金回报所带来的任何收益
(2003年,任何高于6.5\%的回报都是不可持续的)
\end{verbatim}


\begin{verbatim}
公式中的投资资本等于:
资产总额
减去 现金(以及短期投资和非生息的流动负债)
加上 过去降低投资资本的会计扣除
\end{verbatim}

ROIC的优点在于,它在扣除所有合理费用之后,反映了公司从其经营活动中所获利润,以及公司在利用股东资金获取回报方面有多大的效率。ROIC只达到10\%就很有吸引力了;即使6\%或7\%也会有吸引力,如果该公司有一个好的品牌、目标明确的管理层,或者只是遇到了暂时的困难。

\subsection{看清道路}
还有其他更多通往耶路撒冷的道路。

一些优秀的证券组合经理(比如Dreman价值管理公司的戴维·德尔门和Third Avenue基金公司的马丁·惠特曼)都把目光集中于资产、利润或现金流的乘数非常低的公司。也有人(比如,Royce基金的查尔斯·罗伊斯和富达低价股基金的乔尔·蒂林哈斯特)追逐的是价值被低估的小公司。想简要了解如今最受尊敬的投资者巴菲特是如何选择公司的,请参见下面的专栏内容。

有一种方法会有所帮助:看哪一位优秀的职业货币经理与你持有相同的股票。如果经常会出现一两个这样的人,请访问这些基金公司的网站,并下载其最新的报告。通过查看这些投资者还拥有其他的股票种类,你能更多地了解这些股票的共同点。通过阅读经理的评论,你会知道如何去改进自己的方法。

无论他们使用什么样的技巧来挑选股票,成功的投资专家都有两个共同点:首先,他们遵守既定的约束,并一贯坚持自己的行为,拒绝改变自己的方法,即便这种方法已不再流行;其次,他们大量思考的是做什么以及如何去做,而很少去关注市场情况如何。


\subsection{巴菲特的做法}
格雷厄姆最优秀的学生沃伦·巴菲特已经成为了全球最成功的投资者,其做法是在格雷厄姆想法的基础上设计出新的窍门。巴菲特及合伙人查尔斯·门格(Charles Munger)把格雷厄姆的“安全性”和远离市场的观点,与自己强调未来增长的创新做法结合起来了。下面是对巴菲特方法的简要介绍。

他寻求的是“特许经营”的公司,该公司有很强的消费品牌,有很容易被人们理解的业务活动,有强有力的财务状况,以及有近乎于垄断的市场——比如H \& R Block 、吉列和华盛顿邮报等公司。巴菲特喜欢在公司丑闻、巨额亏损和其他坏消息似乌云飘过时迅速下手购买其股票——比如,1987年可口可乐公司推出的“新可乐”遭到惨败、公司股价崩盘后,他迅速买入了该公司的股票。他还喜欢公司的经理具备下列行为:制定并实现合理的目标;从自己内部而不是通过并购来拓展业务;明智地进行资本分配;不给自己派送上百万美元的股票期权奖金。巴菲特坚持认为,利润的增长应该是稳步和持久的,这样,公司未来的价值会比今天更大。

在自己的年度报告(参见www.berkshirehathaway.com)中,巴菲特完全袒露了自己的思想。或许,没有哪一位投资者(包括格雷厄姆在内)会公开透露更多关于自己投资方法的信息,或公开发表如此通俗易懂的文章。(巴菲特有一句经典名言:“当声誉卓越的管理层经营一家企业而搞得一团糟的时候,人们心目中将只会留下企业的形象。”)每一位聪明的投资人都可以(和应该)通过阅读这位大师自己的语言来获得更多的知识。


\section{第16章点评}
\subsection{对可转换债券的热情}
尽管可转换债券被称为“债券”,但它们的行为却类似于股票,原理类似于期权,而且显得有些模糊。

如果你拥有一种可转换债券,你同时也拥有一种期权:你可以继续持有该债券并获取利息,你也可以按事先规定的价格用它来换取债券发行公司的普通股。(期权的所有者可以在特定期限内,按规定的价格购买或出售另一种证券。)由于可以转换成股票,因此,可转换债券所支付的利率,要低于大多数不可转换的同类债券。另一方面,如果公司的股价急速上升,可以转换成该股票的债券的业绩,要大大好于传统的债券。(相应地,当债券市场下跌时,一般的可转换债券——其利率较低——的结果会更差。)

根据Ibbotson Associates提供的信息,从1957年到2002年,可转换债券的年均回报率为8.3\%,只比股市的总体回报率低2\%。然而,可转换债券的价格更平稳,亏损要更小一些。与股票相比,收益更高,风险更低:难怪华尔街的销售人员经常把可转换债券说成“两方面都是最优的”投资。可是,聪明的投资者很快就会意识到,可转换债券比其他大多数债券的收益更低,风险更大。因此,出于同样的逻辑以及相同的理由,可转换债券也可以被称为“两方面都是最差的”投资。至于你支持哪一方的观点,取决于你如何去使用可转换债券。

实际上,可转换债券更像股票,而不太像债券。可转换债券的回报与标准普尔500指数的回报的相关性大约为30\%。因此,可转换债券与大多数债券的行为是不一致的。对于大部分或全部投资都属于债券的稳健投资者而言,增加一个分散化的可转换债券组合是一种明智的做法:它既可以获得股票一样的回报,又不用直接承担股票投资的可怕风险。你可以将可转换债券称为“胆小者的股票”。

正如Advent资本管理公司的可转换债券专家巴里·纳尔逊(F.Barry Nelson)所指出的,自从格雷厄姆所处的时代以来,这个规模大约为2 000亿美元的市场已经兴旺起来了。现在,大多数可转换债券都属于期限在7~10年间的中期债券;大约有一半属于投资级的债券;而且,许多都附带有一些赎回保护条款(以防止债券被提前赎回)。所有这些使得它们的风险比以前有所下降。

少量可转换债券的交易是十分昂贵的,而且除非你在这个行业本身的投资大大超出10万美元,否则要做到分散化也是不现实的。幸运的是,如今,聪明的投资者可以方便地购买到低成本的可转换债券基金。富达和先锋等共同基金的年费还不到1\%,其他一些封闭式基金也可以按合理的成本购买到(而且,有时可以按低于净资产的价值购买到)。

在华尔街,可爱与复杂总是形影不离——可转换债券也不例外。在各种新的可转换债券中,有大量由首字母构成的昵称,比如:LYONS、ELKS、EYES、PERCS、MIPS、CHIPS和YEELDS等。这些复杂的证券给你的潜在亏损设立了一个“下限”,但同时也给你的潜在收益设立了一个上限,而且它们经常会迫使你在固定的日期将其转换成普通股。与声称保证没有亏损的大多数投资一样(参见下面的专栏内容),为这些东西承担麻烦一般都是不值得的。不购买这些设计复杂的证券,你完全能避免遭受损失——其做法是,理智地将自己的整个证券组合分散于现金、债券、美国股票和外国股票之中。



\section{第17章点评}
\subsection{有更多的事情发生了变化……}
格雷厄姆重点分析了四个极端的案例:

\begin{itemize}
\item 股价被高估的一家“摇摇欲坠的大公司”
\item 类似于帝国大厦的一家综合大企业
\item 一次蛇吞象的并购
\item 一家根本不值钱的公司所进行的股票首次公开发行
\end{itemize}

过去几年内,格雷厄姆介绍过的这些案例层出不穷。下面给出几个例子。

\subsection{难以看透的朗讯公司}
2000年中期,投资者拥有的朗讯技术公司(Lucent Technologies Inc.)的股票是最多的。该公司资本市值为1 929亿美元,在美国最具价值的公司中名列第12位。

\begin{figure}[H]
\centering
\includegraphics[width=\linewidth ,totalheight=0.95\textheight , keepaspectratio]{朗讯技术公司的财务信息.jpg}
\caption{朗讯技术公司的财务信息}
\end{figure}

所有的单位都是百万美元。

* 包括商誉在内的其他资产。

资料来源:朗讯公司的季度财务报告(表10-Q)。

对这位巨头的估价合理吗?让我们来看一看截止于2000年6月30日的季度财务报告所反映的一些基本信息。

更加仔细地阅读朗讯公司的财务报告,会感觉到警铃之声不绝于耳:

\begin{itemize}
\item 朗讯以48亿美元刚刚收购了一家光纤设备供应商Chromatis网络公司——这48亿美元中,有42亿美元属于“商誉”(高于账面值的成本)。Chromatis有150名员工,没有客户,没有任何销售收入,因此“商誉”这一项记录是不恰当的;或许,更为准确的称呼应该是“所怀有的希望”。如果Chromatis尚未成熟的产品还没成功,朗讯就必须减记这笔商誉,并且将其从未来的利润中扣除。

\item 有一条注释透露,朗讯已经向购买自己产品的人提供了15亿美元的信贷,朗讯还向在别处借款的客户提供3.5亿美元的担保。这些“客户融资”的总额,在一年之内增加了一倍——这表明,购货方已经快要拿不出资金来购买朗讯的产品了。如果这些客户没有钱偿还朗讯的债务,情况会如何?

\item 最后,朗讯将开发新软件的成本作为“资本资产”。这本来就不是一笔资产,难道这不是应该从利润中扣除的一项日常业务费用吗?
\end{itemize}

结论:2001年8月,朗讯关闭了Chromatis 这一部门,原因据说是其产品只有两个客户感兴趣。2001年财务年度,朗讯亏损了162亿美元;2002年财务年度,又亏损了119亿美元。这些亏损额中包括:35亿美元的“不良债务和客户融资备抵”,41亿美元的“与商誉相关的减损”,以及3.62亿美元“与资本化软件相关的”费用扣除。

朗讯的股价从2000年6月30日的51.062美元,降到了2002年年底的1.26美元——两年半之内,其市值几乎亏损了1 900亿美元。

\subsection{并购魔术师}
说到泰克国际有限公司(Tyco International Ltd.),我们只能转述丘吉尔说过的一句话:有史以来,从没有过如此少的几个人,能够欺骗如此多的公众。从1997年到2001年,这家位于百慕大的综合大企业总共花了370多亿美元(其中大多数为泰克股份)去并购其他公司,就像伊梅尔达·马科斯(Imelda Marcos)买鞋子那样频繁。根据其年度报告,2000年财务年度,泰克并购了“大约200家公司”——平均算下来每两天要收购不止一家公司。

结果如何呢?泰克的规模急剧增大;在5年内,销售收入从76亿美元,增加到了340亿美元,营业利润也从4.76亿美元的亏损转变成了62亿美元的盈利。难怪2001年年底,该公司的总市值高达1 140亿美元。

\begin{figure}[H]
\centering
\includegraphics[width=\linewidth ,totalheight=0.95\textheight , keepaspectratio]{泰克国际有限公司的财务信息.jpg}
\caption{泰克国际有限公司的财务信息}
\end{figure}

所有数据都来自于原始报告,单位是千万美元。

“并购”总额中,不包括权益联合的交易。

资料来源:泰克国际的年度报告(表10-K)。

然而,泰克的财务报告至少也与其增长速度一样,令人难以想象。几乎在每一年,都报告有数亿美元与并购相关的费用产生。这些费用主要包括三类:

\begin{enumerate}
\item “并购”、“重组”或“其他一次性的”成本。
\item “长期资产损耗的费用”。
\item “购买在途研发产生的冲销额”。
\end{enumerate}

为了简便,我们将第一种费用简称为MORON,第二种简称为CHILLA,第三种简称为WOOPIPRAD。它们是在什么时间出现的呢?

正如你所看到的,MORON费用(应该是一次性的)在5年中出现了4次,其总额高达25亿美元。CHILLA也经常出现,并且总额超出了7亿美元。WOOPIPRAD大约有5亿美元。

聪明的投资者可能会问:
\begin{itemize}
\item 如果泰克借助并购求增长的策略是一个如此巧妙的想法,那么它为什么平均每年要花7.5亿美元才能将这项业务完成?
\item 如果(正如人们明显看到的)泰克不从事生产活动,而只是收购从事生产活动的其他公司,那么MORON这一项费用为什么会是“一次性的”?它们难道不正是泰克正常业务成本的一部分吗?
\item 随着以往并购的会计费用使每一年的利润都被看好,谁知道下一年的情况将会是怎样的?
\end{itemize}

实际上,投资者甚至不知道泰克过去的利润是多少。1999年,在美国证券交易委员会对其进行会计审核后,泰克反过来又在1998年的费用中增加了2.57亿美元的MORON费用——这意味着那些“一次性的”成本实际上又在1998年出现了。与此同时,泰克公司更改了1999年最初所报告的费用:MORON这一项降为9.29亿美元,而CHILLA这一项上升到5.07亿美元。

显然,泰克的规模正在扩大,但是,其利润率也在增长吗?外人无法准确地知道。

结论:在2002年财务年度,泰克亏损了94亿美元。其股价在2001年年底收于58.9美元,而2002年年底只有17.08美元——一年之内亏损了71\%。

\subsection{蛇吞象}
2000年1月10日,美国在线公司和时代华纳公司宣布,它们将进行一项初始价值高达1 560亿美元的合并。

截止于1999年12月31日,美国在线拥有103亿美元的资产,公司前12个月的营业收入为57亿美元。另一方面,时代华纳拥有521亿美元的资产,以及273亿美元的营业收入。除了股票估值这一项之外,从其他各方面看,时代华纳的规模都要大许多。由于美国在线仅凭自己所在的互联网行业就使得投资者痴迷于它,因此,其股价令人瞠目地达到了其利润的164倍。时代华纳从事有线电视、电影、音乐和杂志等多项业务,其股价大约只有利润的50倍。

在宣布这项交易时,两家公司都称这是一次“平等的战略合并”。时代华纳的董事会主席杰拉尔德·莱文声称:“对于与美国在线时代华纳有关系的任何人来说,都将面临着无数的机会。”——尤其是对于其股东而言,他补充说。

想到其股票最终将贴上互联网宠儿的标志而欣喜若狂,因此,时代华纳的股东以压倒性的优势批准了这笔交易。然而,他们却忽视了这样几点:

\begin{itemize}
\item 根据这次“平等的合并”,美国在线股东将拥有公司合并后55\%的股份——尽管时代华纳的规模是美国在线的5倍。
\item 3年内,美国证券交易委员会曾两次调查美国在线营销成本的会计记录是否恰当。
\item 美国在线总资产中,有将近一半(49亿美元)是由“待售股份”组成的。如果公开上市的技术股下跌,这将会使该公司的基础资产大多化为泡影。
\end{itemize}

结论:2001年1月11日,两家企业最终完成了合并。2001年,新组建的美国在线时代华纳亏损了49亿美元,而且在2002年又亏损了987亿美元(这是有史以来所有公司亏损中最大的一笔金额)。大多数亏损来自于美国在线价值的减记。到了2002年年底,莱文当初预计会面临着“无数”机会的股东没有任何收获,相反,自从交易首次宣布以来,他们的股票价值大约亏损了80\%。

\subsection{投资幼儿园会失败吗}
1999年5月20日,eToys公司向公众发售了8\%的股份。华尔街最有名气的4家投资银行——高盛、BancBoston Robertson Stephens、Donaldson,Lufkin \& Jenrette和美林——以每股20美元的价格承销了832万股,为公司筹集了1.664亿美元资金。该股票价格急速上涨后,收于76.5625美元,在首个交易日就获利282.8\%。按照这一价格,eToys(它拥有1.02亿股)的市值高达78亿美元。

股票购买者得到了什么?前一年,eToys公司的销售额上升了42.61倍,仅最后一个季度,它又增加了7.5万个客户。然而,经营20个月之后,eToys的总销售额为3 060万美元,并且有3 080万美元的净亏损——这意味着,eToys每卖出1美元的玩具,就要花费2美元。

IPO招股说明书还透露,eToys将使用出售股票的部分收入来收购另一家网络公司BabyCenter——前一年,BabyCenter的销售额为480万美元,亏损为450万美元。(要收购这家公司,eToys只需支付2.05亿美元。)eToys还要“预留”4 060万份普通股,以便将来出售给其管理层。因此,如果eToys能够赚钱的话,其净收益将在1.43亿份股票之间进行分配,而不是在1.02亿份股票之间进行分配——这将使得未来的每股收益下降近三分之一。

把eToys与其最大的竞争对手玩具反斗城(Toys“R”Us)进行对比后,会使人感到非常惊讶。前3个月内,玩具反斗城公司获得了2 700万美元的净收益,而且其销售额是eToys全年的70多倍。然而,从图17-3中可以看到,eToys的股市估价比玩具反斗城高出近20亿美元。

结论:2001年3月7日,在公开上市后不久,净亏损超过3.98亿美元之后,eToys公司提出了破产保护申请。该公司的股票在1999年10月曾达到每股86美元的最高价,但最后几乎一文不值了。


\section{第19章点评}
\subsection{格雷厄姆为何会放弃努力}
在《聪明的投资者》这本书中,格雷厄姆对这一章内容的改动或许是最大的。在本书的第一版中,本章的内容只是两项内容中的一项——两项内容一共接近34页。最初的内容(“作为企业所有者的投资者”)涉及股东的投票权、评价公司管理层水平的方法,以及观察公司内部人员与外部投资者之间利益冲突的方法等。可是,在最后修订的这一版中,格雷厄姆将整个内容压缩成了只有关于股息的几页内容。

格雷厄姆为什么将原来的论述砍去了四分之三以上?经过了几十年的劝告之后,格雷厄姆显然放弃了希望:在监控公司管理者的行为方面,投资者永远都不会感兴趣。

可是,近期广泛出现的丑闻(比如,美国在线、安然、Global Crossing、Sprint、泰克和世界通信等大公司,在管理者行为不当、会计不透明和税收操纵等方面受到的指控)明确地告诉我们,格雷厄姆先前关于内部监督的告诫,比以往任何时候都显得更重要。下面我们将以如今发生的事件为例重新探讨这一话题。

\subsection{理论与实际的对比}
在本书的第一版(1949年)中,探讨“作为企业所有者的投资者”这一项内容时,格雷厄姆认为,从理论上讲,“股东这一群体拥有最大的权利。作为绝大多数,他们可以雇用和解聘公司管理层,因此,他们能让公司管理层完全服从于自己的意愿。”但是,从实践来看,格雷厄姆说:

\begin{quote}
“股东完全是无能的。作为一个群体,他们既没有表现出智慧,也没有表现出警觉。他们对管理层所建议的东西都去投票赞成,而不管管理层的的记录有多么糟糕。……想使一般的美国股东独立地采用理智的行动,惟一的办法就是对其当头一棒。……我们不得不指出这样一个似乎矛盾的事实:耶稣似乎是比美国股东更为现实的商人。”
\end{quote}

格雷厄姆想要你认识到一个基本但却十分深刻的事实:当你购买了某公司的股票时,你就成为该公司的所有者。包括CEO在内,公司的所有管理者都是在为你工作。公司的董事会必须对你的问题做出回答;公司的现金属于你;公司的业务是你的财产。如果你不喜欢公司的管理方式,你有权要求解雇管理者,更换董事,或者是将公司财产出售。“股东们”,格雷厄姆说,“应该明白过来了。”

\subsection{聪明的所有者}
如今,投资者已经忘记了格雷厄姆的忠告。他们在购买股票上花费大量的精力,在出售股票上花费很少的精力,然而,在拥有股票上却没有花任何精力。“毫无疑问”,格雷厄姆提醒我们,“无论是即将成为股东,还是已经成为股东,都需要有细致的判断。”

这样,作为一位聪明的投资者,你如何才能成为一位聪明的所有者呢?格雷厄姆首先告诉我们,“股东只需要关注两类基本问题。

\begin{enumerate}
\item 企业管理层是否具备应有的效率?
\item 普通的外部股东的权益是否得到了适当的认可?”
\end{enumerate}

将公司的盈利能力、规模和竞争力等方面与同类公司进行对比,就可以判断公司管理层的效率。如果你得出的结论是管理层缺乏效率,那么该怎么办呢?这时,格雷厄姆认为:

\begin{quote}
少数重要的股东应该确信公司需要改革,并且朝着这一方向努力。其次,普通的股东应该公正地阅读代理材料,并对双方的观点进行权衡。他们至少应该知道公司的业务何时开始出现问题,而且不能只听信现有的管理层以虚假的陈词滥调所做的辩护。第三,如果数据清楚地表明公司的经营结果大大低于平均水平,那么最好是有这样一种惯例:请公司外部的企业策划师(business engineer)来对管理层的政策和能力做出评判。
\end{quote}

什么是“代理材料”,格雷厄姆为什么坚持认为应该阅读这份材料?在发给每位股东的代理报告中,公司会宣布股东大会的日程,并详细披露其管理者和董事的薪酬和股份,以及公司内部人员和公司之间交易的相关信息。股东要投票表决挑选出审计师事务所以及公司的董事。如果你根据常识来阅读代理报告,那么这份材料就类似于煤矿中的金丝雀——一个早期预警信号(金丝雀对于瓦斯的敏感性远远高于人类,当瓦斯浓度过高时,金丝雀就会停止歌唱或死去,因此在煤矿中能够起到预警作用——译者注)。

然而,一般情况下,有三分之一到二分之一的个人投资者不会针对代理报告中的问题投票。他们会去阅读代理报告吗?

理解并对代理报告进行投票,是对信息进行跟踪的聪明投资者的基本行为,而且,根据自己的良心投票是一个好公民应有的行为。无论你拥有公司股份的10\%,还是只拥有微不足道的100股(只占公司股份的百万分之一),这并不重要。如果你从未阅读过自己股票的代理报告,而公司破产了,那么你只能怪自己。如果你阅读了代理报告,并发现了一些使人感到不安的东西,那么:

\begin{itemize}
\item 投每位董事的反对票,以让其知道你不同意他们的行为
\item 参加股东大会,并阐述自己的权利
\item 找一个针对股票业务的网上信息公告栏(比如,在http://finance.yahoo.com这一网站上的公告栏),号召其他投资者加入到你的行动中来
\end{itemize}

格雷厄姆的另外一个想法对如今的投资者会有帮助:

\begin{quote}
……挑选一位或多位职业董事和独立董事是有好处的。这些人必须有广泛的商界经历,能够从不同的专业角度来观察企业的问题。……他们要提供一份独立的年度报告,报告直接面向股东,并且要包含在重大问题上的一些观点——这些问题是企业所有者所关注的:企业给外部股东带来的结果,符合正常管理条件下的预期结果吗?如果不符合,那么原因是什么——以及应该怎么办?
\end{quote}

人们可以想象,格雷厄姆的建议将在公司密友和高尔夫伙伴之间造成多大的恐慌(如今的“独立”董事大多是由他们来担任的)。(我们并不认为这会使得他们的脊背发凉,因为大多数独立董事似乎没有脊梁。)

\subsection{到底是谁的钱}
现在让我们来看格雷厄姆的第二个标准:管理层的行为是否最有利于外部投资者。管理者总是告诉股东,他们最了解公司的资金应该怎么使用。格雷厄姆正好看穿了管理者这种糊弄人的话:

\begin{quote}
公司的管理层有可能把业务经营得很好,但却不能给外部股东带来理想的结果,因为公司的效率取决于业务活动,不一定反映了资本的最有效利用。有效经营的目标是以最低的成本进行生产,并且出售最赚钱的东西。有效融资要求股东资金的使用最符合他们的利益。这是目前的管理层不感兴趣的问题。实际上,公司管理层总是想尽可能从所有者那里获得资本,以尽量减少自己的财务问题。所以,一般的管理层都会利用过多的资本来开展经营——只要股东许可的话(情况通常是这样的)。
\end{quote}


20世纪90年代末和21世纪初,一些主要的技术型公司的管理层,将“家长最懂行”这种态度发展到了新的极致。其理由是这样的:既然我们能够替你进行投资,并将投资转变为股价的上涨,你为什么要求支付股息呢?仅看一看我们的股价上涨得有多快——难道这还不能证明我们比你更能使资金增值吗?

难以置信的是,投资者完全接受了这种观点。“家长最懂行”因此而成为一种信条:到了1999年,首次公开发行股票的公司中,只有3.7\%在当年支付了股息,而20世纪60年代所有IPO支付股息的平均比重为72.1\%。...

可是,“家长最懂行”只不过是一句假话。尽管有些公司使自己的资金发挥了很好的作用,但更多的公司则面对着其他状况:有的只是把资金浪费了,而有的资金因增长过快超出了应有的需求。

关于第一类公司,请看这样几个例子。冒然地进入了食品杂货业和汽油行业之后,价格线网络公司(Priceline.com)在2000年亏损了6 700万美元;由于对Webvan和Ashford等网络公司的“投资”遭遇巨大失败,亚马逊网络公司给股东的财富至少造成了2.33亿美元的损失。迄今为止,最大的两笔亏损记录(2001年JDS Uniphase公司的560亿美元的亏损,以及2002年美国在线时代华纳公司990亿美元的亏损),是在这些公司不愿支付股息而决定与股价高估得令人恐怖的企业合并时发生的。

就第二类公司而言,我们来看这样几个例子。2001年年底,甲骨文公司积累了50亿美元的现金。思科公司至少储存了75亿美元的现金。微软积累的现金高达382亿美元——而且还在以平均每小时200万美元的速度增加。不知道比尔·盖茨预计将来会有多少困难?

这些事例清楚地表明,许多公司都不知道如何将闲置现金用来赚取额外的回报。统计方面的证据能够告诉我们什么呢?

\begin{itemize}
\item 罗伯特·阿诺特和克利福德·阿斯尼斯等货币经理的研究表明,当期股息较低的时候,公司未来的利润最终也会较低。当期股息较高的时候,未来的利润也较高。在10年期内,股息较高的公司的平均利润增长率,比股息较低的公司要高出3.9个点。
\item 哥伦比亚大学的会计学教授多伦·尼斯姆和阿米尔·齐夫的研究发现,增加股息的公司不仅有更高的股票回报率,而且“其股息的增加还会使得此后至少未来4年内的利润率(更高)”。
\end{itemize}

总之,当管理者说自己能够比你更好地使现金发挥作用时,他们的话语大多数是错的。支付股息不一定保证是最好的结果,但是这的确可以使得一般的股票回报得以改进,因为这至少从管理者手中挤压出了一部分可能会被浪费或储存起来的现金。


\subsection{贵买贱卖}
公司以闲置现金买回自己的股票是较好的办法,这一观点如何?当公司购回其中的一些股票时,现有的股票数量会下降。即使公司的净收入不变,其每股收益也会上升,原因在于,其总利润将分摊到更少的股份数之中。反过来,这会使得股价上升。更有利的是,与股息不同,回购股份对持股的投资者来说是免税的。因此,它在使股票升值的同时,不会加重税收负担。所以,如果股价便宜,将闲置资金用于回购股票,是利用公司资本的一个很好的方法。

从理论上讲,所有这些都是正确的。遗憾的是,在现实世界中,股票回购只会带来一种不祥的后果。由于发放的股票期权已占高管薪酬的很大一部分比重,因此,许多公司(尤其是高科技行业的公司)必须向行使股票期权的管理者发售上百万份股票。但是,这会抬升现有股份的数量,并降低每股收益。为了抵消这种稀释作用,公司必须回过头来在公开市场回购上百万份股票。2000年,公司全部的净收入中,竟然有高达41.8\%的净收入用于回购自己的股份;而1980年,这一比重只有4.8\%。

让我们来看软件行业的巨头甲骨文公司的情况。从1999年1月1日到2000年5月31日,甲骨文向其高管发行了1.01亿股普通股,并且向员工发行了2 600万股普通股——筹集到了4.84亿美元的资金。与此同时,为了防止行使股票期权给每股收益带来的稀释作用,甲骨文花费53亿美元(占当年总收入的52\%)回购了2.907亿股。甲骨文向公司内部人员出售股票的平均价为每股3.53美元,而回购股票的平均价为每股18.26美元。贵买贱卖:这是“提升”股东价值的方法吗?

2002年,甲骨文的股价还不到其2000年最高值的一半。既然股价更便宜了,甲骨文会抓紧时间回购更多的股票吗?从2001年6月1日到2002年5月31日,甲骨文将回购的股票压缩到28亿美元。这显然是因为其高管和员工这一年行使的股票期权较少。在其他十几家技术型公司中,也明显存在着这种贵买贱卖的现象。

这是为什么?是因为有两个令人意外的因素在发挥作用:

\begin{itemize}
\item 高管和员工行使股票期权会给公司带来税收上的优惠(国税局将行使的股票期权看做公司的“薪酬费用”)。比如,在2000~2002年财务年度,甲骨文因为公司内部人员兑现期权而获得了16.9亿美元的税收优惠。1999年和2000年,由于其高管和员工获取了19亿美元的股票期权利润,Sprint公司自己也获得了6.78亿美元的税收优惠。
\item 薪酬中股票期权占很大比重的高管,从自身利益出发,必然会赞成股票回购,而不赞成股息派发。为什么?从技术上讲,期权的价值会随着股价波动幅度的增加而增加。但是,股息会抑制股价的波动性。所以,如果管理者增加股息,就会降低自己拥有的股票期权的价值。
\end{itemize}

难怪CEO们更愿意回购股票而不愿意支付股息了。不管股价如何波动和高估,也不管这会浪费外部股东多少资金。

\subsection{让股票期权公开}
最终,懒散的投资者只能让公司肆无忌惮地给高管发放高额的薪酬。1997年,苹果电脑公司的共同创始人史蒂夫·乔布斯以“临时”首席执行官的身份重返公司。由于已经很富有,乔布斯坚持每年只要1美元的现金薪酬。1999年年底,为了感谢这位“在过去的两年半内没有获取薪酬”的CEO,公司董事会送给了乔布斯一架Gulfstream喷气飞机——这就花费了公司9 000万美元。接下来的一个月,乔布斯同意从头衔中取消“临时”两个字,这样董事会分给了他2 000万美元的股票期权。(此时,乔布斯总共持有2 000股苹果股份。)

分配这种股票期权的原则是使管理者的利益与外部投资者的利益一致。如果你是苹果股份的外部持有者,那么只有当苹果的股票获得很好的回报时,你才允许公司的管理者得到奖励。对于你和公司其他的所有者而言,任何其他做法都是不公平的。然而,正如先锋基金的前董事会主席约翰·博格尔所说的,几乎所有的管理者都会在行使期权后,立即将所获股票出售。一下子出售上百万份股票以获取眼前的利润,这种行为怎么可能与忠实于公司的长期股东的利益是一致的?

以乔布斯的情况为例,如果苹果公司的股价仅按5\%的年率增长到2010年年初,他就能够将股票期权兑现为5.483亿美元。换句话讲,即使苹果的股票回报率还不到整个股市长期平均回报率的一半,乔布斯也会有5亿美元的意外收获。乔布斯的利益与苹果公司股东的利益相一致吗?或者说,苹果公司的董事会是否滥用了公司股东对自己的信任?

仔细阅读代理报告之后,聪明的所有者将会投票否决下列高管薪酬计划:利用股票期权将公司3\%以上的现有股份变成管理者所有。凡是不与公平和持久的公司优异结果(比如,至少5年内的股票回报要高于同行业的平均水平)挂钩的股票期权计划都应该被否决。如果他给你带来的结果不理想,那么这位CEO就不应该使自己变得富有。

\subsection{最后一点思考}
让我们回头来看格雷厄姆的建议:公司的每一位独立董事都应该以书面形式向股东提供报告,说明企业是否从实际所有者的利益出发在开展经营。如果独立董事还应该评判公司股息政策和股票回购政策的合理性,情况会如何?如果他们应该准确地说明自己是如何判断公司高管薪酬的合理性的,情况会如何?如果每一位投资者都成为聪明的所有者,并且去实际阅读公司的财务报告,情况又会如何?



\section{第20章点评}
\subsection{首先,不要陷入亏损}
什么是风险?

你在不同的时间问不同的人,得到的答案都是不一样的。在1999年的时候,风险并不意味着赔钱了,而是指赚的钱比其他人少。当时许多人所担心的是,在野餐会上突然遇到某个人,而这个人通过进行网络股的短钱交易比自己赚钱更多、更快。随后,到了2003年的时候,风险的含义突然变成了:股市有可能继续下跌,直到将你手中所剩下的一点财富消耗殆尽。

虽然风险的含义看上去几乎与金融市场本身一样,处于捉摸不定和不断变化之中,但是它还是具有某种固定的和持久的特征。在牛市中投下最大赌注和赚钱最多的人,在终将到来的熊市中也伤得最狠。(现在的“正确”使投机者更急于承担更大的风险,因为他们信心倍增。)一旦你输掉大笔资金,你就必须下更大的赌注才能把本追回来,这就好比赌徒们在每一次输钱之后,拼命地把所赢的钱也一同投下去一样。除非你非常幸运,否则,这种做法将会导致灾难。难怪,当有人要求Wertheim \& Co.公司传奇式的金融家克林根斯坦(J.K.Klingenstein)根据自己长期的从业经验对如何致富做出总结时,他直接回答:“不要陷入亏损。”

假设你认为某种股票每年可以上涨10\%,而市场每年只能上涨5\%。遗憾的是,你的热情太高,因此支付的价格过高,所购买的股票在头一年就亏损了50\%。即使该股票后来的收益是市场的2倍,你也要花16年多的时间才能赶上市场——原因就在于,你一开始支付的价格太高,因而亏损太大。

投资必然会出现一定的亏损,这是无法避免的。可是,为了成为一位聪明的投资者,你必须确保自己永远不会使全部或大多数资金出现亏损。印度的财富女神拉希米(Lakshmi)常被人们描绘成以足尖站立,随时准备疾驰而去。为了象征性地留住她,其崇拜者会用绳子将其雕像捆住,或者是将雕像的脚用钉子固定在地上。对聪明的投资者而言,格雷厄姆的“安全边际”起着相同的作用:通过拒绝购买价格过高的证券,你就能降低财富消失或突然毁灭的机会。

...

\subsection{风险不在于我们的股票,而在于我们自身}
风险存在于另一个方面:在于你自身。如果你高估自己对投资的真正理解,或者夸大自己在应对价格临时下降时的能力,那么无论你买什么股票,或者无论市场如何,都没有关系。最终的金融风险不在于你从事了何种投资,而在于你是何种投资者。如果想知道真正的风险在何处,请走进离你最近的一个卫生间,并且站在镜子前面。这就是风险——你从镜子中看到的这个人!

当你从镜子中观察自己时,你应该注意什么?诺贝尔经济学奖获得者、心理学家丹尼尔·卡尼曼解释了好的决策所具备的两个特点:

\begin{itemize}
\item “非常准确地把握信心”(对这种投资的理解,有我所认为的那么好吗?)
\item “正确预知将来的遗憾”(如果我的分析最终被证明是错误的,我该怎么办?)
\end{itemize}


为了了解你的信心是否准确地得到了把握,对着镜子问自己:“我的分析在多大程度上是正确的?”仔细考虑这样一些问题。

\begin{itemize}
\item 我有多少经验?过去在这些决策方面我的表现如何?
\item 过去从事过这方面尝试的其他人的业绩一般如何?
\item 如果我准备购买,其他人准备出售。下列情况有多大的可能:我所知道的某种东西,是做交易的其他人(或公司)并不知道的?
\item 如果我准备出售,其他人准备购买。下列情况有多大的可能:我所知道的某种东西,是做交易的其他人(或公司)并不知道的?
\item 我是否进行了下列计算:这笔投资需要上涨多少,才能弥补我的税款和交易成本?
\end{itemize}


按下来,对着镜子思考,你是否能够正确地预知将来的遗憾。首先问自己:“如果我的分析最终被证明是错误的,我能完全理解将会产生的后果吗?”通过下列三点来回答这个问题:

\begin{itemize}
\item 如果我是正确的,有可能赚很多钱。但是,如果我是错误的呢?从以往类似投资的结果来看,我能承受多大的亏损?
\item 如果这项决定最终是错误的,还有其他一些投资帮我度过难关吗?当我正在考虑的这种投资出现价格下跌时,我已经持有的股票、债券或基金的价格肯定会上涨吗?这笔新的投资会使得我的资本风险过大吗?
\item 当我告诉自己“你对风险有很强的承受力”的时候,我是怎么知道的?投资曾经使我亏过许多钱吗?亏了之后的感受如何?是购买了更多的投资,还是畏难而去?
\item 出了问题时,我能完全凭借自己的毅力来消除恐慌吗?或者说,我通过投资分散化、签订投资合同和成本平均法事先控制好了自己的行为吗?
\end{itemize}

用心理学家保罗·斯洛维奇的话来说,你应该始终记住,“风险来自于两种等同的要素——可能性和后果。”投资之前,必须确保切实评估自己判断正确的可能性,以及错误发生时将对后果做出怎样的反应。

\subsection{帕斯卡的赌注}
投资理论家彼得·伯恩斯坦(Peter Bernstein)有另一番总结,他谈到了法国伟大的数学家和神学家帕斯卡(1623~1662年)。帕斯卡进行过一次思维试验,试验中无神论者必须就上帝是否存在打赌。打赌者必须事先投下的赌注,是他这一生的行为举止;打赌者的最终回报,是死后自己灵魂的命运。在这一次打赌中,帕斯卡声称,“理智不可能决定”上帝存在的可能性。无论上帝是否存在,都只有信仰(而不是理智)才能回答这一问题。可是,尽管帕斯卡打赌的可能性如同掷硬币一般,但其后果是相当清楚的,并且是完全确定的。正如伯恩斯坦所解释的:

\begin{quote}
假设你的行为好比上帝是存在的,而且(你)过着行为端庄和有节制的生活,而实际上上帝并不存在。这样,你将错过人生中一些带来快乐的东西,但是也会获得一些回报。现在,假设你的行为好比上帝并不存在,并且整个一生都在罪恶、自私和情欲中度过,但事实上上帝是存在的。这样,在你相对短暂的一生中,你享受了快乐和兴奋,但是,当最后的审判到来时,你就会面临巨大的麻烦。
\end{quote}

伯恩斯坦得出的结论是:“在不确定条件下做决定时,后果比可能性要重要得多。我们永远无法知道未来的情况。”所以,正如格雷厄姆在本书的每一章都提醒过的,聪明的投资者一定不能只关注分析的正确性,还必须防止分析结果最终出现错误时的损失——因为,即使是最好的分析,最终也有可能出错。在你的整个投资活动中,至少在某个时间出现一次失误的概率几乎是100\%,而且这种情况是完全无法避免的。然而,你的确可以对错误造成的后果加以控制。1999年,许多“投资者”几乎将所有的资金都投入到了网络股之中;《货币》杂志1999年对1 338名美国人进行的网上调查表明,他们之中近十分之一的人在网络股中投入了至少85\%的资金。由于忽视了格雷厄姆对安全边际的要求,这些人在帕斯卡打赌中下错了赌注。可以肯定的是,他们知道取胜的可能性有多大,但是,他们没有采取任何措施来防范错误发生时的后果。

只要你始终进行分散化的投资,并拒绝花大量的资金去追逐市场上的新宠,你就能保证错误所带来的后果永远不会是灾难性的。无论市场如何引诱你,你将始终能够冷静而充满信心地说:“这,也将成为过去。”



\chapter{二十一世纪前期对世界各地葬礼的观察}
本章划线内皆引用自参考资料2:
\hr
早在2000多年前,古希腊历史学家希罗多德就写过两种文化对彼此的丧葬传统感到震惊的情形,可谓历史上对此现象最早的记录之一。在他的描述中,波斯帝国的统治者召集了一些希腊人到他面前。希腊人有火化遗体的传统,国王便问道:“什么样的奖赏能让你们吃掉父亲的遗体?”这个问题让希腊人惊慌不已,连连解释无论多么丰厚的赏赐都不足以把自己变成食人族。第二天,国王召集了因食用遗体而闻名的卡拉提亚人。国王问他们:“什么样的奖赏能让你们焚烧父亲的遗体?”卡拉提亚人表示这种行径“太可怕了”,请求国王不要再提及此事。
\hr

\section{美国科罗拉多州的克雷斯通镇}
\hr
8月的一个午后,我收到一封期待许久的来信:

凯特琳:

我们社区重要的一员劳拉今早离开了我们。她有心脏病史,刚刚过完75岁生日。不知道你现在身在何处,但欢迎你加入我们。

史蒂芬妮

劳拉的死完全出人意料。周日傍晚,她还在当地的音乐节上纵情起舞,周一一早就被发现倒在厨房的地板上没有了呼吸。火化仪式定在周四上午举行,届时她的家人都会出席,我也会在场。

...

劳拉的家人用布制的担架抬着劳拉穿过一片黑眼花,来到火葬用的火化台旁。一记嘹亮的锣声响彻天空。我沿着停车场旁的沙石小路向前走,一名笑容满面的志愿者递给我一捆新鲜的杜松枝。

劳拉平躺在一块金属炉箅\footnote{bì,蒸锅中的竹屉。}上,左侧和右侧各有一面光滑的白色水泥壁,上方则是一望无际的科罗拉多苍穹。我在这里见过两次空空如也的火化台,但这一次,遗体的出现令我理智、清晰地认识到这种仪式的目的。凭吊的人一个接一个地走上前来,把手中的杜松枝放到劳拉身上。作为唯一不认识劳拉的人,我有些犹豫——我管这个叫葬礼尴尬症。但我既不能一直把杜松枝拿在手里(太明显了),又不能放进背包(太没品了),于是我只好上前把枝条放在遗体上。

接下来,劳拉的家人(包括一个八九岁的小男孩)用矮松木的松枝和云杉原木把柴堆围起来。这两种木材越烧密度越高,所以通常是火化的首选。劳拉的伴侣和她已经成年的儿子手持火把站在角落里。随着信号出现,他们一起走向劳拉,将柴堆点燃。此时,太阳刚好从地平线上升起。

当火焰开始吞噬劳拉的身体时,白色的浓烟像小型旋风似的旋转上升,然后逐渐消失在清晨的天空中。

火化的味道让我想起爱德华·阿比作品中的一段描述:

\begin{quotation}
火焰。在我看来,杜松枝燃烧时发出的香味是世界上最甜美的,我怀疑但丁笔下的天堂都没有能与此媲美的熏香。就像雨后灌木丛的味道,只需闻一下,就能引发魔法般的通感:那是美妙的音乐,那是美国西部的广袤、光明、澄澈和神秘。希望火焰永远燃烧下去。
\end{quotation}

几分钟之后,螺旋状的浓烟不见了踪影,取而代之的是明亮的红色火焰。火焰攒足了能量,足足蹿到6英尺高。参加葬礼的一共有130人,他们全围绕在火化台旁,无声地看着眼前的一切。唯一的声响来自树枝断裂时的“噼啪”声,仿佛劳拉的每一段记忆都随着声响飘散在晨空中。

在科罗拉多州的克雷斯通进行的这种火葬仪式已有上万年的历史。古希腊人、古罗马人及印度教徒,都因使用最朴实的秘术——火——来消除肉体和净化灵魂而闻名于世。但火葬本身还可以追溯到更古老的时期。

20世纪60年代,一名年轻的地质学家在澳大利亚内陆发现了一具火化过的成年女性遗骸。根据他的预测,这具遗骸距今已有20000年。事实上,后续研究表明这名女性应该生活在42000年前,比原住民到达澳大利亚的推测时间还要早22000年。她应该是居住在布满植被的山坡上,和一群巨型生物(袋鼠、袋熊以及其他尺寸超常的啮齿类动物)为伴,以鱼肉、植物的种子和巨型鸸鹋\footnote{ér miáo,鸟纲,鸵鸟目,鸸鹋科。}的蛋为食。她死后,族人便把她(现在被称为“芒戈女士”)火化。没有焚烧完全的遗骨将被捣碎并进行二次火化。最后,族人用红色泥土把遗骸包裹起来埋在地下,一晃就是42000年。

...

美国其他地方也一样:同样规模的小镇,同样悲伤的人群,同样的露天火化台。但这显然不是事实。克雷斯通其实是美国,也是西方世界中唯一以社区为基础来实施露天火化的地方。

这种令人心潮澎湃的火化方式并非一直存在于克雷斯通。...“我们执着于用火化台进行露天火葬。”史蒂芬妮解释道,...

和史蒂芬妮一起工作的还有保罗·克鲁本伯格。他也很讨人喜欢,有着一口厚重的荷兰口音。他们二人带着火化台四处奔走,直接在别人家里进行火化。为了不让镇政府发现,两个人练就了一身速战速决的本领。就这样,他们用这套可移动设备完成了七场火葬。

“我们只要在你家院子的角落里把设备组装好就能开工。”保罗说道。

他们这套可移动火化台设备技术含量并不高,主要用煤渣砖制成,外加一个炉箅。每次火化时产生的高温都会让炉箅弯曲变形。“我们不得不开车碾过去,直接把它轧平。”史蒂芬妮说,“现在看来,我们当时确实很疯狂。”她边说边笑,并不觉得以前的做法有何不妥。

2006年,二人决定寻找一处固定场所来执行露天火葬。克雷斯通堪称完美之选:地理位置足够偏远,距离丹佛市中心约四个小时车程,全镇人口只有137人(周边地区人口1400人)。这种边缘性让克雷斯通具备了一种“老子的事儿政府最好别插手”的自由派气质,大麻和妓院在这里都是合法的(并不是说已经开了好几家妓院,只是说可以有)。

克雷斯通吸引了形形色色的“朝圣者”,人们从世界各地来到这里寻找精神家园。当地的天然食品店里贴着各种各样的广告宣传单:气功班、影子智慧班、面向儿童的“潜能开发”灵修班、北非民族舞蹈班,还有什么“魔法森林圣集会”。克雷斯通的本地居民不乏嬉皮士和信托自由儿\footnote{指来自富裕家庭的年轻人。},但大部分都是严肃的终身信徒。这里有佛教徒、苏菲主义者、卡迈尔修女等。刚刚过世的劳拉就是印度哲学家室利·阿罗频多的忠实追随者。

史蒂芬妮和保罗首次申请火葬固定用地时,就遭到了多个业主的反对。“他们是一帮烟民,就这德行。”保罗告诉我。这些业主义正词严地警告他们“别打我地盘的主意”。在史蒂芬妮看来,他们就是一群“守财奴”,根本不在乎二人提供的无林火风险、无异味、无水银或其他有毒物的证据。这些烟民集体给镇政府和环保局写信抗议。

为了抗争,史蒂芬妮和保罗把自己的业务进行了合法化。他们成立了一个名为“克雷斯通临终计划”的非营利机构,一刻不停地收集了400多个签名(相当于周边人口的1/3)。他们把法务文件、科学检测报告等材料通通收集起来,装订成一本厚厚的资料册。他们甚至走访了当地每一户居民,认真倾听大家的担心和顾虑。

一开始,他们遭到了人们的严重抵触。在一次反对露天火葬的集会上,有人叫他们“邻里互助火化二人组”。当史蒂芬妮和保罗提议(其实是个玩笑)在当地游行活动中使用卡通气模进行宣传时,有一家人走过来抗议说这种行为“简直是大不敬”,因为气模上带有火焰形状的纸质装饰品。

“镇上的居民甚至担心露天火葬会导致交通堵塞。”史蒂芬妮说道,“要知道在克雷斯通,一条街上同时出现6辆车都能被看作交通堵塞。”

保罗解释道:“这是因为人们充满了恐惧。‘这会不会造成空气污染?’‘你不觉得这很变态吗?’‘一想到与死有关的东西,我就害怕。’你必须保持耐心,仔细倾听他们的想法和需求。”

哪怕面临诸多法律困境,史蒂芬妮和保罗也没有退缩,因为他们坚信露天火葬能够启发整个社区(一些居民因为自己有机会被露天火葬而激动不已,要求史蒂芬妮和保罗把设备组装在自家的停车道上)。“如今,有多少人能提供真正让别人产生共鸣的服务?”史蒂芬妮说道,“我只做能唤起心灵共鸣的事,不然就不做。”

二人终于给自己的火化台找到了一个稳定的家。那是小镇外的一处空地,距离主路几百码远。这块土地是禅宗团体“龙山庙”捐赠给他们的。史蒂芬妮和保罗毫不遮掩地把火化台放在外面。当你沿着这条主路驶向克雷斯通时,你就会看到一个印有火焰图案的金属标志,上面写着“火化台”三个字。这个标志出自当地一位种土豆的农民之手(这个人同时兼任验尸官),可以说是一个明显的地标了。

火化台搭建在一片沙地之上,四周环绕着一片竹林。竹子的枝条形态各异,好似书法的笔画。目前,这里已经火化了50多人,包括把他们称为“邻里互助火化二人组”的那个人,他在临终前改变了自己的看法(反转无误)。

在为劳拉举行火葬的3天前,克雷斯通临终计划的志愿者来到她家帮忙。他们和劳拉的友人一起打理遗体,帮他们用低温毯裹好遗体以防腐化。他们还给劳拉穿上由天然面料制成的寿衣——化纤面料不适合用在火化台火葬中,比如涤纶。

克雷斯通临终计划会协助死者家属进行所有的葬礼后勤工作,但不为其提供资金支持。死者家属也可以选择露天火葬之外的火化方式:传统土葬(遗体经过防腐),自然土葬(没有墓穴或遗体不经过防腐),去几个镇子之外的火葬场进行火化。不管选择哪一种,志愿者都会协助家属进行相关的后勤工作。保罗把最后一种称为“商业火化”。

史蒂芬妮抗议道:“保罗,你应该叫它‘传统火化’。”

“不不,”我争辩道,“‘商业火化’这个称呼很贴切。”

克雷斯通让身为殡葬业专业人员的我备受鼓舞——这也是我多次来到这里的原因,但也让我感到些许(近乎嫉妒的)忧伤。这里的人们拥有一个庄严的火化台,就在浩瀚苍穹之下,我却只能把家人送到郊区厂房里那个又脏又吵的火葬场。如果我自己的殡仪馆能有一处如此壮观的火化场所,我发誓会请一个迪吉里杜管乐手在现场演奏。

以火化炉作为主要设备的工业火化首先出现在19世纪末期的欧洲。1869年,一群医学专家聚集在意大利佛罗伦萨,宣称土葬存在卫生问题,应该用火化取而代之。几乎是与此同时,一股提倡火化的风潮也席卷了美国,领头人是奥克塔维厄斯·B.弗洛汀汉姆牧师,他认为化作一堆“白色灰烬”的遗体比“高度腐烂”的遗体要好得多。

约瑟夫·亨利·路易斯·查尔斯·德·帕尔姆男爵是第一个在美国经历“科学的现代化”火葬的人。这名男爵是一位身无分文的奥地利落魄贵族,死于1876年5月。用《纽约论坛报》的话来说,“他主要是因为变成了一具尸体而出名”。

火化安排在当年12月,也就是他死去的六个月之后。在这半年里,他的遗体被注入砒霜以防腐烂。当砒霜逐渐失效,不足以抵抗大自然力量的时候,当地的一个殡葬人直接把器官从他体内拽出来,再用黏土和石炭酸的混合物涂满他的全身。在从纽约开往宾夕法尼亚州(火葬场所在地)的火车上,这具已然木乃伊化的尸体还在行李车厢中消失了一段时间。历史学家斯蒂芬·普罗特劳称这段插曲为“令人毛骨悚然的捉迷藏”。

这场具有划时代意义的火化在宾州的一处火葬场进行。这个火葬场由私人诊所改造而成,主要设备是一个火炉,炉箅下方装满了用作燃料的煤块。这样一来,火焰不用接触尸体,光凭高温就能让尸体“分崩离析”。尽管医学专家们一再强调这场火化是“一次严格遵循科学和卫生原理的实验”,德·帕尔姆男爵的遗体还是被撒上香料,安放在铺满了玫瑰、棕榈枝、报春花和松树枝的灵床上。当遗体刚被放入火炉时,观察人员称自己闻到了一股独特的烤人肉味,但很快这个味道就被花朵和香料的芬芳掩盖住了。一个小时之后,德·帕尔姆男爵的遗体上笼罩了一层闪闪发光的玫瑰色薄雾。接着,这层光芒逐渐转变为金色,最终变成了透明的亮红色。遗体就这样烧了两个半小时,直到从白骨化作灰烬。现场的记者和评论员后来在报道中写道,这次实验性火化“第一次实现了把人放在烤箱中进行安全、无味的烘烤”。

从那之后,火化设备变得越发庞大,运转速度越来越快,效率越来越高。大约150年之后,火葬的受欢迎程度达到了历史新高(美国2017年的火葬人数可能超过土葬人数,这在历史上还是头一遭)。但是,火葬的仪式和美学还是老样子。我们的火化设备还在使用19世纪70年代发明出来的款式——用钢铁、水泥和砖块建成的,重达24000磅的庞然大物。这些“怪物”每个月都要消耗上千美元的天然气,向大气中排放大量的二氧化碳、烟尘、二氧化硫,以及毒素含量极高的水银(主要来自遗体内的填充物)。

大城市里的火葬场大多存在于郊外工业区中不起眼的厂房里。...有的火葬场就位于墓地,...

有些火葬场把自己打造成“为生命喝彩”型或“火葬纪念堂”型机构。在那里,死者的亲朋好友聚集在空调屋里,透过玻璃窗目送遗体消失在火化炉的金属门后。火化机则躲藏在墙体后方,这样就不会被来宾看到,虽然它们和你在其他火葬场里见到的工业用火化炉没什么区别。这种伪装让还活着的人越发远离死亡这一事实,也越发远离火化方式笨重陈旧、污染环境这一真相。如果想让死去的母亲享受到“火葬纪念堂”型机构提供的优等待遇,你至少要花上5000美元。

我并没有说选择露天火葬就能解决这些问题。在印度、尼泊尔等将露天火葬作为主要殡葬方式的国家,每年的火化量高达百万场,耗费近5000万棵树,并释放出大量的黑碳气溶胶。黑碳气溶胶是继二氧化碳之后,造成气候变化的第二大人为因素。

但是,克雷斯通模式可以。克雷斯通临终计划接到了多个来自印度的殡葬业改革者的电话,想把他们的设备设计和操作方法引入印度——让火化台的位置远高于地面,这样就能减少木材用量和污染物排放量。既然我们能够改变古老的、与宗教和国家紧密相连的殡葬方式,那我们也有可能推动对现代工业化火葬的改革。

劳拉在克雷斯通生活了很多年,火化当天,全镇的人几乎都赶来参加她的葬礼。她的儿子杰森注视着火焰,首先打破了沉静。“妈妈,谢谢你爱我,”他用优美的嗓音说道,“不要再担心我们了,朝着自由飞翔吧!”

火焰继续燃烧。一位女士开始讲述她11年前一个人来到克雷斯通的情景,那时她已经被慢性疾病折磨了很多年。“我搬来克雷斯通是为了寻找快乐。起先,我认为是这里的云朵和天空治愈了我,但现在看来,应该是劳拉。”

...就在来宾一一发言时,火焰蹿上了她裸露的肌肤和软组织,将包含于其中的水分一层层抽干,留下越发干瘪、枯萎的肉体。她的内脏因此暴露无遗,眼看着就要被火焰吞噬。

对毫无经验的旁观者而言,这绝对是令人惊悚的一幕。好在机构的志愿者警惕性很高,把炉箅上的遗体挡得严严实实。他们的动作专业、优雅,确保来宾既闻不到异味,也看不到半熟的脑袋和焦黑的手臂。“我们并不是把遗体掩藏起来,”史蒂芬妮解释道,“但这里大多数的火葬都向全镇人开放。你不知道谁会来,也不知道他们会如何对待因目睹火葬而产生的心理动荡。有时,人们甚至把火化台上的遗体想象成自己。”

随着仪式的继续,志愿者悄然来到火化台旁边,往里面添加木材。整场火化使用了42.6立方英尺的木头,也就是1考得的1/3。

火焰终于碰到了劳拉的骨头。首先燃烧的是膝盖、脚踝和面部骨骼,烧了好一会儿才轮到骨盆和四肢。骨骼中的水分被蒸发掉,有机物也逐渐荡然无存。在这个过程中,她的遗骨先是经历了从白色变成灰色和黑色的过程,然后又从黑色变回白色。木材的重量让她破碎的遗体从炉箅穿过,径直落在下方的地面上。

...

离劳拉葬礼结束还有一个小时,此时的氛围已经不像之前那么哀伤了。最后一个人的发言要是放在一个半小时之前,肯定属于不合时宜。“你们刚才一直在说劳拉是个多么好的人,这我同意,她确实是个好人。但在我看来,劳拉永远都是一个狂野的疯婆子、一个派对女郎,我要为她喝彩。”

“嗷嗷嗷嗷嗷嗷!”她大声尖叫着。周围的人也随她一起号叫起来,我自己也小心翼翼地喊了一嗓子。要知道,我刚才连杜松枝都不敢往劳拉身上放。

到了早上9点30分,只有史蒂芬妮和我(以及劳拉的部分遗骨)还留在现场。我们坐在木头长凳上说话,此时火化台上只剩下三根木头,它们在灰烬中闪烁着轻柔的余光。消防队提供的红外线温度计显示,这些未尽的余火达到1250华氏度。

史蒂芬妮通常是第一个到达,也是最后一个离开火化现场的人。“我喜欢这份宁静。”她解释道。

史蒂芬妮沉默了一会儿,突然起身走到火化台旁,拾起一片金属炉箅仔细看了看。“这是保罗新设计的火花防护装置,能够把灰烬牢牢锁住,这样就不会让风吹走。烧剩下的木块也会被固定住。但上面要是有没熄灭的火星呢?”

不出几分钟,史蒂芬妮就给消防队打了电话,商量对火花防护装置进行检查和测试的事宜。精力充沛的史蒂芬妮不允许自己有片刻清闲。我非常好奇,究竟是什么样的恒心,能够让她十几年如一日地坚持工作,最终把露天火葬变为现实。“这是一个让人心力交瘁的过程,我们不得不耐心等待,直到人们接受我们。这太难了,我真想强行拉他们入伙。”

...


露天火葬受欢迎到什么程度呢?为了获得火葬资格,有人甚至在克雷斯通购买了地产。一名罹患宫颈癌的女士在42岁临终之前买下了一小块地皮。在她去世后,她12岁的女儿亲手给她的遗体做火化前的准备。
放眼全世界,人类对火焰的渴望不足为奇。印度人会把逝去的家人带到恒河边进行露天火葬,父亲的遗体由长子点燃。随着火焰的温度不断升高,死者的肉身也逐渐消失殆尽。当进行到一定时刻,负责火化的人会过来把头骨敲裂。印度人相信,头骨开裂后,人的灵魂才能从躯壳中释放出来。

放眼全世界,人类对火焰的渴望不足为奇。印度人会把逝去的家人带到恒河边进行露天火葬,父亲的遗体由长子点燃。随着火焰的温度不断升高,死者的肉身也逐渐消失殆尽。当进行到一定时刻,负责火化的人会过来把头骨敲裂。印度人相信,头骨开裂后,人的灵魂才能从躯壳中释放出来。

有人这样描述过自己双亲的火葬:“在此(敲碎头骨)之前,你惊慌得浑身发抖,眼前的这个人几个小时前明明还活着。但当骨头碎裂的那一刻,所有的痛苦都不见了,因为你意识到正在燃烧的不过是一具空壳。”灵魂获得了自由,就像现场播放的宗教歌曲中唱的那样:“死神,你以为征服了我们,但我们正在引吭高歌,赞美那熊熊燃烧的柴火。”

...

劳拉葬礼后的第二天清晨,我再次来到仪式场地。拴在火化台边上的两只大狗热情地迎接了我。麦格雷戈比我到得还早,此时正在用筛子仔细地清理劳拉的骨灰。他是劳拉的弟弟,自愿承担起清理骨灰的任务。骨灰和木头渣混在一起,大概有4.4加仑。他从灰烬中拣出最大的几块遗骸,分别是股骨、肋骨和头骨。有些家庭会把遗骨拿回家当作遗物保管。

与商业火化相比,露天火化产生的灰烬要多得多。前者留下的骨灰也就是一罐福尔杰牌咖啡那么多。加利福尼亚州要求我们用“骨灰研磨机”把残留的遗骨磨碎,达到“无法用肉眼识别”的状态,因为该州的法律不允许我们把形状明显的大块遗骨交给死者家属。

劳拉的几个朋友分走了一些骨灰,剩下的将撒在火化台附近的山丘或远处的山林里。“她肯定喜欢我们这样做,”杰森说道,“这样她就无处不在了。”

我问杰森,昨天的葬礼是否让他变得和以前不一样了。“我上次来克雷斯通时,妈妈带我参观了一下这个火化台。我当时就傻了,以为我得自己一个人坐在这里把她火化掉,这太吓人了。葬礼前三天,我已经害怕得不知该怎么办才好了。但是,她说过:‘这是我的选择,你来不来都可以。’”

杰森说,当他来到母亲家中参加守灵仪式的那一刻,他改变了之前的想法。特别是在火化仪式上,他意识到整个镇子的人都陪在他身边。在谈话和歌声中,他安心地接受了母亲的朋友们给予他的情感支持。“我非常感动,觉得一切都不一样了。”

...

\hr

\section{印度尼西亚的南苏拉威西岛}
\hr 

在印度尼西亚的一处偏远地区,当地人会和死者的遗体共同生活一段时间,时间长到你我都不敢想象——这里简直就是研究人与尸体关系的天堂嘛。但我一直认为自己不太可能到那么远的地方去。这个想法伴随了我很多年,直到我发现自己忘记了一个关键人物——保罗·库多那里斯博士。

春日里的一天,我坐在保罗博士的家中。保罗是一名研究死亡的学者,也是洛杉矶邪典圈子里的瑰宝,可谓久负盛名。说是“坐”在他家,实际上我是直接坐在他家的硬木地板上。保罗把自己在洛杉矶的家称为“摩洛哥海盗城堡”,里面没有家具,只有大量的动物标本和文艺复兴风格的画作,以及几个吊在天花板上的中东灯笼。

“8月,我要去塔纳托拉雅参加马聂聂节。”保罗满不在乎地说。这是他独有的腔调。在过去的12年中,他走访世界各地进行拍摄,不管是卢旺达的墓穴、捷克的人骨教堂,还是全身贴满黄金树叶的泰国僧侣木乃伊,全收录在他的镜头里。他是一个什么样的人呢?为了前往玻利维亚的蛮荒地带,他竟坐上一架在第二次世界大战时期用来运输冷冻肉的伞兵机。跟他一起的还有一个农民、一头猪、一只羊和一只狗。途中,飞机遇到气流颠簸,吓得几只动物在机舱里乱跑。正当他们两个人左扑右跳地要把它们抓回来时,副机长扭过头冲他们大叫:“别闹了,再动我们就要坠机了!”对保罗这种人来说,搞定塔纳托拉雅的旅程应该很容易。

保罗还邀请我和他一起去:“丑话说在前头,这趟旅行会让你受不少罪。”

几个月之后,我们首先来到印度尼西亚最大的城市雅加达。印度尼西亚由17000多个岛屿组成,是全球第四大人口国(排在中国、印度和美国之后)。

转机之前,我们到海关进行入境检查。

“你们要去印度尼西亚哪里?”一名年轻的女性工作人员笑着问我们。

“塔纳托拉雅。”

她的脸上掠过一丝调皮的神情。

“是去看死尸吗?”

“是的。”

“你们竟然是认真的?”她看起来吓了一跳,好像之前那个问题只是客套话,“你们不知道吗,那些尸体会自己走路。”

“其实,是家里人把他们立起来的。他们又不是僵尸,不会自己走路。”保罗说道。

“那我也害怕!”她看了一眼旁边的同事,紧张地笑了几声,然后在我们的护照上盖好章。

到达南苏拉威西岛首府望加锡时,我已经有39个小时没合眼了。当我们走出机场,踏入潮湿的空气中时,保罗像个明星似的被一大群人瞬间围住。我之前忘了说,保罗本人和他的家一样,充满了异域风情——我可是带着最崇高的美学敬意说出这番话的。他留着厚厚的一头脏辫儿,巫师风格的山羊胡上串着串珠,浑身上下都是文身。此时,他身穿一件华丽的紫色天鹅绒长袍,头戴一顶高礼帽,帽檐上还别着一个雪貂头骨。没人知道他的年龄。...

正在马路边上等候拉客的出租车司机停下生意,纷纷凑过来打量保罗的文身和帽子上的骨头。保罗的奇装异服成了他的通行证。一般人无法进入的秘密修道院和墓穴,他都能畅通无阻,因为人们被他的打扮惊呆了,实在不知道该如何拒绝他。

我们连回酒店小睡一会儿的时间都没有,直接包车前往北部,车程长达八个小时。绿色的稻田在公路两侧延展开,一群水牛在泥浆里笨重地嬉戏。

在塔纳托拉雅的偏远山区,人们曾经信奉名为“Aluk to Dolo”(祖先之道)的泛灵信仰,直到20世纪初才在荷兰传教士的影响下改信基督教。

我们的多用途车很快就进入了山区。弯弯曲曲的盘山路只有两条车道,司机师傅一路都在疯狂地躲避无穷无尽的卡车和摩托车。虽然我不会讲印尼语,但我成功地用世界通用的肢体语言向司机师傅表达了“说真的,哥们儿,我要吐了”这个意思。当我们到达塔纳托拉雅时,我已经因睡眠不足产生了幻觉。但是保罗,这个已经在飞机上睡了好几觉的家伙,坚持要赶在天黑前去附近的墓穴拍照。

于是,我们驾车来到龙达墓穴。这里没什么人,只见旁边的峭壁上有一个摇摇晃晃的脚手架,上面摆放了几具形状各异的乌鲁木棺材,有船形、水牛形和猪形。放射性碳定年法结果显示,塔纳托拉雅地区早在公元前800年就开始使用这种棺材。白花花的头骨从破裂的棺木中露出来,像好管闲事的邻居似的观察我们。一等到棺木彻底朽化,遗体就会顺着峭壁跌落下去。

更超现实的是,棺材旁边放了一排排“托托”。托托是按照真人容貌和比例打造的木质雕像。它们“端坐”在岩壁上,像一群正在开会的村委会成员。这些雕像象征着散落在墓穴里的那些无名遗骨的灵魂。老一点儿的托托做工比较粗糙,有着不成比例的大眼球和凌乱的假发。新的托托更加写实,脸颊的线条柔美,皮肤上画着以假乱真的斑点和血管,看起来着实让人动容。它们穿戴齐全,眼镜、衣服、首饰,样样不缺,仿佛为迎接我们的到来特意打扮了一番。

黑暗的墓穴中,一颗颗骷髅头嵌在石壁的夹缝和裂纹里。它们有的被码成金字塔形,有的天灵盖朝下放着,有的一看就是漂白过,有的长了一层苔藓而变得绿油油的,还有一些正扬扬得意地叼着烟。最后映入我眼帘的是一块同时抽着两根烟的下颌骨(头骨的其他部分已经不见了)。

保罗示意我跟他进入一个小一点儿的洞穴,我猜应该是另一处墓室。我在黑暗中眯着眼睛弯着腰,摸索着走了一段时间,随即发现前面的通道需要我整个人趴在地上爬过去。

“我就不往前走了,在这儿等你就行。”

保罗继续向前爬行。不久,紫色长袍的后摆就消失在前方的洞穴中。他经常潜入洛杉矶郊区的废矿,是这方面的老手了。

此时,我的手机只剩下2\%的电量,我只好关掉这唯一的光源,和骷髅头一起呆坐在黑暗中。不知道过了5分钟还是20分钟,一道烛光划破了黑暗。来者是几名印度尼西亚本土游客,一个母亲带着几个十几岁的孩子,一家人从雅加达来这里旅游。估计在他们眼中,此时的我就像一只被堵在车库门前无路可逃的负鼠,狼狈的样子在车灯下一览无余。

其中一个男孩子走过来,用英语优雅地对我说:“女士你好,很抱歉打扰你。烦请你看向镜头,我们想拍张照发到照片墙。”

接下来就是一阵闪光灯。我的照片应该不久就会出现在“龙达”这个标签里。尽管这是一个古怪的时刻,但我理解他们:在满是骷髅头的墓穴中发现一名6英尺高、身穿波点连衣裙的白人女郎,确实值得在照片墙上记录一下。这家人用不同的姿势又跟我拍了几张合影,然后就离开了。

在兰特包的酒店里昏睡了14个小时之后,我终于满血复活。我和保罗在酒店大堂见到了导游阿古斯和司机。阿古斯是一个身材结实、容貌俊朗的男人,在过去的25年里一直给德国和荷兰的游客担任导游,主要负责以丛林深度徒步和漂流为主的行程。但在最近几年,他和保罗建立了以死亡为主题的特殊业务关系。阿古斯告诉我们,马聂聂节明天才开始,今天的行程——观摩塔纳托拉雅的传统葬礼——就算是节日前的开胃菜了。

我们坐在阿古斯的多用途车里,穿梭于青翠的群山之间,驶过了无数条泥泞的道路。有那么几公里,我们一直跟在一辆摩托车后面,摩托车的后座上用绿色的尼龙绳绑着一头黑毛猪。我往前探了探身子:它还活着吗?正当我琢磨的时候,4只猪蹄像游泳似的在空中乱蹬。

阿古斯发现我看得起劲,于是说道:“猪在摩托车上可没有人老实,它们总是扭来扭去的。”

这头猪和我们的路线一致,也是往葬礼的方向走。这样看来,我们双方只有一方能活着回来。

震耳欲聋的鼓声和镲声让我们先听到了这场葬礼,之后我们才融入遗体后方的人群中一窥究竟。遗体盛放在一个缩小版的长屋里。长屋是当地的传统民居建筑,你在其他任何地方都不会看到这种房屋结构:屋脊两侧的尖顶一直向天空延伸,像两个高跷高高耸立在屋顶上。一共有35名年轻人扛着这间装有遗体的棺房。请参看下图\ref{fig:ta_na_tuo_la_ya_de_zang_li}:

\begin{figure}[H]
\centering
\includegraphics[width=\linewidth ,totalheight=0.95\textheight , keepaspectratio]{塔纳托拉雅的葬礼.jpg}
\caption{塔纳托拉雅的葬礼}
\label{fig:ta_na_tuo_la_ya_de_zang_li}
\end{figure}

人们一拥而入进了院子中央,而抬棺的队伍此时还在外面缓慢前进着。棺房比想象的要沉,小伙子们只能每走半分钟就把它放下来休息。

院子中央有一头强壮的水牛,它神情凝重,仿佛已经猜测到了自己的结局。它被一根短绳子拴在旁边的木桩上,可怜的模样...

游客(我猜的,他们有几个人皮肤白皙,带着西欧口音)统一集中在院子远处的角落里,所在之处和主场地隔着一排栅栏。这个措施反映出塔纳托拉雅死亡旅游业的一个棘手问题:如何才能既让游客近距离参观,又不让他们离得太近。我们也在这个“看台后排区”,但我不觉得有什么不妥。我找好地方坐下,保罗拿出相机准备拍照。为了让自己在闷热潮湿的天气里好过点儿,保罗今天换了一身行头:一件牛仔上衣、一条牛仔工装裤、一个警长徽章、一双波点短袜和一顶牛仔帽。

有些游客不太明白“后排区”的意义。一对夫妇直接坐进了留给死者家属的贵宾区。但当地人太客气了,没好意思把他们赶走。一个染着扎眼金发的德国老太太径直走到院子中央,用iPad对着小孩的脸就是一顿猛拍,还一支接一支地抽着红色万宝路烟。我真想找根拐杖钩着她的脖子把她弄走。

塔纳托拉雅的旅游业在最近几十年才兴起,20世纪70年代之前几乎没有人听说过这个地方。当时,印度尼西亚政府重点开发巴厘岛和爪哇岛的旅游业(事实证明他们做得很成功),忽视了塔纳托拉雅的独到之处——观赏性和仪式性兼具的传统葬礼。塔纳托拉雅不想再被其他印度尼西亚人看作“割取别人首级进行黑巫术”的地方,而是希望以保存良好的传统文化而闻名。

遗体终于被抬进了院子。抬棺人将棺房举起再放下,如此反复,口中哼着歌谣。他们一刻不停,直到全身的力气用尽之后才把棺房放到地上。他们喘了几口粗气,接着又把棺房扛在肩上,重复之前的动作。这种力量的交替让我看得入迷,西式葬礼的抬棺仪式一下子就显得有些古板。

这具遗体生前属于罗文纳斯·林汀。他是一位政府公职人员,也是一个农民,是村里的重要人物。我身后就放着一张5英尺高的照片,上面是林汀先生的脸部特写。照片里的人大概60岁,留着细长胡须,穿着时髦的蓝色西装。

身穿特色串珠服饰的小孩子在院子里追逐打闹,不时给搬运活猪的人让路。这些猪绑在竹板上,不断地发出尖锐的叫声。几个人把猪抬到屋后宾客看不见的地方。主屋的大门前挂着一个门帘,上面画满了迪士尼公主。在贝儿公主、爱丽儿公主和爱洛公主的注视下,这些猪穿过了通向屠宰场的大门。不知道我们早些时候碰见的黑毛猪是否也在其中。

塔纳托拉雅的葬礼可不是普通的自带酒水型(这里的“水”是水牛的“水”)活动。每一只献祭的动物都来自不同的人家,而且有记录可查。因为这套份子体系,人们通常不会错过别人的葬礼。就像阿古斯说的:“你在我妈妈的葬礼上送来一头猪,那我也要在你家举行葬礼时还一头。”看来,塔纳托拉雅式葬礼和美国葬礼一样,也存在过度花费的问题,因为不愿意被人看作不尊重死者。

到目前为止,这场葬礼上的仪式看上去都很繁复\footnote{繁多复杂。},但阿古斯说这已经比从前简单多了。阿古斯的父母从出生起就信奉阿鲁克教,但父亲在16岁时皈依了天主教。对此,阿古斯的解释是:“阿鲁克教里有7777种仪式,这太复杂了,人们只好改信别的。”我可不觉得天主教的仪式少,但就这样吧。

当牧师走到麦克风前开始布道时,人们安静下来。我听不懂他在讲什么,但能听出有几次他停下来,大声呼喊死者的名字向其致意:“罗——文纳斯,林——汀——!”他连续不停地讲了20多分钟,内容大多是重复的。很多人都准备离场,这时他贴近麦克风大吼了一声:“CO——E——!”那个动静活像是死亡金属乐队的主唱发出来的。告诉你,如果你恰好坐在音箱边上,却没在“CO——E——!”这个词出现时及时逃开,后果一定惨不忍睹。阿古斯告诉我,“COE”类似“好好听着”。据说,塔纳托拉雅的葬礼悼词在近几年深受电视节目的影响(舞蹈风格和服装设计也是)。

按照西方医学对死亡的定义,罗文纳斯于5月底去世,也就是葬礼前3个月。但按照塔纳托拉雅的习俗,他其实还活着。虽然已经没有了呼吸,但塔纳托拉雅人认为他只是处于类似发烧这样的生病状态。这个状态会一直持续到人们献上第一个活祭,通常是一头水牛或一头猪。献祭之后,罗文纳斯就可以ma’karu’dusan(咽下最后一口气),和作为祭品的动物一起死去。

人类学家迪米特里·辛吉罗尼斯在塔纳托拉雅进行了两年的田野研究,其间和一位名叫奈拉雅的当地老人建立了深厚的友谊,老人把迪米特里看作亲儿子一般。9年之后,迪米特里重返塔纳托拉雅,本以为能给奈拉雅一个惊喜,没想到她已经在两周前去世了。迪米特里来到奈拉雅家,她的一个亲戚把他领到后屋,然后告诉奈拉雅,迪米特里“回来了”。

\begin{quotation}
我端详着她的脸,然后盘腿坐下,在她耳边轻声问好。虽然一侧脸有腐烂、脱落的迹象,但她看上去十分宁静、安详……她只是“睡着”(mamma’)了,但她“知道”(natandai)我来看她了。不仅如此,她听得到也看得见我。她没有“死”(mate),只是“生病”(hot)了。因此,她还是“能够感受到一切”(nasa’dinganapa-apa)。
\end{quotation}

按照塔纳托拉雅的习俗,遗体在葬礼之前要放在家中。这听上去好像没那么震撼,但让我告诉你,在家中这段时间少则几个月,多则好几年。在此期间,死者的家人负责把遗体制成木乃伊,还要给他送饭、换洗衣物,时不时还要跟他说说话。

保罗第一次来塔纳托拉雅时,问阿古斯是否觉得家里待着一个已经死掉的亲戚很恐怖。阿古斯听了后放声大笑:“我小时候,我爷爷的遗体在我家待了7年。我跟我哥哥,我俩和他睡在同一张床上。每天早上,我们都给爷爷穿好衣服,然后把他靠墙立着,晚上睡觉时再把他放回床上。”

根据自己的所见所闻,保罗认为在塔纳托拉雅,死亡并不是一道强行把生者和死者隔开的“难以逾越的鸿沟”,而是一个模糊的界限。塔纳托拉雅人万物有灵的信仰也强调人类和非人类,不管是动物、高山还是尸体,两者之间都没有绝对的隔阂。

牧师总算讲完了,最后一嗓子“好好听着”也逐渐从音箱中消失。保罗轻轻走到我身边,低声说:“他们杀死那头牛后,应该顺手把某些游客也解决了。”

这时,两名男子好像收到信号一般,同时向院子里的水牛走去。一个人把一根蓝色的绳子从牛的鼻环中穿过。他的动作很轻柔,一边操作,一边用手挠着牛的下巴。水牛貌似没有意识到自己成了全场的焦点。另一个人蹲下来,把它的两只前蹄拴在地上的木桩上。

我不确定自己还在等待什么,也许是一段吟唱,也许是死者家属的相聚?这时,第一个人牵起牛头,从腰带里抽出一把砍刀瞬间割开了牛的喉咙,整个过程只有几秒钟。水牛向后仰起身躯,健壮的肌肉和牛角清晰可见。它想逃跑,却被绳子牢牢地固定在原地。一道鲜红的刀口出现在它的喉咙上,但没有血流出来,看来刀口不深。

好几个人冲上前拽住牛鼻环上的绳子,但没能制服它。水牛又踢又踹,猛烈地摆动自己的身躯,喉咙上的伤口撕裂开来,暴露在众人面前,这个场面让人难以直视。一个人拿出砍刀,朝牛脖子再次砍了下去。这一次,鲜血瞬间喷溅出来。

水牛发疯似的腾空跳起,力量大到挣脱了绳子和木桩的束缚,跌跌撞撞地冲向右侧的人群。现场一片混乱,尖叫声此起彼伏。我用小型摄像机拍下的视频记录下来的全是沉重的呼吸声和踩踏地面的闷响。我周围挤满了慌张的人,推搡之中,我不小心割破了手。

当时,我确信会有人(很可能是我自己)死于水牛的报复行动。最后,司仪成功把牛抓回了院子中央。不久之后,水牛倒地死去,喉咙上的伤口汩汩地往外冒着血。人群中有人哭有人笑,两种截然不同的声音混合成一种缤纷的复调。刚才的惊心动魄向葬礼注入了一丝活力。

\begin{center}
* * *
\end{center}

阿古斯正在打电话,语气很激动。

“出什么事儿了?”我问保罗。

“我们得带头猪过去。”

“我们去哪儿找猪啊?”

“阿古斯说他可以帮忙。不带头猪过去不礼貌。”

阿古斯的车已经坐满了,里面有我、保罗、阿古斯、司机和阿托——一个搭顺风车去隔壁村庄的15岁少年。车上已经没地方放猪了。

阿古斯挂断电话,向我们宣布道:“明天,我的一个朋友会骑摩托车把猪送来。”

阿托一路上都在疯狂地发短信。和一群成年人挤在一辆车里,换作哪个青少年都会这样吧。阿托家将在马聂聂节上给他叔叔和曾祖父开棺,这两个人在阿托还没出生的时候就去世了,因此阿托只能和他们的遗体相逢。

我们到达了目的地。这是一个由多个独立小村庄组成的村镇,没有所谓的中心。大部分村民都以种植水稻为生,包括我们此行的东道主。他们几户人家住在传统的长屋里,一共七栋,中间是公共生活用的庭院。院子里,一群红冠大公鸡正被几只瘦骨嶙峋的狗追着跑,边跑边“喔喔”地叫,狗后面还跟着几个笑哈哈的小孩。女人们正在用细长的竹竿敲打刚收获的稻谷,动作一刻不停。

村子里陆陆续续来了不少人,他们开始帮忙清理房子状的墓室。村子现在共有十几个墓室,都在自家屋子附近。每个墓室的门上都挂着一把大锁,这在以前是没有的。这么做不是为了提防邻居,而是因为几年前有人从村里偷走一具干尸,卖给了兰特包的一个收藏家。后来,村民打听到了是哪个收藏家,又去人家那里把干尸偷了回来。

几个人正聚在一起讨论如何给墓室通风。两年前,村里安葬了一个名叫约翰·汉斯·塔皮的村民。现在,他的深色木质棺材翘开了一个角,一打开墓室的大门就能看见。塔皮的儿子担心墓室里的空气过于潮湿:“希望父亲没什么事。但愿他还是干燥的,没有腐烂。”

今年的马聂聂节对约翰·汉斯·塔皮可谓意义重大。他的儿子认为,父亲在两年前去世时,家里在物质方面做得不够好——当时家里穷,没有杀牛献祭。他对此一直耿耿于怀,坚信没有活祭的话,“父亲就不能转世再生了”。不过,这一切将在这周改变:献祭的水牛已经选好了,正拴在附近的空地上呢。

两个墓室已经打扫完毕。一个女人敞开墓室的门,拿着一大罐柠檬味清新剂往里喷。

路的尽头有一户人家刚刚宰完猪,正等着新教牧师前来给他们新建的、能装下六个家庭成员的墓室赐福。他们邀请我们共进晚餐。

切成块的猪肉盛放在竹盘里,正架在火上烧烤。猪是在炉子旁边杀掉的,现在我们就坐在干掉的猪血旁吃晚饭,几只苍蝇绕着我们飞来飞去。不远处的竹架上挂着几只猪蹄。一只小狗冲进来,叼起一块滴着血的猪下水\footnote{猪下水,也叫猪下货、猪杂、猪杂碎,一般指猪内脏,或部分其他器官。}就往外跑。“哎哟!”厨子冲小狗大叫了一声,但没有阻止它带走战利品。

一个女人递给我一片竹叶,上面放了一个热乎的粉色饭团。这时,有人把竹盘上的肉从火上取下来,一大堆肥肉烧得滋滋作响。吃到一半时,我把这盘肉拿到跟前仔细瞧了瞧:焦脆的脂肪表皮上,一个个毛囊清晰可见。这可是动物尸体上的一块肉啊,我突然反应过来,随即感到一阵厌恶。

我虽然在人类遗体上花了很多时间,但不熟悉动物尸体,不是包装在保鲜膜和塑料泡沫盒里的动物我就不认识。法国人类学家努艾莉·维亚莱斯的一篇关于法国食物体系的文章中有一段话,我觉得也适用于西方其他国家:“屠宰被认为应该是工业化的,即大型化、统一化。不能存在暴力(最好是无痛的),也不能被人看见(最好根本不存在)。这几点即使做不到也要做到。”

\begin{quote}
即使做不到也要做到。
\end{quote}

饭桌上有一位老奶奶,年龄大到眼睛里蒙上了一层白膜。她拿起一小撮米饭,看向屋外山谷的方向。她不怎么与坐在旁边的人交流,也许她已经说不动话了。阿古斯用沾满猪肉残渍的手指捅了我一下,然后悄声说道:“那个新墓穴应该就是给她准备的。”阿古斯在调侃她,但说的也算是事实。这位老奶奶很快就会踏上祖先走过的路,搬到她的新房子里,那个“没有火也没有炊烟的房子”。

夜里,我们的猪乘着摩托车到了。它一着地就钻进其中一栋木屋,狼吞虎咽地吃着泔水。它完全没有意识到,因为我和保罗,这里将成为它的葬身之地。

这天晚上,我们都住在长屋里。长屋从外面看上去很宽敞,但我们进屋爬上梯子后却惊讶地发现,楼上只有一个没窗户的单人间。好在房间的地板上已经铺好了被褥,我们躺下后很快就睡着了。后半夜时,我们才意识到这里不是单人间,墙壁上的木闩拆下后,推开就可以看到另外三间屋子。一整晚都有人蹑手蹑脚地在我们房间周围进进出出。

\begin{center}
* * *
\end{center}

第二天一早,一阵阵凄凉的铜锣声从路边传来,正式宣告马聂聂节拉开帷幕。

我见到的第一具木乃伊戴了一副20世纪80年代的飞行员墨镜,镜框还镶着金边。

“哎哟,”我心想,“这哥们儿跟我中学的代数老师一个风格。”

一个年轻人把这具干尸立起来,另一个人拿起剪刀,把他身上的蓝色上衣从领口一路向下剪开,直到露出内裤边缘。木乃伊的上身和双臂就这样暴露在空气中。这个人虽然已经死去八年,但遗体保存得非常完好,表面上没有肉眼可见的伤口或裂痕。跟他隔着两具棺材的哥们儿就没那么走运了:全身上下都皱巴巴的,只有一层薄薄的干皮勉强搭在骨头上,要不是裹了一条镶有金边的毯子,估计早就散架了。

刚才那具木乃伊被放在地上,头下垫了个枕头,身上只剩下平角短裤和飞行员墨镜。一张8英寸×10英寸的遗像照片摆在它身旁。照片上的他看上去一点儿也不像我的代数老师,人在八年间的变化还真大。

几个女人在他旁边跪下,一边轻抚他的脸颊,一边大声哭着呼唤他的名字。当哭喊声逐渐停下时,他的儿子手捧一套工具刷(就是五金店里卖的那种)走了过来,开始清洁遗体。他用刷子轻轻扫过父亲粗糙的皮肤,每一下都很利落,并且充满爱意。一只蟑螂从父亲的平角短裤里蹿了出来,但他好像并不在意,继续手里的动作。这种打理遗体的方式我还是第一次见。

10分钟前,阿古斯接到一个电话,对方说有一家人正在河边一处难以到达的墓室中给死者开棺。我们迅速沿着水稻田旁的土路朝那个方向跑去。路的尽头是一条棕黄色的泥沟,附近没有垫脚的石头,也没有桥。我们只好一边抱怨,一边从泥里蹚过去,其间我还摔了个屁股蹲儿。

到达目的地后,我们看到至少有40具遗体被抬了出来,在地上一字排开。有些遗体裹着颜色鲜艳的布料,有的放在狭长的木质棺材里,还有的用印满卡通图案——凯蒂猫、海绵宝宝,以及一大堆迪士尼卡通人物——的毯子包着。这家人在遗体间走来走去,商议要先给谁开棺。有些死者他们已经不记得是谁了,有的则是重点关照的对象,比如某人亲爱的丈夫或某家的宝贝女儿,他们的家人都迫不及待地想再见到他们。

一个母亲正在打开16岁儿子的棺材。最先露出来的是一双弯曲变形的小脚,然后是手,看上去保存良好。站在棺材两侧的人试着抬起男孩,他们动作轻柔,确保不会碰坏遗体。他们成功地把男孩立了起来。男孩的躯干完好无损,但头部已经白骨化了:牙齿清晰可见,浓密的黑发还连在头皮上。但他的母亲毫不介意,此刻正欣喜若狂地看着他。这份喜悦也许是一时的,也许她从来都是这样,但不管如何,现在她正握住他的手,轻抚着他的面庞。

附近有一个人在用刷子清洁父亲的遗体。父亲的脸被蜡染布料制成的裹尸布染成了粉红色。“他是个好人,”他说道,“他有八个儿女,但他从没打过我们。我很难过,但又很开心,因为我现在能照顾他了,就像他以前照顾我那样。”

塔纳托拉雅人会把自己接下来的动作讲给遗体听:“现在,我要把你从坟墓里移出来。”“抱歉只给你买了烟,我最近手头有点儿紧。”“现在,我要脱掉你的外衣。”“你女儿和女婿从望加锡过来看你了。”
我们在这个河边墓穴碰到了这家的族长,他对我们的到来和送上的香烟表示感谢。他允许保罗拍照,也同意我问问题,但条件是答应他的一个请求:“如果你们在村子里碰见外来人,不要告诉他们这个地方,这里是我们的秘密。”

我想起前几天葬礼上那个粗鲁的德国女人,一边抽烟,一边把iPad贴在别人脸上乱拍。我担心自己也变得跟她一样——我们渴望把期待已久的事物看个遍,但不承想这让我们变成了不受欢迎的人。

\begin{center}
* * *
\end{center}

我们穿过稻田返回主路,正好赶上东道主开棺打理自己家的遗体。我认出一个在兰特包当平面设计师的同龄人。昨天深夜,他骑摩托车赶了回来,在我们睡觉的房间钻进钻出。他指着一具包在金色布料里的骷髅说道:“这是我兄弟,17岁时骑摩托车出车祸死了。”然后指了指旁边的遗体,“那是我爷爷。”

我们下方的山坡上有一家人在野餐。用餐完毕后,他们把死去七年的祖父平放在方格桌布上。这是他祖父第二次参加马聂聂节,遗体状态极佳,看不出有任何损坏。这家人先用软刷清洁他的面部,再把他翻过来清理脑后干裂开的皮肉。拍摄全家福时,祖父被立在中间,家庭成员纷纷围过来,有的表情严肃,有的面带笑容。我正津津有味地看着,一个女人招呼我过去一起合影。我朝她摆摆手,意思是“还是不用了吧”,但他们坚持要叫上我。于是,一张奇特的合影出现在印度尼西亚的深山里,上面是我、一个塔纳托拉雅家庭和一具焕然一新的干尸。

我之前听说把遗体制成木乃伊通常发生在气候寒冷干燥的地区,没想到在湿润多雨的印度尼西亚也能这么做。那么问题来了:这里的遗体究竟是怎么变成干尸的呢?不同的人会给你不同的答案。有人说他们用的是自古流传下来的方法——把油灌入死者的嘴巴和喉咙里,再用特制的茶叶和树皮覆盖全身。茶叶的茶多酚和树皮将皮肤中的蛋白质收紧,从而强化了皮肤的柔韧性和硬度,这样就能够更有效地抵挡细菌入侵。这个过程与标本师对兽皮进行处理使之长久保存的过程类似(所以有一种皮革叫“植鞣皮”)。

塔纳托拉雅现在还流行一种新潮的干尸制作方式:把防腐师的好朋友福尔马林注射到遗体中。有一个女人告诉我她无法接受这种方式,因为不愿意死去的家人被针扎。“但我知道有人这么做。”她压低声音说,好像发现了什么见不得人的事。

生活在塔纳托拉雅这部分的村民可以称得上业余人体标本师了。既然他们现在和北美人民都用同一种化学制剂给尸体防腐,我不明白为什么西方人依然会被他们的习俗吓个半死。原因应该和尸体干燥的程度无关,而在于塔纳托拉雅人不把遗体密封在棺材里,也不把棺材搁置在水泥做成的地下墓穴里,而是大大方方地让遗体和活人待在一起。

昨天,我碰到了约翰·汉斯·塔皮的儿子,今天我要去见约翰·汉斯·塔皮本人。他被平放在地上晒太阳,身上只穿了一条平角裤,手腕上戴了一块金表。他的胸腔和腹腔注满了福尔马林,因此这两处保存得异常完好。相比之下,他的面孔已经变成黑色,上面布满了斑点,白色的骨骼依稀可见。这时,他的家人把刷子伸到他的平角裤里,开始清洁他那干瘪的阴茎。不出意料,家人的脸上写满了尴尬。他们互相开了个玩笑,然后继续手里的活儿。

小孩子们在木乃伊之间跑来跑去,他们有时会在某具遗体面前突然停下,仔细瞧一瞧之后再伸手捅一下,然后大叫着跑开。一个5岁的女孩爬了上来,和我一起坐在一个墓室的屋顶上看着他们玩闹。我俩没有说话,心照不宣地享受着俯视他人的奇妙快感。

阿古斯发现我在屋顶上,向我喊道:“喂,我在想我之后是不是也会变成这样。肯定会的,是吧?”

我们回到留宿的那栋房屋后,一个4岁的男孩过来偷看我们吃饭。他从篱笆后面探出头,每次我回头冲他做鬼脸时,他都开心地大叫。这时,他妈妈过来告诉他不要打扰我们,于是他给自己找了个小刷子。他穿过院子走到一堆干竹叶旁坐下,用刷子全神贯注地在地上刷来刷去,保证每一条小裂缝都被刷到。如果马聂聂节一直存在,长大成人的他也一定会去清洁某具遗体,遗体的主人说不定就是我们这次碰见的某个村里人。

\begin{center}
* * *
\end{center}

第二天清晨,约翰·汉斯·塔皮已经换上了一套新衣服...。他今天要被搬去路尽头的一间新墓室里,墓室的墙壁是浅蓝色的,屋顶上立着一个白色十字架。里面的装饰走的是多元文化路线:除了传统的水牛符号之外,还有玛利亚之心、耶稣祈祷像以及《最后的晚餐》全景图。

家里人把约翰立起来,跟穿着新衣的他拍了最后一张合影,便把他放回棺材。他们把闪亮的黑皮鞋从他脚上脱下放在旁边,给他裹好毯子,随后合上了棺材盖。棺材两侧被擦拭干净之后,他们就把棺材扛在肩上出发了。鼓声和诵经的声音伴随了他们一路。

我往车上装东西时,阿古斯走了过来。“知道吗,现在就有具尸体在这栋房子里。”他指着跟我们木屋只隔了10英尺的一栋传统房屋说道。一位名叫山达的70岁老太太两周前死在这栋房子里。这家人一直瞒着我们,就是想看看我们知道后会有什么反应。

“你想见见她吗?”阿古斯问道。

我慢慢地点了点头。不知为什么,我觉得我们应该就睡在尸体隔壁。

“嘿,保罗,”我叫醒正在屋里补觉的保罗,“跟我下楼,给你看样东西。”

在阿古斯的指导下,我们给山达带来了一些吃剩的食物——她应该知道是我们送的。我们钻进里屋,看到山达躺在一张竹席上,身上盖着一条绿色的格纹毯子。她穿着一件橘色的衬衫,脖子上围了一条粉色的围巾,身边还放着钱包和食物。她的脸上围着一块布,面部皮肤有着橡胶般的质感,这在防腐后的遗体上很常见。

山达是由一个本地的专业人员用福尔马林防腐的。她的家人没有亲自操作,因为觉得福尔马林太“辣眼睛”。山达一家子都是成功的稻农,没有时间按照传统天天照料她的遗体。

她会在家里一直住到搬进墓室。这期间,家人给她端茶倒水,她在梦里与家人相会。距离她穿过生与死那条模糊的界限只有两周,因此家里人决定,等味道散去后就跟她一起睡。

阿古斯小时候跟他祖父的遗体睡了七年。他耸了耸肩,说道:“这种事情我们已经习惯了,这就是生与死。”

\begin{center}
* * *
\end{center}

在前往印度尼西亚之前,我试图在网上查询与塔纳托拉雅有关的信息,想先看看会有什么样的丧葬仪式在等着我。搜索出来的信息很少,起码没什么英语的。

照片也没有几张,其中最清晰的一张来自英国报纸《每日邮报》。我不知道他们是从哪儿得到的这张照片,因为他们显然没有派记者去。页面上的评论区很有意思,一条评论写道:“天哪,他们对这些逝者做了些什么呀?”另一条评论则写着:“讲真,这是对死者的大不敬。”

确实,如果说把莎莉姨妈从明尼苏达的坟里挖出来,把她放到高尔夫球车上在郊区民宅附近转一圈儿,那么是的,这的确是对死者的大不敬。但这些心智不健全的网友没有意识到,人死后,亲情还会继续存在。塔纳托拉雅人不但不觉得开棺掘尸是对死者的不尊重(事实上,这是对死者最大的尊重),而且认为这是与逝去的家人维持情感的一种方式。

作为一名殡仪业者,我总会碰到各种各样关于母亲遗体的问题。你不知道我听到了多少次:“我妈妈11年前在纽约州北部去世,防腐后葬在我们的家庭墓地里,你能告诉我她现在变成什么样了吗?”这个问题的答案取决于多个因素:天气、土壤、棺材、化学品。所以,我给不出一个完美答案。当我看到塔纳托拉雅人与母亲的遗体互动时,我意识到他们不需要通过殡仪人员了解遗体的状况。他们很清楚自己的母亲死后是什么样子,哪怕她已经死了11年。相较于与死去的母亲重逢,还是人类的胡思乱想更可怕一些。


\hr 

\section{墨西哥的米却肯州}
\hr
一具头戴黑色圆顶高帽、嘴里叼着雪茄的骷髅疯狂地挥舞着细长的双臂,朝华雷斯大街俯冲下来。这个庞然大物高达15英尺,在黑压压的人群中显得极为壮观。一群打扮成卡特里娜骷髅的男人和女人跟在它身后,一边欢呼雀跃,一边舞蹈。华丽优雅的卡特里娜骷髅是这里的标志性形象。当身着阿兹特克勇士戏服的旱冰方阵表演旋转时,一大片金色亮片从礼花炮中射出,街上的几万名观众顿时爆发出热烈的欢呼。

如果你看过2016年的詹姆斯·邦德电影《007:幽灵党》,就会认出这充满了鲜花、骷髅、魔鬼和大型卡通气球的狂欢,正是墨西哥城一年一度的亡灵节大游行。这部电影的片头,就是西装革履的邦德戴着骷髅面具穿过嘈杂的人群,与一名同样戴着面具的女子溜进了旅馆。

不过请注意,这部邦德电影并没有从亡灵节大游行中获取灵感,亡灵节大游行反倒是因为这部电影才存在的。墨西哥政府担心世界各地的人们观看完影片之后,会来墨西哥参加其实并不存在的游行,于是现招募了1200名志愿者,用一年的时间打造出这场四个小时的盛会。

亡灵节为每年11月的头两天。在这两天里,死去的人会重返人间,与活人一同享受人间的乐趣。在有些人看来,这个愚蠢的游行把本应以家庭为中心的、私人化的节日庆祝商业化了。另外一些人认为,这恰好代表了亡灵节正在向更加世俗的全国性节日演变,在全球的瞩目下毫不畏惧地展现墨西哥的历史文化。

游行结束后,我们艰难地走在装饰亮片成堆的街道上。跟我同行的是莎拉·查维斯,她是我建立的非营利组织“死亡新秩序”的总监。她发现这里到处都有亡灵节的装饰品,不管是在家里还是在商店,卡特里娜骷髅和鲜艳的骷髅剪纸随处可见。

“噢!”听起来她好像突然想起一件重要的事,“忘了告诉你,我们酒店旁边的星巴克在卖亡灵面包。”亡灵面包就是一种撒满糖霜的、带有立体骷髅造型元素的圆面包。

明天,我们就要动身前往西部的米却肯州。那是比较偏远的地区,当地人有着漫长的庆祝亡灵节的传统。但是在墨西哥,亡灵节在20世纪的一段时间里并没有受到普遍欢迎。到了20世纪50年代,墨西哥的城镇居民仍然把亡灵节视作老掉牙的民俗节日,认为只有文明社会里的边缘人才会庆祝这种节日。

这种情况不久之后就迎来了反转,主要原因之一就是万圣节从美国南部传入了墨西哥。在20世纪70年代早期,用记者玛利亚·路易莎·门多萨在她的文章中的话来说,万圣节习俗的核心就是“光彩夺目的聚会”。“黑猫、南瓜和戴着尖顶帽坐在扫帚上的女巫如果只是出现在侦探小说里,我倒觉得还好,因为这些东西跟我们的文化一点儿关系都没有。”门多萨继续写着,“普通的墨西哥人忽视了那些乞讨要饭、靠擦车窗讨生活的孩子。在富人区,我们的中产阶级模仿美国得州人,让孩子们穿上可笑的服装去别人家里要施舍。要知道,这些孩子从未空手而归。”

学者克劳迪奥·龙尼茨对这一时期是这样描述的:亡灵节“变成了国家形象的统一代表”,站在了“美式万圣节狂欢”的“对立面”。那些曾经抗拒亡灵节(或者生活在从不庆祝亡灵节的地区)的人逐渐把庆祝亡灵节视作墨西哥的传统。亡灵节不仅回归了主流城市(瞧瞧那个邦德引发的大型游行),还成为丧失权利的弱势群体的发声通道。他们在亡灵节上悼念那些远离公众视野的人,包括性工作者、本土性少数人权倡议团体和死在美墨边境线上的同胞。在过去的40年间,亡灵节成为整个墨西哥的流行文化、旅游文化和反抗文化的代名词。墨西哥本身也因成功地打造了全民参与的祭奠文化而走在世界前列。

“我小时候,跟我一起生活的长辈都是一些自我厌恶的墨西哥人。”莎拉说道,此时我们正坐在米乔坎的酒店房间里,“他们生活的环境告诉他们,他们没什么可自豪的,他们应该对自己的一切感到羞愧。他们应该融入。要想在美国获得快乐,就得变得和白人一样。”

20世纪初期,莎拉的祖父母从墨西哥蒙特雷搬到东洛杉矶的查维斯山谷区定居。20世纪50年代,政府通过公函告知该地区的近2000个家庭——大多都是低收入的墨西哥裔美国农民——他们必须卖掉自己的房产,给政府的公共住房项目腾地方。政府保证,在项目完成后,不仅会在原地新建学校和操场,还会优先给予搬迁居民回迁的机会。这些家庭只好搬走,原先的社区也被拆毁。然而,洛杉矶政府取消了公共住房项目,转而与纽约的商人共同兴建道奇体育场。包括罗纳德·威尔逊·里根在内的体育场项目支持者公然把反对者称为“痛恨棒球的家伙”。

由于一项歧视性的住房政策,从查维斯山谷区搬离的墨西哥裔美国人被赶到洛杉矶东部。在搬迁的过程中,莎拉的父母步入成年,二人在19岁的时候生下了她。

“直到今天,我的奶奶、姑姑和叔伯一谈到查维斯山谷区,还是会伤心得不得了。他们特别、特别想念那个地方。”

莎拉出生后,她的家人不允许她学西班牙语。她的浅色肌肤让她成为家里最受宠爱的孙辈。她墨西哥人的一面只能留在家里。在洛杉矶长大的莎拉,不停地游走在与她关系疏远的母亲、在好莱坞从事服装道具营生的父亲(时至今日,他都认为自己是“美国原住民”,而不是墨西哥人)和祖父母之间。莎拉只当自己是个碰巧带有墨西哥血统的美国人,家里的墨西哥文化氛围和自己的关系不大。

2013年,莎拉已经在学前班和幼儿园当了10年的老师。她与自己工作上的搭档鲁本坠入爱河,不久后,两个人决定是时候要个孩子了。莎拉很快就怀孕了。对她来说,有了这个孩子就意味着“一个真正的家庭,我的家庭,一个天选之家,没有谁能从我手里夺走”。

可惜事与愿违,莎拉的儿子在六个月大的时候胎死腹中。在接下来的日子里,莎拉心里“放不下任何人、任何事”。她与父母渐行渐远,觉得自己孤独极了。有那么几天,她甚至想消失在房屋后面的橙树林里。她不停地自责:是不是我搬东西的姿势不正确?我是不是吃了不该吃的东西?“女人的本质是带来生命之人,”莎拉说道,“我的身体却是一个坟墓。”

她觉得对朋友和同事来说,自己就是一个行走的辐射源。她知道,在人们理想中的世界里,孩子是可爱的,不会受到伤害。“这个社会要求我藏起悲伤,”她继续说道,“人们不想面对这种恐惧,而我恰好就成了这种恐惧的代言人。我是一个恶魔。”

莎拉在网上搜遍了那些遭受丧子之痛的母亲的故事。她找到了几个好心人建立的网站,但它们都带有一些基督教的意味(例如,“我的天使躺在上帝的臂弯中”),里面的故事也是老生常谈,一直在绕弯子。在她看来,这些看上去很美的话不过是一堆空洞的陈词滥调,无法捕捉到她的悲痛和渴望。

在寻找安慰的过程中,她将目光转向了自己的文化与传统。“莎拉,你是墨西哥人,你来自无疑是这个世界上与死亡关系最紧密的文化。”她想,“你的祖先会如何处理这种悲剧呢?”
\begin{center}
* * *
\end{center}
墨西哥诗人奥克塔维奥·帕斯有一段著名言论。他说,当纽约、巴黎、伦敦等西方城市的居民还在担心频繁提起“死亡”这个词会让自己“咬到嘴巴”时,“墨西哥人已经无时无刻不在谈论它、嘲讽它、爱怜它、取悦它,甚至与它同眠。死亡是墨西哥人最喜欢的玩乐之一,也是他们永恒的挚爱”。

不过,这并不意味着墨西哥人不惧怕死亡。他们与死亡的联结源自长达几个世纪的暴行,可谓来之不易。“墨西哥没有成为一个令人骄傲并强大的帝国,”克劳迪奥·龙尼茨解释道,“这个国家不断被列强和独裁者欺压、侵略、占领、分裂、敲诈。到了20世纪,当西方世界对墨西哥的镇压和对死亡的抗拒双双达到顶峰时,墨西哥与死亡之间愉快、亲密的联系逐渐成为国家形象的奠基石。”

对莎拉而言,接受儿子的死不等于抹杀对死亡的恐惧,她知道自己不可能不惧怕这必然的命运。她只是想适应死亡,只是想开诚布公地谈论死亡。就像帕斯说的,时刻“谈论它、嘲讽它、爱怜它”。

许多来自移民家庭的孩子发现自己越发远离家族所传承的文化传统,就像莎拉经历的那样。臭名昭著的美国殡葬体系通过立法和出台管理条例,大肆干预多元化的丧葬习俗并强制其与美式规范同化。

莎拉最初是通过画家弗里达·卡罗的作品与墨西哥文化建立起联系的。弗里达·卡罗是墨西哥的“痛苦的女主人”,对莎拉的影响最为深远。在1932年的画作《墨西哥和美国边境线上的自画像》中,卡罗站在假想出的墨西哥和底特律的边境上,脸上充满了蔑视。那时,她和作为壁画画家的丈夫迭戈·里维拉正在底特律生活。边境线上的墨西哥一侧画满了骨头、废墟、植物、花朵和深深植入土壤中的粗壮的根茎,底特律一侧则是工厂、摩天大楼和滚滚浓烟——一个掩盖了大自然生死循环的工业城市。

在底特律生活时,卡罗怀孕了。她把这个消息写信告诉给了自己的前任医师利奥·埃劳塞,他们二人在1932年到1951年一直密切通信。她担心这次怀孕有很大的风险,因为在那次电车事故中,她的部分骨盆粉碎性骨折,子宫也被刺穿。卡罗在信中写道,底特律的医生“给我开了打胎用的奎宁和高浓度蓖麻油”。但这些化学药品没能终止妊娠,医生也拒绝为她进行堕胎手术。卡罗很有可能要冒着生命危险分娩。她请求埃劳塞给那名底特律医生写信:“堕胎是违法的,他应该是害怕做违法的事。但再这样拖下去,恐怕会错过手术的最佳时机。”我们不知道埃劳塞有没有按照卡罗说的做,但两个月之后,她遭遇了严重的流产。

卡罗以这次经历为灵感创作了《底特律的流产》。画里的她赤裸着躺在医院的病床上,鲜血染红了床单。几个物体飘浮在她的周围,分别是男胎(她的儿子)、手术工具以及具有象征意义的蜗牛和兰花,均由红色丝带做成的脐带连在她的肚子上。底特律了无生气的工业建筑构成了天际线,突兀地矗立在画面后方。艺术史学家维克多·萨穆迪奥·泰勒称,除了表达对底特律发自肺腑的憎恶,哀诉发生在自己身上的可怕不幸,“这幅画作是卡罗第一次有意识地画出自己的故事,画出自己最隐秘、最痛苦的内心”。

莎拉一直淹没在“上帝对你早有计划”这种空洞的说辞中,卡罗的画作和书信里的直白表达对她来说无疑是一种安慰。她在卡罗的身上,看到了另一个因为孩子、身体而被迫与命运的无常进行抗争的墨西哥女人。卡罗用画笔展现出这种痛苦和困惑,毫不羞耻地描绘自己的身体和哀痛。

莎拉的儿子死于2013年7月。同年11月,她和自己的伴侣——同样是墨裔美国人的鲁本在亡灵节时来到墨西哥。“我们不是游客,不是来看‘死亡’的。”莎拉说道,“我们每天都与死亡在一起。”

穿行在精致的祭坛和骷髅装饰品之间,莎拉感受到了在加利福尼亚州从未体会过的冲击与平和。“来到墨西哥后,我发现这是一个能让我放下悲伤的地方。这里认可我的悲伤,我不再是那个让别人不舒服的人。我终于可以喘口气了。”

莎拉和鲁本来到了以收藏木乃伊而闻名的瓜纳华托。19世纪末,当地人要给葬在坟场的死者缴一笔“坟墓税”来获得“永久”埋葬的资格。如果死者的家人交不起这笔钱,那么死者的遗体就会被挖出来给刚死的人腾地方。在一次例行的挖掘作业中,工作人员惊讶地发现挖出来的不是白骨,而是一具具“表情狰狞、形状怪异的干尸”。原来,土壤中的化学物质在当地的气候条件下能够天然地将遗体木乃伊化。

瓜纳华托市政府在接下来的60年里,不断地挖出了木乃伊。轻度干尸化的遗体被直接火化,完全呈木乃伊形态的则作为展品收藏在木乃伊博物馆。

20世纪70年代,作家雷·布拉德伯里到访该博物馆,并以陈列在里面的木乃伊为灵感创作了一个故事。他之后写道:“这个博物馆让我的心灵饱受创伤。我害怕极了,只想赶紧逃离墨西哥。我接连做噩梦,梦见自己死后被当作展品,和一屋子被金属丝线固定的干尸关在一起。”

由于这些木乃伊是天然形成的,不是出于防腐的目的刻意而为之,因此他们都大张着嘴巴,手臂和脖子呈扭曲状。人死后,遗体会重新回到“肌肉松弛的状态”,放松的下颌导致嘴巴张开,眼睑失去张力,四肢的关节也变得极度松垮。总之一句话,遗体不会自己使劲,它们不再按照活人的规矩来。瓜纳华托木乃伊的吓人形态不是为了“吓唬”布拉德伯里先生而人为制作的,那只是人死后正常的生理变化现象而已。

这些木乃伊现在还在展出。莎拉一点儿也不觉得它们可怕,她走到一个黑暗的角落,在一个身穿白色连衣裙、躺在天鹅绒垫子上的木乃伊女童跟前停下。“她看上去像一个被光芒围绕的天使,那一刻,我觉得自己可以永远站在那里看着她。”

莎拉默默地流着眼泪,一个女人看到后递给她一张纸巾,然后轻轻地拥抱了她一下。

博物馆里其他的木乃伊儿童也有自己的专属道具,比如拿着权杖和戴着王冠的Angelitos(“小天使”)。在20世纪早期的墨西哥以及其他拉丁美洲国家,人们把死去的婴儿和小孩视作能与上帝沟通的神体,类似于圣徒。这些“小天使”已经不再背负原罪,能够给尚在人间的家人带来恩惠。孩子的教母负责打理遗体,她把遗体清洗干净,然后给遗体穿上小号的圣徒服装,再把蜡烛和鲜花摆在遗体周围。孩子的母亲在这个过程结束之后才能见到遗体。此时,呈现在她眼前的是一个不再带有世间忧伤的天人,已经准备好回到上帝身边了。

孩子的家人会邀请朋友和亲属参加葬礼聚会。这么做不仅是为了缅怀死去的孩子,也是为了给这个孩子留下好印象以获取恩惠——别忘了,这个孩子现在已经拥有了灵力。有时候,人们甚至会带着孩子的遗体参加一个又一个聚会,遗体由同龄的小朋友抬着,后面跟着父母和亲属。“小天使”通常会出现在拥有恢宏场景的照片或画作中。

虽然莎拉不相信圣徒和来世,但是这种文化承认孩童的死亡,这一点深深地打动了她。“他们对待这些孩子的方式太特别了,很多事就是专门为这些孩子做的。”她说道。不管是聚会、作画,还是游戏,这里的人们为死去的孩子做了很多,而不是沉溺在孤独和无尽的缄默中。
\begin{center}
* * *
\end{center}
每年11月1日的夜晚,生与死的界限变得薄弱而模糊,死者轻易就能跨越这道坎儿。在米却肯州的小城圣达菲拉古娜,一群上了年纪的妇女正端着亡灵面包和水果,拜访那些在今年失去亲人的邻居,铺满鹅卵石的小道上到处都是她们走家串户的身影。

我低头钻过画有金盏花的门帘进到玄关。大门正上方挂着一个相框,里面是赫尔海的照片。赫尔海死的时候才26岁。照片上的他反戴一顶棒球帽,身后贴着好几张乐队的海报。“活结乐队?这个我可不好评价,赫尔海。”我琢磨着,同时在思考对一个死人的音乐喜好评头论足是不是不太妥当。“噢噢,那是错配乐队!算你有眼光。”

穿过玄关就是赫尔海的祭坛,一共三层。里面每一个祭品都是他的家人和朋友送来做诱饵的——诱使他在这一晚回家。赫尔海在今年身亡,因此他的家人先在家中设立祭坛,之后再把祭品带去他的坟墓。只要他的家人经常去扫墓,邀请他与生者重逢,他就每年都能回家看看。

祭坛的最下方有一个黑色的高脚杯,里面放着柯巴脂香料,辛辣的气味在空气中飘荡着。水果和面包足足堆了3英尺高,旁边放满了装饰用的蜡烛和金盏花。这一晚,随着越来越多的街坊邻居过来送上自己的那一份祭品,这个祭品台只会变得越来越高。重返人间并不意味着赫尔海要变成一具活跳尸,他会以灵魂的形态回到家中,用灵体特有的方式吃掉那些水果和面包。

摆在祭坛正中间的是赫尔海最爱的白色T恤,上面印着一个表情悲伤的小丑和手写体的“小丑”字样。等待他回家的还有一罐可乐(可乐的魅力我太了解了——这听起来也许有些不着调,但如果家里有一瓶健怡可乐等着我,我也会起死回生的)。T恤的上方挂着一些传统的基督教饰品,我看到了几个圣母玛利亚和一个钉在十字架上的血淋淋的耶稣。天花板上悬挂着色彩缤纷的剪纸,形状是脚踩自行车的骷髅。
祭坛旁边围了十几个人,他们都是赫尔海的家人,正在为迎接客人做准备。他们今晚很有可能要忙活到深夜。几个身穿闪亮公主裙的小家伙跑来跑去,脸上画着卡特里娜骷髅花纹,手里还拿着挖空内瓤的小南瓜,用来装大人给的糖果。

莎拉早就买好了满满一袋糖果。消息不胫而走,一大帮脸上画有骷髅彩绘的小孩提着点了蜡烛的南瓜灯把她团团围住。“女士,女士,谢谢你!”莎拉顺势蹲下,带着幼儿园老师特有的沉着和爱意给他们发糖。“我以前也在亡灵节的时候给班里的小孩做南瓜灯,里面也有蜡烛,跟他们现在手上拿的一模一样。可惜一点儿火光就能让管理部门把活动叫停。”她苦笑了一下。

圣达菲拉古娜是布雷佩查人的故乡,这一原住民以建造造型独特的金字塔和制作用蜂鸟羽毛制成的装饰画而闻名。1525年,布雷佩查人正因天花肆虐而变得脆弱不堪。当听闻勇猛善战的阿兹特克人已成为西班牙人的手下败将时,布雷佩查的领袖只好宣誓效忠西班牙。时至今日,这个区域的学校依旧在实行布雷佩查语和西班牙语双语教学。

许多现在流行的葬礼仪式元素,例如音乐、熏香、花朵和食物,布雷佩查人早在16世纪西班牙入侵之前就使用了。到了西班牙殖民时代,根据一名多米尼加男修士的记载,布雷佩查人很乐意接纳天主教的万圣节和万灵节,因为这跟他们已有的纪念亡者的节日完美契合。

在接下来的几个世纪里,殖民者想方设法地根除原住民祭奠死者的习俗,因为这“吓坏了杰出的精英阶层,他们竭尽全力要把死亡从社会生活中赶走”。1766年,皇家罪案处禁止原住民进行扫墓活动,强行切断了他们与死者的联结。不过,原住民总能找到钻空子的方法,这个习俗也就保留下来了。

我们来到另一户圣达菲拉古娜人家,只见门口的牌子上用布雷佩查语写着“欢迎回家,柯奈李奥神父”。柯奈李奥神父的祭坛占满了一整间屋子。我把橘子和香蕉摞在水果堆的最上方。这时,这个家族里几个年长的妇女围过来,给我们递上一大碗还冒着蒸汽的猪肉炒番茄和几杯玉米粥——一种用玉米、肉桂和巧克力做成的热饮。对死者家属而言,这一晚不能单方面地接受别人送来的慰问品,邻里间的这种交流是相互的,他们也要有所给予。

坐在屋子一角观察我们的正是柯奈李奥神父本人——其实这是一具真人大小的塑像。柯奈李奥神父坐在折叠椅上,身披一件斗篷,脚蹬一双黑色的高帮靴子,头上的白色牛仔帽遮住了半张脸,一副正在睡午觉的样子。

祭坛的中间是一个相框,照片里的柯奈李奥神父也戴着一顶白色牛仔帽,和塑像头上的是同一顶。相框后方的墙上挂着一个木质十字架,再往上则挂着一串色彩鲜艳的骷髅头糖块和……面包圈。“莎拉,往祭坛上挂面包圈正常吗?”

“正常。”莎拉答道,“你之后会看见更多的面包圈。”

又拜访了几户人家之后,我问莎拉哪户人家的祭坛让她最受触动。“让我最开心的不是祭坛,而是和孩子们在一起的时候。”她边说边示意我看向旁边的一个小男孩,这个小孩三四岁的样子,身穿超人的服装,手里拿着用南瓜做成的小篮子,“那感觉又苦又甜。如果我的儿子还活着,他现在应该和那个孩子一样大了。”话音刚落,“小超人”就害羞地把篮子伸过来要糖。

\begin{center}
* * *
\end{center}

我们的墨西哥之旅还在继续。现在,我们一路向南前往辛祖坦小镇。亡灵节期间,这里会举办热闹非凡的街头文化节。小贩们用大大的金属锅煎着猪肉和牛肉,街边店铺门口的音响里传来震耳欲聋的音乐,孩子们在街上兴高采烈地放爆竹。镇子的边缘地带有一处小山丘,顺着斜坡走上去,就来到了镇里的公共墓地。

在11月1日这天夜游墓地,不愁没有启示性的收获。为了迎接死者回家,墓地里有成千上万支蜡烛在燃烧,都是人们用一年的时间积攒下来的。一个小男孩认真地把祖母坟墓上熄灭的蜡烛重新点燃或换上新的,小手在上百支蜡烛之间忙碌着。烛光混合着金盏花和熏香的香气,给一座座坟墓笼罩上一层金色的薄雾。

最近几年,许多美国城市也开始举办亡灵节庆祝活动,其中就包括在好莱坞永恒墓园举行的大型庆典。好莱坞永恒墓园距离我在洛杉矶的殡仪馆只有几分钟的车程,我已经参加过好几次了。那里举办的庆典规模宏大、执行到位,但情感上远不及辛祖坦。此时此刻,站在辛祖坦的墓地里,就像站在一颗光芒四射、怦怦直跳的心脏中间,充满了安全感。

...

莎拉站在1岁的马可·安东尼奥·巴里加的墓前。墓碑上的照片上,一只白鸽从马可的头顶飞过。马可的坟墓高达7英尺,简直像个堡垒。可以说,坟墓有多高,他的父母就有多悲伤。马可死于20年前,但蜡烛和鲜花依然堆满了他的坟墓,看来丧子之痛从未从他父母的心头散去。

我在来墨西哥之前就知道莎拉失去过一个孩子,但不清楚来龙去脉。我们俩在酒店房间说话时,莎拉向我倾诉了这场灾难背后的故事。

莎拉第一次做超声波检查时,一直和她拉家常的护士用探头在她肚子上扫了几下,随即陷入了沉默。“我去叫医生。”过了一会儿,护士说道。

第二次做超声波检查时,医生彻底惊呆了。“呃,这只脚是内翻足,”医生把看到的影像描述给莎拉,“这只手有三根手指,另一只手有四根。心脏发育不良。哦,再看看这个——他有两只眼睛!大多数这样的胎儿不会有两只眼睛。”最后,医生给了莎拉致命一击,“我认为这个孩子活不到你临盆的时候。”

莎拉的孩子患有13-三体综合征。这是一种罕见的染色体异常,会导致孩子智力发育不全和身体畸形。绝大部分患有此病的孩子熬不过出生后的头几天。

第二个医生直白地告诉莎拉:“如果你是我的妻子,我会建议你放弃这个孩子。”

第三个医生给了莎拉两个残酷的选择:第一个是在医院引产,引产之后,孩子只能在子宫外短暂地存活一段时间,然后死去;第二个是终止妊娠。“洛杉矶有一名医生可以帮你。”医生告诉莎拉,“她通常不会给处在怀孕晚期的人进行手术,但我可以帮你和她谈谈。”

发生这一切时,莎拉已经怀孕6个月了。她与那位洛杉矶的医生预约了手术时间。为了不让自己到时过于痛苦,她试图在情感上疏远自己的孩子,但小家伙一直在她的肚子里动来动去。她不想让自己的孩子被夺走:“这不是我身体里的一块异物,这是我的孩子。”

在怀孕6个月时终止妊娠,莎拉需要在3天内接受3次手术。当莎拉和鲁本走向诊所时,一群反对者挡住了他们的去路。“里面有一个女人非常恶毒,不断地尖叫着说我是谋杀犯。我实在受不了了,只好走到她面前吼道:‘我的孩子已经死了!你再这么说一次试试!’”

莎拉和鲁本在诊所里等候了一个小时。在这期间,屋外抗议的声音不断地传进他们的耳朵:“喂,那个肚子里有死孩子的女士!听着,我们还是能拯救你的!”

这3天是莎拉和鲁本人生中最糟糕的3天。最后一次进行超声波检查时,莎拉别过脑袋,努力不去看显示屏。鲁本没有,他看到孩子的手动了几下,好像在挥手和他们告别。

莎拉听到隔壁房间传来痛苦的抽泣声,一个女孩因为怀孕想结束自己的生命。“我不要这个东西!我不要!”她尖叫道。

“我很想过去安慰她,告诉她我可以领养她的孩子。”莎拉回忆道,“但那不是我真正想要的,我只想要这个孩子——我自己的孩子。”

最后一场手术时,所有的工作人员都聚集到莎拉的手术台周围,告诉她他们为她的遭遇感到非常遗憾,并承诺一定会照顾好她。“我在这个诊所收获了人们最美好的善意,”莎拉说道,“虽然对我来说,这里是我孩子的死亡之地。”

即便是在若干年后的今天,丧子之痛仍然像巨石一般压在她的心头。在辛祖坦公墓,正当莎拉盯着马可宝宝的照片看时,鲁本温柔地抚摩着她的后背。莎拉打破了沉默:“父母总想炫耀自己的孩子,因为他们为孩子感到骄傲。但如果孩子死了,他们也就没有炫耀的机会了。但是看看这里,他们依然有机会展现出自己对孩子的爱,展现出自己的骄傲之情。”

孩子死后,莎拉没有感受到任何与骄傲有关的情绪。相反地,她不得不保持“体面”,将悲伤吞进肚子,生怕自己内心的痛苦流露出来给别人添堵。

西方国家的殡仪馆超爱“体面”这个词,美国最大的殡葬公司甚至把这个词注册成了商标。在通常情况下,“体面”意味着保持沉默、故作镇静和流于形式——守灵时间严格规定为两个小时,然后迅速带死者家属去墓地,并赶在棺木入土前让家属离开。

...

克劳迪奥·龙尼茨曾写道,采纳或适应亡灵节的传统可以在情感上拯救墨西哥的北方邻居,墨西哥人“拥有治愈的力量,尤其能治愈美国最严重的慢性疾病——对死亡的抗拒……以及对痛失所爱之人的孤立”。

\begin{center}
* * *
\end{center}

在墨西哥的最后一天,我们回到墨西哥城,拜访了弗里达·卡罗的故居——著名的“蓝房子”。卡罗在这里出生,47岁时在这里逝世。“这听上去也许有些奇怪,但我来这里是为了感谢她。”莎拉解释道,“弗里达帮助了我,‘蓝房子’是我的圣地。”

“我觉得大多数母亲多少都曾害怕被孩子束缚。”莎拉说,“我非常在意自己能做哪些事,可以去哪些地方旅行,可以到哪些地方进行‘朝圣’,因为我没有孩子。我对我拥有的所有时间都很在意,因为我付出了可怕的代价,才获得这些属于自己的时间。”

“蓝房子”里展出了一幅名为《弗里达与剖腹产》的画作。在这幅未完成的作品中,肚子被剖开的卡罗躺在一个足月婴儿的旁边。莎拉看到这幅画时,惊讶地屏住了呼吸:“这是我见过的第一幅卡罗的真迹。这就像在网上交到了知心好友,然后在现实中与他们相见。这种感觉很令人动容。”

弗里达·卡罗对生育的真实态度其实并不完全为人所知。许多传记作家为了维护她的神圣形象,把她用药物进行流产一事包装成一个热切的母亲不幸遭遇“意外流产”。另外一些传记作家坚称卡罗对孩子没有兴趣,“身体状况欠佳”只不过是她的借口,用来躲避社会传统对女性生儿育女的期待。

在楼上卡罗那间不大的卧室里,摆放着一个前哥伦布时期的骨灰瓮,里面是卡罗的骨灰。她的死亡面具摆在单人床上,古怪又诡异,像是在提醒人们,她就是在这间屋子里流血死去的。卡罗的床头上方悬挂着一幅画:死去的婴儿裹着白布,头戴花冠,枕在一个缎面枕头上——一个“小天使”。

\hr

\section{美国北卡罗莱纳州的柯洛威}
\hr

这头灰鲸体形庞大,身长55英尺,体重超过36吨,尾巴长达10英尺。它出现在距离加利福尼亚州海岸十几公里的海域,随着身体逐渐浮出水面,它微弱地喷出一道水柱。这是它最后一次这样做了。这头65岁的庞然大物没有逃过死神的魔爪,一动不动地浮在水面上。

有些鲸死后会迅速下沉,但这头灰鲸将一直处于漂浮状态。它体内的肌肉组织和蛋白质不断地分解,内脏逐渐腐烂变成液体。由此产生的腐败气体越积越多,最终填满了脂肪外层,把这头灰鲸变成了一个尸体气球。如果这时在它的身上戳一个洞,它体内的气压能把烂成糊状的内脏喷出几米高。不过,灰鲸的皮肤强韧有力,降低了气体泄漏的速度。随着气体逐渐排出体外,这头灰鲸缓慢地沉入水中。慢慢下沉了大约1公里之后,它落在了柔软的海床上。

这里是深海区,黑暗、冰冷,因为阳光无法照射到如此深邃的地带。我们这位灰鲸朋友不是来这里“安息”的,也不是来乘凉或是享受寂静、黑暗的——它的遗体即将成为一场持续几十年的盛宴。一套完整的生态系统将以鲸的尸体为中心形成,海洋科学界把这个过程称为“鲸落”,类似于给那些外星人模样的海洋原始生物开了一家餐厅。

四处游动的食腐动物循着气味而来,抵达灰鲸尸体并大快朵颐。睡鲨、八目鳗、螃蟹、银鲛,这些长得像异界来客似的家伙都是典型的深海生物。它们疯狂地吞噬着腐肉,一天最多能进食130磅。

当有机物被吃干净之后,另一批生物就占据了剩下的尸体及其周边区域,原本是一片不毛之地的海床突然变得热闹起来。鲸骨上积满了厚厚一层红色的深海蠕虫,每平方米约有45000条。这种蠕虫的拉丁文名称叫“Osedax”,意为“噬骨者”。它们没有眼睛,也没有嘴巴,直接钻入骨头提取骨髓中的油和脂肪,绝对名副其实。科学家最近发现,存在于鲸落中的硫细菌与深海热液喷口处的细菌非常相似。

鲸落的现场变成了《美女与野兽》插曲《做我们的贵客》中的盛宴场景。这场纵情声色的欢闹派对可以持续数十年,使得无数海底生物拥有了“菜好吃,吃不停”的体验。鲸落是死后造福后世的典型,美丽而壮烈,又合情合理——死去的动物把遗体奉献了出去,作为其他生物的生存来源。鲸鱼的遗体仿佛在招呼道:“快来尝尝这块灰色的玩意儿,好吃极了。”如此看来,鲸简直是死者界的楷模。

...

鲸一生都在为周围的环境做贡献。鲸以鱼和磷虾为食,这就让人类产生了一种错觉:鲸的数量减少等于鱼和虾的数量增加。这个等式似乎完美地解释了为什么捕鲸业在20世纪至少屠杀了300万头鲸鱼。

事实上,虽然鲸的数量减少了,但鱼的数量并没有增加。当鲸潜入深海区捕食一段时间后,需要浮到水面上换气并进行排泄。鲸的粪便富含铁元素和氮元素,为浮游生物提供了充足的养分——你也许已经猜到了,浮游生物正是鱼和磷虾的美食。不管是生是死,鲸都是生态链不可或缺的一部分。

你也许会本能地联想到自己,不然怎么会有越来越多的人说:“别把后事搞得那么复杂。等我死了,你们直接挖个坑把我埋了就行。”

这其实是一种明智的做法。让你的遗体回归自然,应该是最便宜、最绿色的丧葬方式了。你将为大地提供养分,而你生前消耗的植物和动物正是得益于土地的营养。

1英亩的土壤中包含2400磅真菌、1500磅细菌、900磅蠕虫、890磅节肢动物和藻类,还有130磅原生动物。土地蕴含着勃勃生机,尸体也是(就藏在角质层——也就是死皮——下面)。虽然尸体只埋在地面以下几英尺,微生物却已经开始工作了。数万亿生活在你体内的细菌把你的内脏变成液体,随之产生的气体把皮肤撑破,导致尸体内外的微生物疯狂地进行重组。最终它们将我们的肉身与土壤结合在一起。

大地赋予了我们生命。威廉·布莱恩特·洛根说过:“我们还回去的肉身还不足以作为回报。”虽说如此,但至少我们已经开始报答大地母亲了。

\begin{center}
* * *
\end{center}

“卡特里娜,你会如何描述你们在这里的工作呢?”

卡特里娜思考了一下:“我们在这里做实验。”

“什么样的实验?”

“等等,我想换个词。‘实验’让我听起来像个疯狂的科学家。”

“那你想用哪个词?”

“我们在这里‘堆肥’。不不不,这听上去也挺吓人的。”我等着她重新遣词造句。

“这么说吧,我们在改善传统的‘堆肥处方’。”卡特里娜说道,显然她对这种表述也不甚满意。

如果你是卡特里娜·斯贝德,这个被《纽约时报》描述为“把尸体变为肥料”运动的领军人物,你也会小心翼翼地组织语言。因为你需要一套巧妙的推销词,让人们不把你的绿色创新型殡葬企划与《超世纪谍杀案》中丧心病狂的商业丑闻混作一谈。

卡特里娜和我行驶在阿巴拉契亚南部蓝岭山脉的崎岖小路上,山下就是田纳西州和北卡罗来纳州的边界线。和美国其他地区一样,现代殡葬业在这里深深地扎了根,取代了其他形式的丧葬仪式和处理后事的方法。但该地区地理位置偏僻,再加上宗教和贫困等因素,现代殡葬业用了很长时间才打开这里的市场,比在美国其他地方用时都要久。

我们来到一条人迹罕至的小路上,在一扇大门前停下了车。切丽尔·约翰逊博士正和一群本科生在门口等我们。她的学生都管她叫“J博士”。J博士是西卡罗来纳大学法医骨学研究站的负责人。你也许听说过,这种机构通常被叫作“尸体农场”,主要以法医病理学研究和执法培训为目的研究人类尸体的分解过程,里面的尸体都是捐献给科研事业的遗体。但是,J博士很快指出:“‘尸体农场’这个叫法并不准确,农场是种粮食的地方,但我们不种尸体。考虑到我们的最终产品,我觉得应该管这个地方叫‘白骨农场’,对吧?”

我用余光瞄到几个看起来像是土坟的东西,每一个都用银色的布盖着。他们把捐献的遗体放在这儿?就放在停车场?我琢磨着。迄今为止,我见过的尸体数不胜数,它们大都躺在消过毒的白色操作台或者轮床上,完全不具有威胁性。但当你在“本不应该”出现尸体的地方看见尸体时,你会感到一阵不安,就像在超市里碰见你的化学老师一样。

“不,”简单介绍了这里的情况之后,J博士告诉我,“这些不是人类的尸体,是黑熊。它们都是在路上被车轧死的,有时自然资源部一年能给我们送来15~20头。黑熊皮毛的颜色很深,晚上开车一不小心就会撞到它们。”

埋葬黑熊并进行研究是本科生的课题。当熊的尸体只剩下白骨后,学生们随即便建立系统网格并采集骨骼样本,然后带回实验室研究。成功地完成黑熊尸体研究的学生将获得研究人类的资格。人类的尸体不是在停车场,而是放置在山坡上一处58英尺×58英尺的空地上。为了防止不速之客,比如郊狼、熊、喝大了的大学生等,空地用电网围了起来。

我们一行人来到这里,J博士打开门上的锁。进入这片区域之后,我发现自己没有闻到刺鼻的气味,也感受不到死亡的气息。不仅如此,这块位于北卡罗来纳山间的空地如画般美丽,斑驳的阳光透过茂密的枝叶,洒在茂盛的灌木丛上。目前,有15具尸体在这里享受死后的时光:土里有3具,地面上有12具。

地上有一具身穿紫色波点睡衣的女尸,春天时的一场暴风雨让她的骨头散落得到处都是——她的脑袋正待在她的股骨附近。在她左侧几码开外的地方躺着一个男人。这个人刚过世不久,下颌骨大张着,眼看就要掉下来了,全靠薄薄的一层皮肉将其固定。如果蹲下来近距离观察,你就会看到他的脸上有一层琥珀色的绒毛。

卡特里娜指着山坡上一具开膛破肚的尸体:“几个月前,我看见他的时候,他的山羊胡完好无损,大理石般的蓝色皮肤堪称完美,虽然他闻起来不怎么样。”把我们带过去后,卡特里娜向我们道歉,“不好意思,他确实闻起来不太好。”

卡特里娜第一次想到用尸体制作堆肥时,她正在写自己的建筑专业硕士论文。其他学生都在钻研雷姆·库哈斯和弗兰克·盖里的作品,卡特里娜却忙着设计一处“让城市的死者安息的地方”。在她看来,她未来的客户是那些不介意在钢筋水泥中生活,但希望死后回归大自然、“让肉体化作土地”的都市人。

想实现“让肉体化作土地”这一愿望,有一个直截了当的方法,那就是增加自然土葬或保护型土葬场所的数量——没有防腐,没有棺材,也没有水泥墓穴,遗体直接埋进土坑就完事了。既然如此,卡特里娜为何要选择堆肥这种相对复杂的方式呢?她的回答很正确,过度拥挤的城市不可能把大片已开发的昂贵土地提供给死人。于是,卡特里娜把自己的改革目标从土葬转向火葬。

致力于在城市建造尸体堆肥中心的“城市死亡计划”脱胎于卡特里娜的论文。从北京到阿姆斯特丹,这些中心将遍布世界各地。建筑的核心部分是两个半层楼高的平台,由细腻、保温的水泥制成。平台的周围是若干条坡道,哀悼的人们可以顺着坡道把遗体搬到平台顶端。到达顶端后,遗体将被安置在碳含量极高的混合物中,只需4~6周就能完成分解(包括骨头以及其他所有组织),从而完全融入土壤之中。

当你把高氮物质(厨余垃圾、杂草或者……人类尸体)与高碳物质(木块或锯末)混合在一起时,就会发生堆肥反应。往里面加入水和氧气之后,混合物中的微生物和细菌便开始分解有机组织,释放出热量,就像给饭菜加热一样。混合物的温度通常能达到150华氏度,足够杀死大多数病原体。如果氮元素和碳元素的含量达到正确的比例,混合物的分子将重新组合,创造出肥沃的营养土。

“经历了这4~6周,你就不再是人了。”卡特里娜解释道,“在这段时间内,你的分子转换成了其他分子。也就是说,你变异了。”分子的这种转换让卡特里娜备受启发,她给这个过程起名为“重组”(对普通群众来说,“尸体堆肥”这几个字太重口味了)。重组完成后,死者家属可以把土壤置于家里的花园里,生前热爱园艺的母亲又能亲自哺育新的生命了。

卡特里娜99\%确信我们能够重组人类。她的顾问委员会里有一大批土壤科学家,他们都认为她应该有100\%的信心,毕竟他们从事家畜堆肥已经很多年了。分解1000磅公牛所用的化学和生物过程,应该也能作用于仅有180磅重的人类。不过,卡特里娜需要通过活生生(或者说死翘翘)的人体实验来印证这个设想。

J博士和骨学研究团队就是在这个时候参与进来的。J博士虽然对卡特里娜利用尸体制作堆肥的想法很感兴趣,但是没有立即安排实验。不久之后,J博士碰巧从学校的回收项目中得到了大量废弃木材。很快,她又接到一个通知,一具新的遗体马上就要送到骨学研究站。这时,她才发短信给卡特里娜:“我们拿到了一具尸体。现在开始实验如何?”

2015年2月,这具遗体身下铺满碎木块,出现在骨学研究站实验用地的山坡底部。遗体的主人是一位78岁的老妇人,我们给她起名叫琼恩·坎普斯特\footnote{坎普斯特的英文含义是堆肥的意思。}。一个月之后,J博士收到了第二具遗体,是一个大个子男人(我们叫他约翰·坎普斯特)。实验团队把他带到山坡顶,放入苜蓿和碎木块的混合物中,然后用银色布料盖起来。这个实验并不复杂,只需这两具遗体回答一个问题:你们会不会变成堆肥?

还有一个小时,一具新的遗体就将抵达骨学研究站。这个人叫弗兰克,60多岁。他在这周早些时候死于心脏病,临死之前决定把自己捐献给研究人类腐化过程的机构。

“弗兰克的家人知道堆肥这件事吗?”我问J博士。

“我跟他的兄弟鲍比谈过几次。”J博士回答道,“我很明确地告诉他:‘你完全可以拒绝,之后我们会让弗兰克参与普通的法医学研究。’但他表示这个研究才是弗兰克的心愿所在。说实话,当你签字确认要把遗体捐到我们这种机构时,就等于接纳了所有的可能性。”

为了迎接弗兰克的到来,我们把一大堆松树和枫树的碎木块运到山上,大约有5加仑那么多。卡特里娜没有被这个运动量吓到。她身材瘦高,留着小精灵的发型。虽然已经快40岁了,但她看起来就像是高中足球队里的明星球员。她拎起一筐木头,几乎是蹦蹦跳跳地上了山顶。

一个身材魁梧的金发男生同时拎了四筐木头,每只手上两筐。

“你是这里的学生吗?”我问道。

“是的,夫人,我是法医人类学专业的大四学生。”他用一种慢吞吞的语调说道。我说服自己,“夫人”只是他们南方人的习惯用语,跟我的年龄无关。

在北卡罗来纳的艳阳下搬木头纯粹是个体力活(我必须得说,我英勇地完成了这个任务),完全体会不到清扫骨灰时的那种禅意。

中午11点时,我们已经把碎木块堆成了一个2英尺高的长方体,就差让志愿者——也就是我们的弗兰克——躺进去了。说曹操,曹操就到,一辆深蓝色的货车驶入了停车场。从车上下来两个人,他们都穿着紧身卡其裤和与货车同色的POLO衫,衣服上印着“克罗威殡仪馆”的商标名。这二人是父子档,白头发的是老克罗威,金发飘飘的是他的儿子。

克罗威父子之前从未来过骨学研究站,于是J博士带他们参观了一圈。父子二人的脸上满是纠结和困惑,应该是在思索如何才能顺利地穿过堤岸和灌木丛,把弗兰克搬到山上。老克罗威首先发话了:“他是个大块头。”

人类经常死在比较麻烦的地方(轮椅、浴缸、后院仓库、又高又陡的台阶上等),殡葬人的工作就是把他们从这些地方移走,不让他们直接烂在那里。殡葬业以平复死亡带来的骚乱为荣,他们看重回归秩序,而不是制造混乱。

我问老克罗威,这是不是他经历过的最奇怪的一次遗体运送。

他回头看了看,然后用干巴巴的语气说了一句“yes”(是的)。就这一个单词,没了。

运送弗兰克的路线计划好了:不仅有落脚的地方,还不会踩到山上其他的死者。在骨学研究站,雨水和小动物让尸体腐化的过程变得一团糟,人们一不留神就会踩到不知是谁的腓骨。

克罗威父子在大门外组装好轮车,把装在医院蓝色尸袋里的弗兰克放了上去。闪电蓝色的尸袋,在绿色和棕色绘成的北卡罗来纳的夏日里,显得格外刺眼。弗兰克大脚趾上的标签写着“西卡罗来纳大学——城市死亡计划”。卡特里娜拿起标签看了看,嘴角闪现出一丝微笑。后来她告诉我,当看到项目的名称出现在印刷品上时,她感到自己被正式认可了。

老克罗威和J博士聊了起来。出乎我的意料,老克罗威没有让J博士“再跟我讲一遍,你们这帮疯子都在这里干啥”,而是讨论起技术问题:“也就是说,你用苜蓿加速氮元素的分离?”原来,克罗威老爸自己就是堆肥专家,流程上的技术细则说得头头是道。在公司制的殡葬业,有人把自然土葬描述为“嬉皮士的胡闹,我们的客户根本看不上”。因此,看到一位传统的殡葬人站在了我们激进派这一边,我非常开心。

但是,对可怜的卡特里娜而言,她要赢得的不仅是殡葬业的支持。知名博客作家麦克·亚当斯在脸书上发布了一篇关于卡特里娜的文章。在这篇点击量已达11000多次的文章中,亚当斯认为堆肥项目的最终目的是为城市人口种植更多的粮食。他认为,既然新的世界秩序需要稳定的人类遗体供应来解决粮食不足的问题,那么一定会导致“对老年人强制实施安乐死,获取其遗体进行堆肥”。亚当斯声称:“政府将用这个项目漂绿\footnote{因为经济利益而进行的虚假的环保宣传。}大规模屠杀行为。”

了解卡特里娜的人都知道,她是个热心环保的人士,和伴侣及两个孩子一同住在西雅图,把她说成大规模屠杀的策划者简直荒唐透顶。虽然如此,但公共关系危机依然存在:有一个坚信遗体应该回归大地的人,就有一个把卡特里娜的项目视作堕落的人。

是时候把弗兰克推上山了。这是一次团队合作,我们首先开始了一场头先还是脚先的冗长讨论。在争论的过程中,我抬头看到山上有颗骷髅头正观察着我们这帮愚蠢的活人。

当弗兰克终于抵达山坡顶端时(头先),我们把蓝色的尸袋放在碎木块上,然后拉开拉链,一个高大、强壮的男人出现在我们面前。弗兰克浑身赤裸,只穿了短裤和袜子。我们抬起他的身体左侧,轻轻把尸袋从他身上脱下来。现在,碎木块和弗兰克全部就位,想后悔也来不及了。

弗兰克有着白色的山羊胡和一头披肩长发,左臂优雅地搭在脑后,就是《泰坦尼克号》中杰克给露西作画时露西摆的那种姿势。他的双臂和躯干布满文身:巫师、巨蟒、宗教符号,胸口上还有一只狂奔的霸王龙,墨水的颜色已经褪成深绿色。

本科生回到山下去取更多的苜蓿,山顶上就剩下我和卡特里娜两个人。这是我们二人在这个早上的第一次独处。

卡特里娜凝视着弗兰克,眼中泛着泪光:“知道吗,这个人不是偶然出现在这里的。他是自愿要来的。”她停下来深吸一口气,然后继续说道,“我非常感激他。”

卡特里娜把一小堆绿色的苜蓿和木块混在一起,在弗兰克的脸上铺满。这是堆肥的第一步。

我随即加入,和她一起用混合物把弗兰克的脖子和上半身埋起来,看上去像是给他盖了床被子。“我们给他做了一个舒服的小窝!”卡特里娜评价道,这时,她停下手里的动作,用自责的口吻说道,“J博士说过,与遗体一起工作时,我们不能多愁善感。别再感情用事了,卡特里娜。”

这我有些说不准。今天早些时候,J博士跟我聊起一个把遗体捐赠给骨学研究站的80岁老人。老人死后,他的老伴和女儿开着自家的卡车把遗体送来。得到允许后,他们还亲自在山上给他选了一个安置点。然而,六个月之后,老太太也离世了。临终前,老太太要求把自己的遗体放置在丈夫附近的区域。骨学研究站同意了她的请求。这对老夫妇携手走过人生,现在也将一同化作泥土。这怎么能不让人多愁善感?

J博士不认为自己的态度有错:“我习惯对捐赠来的遗体直呼其名,比如某某先生或者某某女士,也就是直接叫他们的本名。我没有理由不这样做。这些人虽然死了,但身份不变。有些机构不认同我的做法,认为我不专业,没有和死者保持距离。我完全反对这种看法。直呼其名赋予了死者人性。有些人临终之前和我见过面,我认识他们,他们是人。”

J博士的观点反映了对科研用遗体的新主张,那就是把捐献的遗体视为一个人,而不是某具无名尸。印第安纳大学医学院西北分部副主任小厄内斯特·塔拉里科也持有同样的观点。他的学院经常接收捐献遗体,让学生用来练习解剖。塔拉里科第一次接收遗体时,发现自己无法接受当时人们固有的思维模式:捐来的遗体就是一堆无名的肉块,可以直接用编号或者外号称呼。

于是每年1月,塔拉里科都会给这个学期将要用到的6具遗体举行纪念仪式。令人震惊的是,参与者不仅包括医学院的大一新生,还包括遗体捐献者的家属。丽塔·伯莱利把自己丈夫的遗体捐赠给了印第安纳大学。不久后,学生给她寄去一封信,表示希望多了解一些她丈夫的生平。她对此很是惊讶:“他们还想要一张他的照片。我号啕大哭,连信都没有读完。”

死者家属可以自愿选择是否参与。而对学生来说,他们将学会克服现代医者所面临的一个难以跨越的障碍——与病人家属坦诚地谈论死亡。这些学生甚至把这些遗体称为“他们的第一个病人”。《华尔街日报》在对该学院的一篇报道中引用了大一新生拉尼娅·卡欧吉斯的话:“把遗体想象成编号也许会让我们好受一些,但合格的医生是不会这么做的。”

既然这是当前的潮流,我问J博士,当轮到她自己脱离尘世的喧嚣时,是否会把遗体捐献给骨学研究站。她的回答是原则上会,但担心学生的反应。了解遗体捐赠者的生平和能对遗体直呼其名是一回事,眼看着教过自己的教授逐渐腐烂是另一回事。不过,真正阻碍J博士捐献自己遗体的是她的母亲。她母亲那一代人认为,只有在教堂里举行葬礼才算得上体面,因此竭力反对自己的女儿把遗体捐给研究人体腐烂过程的机构。只要母亲还活在世上,仍对遗体捐献持反对意见,J博士就不会这样做。

但就在最近,J博士的母亲就如何处理自己的遗体一事发表了意见:“真不明白我们为什么要在火葬和土葬上大费周章。难道就不能把我们带到森林里,让我们自然地腐烂吗?”

“妈?”J博士开口道。

“什么事,亲爱的?”

“你知道这就是我的工作,对吧?你知道骨学研究站就是一个能让你在森林里自然腐烂的地方,对吧?”

埋葬弗兰克的木块堆已经达到了3.5英尺高,看起来特别像传统的维京人墓冢。那个身材魁梧的金发本科生用木桩和铁丝网做成的围栏把堆肥圈了起来,防止这堆混合物(或者说防止弗兰克)滑落到山下。这和城市里的堆肥过程大相径庭,但此处蝉鸣鸟啼,树影间阳光闪烁,尸体能够在这里腐烂,绝对符合美学的定义。

另外几个本科生朝我们走过来。他们一个个大汗淋漓,浑身沾满木屑,手里拎着回收再利用的“泰迪猫”牌猫砂包装桶,里面盛满了水。我们把一共12加仑的水浇注在堆肥上,潮湿的环境将加快微生物和细菌的反应速度。当我们对整个过程进行拍照记录时,有人提议撕掉水桶上“泰迪猫”的标签,不然有一种“‘泰迪猫’隆重推出人类尸体堆肥产品!”的即视感——骨学研究站和“泰迪猫”应该都不愿意看到人们产生这种联想。

当水从堆肥的顶端浇下去的时候,卡特里娜将其视作一种仪式,可以用在未来的堆肥中心里。她不希望“城市死亡计划”的堆肥中心和如今的殡仪馆一样,鲜少让死者家属参与。她希望给堆肥浇水这一环节能够像点燃火葬柴堆、按下火化按钮,或者给棺材铲土那样,充满仪式感和力量。当我们给弗兰克所在的堆肥浇水时,我感到这就是一个仪式:不管对弗兰克还是对整个社会,这都象征着一个新的开始。

\begin{center}
* * *
\end{center}

我们在镇里的体育酒吧吃过午饭,又回到了骨学研究站。弗兰克不是我们今天来到骨学研究站的唯一原因,我们还要看看最早的两具捐献尸体——琼恩·坎普斯特和约翰·坎普斯特。J博士今天就要把他们从堆肥里挖出来,检查实验进度。

在去往山顶的途中,J博士向卡特里娜宣布道:“之前忘了告诉你,寻尸犬没有发现这几处堆肥。”卡特里娜的眼睛一下子亮了。

作为一名法医人类学家,J博士为无数起发生在山区森林里的失踪案件提供过咨询服务。在目睹警方搜寻尸体时的困难之后,J博士成立了骨学研究站,旨在培训执法人员搜索与救援志愿者和他们的工作犬。骨学研究站为这些专业人士提供了巨大的便利条件,因为这里有货真价实的腐烂尸体,完美模拟了现实中野外弃尸的状况。在为期一周的培训结束后,J博士给每个人都发了一包被她称为“脏脏土”的东西——来自腐烂尸体下方的泥土。有了这个,他们在返回工作岗位之后还能继续训练寻尸犬。“你不知道他们在拿到这些泥土以及混有泥土的衣服碎片时有多激动,就像拿到了圣诞礼物一样。”J博士告诉我。我不禁联想到一首古老的圣诞礼赞:“我的真爱送了我,两只斑鸠和一包死人身下的土。”

为什么寻尸犬没有发现堆肥是件大事呢?寻尸犬的嗅觉灵敏,能轻易发现旷野或浅坟里的尸体。但堆肥中的湿度、曝气、碳元素和氮气能将尸体的气味密封在堆肥里。卡特里娜知道,如果用来哀悼和举行仪式的堆肥中心散发出人体腐烂的恶臭,是不会赢得公众支持的。寻尸犬没有闻到堆肥里尸体的气味,这对“城市死亡计划”来说是个天大的好消息。

我们决定首先检查约翰·坎普斯特。这是一个高大、健壮的老人家,65岁左右,死于今年3月。这样算下来,他在木块和苜蓿的混合物中已经躺了5个月。他被安置在山顶上,意味着他比别人拥有更多的太阳直射时长和更高的环境温度。他的堆肥还用银色的布整个罩了起来。

如果使用正常大小的铁锹和铁铲,不管里面有什么,都会存在损坏的风险。于是,我们全换上手持型迷你铲和大号塑料耙。我们一丝不苟地进行挖掘工作,但手上亮紫色和亮黄色的铲子让我们看起来像是用埋尸体的土堆城堡的变态。

突然,我们碰到了骨头。J博士走过来,用一把精致的刷子轻轻地扫落上面的碎屑。约翰·坎普斯特的左锁骨出现在我们的眼前。

卡特里娜大吃一惊:“说实话,我以为这里什么都没有。我以为不管怎么挖,挖出来的只有……土。”

J博士微微一笑:“我倒是希望这里有一些东西。”

“等一下,”我插话道,“我们的目的是4~6周内,堆肥用的尸体能够完全分解。您为什么还希望有骨头留下来呢?”

卡特里娜抢先回答:“因为J博士还有其他的目的,她需要这些骨头。”

原来,J博士热衷于卡特里娜这个项目的同时,还在担心骨学研究站的骨骼储藏量不足。这里的人类学标本离所需的数量还差得很远。一套完整的样本需要体现出性别和年龄上的广度,这样才能进行真实、有效的比较。

J博士相信,如果能掌握正确的挖掘时机,她就能设计出一个新方法,大大缩短从肉体腐化为白骨的时间。要知道,传统的方法是这样的:把尸体放在野外,任由动物和大自然发挥它们的作用。

把约翰·坎普斯特放入木屑堆的第一天,为了提高堆肥的温度,工作人员把一层绿色的苜蓿撒在遗体上——现在看来,这个方法应该是有效的。不过,光有温度还不够,堆肥还需要湿度。当我们把木屑一层一层地从约翰·坎普斯特身上移开后,发现苜蓿吸干了遗体内的水分。约翰·坎普斯特现在成了一具干尸,白纸般的肌肉组织附着在髂骨和股骨上。我轻轻地用刷子把这些干肉片刷下来。尸体堆肥第一课:不要放入过量的苜蓿。

当J博士挖出遗体的头颅和右肩膀时,她发现了一个有趣的现象。这两处都没有撒上苜蓿,大量雨水从山顶滑落,从布罩底部渗进堆肥,浸湿了这两个地方。与其他已经干尸化的部分相比,这两处的骨骼干干净净,一块皮肉都没有,就是颜色有些发暗。而且,他的胸骨上出现了瑞士奶酪上面的那种小洞。也就是说,他的胸骨已经开始腐烂了。

虽然这几处发现着实鼓舞人心,但约翰·坎普斯特远未达到卡特里娜理想中的状态——完全分解并转化为富含营养的土壤。约翰在堆肥里足足待了5个月,但他不仅没有消失,还变成了一具木乃伊。在机械曝气的辅助下,只用4周的时间就可以让一头成年公牛完成堆肥。屠宰场的垃圾堆肥只需5天。这样一比较,人类堆肥还有很长的路要走。

J博士没有灰心:“每次你都能从失败中学到新东西。”她耸了耸肩,指挥我们再把约翰·坎普斯特埋起来(这次我们多加了水,除掉了苜蓿)。

骨学研究站进行的这个实验,让我想起了意大利解剖学教授罗德维科·布鲁奈提在19世纪末尝试建造历史上第一台现代火化机。布鲁奈提的思路非常符合当时工业时代的思潮,被学者托马斯·拉克尔称为“朴素科技现代主义”。

布鲁奈提经历过几次失败的实验,但这些失败象征着“尸体处理史上的新篇章”。放眼今日,使用工业火化机火化遗体已经成为各个发达国家处理遗体的最主流方式。

布鲁奈提用石砖炉火化的第一具尸体是一名35岁的女性。这个实验没有成功。虽然炉内最后只剩下5.5磅重的遗骨,但火化的过程持续了四个小时,在他看来耗时过久。

经过一段时间的思考,布鲁奈提认为应该事先把遗体切成几段来加快火化的速度。于是,在第二次实验中,一名45岁男性的遗体被分成三层放在上次使用的石砖炉里:第一层是四肢;第二层是脑袋、躯干和骨盆;第三层是器官和其他脏器。这次同样用了漫长的四个小时,但只剩下2.5磅重的遗骨。

卡特里娜考虑过这个窍门。许多堆肥专家告诉她:“如果想提高堆肥的效率,最好先肢解尸体。”令人坐立不安的建议还不止这一个。有人让她务必往堆肥中添加粪便做的肥料,还有人给她发邮件建议:“亲爱的斯贝德女士,我对你的项目很感兴趣。我的几次堆肥都很成功,因为我使用了医院丢弃的病人的尿液。不知你是否考虑过这个方法?”

“你回复他了吗?”我问道。

“我非常礼貌地拒绝了他的提议。尿液富含氮元素吗?富含。这是否会加快堆肥的进度呢?也许会。那么,我会把尸体放到里面吗?不会。”

而布鲁奈提不仅没有被肢解尸体这个流程吓到,还决定在第三次实验中使用高温。他把所有尸块混在一起塞入用来制造煤气的焚烧炉里。19世纪,煤气是主要的电力来源。焚烧炉内的温度比石砖炉高出几百华氏度,却比之前多用了两个小时才完成火化(总共六个小时)。结果是,所有的骨骼全部碳化,没有残留任何有机物质。一切让人生而为人的元素全部荡然无存,包括DNA——虽然那时候还没人知道DNA是什么。

布鲁奈提在他1884年的一篇文章中提到了火化:

\begin{quotation}
这是严肃、庄严的时刻,神圣而又壮丽。燃烧的尸体总能将炽烈的情感从我心中唤起。当肉身还以人的形态呈现在火焰中时,我将其视作奇迹,用赞美的目光打量着。当人的形态逐渐消失,全部转化为黑炭后,我的心头只剩一股悲凉。
\end{quotation}

1873年,布鲁奈提已经准备好在维也纳世界博览会上呈现自己的实验结果。意大利区第54号是他的展位,桌上的几个玻璃瓶里盛放着他的实验结果——不同火化程度的骨骼和肌肉组织。

布鲁奈提的火化技术让我们的社会摆脱了腐烂过程,直接将尸体变为无机物。他希望将火化工业化,达到工厂流水线的效率水平,火化过程完成得越快越好。拉克尔指出,布鲁奈提眼中的现代火化“属于科学和技术的范畴”。他所传达的信息很明确:大自然的传统方式缓慢而笨拙,几个月才能完成2000华氏度的火炉几个小时就能完成的事。布鲁奈提的展位上有一个标语“Vermibus erepti—Puro consumimur igni”,即“蛆虫不再吞噬你的肉体,火焰将把你净化”。

布鲁奈提认为只有火焰可以净化肉体,但150年后的卡特里娜和我可不同意这种看法。在诗人惠特曼看来,土壤和大地具备转化的力量——接纳人类的“遗骸”,将其化作“神圣的物质”。惠特曼惊叹于大地的鬼斧神工,用腐化、肮脏、充满病菌的污物创造出崭新、纯洁的生命。既然你的“遗骸”能够发挥更大的作用,为何还要把燃气和火焰作为处理自己肉体的唯一方式呢?

J博士回到临时搭建在停车场的工作棚,从约翰·坎普斯特胸口处的记录器中导出数据并上传至电脑。这些数据记录了堆肥的温度变化。我和卡特里娜留在山上,开始挖掘第二座堆肥,也就是埋有琼恩·坎普斯特的那个。这位78岁的老人死于恶疾的折磨,她的堆肥只用了木块,位于山脚下,没有任何遮挡。

当挖掘到堆肥深处时,我们发现了甲虫的幼虫和蛆。堆肥深处的土壤营养充足,颜色是浓郁的黑色——人们经常把这种堆肥土称为“黑金”。在理想情况下,土里是没有虫子的,不然就意味着堆肥中仍然存在供这些家伙大快朵颐的营养物质。这时,我挖出了琼恩的股骨,上面覆盖着厚厚一层白色的、还未腐化完全的脂肪,看起来和希腊酸奶的质感差不多。随着挖掘工作的继续,我们发现琼恩此时处在腐烂的最后阶段,全身基本上只剩下骨头了。

琼恩·坎普斯特的问题和约翰·坎普斯特的正相反:她所在的堆肥湿度适宜(因此,琼恩才能顺利白骨化),但是氮含量偏低,无法产生将白骨降解为土壤所需的高温。

这样一来,琼恩和约翰这两个实验都没有成功,但这只是卡特里娜系列实验的开始。越来越多的遗体将被送到骨学研究站用于堆肥实验。维克森林大学的谭雅·马什教授要求墓地法专业的学生通读各州法律,寻找在全美50个州合法化人体堆肥机构的方法。西华盛顿大学的琳恩·卡朋特-博格斯是一名土壤科学家兼堆肥专家,即用人类大小的动物遗体(小牛、大型犬、羊、猪等)进行相关实验。现在,还有很多研究致力于探索堆肥过程对遗体牙齿中汞合金填充物有何种作用。要知道,火化会让汞合金向空气中释放毒素,这是与火化有关的最严重的环境污染问题之一。

...

\hr

\section{西班牙的巴塞罗那}
\hr 

美国殡仪馆的审美出奇地一致:20世纪中期流行的矮胖款墙砖,挂满天鹅绒窗帘的房间,来自佳丽牌空气清新剂的刺鼻香气(为了掩盖遗体准备室里的消毒水的味道)。与之相反,巴塞罗那的阿尔蒂玛殡仪馆可以说是谷歌总部与山达基教教堂的混合体。极简抽象风格混搭超现代主义,创造出一种异教氛围。这栋三层楼高的建筑的地板、墙面和天花板全部使用典雅的白色石砖。站在宽大的阳台上,花园的美景便可一览无余。注意,是花园,不是停车场。这里还有一堵从地板延伸至天花板的玻璃幕墙,可以将整个城市的山光海色尽收眼底。自带的咖啡厅还提供免费Wi-Fi。

地中海的阳光从窗户照射进来,白色地板的反光让我不得不眯着眼睛和外表迷人、衣着得体的殡仪馆工作人员谈话,其中就包括乔瑟夫——殡仪馆的负责人,一个风度翩翩的西装男。

除了乔瑟夫,还有63个人在这家运营良好的殡仪馆工作。他们负责收殓遗体、清洁遗体、死亡证明存档,并与死者家属洽谈,举办葬礼仪式。巴塞罗那1/4的遗体都在阿尔蒂玛殡仪馆进行处理,平均每天10~12具。死者家属可以选择土葬或火葬。由于信仰天主教,西班牙对火化的接受程度比不上其他大多数欧洲国家。西班牙的火化率为35\%,巴塞罗那市区的火化率接近45\%。

要想了解巴塞罗那的葬礼仪式,首先得了解玻璃的含义。玻璃意味着透明,意味着直视死亡这个残酷的现实,同时象征着一道明确的分界线——虽然死亡近在咫尺,但你永远也碰触不到。

阿尔蒂玛殡仪馆拥有两个大型的礼拜堂和20间家属室。死者家属可以租下其中一间,在里面与死者待上一整天。很多家庭都是这样做的,从早上一直待到晚上10点殡仪馆关门。重点来了,在这一整天的时间里,殡仪馆一直用玻璃把家属和死者隔开。

你可以选择玻璃的使用方式。如果你选择西班牙式的遗体瞻仰,阿尔蒂玛殡仪馆会将装有遗体的棺材安置在一堵玻璃墙之后,周围摆满鲜花,与百货商店的橱窗陈列非常相似。如果你选择加泰罗尼亚式,殡仪馆就会把棺材放入白雪公主式的玻璃罩里,安置在房间中央。不管选择哪种,殡仪馆都能保证在瞻仰期间,遗体周围一直处于32~42华氏度。

来到幕后,你会看见长长的走廊里摆满了木质棺材,一具具遗体在里面等待着登台亮相的那一刻。每个家属室里都有一扇小型金属门,工作人员就通过这扇门运送遗体。

“为什么玻璃罩属于加泰罗尼亚式?”我问道。

霍尔迪·纳达尔是我的翻译。他是一家出版社的负责人,我第一本书的西班牙语版就是在他的公司出版的。霍尔迪是希腊人左巴式的人物,一有机会就将“及时行乐”付诸实践。只要有他在,你的红酒杯永远是满的,盘子里的鱿鱼和海鲜饭怎么都吃不完。

“我们加泰罗尼亚人愿意与死者更近一些。”这是他的回答。

“所以,你们就像展示动物标本似的,把遗体放在玻璃罩里?你们到底觉得这些尸体会给你们带来哪些危险呢?”这些话我想说但没有说。

事实上,我在西班牙这一周都在接受该国媒体的采访,讨论西班牙殡仪馆把死者和家属分开的问题。阿尔蒂玛殡仪馆也看到了这篇报道,所以他们能够允许我前来拜访简直是个奇迹。他们还愿意采用其他的遗体瞻仰方式,美国的殡仪馆可从没向我表示过类似的意愿,因此我不想糟蹋自己的运气。

不过,这并不意味着我和他们之间没有矛盾。一个上了岁数的员工问我是否喜欢巴塞罗那。

“这个城市很棒,我完全舍不得走。我都想在你们殡仪馆找份工作了!”我开玩笑说。

“考虑到你的一些看法,我想我们不会录用你。”他也开玩笑似的说道,然而语气里带着一丝严肃。

“西班牙也有这句谚语吗,‘亲近你的朋友,更要亲近你的敌人’?”

“有。”他挑起眉毛,“我想我们会照做的。”

我在巴塞罗那遇到的一些人(普通市民和殡仪馆的员工)抱怨当地的殡葬流程过于匆忙。似乎每个人都认为遗体应该在24小时之内安葬,但又说不出为什么。死者家属觉得压力很大,因为殡仪馆急于在短时间内完成所有流程。殡仪馆却认为这些家属“要求不管干什么都要快,绝不能超过24小时”。于是,所有人都困在“24小时”这个怪圈里。有人认为这是历史造成的,也有人表示地中海温暖的气候比欧洲其他地方更容易让遗体腐烂。

20世纪之前,人们普遍认为尸体会带来瘟疫和各种疾病。同样的恐惧在不同的文化之间弥漫,促使发达国家在死者和家属之间建立起一道保护屏障。美国、新西兰和加拿大大力推行化学防腐,巴塞罗那则把遗体用玻璃隔开。

取消“隔离”措施的行动进展缓慢,即使世界卫生组织这样的权威机构已经表明,就算发生大规模死亡事件,“与普遍观点截然相反,没有证据显示尸体存在引发‘疫情’的风险”。

疾病控制中心更是大胆,直截了当地指出“腐烂的模样和气味的确令人不适,但不会对公共健康造成威胁”。

想到这里,我问乔瑟夫,阿尔蒂玛殡仪馆是否允许家属把遗体留在家里并不使用玻璃罩。乔瑟夫坚称很少有家庭提出这种要求,但表示会支持有此需求的家庭,还会派工作人员前往其家中“搭把手”。

我们乘货梯来到楼下的遗体准备室。在西班牙,遗体都是直接送去土葬或者火化,很少接受防腐。不过,阿尔蒂玛殡仪馆仍有一个防腐间,里面只有两张金属操作台。原来,他们只在遗体需要运送至其他城市或境外的情况下才进行防腐。在美国,雄心勃勃的防腐师必须取得殡葬学院的学历和实习证明才能光荣上岗,而在西班牙,你只要接受在职培训即可。阿尔蒂玛殡仪馆以邀请法国著名专家给员工进行防腐培训为荣:“这回我们还请到了给戴安娜王妃防腐的大师!”

遗体准备室里躺着两个一模一样的老妇人:一样的开衫毛衣,一样的十字架项链,一样的木制棺材。两名女性工作人员扶起一个老人,用电吹风机给她吹头发。两名男性工作人员负责另一个老人,此时正在给她的脸和双手搽油。不久之后,他们就要把这两个人送上楼,放入玻璃棺材或“橱窗”中供人瞻仰、悼念。
我问霍尔迪在此之前是否见过这种不带玻璃的遗体。霍尔迪开朗地表示没有,但已经做好了心理准备。“这种展现死亡的方式非常直接,”他说道,“生而为人,我们应该这样做。这给了我们尊严。”

\begin{center}
* * *
\end{center}

胡安负责运营阿尔蒂玛殡仪馆持有的白岩公园公墓,和弟弟乔瑟夫有种“黑白双煞”的既视感。西班牙所有的墓地都属于公墓,但允许阿尔蒂玛殡仪馆这种私人公司在一定期限内承包。电动高尔夫球车在山坡上上上下下,穿梭在墓碑和龛场之间。和很多美国公墓一样,平整的墓碑上堆满了艳丽的鲜花。

不过,这里有一点和美国迥然不同。胡安用对讲机叫来一名守墓人,跟我们一道爬上山坡。山顶上一座坟墓都没有,只有三个密封的洞口。只见守墓人弯腰打开其中一个洞口上沉重的挂锁,接着移走洞口上的金属圆盖。我蹲在他旁边朝里一看,一个深入地底的洞穴随即展现在我的眼前,里面堆满了遗骨和火化后残留的遗骸,都快涌到洞口边缘了。

曾几何时,北美人一度拥有田园牧歌般的公墓,上百具遗体通通堆在公共墓穴里。如今,这却成为这座西班牙公墓的常态。

在白岩公园公墓,死者首先被安葬在地上墓室或全封闭的地下墓穴中。但他们只是“租户”,不是“业主”。这些墓室和墓穴是有租期的,只要租期一到,他们就得离开。

进行安葬之前,死者家属必须租赁至少五年的使用期,让死者有足够的时间腐烂。当腐烂至白骨时,这名死者就会被转移至公共墓穴,给刚死的人腾地方。只有防腐过的遗体不走这个流程(再强调一遍,西班牙很少给尸体防腐),因为防腐过的遗体至少需要20年才会变成白骨。胡安的团队会定期检查遗体的腐烂进度——“啊哦……抱歉兄弟,你还没到时候”。防腐过的遗体将一直待在墓室或墓穴里,直到满足加入“白骨俱乐部”的条件。

这种“循环利用”模式不仅存在于西班牙,大多数欧洲国家也是如此。这让把坟墓视为永恒安息之地的北美人民大为不解。西班牙南部的塞维利亚目前没有可用作坟场的空地,火化率高达80\%(对西班牙来说,这个数字已经很可观了)。这主要是因为政府提供了补贴,火化的自费部分只要60~80欧元。从经济角度来看,死在塞维利亚还是很划算的。

在德国柏林,坟墓的租期一般为20~30年。近年来,墓地的生意不仅面向死人,还面向活人。由于越来越多的人选择火化,一些历史悠久的“老字号”墓地逐渐改造为公园、社区花园,甚至游乐园。这是一个艰难的转变。墓地是拥有文化、历史和社区价值的空间,因此蕴含着巨大的文化功能和修复潜能。国际公共广播电台的一篇报道谈到了这一点:

\begin{quotation}
柏林有一座墓地已将所有墓碑清除,摇身一变,成了当地的社区花园。花园里有一个专为叙利亚难民打造的小型园地,用来种植西红柿、洋葱、薄荷等作物。

墓地入口处的墓碑制作坊,现在已是面向难民的德语培训教室。

“这座墓地曾经是一个安葬亡者的、被遗弃的空间。现在,这里有了美好的用途,能够种植果蔬、培育人才。”改造计划的首席园艺师费德威·塔里凯恩说道。
\end{quotation}

白岩公园公墓也希望能在殡葬服务之外多做一些事情。比如,他们的绿色创新已经斩获了多个奖项。白岩公园公墓全部使用电动汽车,就连请巴塞罗那设计学院的学生设计的银甲虫型灵车也是电力驱动。10公顷的土地上不乏保护松鼠、野猪、蝙蝠等野生动物的措施。为了控制亚洲虎蚊的入侵,他们还专门培育了蝙蝠群落。不过,这种做法让白岩公园公墓招致了一些非议,有报道称他们与蝙蝠、吸血鬼和僵尸为伍。

虽然白岩公园公墓在环保方面做了很多努力,但这里不是自然土葬场所。工作人员把遗体放入木质棺材,然后将其安置在花岗岩制成的墓穴中。一个墓穴通常能堆放六具木棺。这种做法有些令人费解——为什么不直接把遗体埋进土里呢?直接埋入土里能让遗体完全腐烂,不再需要公共墓穴,因此能解放大片土地。“在西班牙,我们一般不这么操作。”胡安回答道。

胡安已经为自己选择了火葬,但他明白其中的矛盾之处。“母亲怀胎九月才诞下一个生命,而工业火化轻轻松松就能摧毁一具肉体。”他思索了一下,“我们也应该让遗体用9个月的时间腐烂。”我小声对霍尔迪说道:“听起来他想要个自然土葬!”

西班牙有着近乎绿色的死亡观。参观白岩公园公墓时,我们穿过了一片树林,里面都是地中海区域的特有树种。白岩公园公墓会给你所在的家庭种一棵树,然后把你家的骨灰分成五堆埋在树的周围,这绝对是字面意义上的“家族树”了。白岩公园公墓是西班牙第一家提供此服务的墓地。

...

白岩公园公墓有两台火化机(也叫“火化仓”),每年火化大约2600具遗体。刚一进操作室,我就惊讶地看见两名西装革履的男士交叉着双臂站在一具挂着十字架吊坠的木质棺材旁,他们身后是已经预热完毕的火化机。“哇,你们在等我们,谢谢!”每次见证火化时,我都很激动,我一直如此。不管已经目睹或亲自操作了多少次,亲临遗体在烈焰中转化的时刻,总能让我感受到一股力量。

胡安带我们简单地参观了一下操作室。里面有一台专门在家庭见证时使用的火化机,已经运转了15个年头。这比美国厂房式火葬场使用的机器高级得多。“炉壁是用意大利大理石做的,底座是巴西花岗岩。”他解释道。

“60\%的家庭都会来见证火化。”胡安说道。听到这儿,我震惊得下巴都要掉到巴西花岗岩上了。

“我没听错吧,60\%?”我感觉有点儿头晕。

这是个庞大的数字,远远高于美国的比例。在美国,很多家庭都不知道有火化见证这项服务。

火化开始之前,胡安把我们带到——重点来了,你们准备好了吗?——操作室外面的三堵玻璃墙之后。这三面墙从地面延伸至天花板,和我在殡仪馆看到的用来隔离遗体的玻璃墙一模一样。“为什么要站在玻璃后面看火化?”我问道。

“这个角度看不见炉里的火焰。”胡安答道。

他是对的。我想尽办法,但无论如何都只能看见火化机的边缘,完全看不到火焰。刚才那两个西装男把棺材推入火化炉,金属炉门关闭后,他们拉上一扇质地上乘的木门,把火化机工业化的外表掩藏了起来。

巴塞罗那是一座“差不多城市”。这里有绿色公墓、动物保护基地和本地特色树种。这里的遗体不用防腐,而且使用木质棺材。差不多算是绿色殡葬的佼佼者,可惜他们要求把木质棺材安置在花岗岩墓穴里。这里有60\%的家庭选择见证火化,殡仪馆还允许家属和死去的至爱待上一整天。在与死者互动方面,这个城市差不多可以说是典范了。然而,他们在遗体瞻仰和火化时,用玻璃墙把家属和死者隔开,把死去的妈妈弄得像个博物馆的展品一样。

我原以为他们使用玻璃的做法能把我衬托得更高尚,但其实不是这样。原因很简单,虽然阿尔蒂玛殡仪馆使用了大理石和玻璃,但他们拥有美国殡葬业梦寐以求的东西——粘在椅子上的屁股。在这里,死者家属的参与度极高。他们可以一整天都待在遗体边上守灵,并且鲜少缺席火化见证仪式:阿尔蒂玛殡仪馆的见证率高达60\%。如此看来,这些玻璃屏障其实拉近了对尸体保持高度警惕的公众和死者的距离,并确保这个距离不至于太近。

火化大概需要90分钟。胡安带霍尔迪来到火化机后方家属止步的区域。胡安打开一扇用铰链拴着的金属窗,让我们一探火化炉里的究竟。只见炉膛顶部猛烈喷射出烈火,一下子吞噬了棺材的上半部分。轮到霍尔迪过来观摩时,我看到他睁大双眼,瞳孔中映射出熊熊火光。

可怜的霍尔迪一直陪我在巴塞罗那游览,不得不与尸体近距离接触了好几次。当我们在市里享受着一顿恨不得有14道菜的饕餮大餐时,我问他对今天的行程有何感受。他想了想,然后回答道:“就像欠账还钱一样。我的公司欠了钱,那我就要还;出去吃顿饭,那我就得结账。生死也是这个道理。当我感到对死亡的恐惧时,我必须提醒自己,是时候还账了,还活着的账。”

\hr 

\section{日本东京}
\hr  

现在是日本早间电视节目Tokudane!的广告时间,几个身穿紫色西装的女人跟着动感的电子乐节拍翩翩起舞,一群动画形象的兔子把一顶假发扣在一个大惊失色的男人头上。广告结束,节目继续播出。在主持人的介绍下,画面上出现了一个身穿白袍、正在寺庙前念经的和尚,他周围堆满了鲜花和香,看样子他正在主持葬礼。

寺庙里挤满了伤心欲绝的悼念者。这时,镜头给了一个远景,葬礼的主角随即出现在屏幕上——原来是19只机器狗,损坏的爪子和断裂的尾巴还得到了几个特写。此时,我正在酒店吃早餐(心形的煎鸡蛋),这一段让我看得入了迷。

1999年,电子巨头索尼公司推出了机器狗Aibo(日语“伙伴”的意思)。这个产品重约3.5磅,能够听从主人的指令并做出反应。Aibo能坐能叫,还能做出小便的动作,着实讨人喜欢。消费者称,这只机器狗让他们不再孤单,并使他们变得更加健康。2006年,索尼宣布Aibo停产,但承诺继续提供维修服务。然而到了2014年,维修服务也停止了,15万只Aibo与它们的主人不得不经历“生离死别”。这个事件催生出一大批机器宠物产品和线上心理疏导小组,最终引发给无法修复的机器狗举行葬礼这一奇观。

节目一结束,被心形的煎鸡蛋喂得饱饱的我径直冲出酒店,去找我在东京的翻译艾米丽(绫子)·佐藤。她建议我跟她在涩谷地铁站的八公像前见面。八公是日本的民间英雄,而且是一只狗(历史上确有此狗)。八公的故事始于20世纪30年代,据说它每天都会在涩谷站接送身为农业教授的主人回家。有一天,教授突发脑出血辞世,再也没有回到车站。然而,八公在之后的九年里依然每天都按时在车站等候,直到最后死去。从跨文化的角度来看,没有比用狗作为集合地点更妥当的了——大家都喜欢忠实的狗狗。

当我抵达八公像时,发现佐藤桑已经等在那里了。佐藤桑是一位快60岁的女性,但看上去还不到40岁。一身干练的西服套装和便于走路的平底鞋让她显得魄力十足。“我的秘诀是每天走1万步。”我跟着她穿梭在迷宫般错综复杂的新宿站地下通道里,好几次我都差点儿迷失方向,只能被衣着考究的东京上班族大军挤着走。“我应该举一个导游领队用的那种小旗子,上面有专门画给你看的骷髅。”佐藤桑微笑着打趣道。

在经过了两个旋转栅栏门、上下了三次台阶、乘坐了四次扶梯之后,我们终于来到了站台。“如果遇到地震,躲在这里会很安全。”佐藤桑说道。佐藤桑提到这个是有缘由的,这一天日本海域发生了一场6.8级地震。我碰到的每一个日本人都会跟我聊起2011年那场地震带给他们的心理影响。在那次灾难中,地震引发的海啸席卷了日本东北部,造成15000多人丧生。

这里的地铁站台用玻璃屏蔽门把乘客和铁轨隔开。“屏蔽门最近才安装上,”佐藤桑解释道,“主要是防止——”她压低声音,“卧轨自杀。”日本是自杀率较高的发达国家之一,佐藤桑继续说道,“结果,清洁工打扫自杀现场的效率越来越高,比如收集尸块之类的。”

在犹太人和基督徒看来——也就是大部分西方人看来——自杀是罪恶的、自私的行为。这个观点很难从大众的脑海中抹去...。

然而,自杀在日本有另一种文化含义。在日本文化中,自杀在某种程度上是一种无私的,甚至高尚的举动。比如,虽是杜撰但尽人皆知的“弃老传说”。讲的是饥荒年间,儿子把年迈的母亲背到树林里遗弃,被遗弃的老妇为了尽到自己的责任,绝不离开树林,自愿冻死或饿死。

局外人认为日本把自杀浪漫化了,称日本盛行“自杀文化”,但事实没有这么简单。我认为,日本人之所以把自我了结视作一种利他行为,是因为不想给别人添麻烦,而不是执迷于死亡本身。此外,作家大原健次郎指出:“外国学者看得懂有关自杀的统计数字,但看不懂数字背后的现象。日本人的自杀只有日本人看得懂。”

就我而言,在观察日本的死亡文化时,感觉就像在照镜子:虽然一切看起来似曾相识,但其实都是失真的镜像。与美国一样,殡葬业在作为发达国家的日本也是一笔大生意:大型殡葬企业占有大量的市场份额,专业的殡仪人士围着老掉牙的设备打转。如果这就是日本殡葬文化的全部,那我也不至于特意来日本走一趟——显而易见,这必然不是全部。

\begin{center}
* * *
\end{center}

幸国寺建于17世纪,如今隐藏在东京市内一条安静的街道上。寺里有一块不起眼的墓地,年代久远的墓碑暗示着几代人都在这里参拜过。一只奶牛猫慵懒地躺在石子铺成的小径上,刚刚还身处现代东京的我们,仿佛一脚踏入了宫崎骏的动画世界。矢岛住持出来迎接我们,他为人随和,身着一袭棕色的袈裟,一头白发剃成圆寸,鼻梁上架着一副眼镜。

矢岛住持极具创新精神,和古色古香的寺院形成了鲜明的对比,尤其是在如何供奉骨灰方面有很多奇思妙想。美国的殡仪人特别害怕美国会出现统一的“火葬文化”,导致防腐业务和棺材销售的利润大幅下降,虽然他们根本不知道什么才是统一的“火葬文化”。而日本人知道。日本的火化率高达99.9\%,全球排名第一,把其他国家和地区远远甩在后面。

全日本只剩下天皇和皇后需要土葬。不过几年前,明仁天皇和皇后美智子宣布死后将选择火葬,打破了日本皇室400年来的土葬传统。

当幸国寺的墓地满员时,矢岛住持本可以再置备一块传统的土葬用地,但他没有这样做,而是在七年前建造了一座名为“琉璃殿”的纳骨堂(用来安置骨灰的独立建筑)。“佛教一直拥有着无与伦比的艺术形态,”他解释道,“把科技引入佛教是再正常不过的事了,我不认为它们存在矛盾。”说罢,他带领我们走进了这栋六边形的建筑。

四周一片漆黑,这时矢岛住持用什么东西碰了一下入口处的键盘。片刻之后,2000尊佛像绽放出蓝色的光芒,从地板到天花板全部都是。“哇噢噢噢!”我和佐藤桑既震惊又兴奋,不约而同地低声感叹。我之前看过琉璃殿的照片,但远没有被360度环绕在闪烁的佛像之中那么震撼。

矢岛住持打开一扇上锁的门,露出佛像后面的空间,里面有600个骨灰盒。“盒上都做好了标记。如果你想找一个叫久保田的人,很快就能找到。”他微笑着说。每一个骨灰盒都对应着墙上的一尊水晶佛像。

来琉璃殿扫墓的家庭首先要在入口处的键盘上输入死者的姓名,或使用装有芯片的智能卡刷卡进入——就像刷卡进地铁站一样。进入大殿之后,所有佛像都亮着蓝光,只有一尊闪烁着白色的光芒——你不用再眯着眼睛苦苦搜寻骨灰盒上的名字,这道白光将引领你找到母亲。

“你们现在看到的都是升级版。”矢岛住持说道,“举个例子,起初我们在大门口只安装了键盘,访客只有先输入死者的姓名才能进入。结果有一天,我看到一个老婆婆花了很长时间才把名字敲进去,之后我们便引入了智能卡。只要她轻轻一刷,立即就能找到死去的家人!”

矢岛住持让我们在大殿中心站好,自己来到控制台前为我们演示。“现在是秋天!”他说道。只见佛像的光芒逐渐从蓝色变为棕黄色,不时出现的几道红色光斑宛如一片片飘落的红叶。“冬天!”灯光随即发出蓝白相间的光芒,象征着冬日的积雪。“流星雨!”一片紫光瞬间袭来,白色的光点在佛像间跳跃,如同定格动画里的夜空。

大部分龛场都失去了创新空间。不管走到哪里,它们看起来都是一个样儿。一排排带有名签的骨灰盒整齐地码放在一望无际的花岗岩骨灰墙上,只能靠摆放的照片、玩具熊、鲜花等物品来彰显个性。

...

“佛教中的来世都是充满了珍宝与光芒的。”矢岛住持告诉我们。

宗教学者约翰·艾什顿和汤姆·怀特把“净土”(东亚佛教概念中的“天国”)描述为“堆满了奇珍异宝,香蕉和棕榈树随处可见,凉爽的清池里长有荷花,野鸟每天要吟唱三遍赞颂佛祖的歌”的地方。

设计琉璃殿时,矢岛住持的想法是创造一个“佛祖眼中的来世”。

琉璃殿的灯光秀经历过几次改动。琉璃殿最早的访客之中有一名灯光师,自告奋勇地承担起设计四季灯光的工作。“看到她的第一版设计时,我以为自己到了拉斯维加斯!”矢岛住持大笑着说,“‘这不是玩具’,我告诉她,看着太夸张了!我们没有采纳那个方案,我希望灯光效果越自然越好。现在,我们仍在不断调整,争取打造出越来越自然的氛围。”

矢岛住持请我们到寺院里喝茶。他递给我一张“外国人专用”的凳子,估计觉得我无法在整个喝茶和谈话期间都能坚持盘腿坐。我告诉他我可以(其实不行,刚坐下三分钟,我的腿就麻了)。

我问矢岛住持为何要把琉璃殿打造成现在这个样子,他慷慨激昂地答道:“我们要行动起来,我们必须做些什么。日本的新生儿越来越少,人们的寿命越来越长。一个家庭需要维护家人的墓碑,但我们没有足够的人手,没法确保每个人的墓碑都能被照顾到。我们必须为那些马上就要无人照料的故人做些事情。”

\begin{center}
* * *
\end{center}
日本国内65岁以上的人口已占全国总人口的1/4。在老龄化和少子化的共同作用下,日本的人口规模在过去五年里下降了近100万。日本女性的平均预期寿命名列全球第一,日本男性位居第三。在“健康期望寿命”(年老但能独立生活)的排名中,日本女性和男性双双拔得世界头筹。随着老龄人口逐渐增多,70岁照顾90岁的现象比比皆是,对护士和护工的需求出现爆发性增长。

我的翻译佐藤桑对此深有体会。她一个人要负责照顾六个老人:自己的父母、丈夫的父母,还有两个叔叔。这六个老人都在八九十岁。几个月前,她的姨外祖母刚刚过世,享年102岁。

这群“白发大军”(或称“银色人才”)工作了一辈子,攒够了存款,没有孩子(有也不多),不愁没有花钱的地方。《华尔街日报》称“日本最热门的商业关键词之一就是‘shukatsu’,即‘生命的终结’,也就是向正在为自己准备后事的人提供临终产品和服务”。

日本殡葬业的利润比2000年增长了3350亿日元(30亿美元)。比如,一家名为“最后的高级成衣”的公司专门为葬礼提供定制款寿衣和遗照摄影服务。

人们通常提前几年来琉璃殿购买属于自己的佛像。矢岛住持鼓励人们经常过来为其他人祈祷,以便更好地面对自己的死亡。“当他们去世后,提前与佛祖相见的人将会迎接他们。”

也有没有为自己提前做好计划的人。他们鲜与亲人往来,死后几周甚至几个月后才被发现,遗体已经在地毯或被褥上留下暗红色的印记。他们就是席卷日本的社会现象“孤独死”的受害者:独居老人死在家中,没有人知道,也没有人给他扫墓。房东甚至会雇用特殊清扫企业来清洁“孤独死”的现场。

矢岛住持说:“建造琉璃殿时,我想起一个没有子嗣的人曾对我说:‘我该怎么办呢?到时有谁会为我祈祷?’”

于是每天清晨,矢岛住持都会在操作台上输入当天的日期。这一天是5月13日,当他按下相应的数字后,几尊佛像发出了黄色的亮光,这些就是在这一天死去的人。矢岛住持点上香,开始为他们诵经祈祷。没有家人记得他们,但他还记得。对于那些没有亲人可以依靠的男男女女来说,琉璃殿的发光佛像就是他们死后的家园。

矢岛不仅是住持,还是一名设计者。“每次祈祷时,我都会冒出各种各样的点子……我不是坐在桌子后做计划的人,我所有的想法都出现在祈祷的时候。”

“如果琉璃殿的骨灰龛位满了呢?”

“那我可能会考虑建造第二个或第三个琉璃殿。”矢岛住持笑着说,“这已经在计划当中了。”
\begin{center}
* * *
\end{center}

20世纪初,日本人普遍把私人火葬场视作藏污纳垢的贼窝(至少媒体是这么认为的)。经常有传言说火化工偷走了死者口中的金牙,或者有陌生人来火葬场盗尸,用偷来的部分做成药物治疗梅毒。那时的火化炉是用木头而非天然气作为燃料,整个过程耗时很久,需要整整一夜。死者家属不得不先回家睡觉,第二天再过来。历史学家安德鲁·伯恩斯坦指出:“为了防止尸体及尸体身上的金牙、首饰、服装被窃,火葬场把火化炉的钥匙交给死者家属,等到领取遗骨和骨灰时再交回。”和巴士站储物柜的原理一样。

相比之下,瑞江殡仪馆显得更为现代化一些。瑞江殡仪馆是一家公立火葬场,建于1938年。他们用汽油做燃料,所有流程一天内就能搞定(而且还不用钥匙)。激进人士认为,应该把火葬场更名为“殡仪中心”,并配套花园般的景观设施,以实现“美学上的管理”。截至目前,瑞江殡仪馆已经运作了80多年,并一直受惠于“美学管理”这个理念。瑞江的建筑群坐落在河堤上,一直向西延伸,南边是花园和游乐场,东边则是两所小学和一所初中。

临海是我拜访的另一家殡仪馆。和瑞江一样,临海也提供全套的殡葬服务。当我到达时,殡仪馆已经为当天的几场葬礼备好了四间礼厅。私人殡仪公司的员工早早就在死者家属抵达之前赶来布置现场,他们带了一大堆装饰品,包括花圈、竹子、盆栽、发光球。社会人类学家铃木光对此的看法是,在现在的日本(和西方),“从准备、安排到操作,商业葬礼的流程全部交由专业人员打理,死者家属只有在付款时才发挥作用”。

一名84岁的受访人告诉铃木光,看到现在的葬礼越来越没有仪式感,他感到痛心疾首。老人抱怨道,20世纪50年代,每家每户都知道应该如何安葬死者,不需要花钱找人替他们操作。“反观现在的年轻人,”老人说道,“像无助的孩子一样,第一件事就是给殡仪公司打电话。过去可没有人这么做,太丢人了。”这时,老人的妻子补充道,最让他们震惊的是,“年轻人从不为此感到羞愧”。总而言之,日本的年轻人不仅对丧葬习俗一无所知,还不把这个问题当回事儿。

而在年轻一代看来,老一辈的传统与迷信无异。老人也提到,每当他缅怀过去的传统时,他的孙女(一名医学院的学生)总是取笑他。“我告诉她,怀孕的女人不能接近死者,猫也不行。据传说,如果有猫跳上死者的脑袋,猫身上的恶灵就会钻入其尸体,让死人复活。为了防止死者变成被妖猫操纵的僵尸,人们尽可能地让猫远离尸体……”

临海的四间礼厅是为四位老婆婆准备的。每一具棺材旁都摆着一个电子相框,里面是老人的遗像。其中一张属于文女士,照片上的她身穿白衬衫,外面套着一件蓝色开衫。

在侧面一间礼厅,田中夫人正躺在薰衣草色的火化纸棺材里。她的遗体没有防腐,为了保持低温,遗体四周摆满了干冰。她的家人围在她身边,一个个都低着头。她的葬礼将于10点开始,一直持续到第二天中午,然后火化。

几名年长的男性聚在吸烟室里抽烟。“我还记得以前没有吸烟室时,”佐藤桑告诉我,“大厅里全是烟味和烧香的香味,混在一起特别难闻。”

葬礼结束后,遗体就要被送至临海内部的火葬场进行火化。这个火葬场由黑色花岗岩打造而成,特别像纽约高级商务楼的大堂,气派极了。如果把美国的火葬场比作一辆破旧的道奇卡车,那么这里简直就是一辆崭新的、闪闪发光的雷克萨斯。10扇银色的大门,10台锃光瓦亮的火化机,机器上一点儿污渍都没有。遗体由不锈钢制成的传送带送至火化机内,比我见过的任何火葬场都干净整洁、有条不紊。

火葬场的门口贴有价目表:火化一个死胎,9000日元;火化遗体的一部分,7500日元;把成人的骨灰分装至若干骨灰盒,2000日元。另一张纸上列有不允许和遗体一起火化的物品清单,包括但不限于手机、高尔夫球、字典、大型毛绒玩具、金属佛像、西瓜。

“等等,西瓜?真的吗?”

“反正上面是这么写的。”佐藤桑耸了耸肩。

这时,抬进了一具遗体,旁边跟着三四名家属,丧主(通常是丈夫或长子)也在其中。他们目送遗体由传送带送进火化机,但没有留下来见证整个火化过程,直接回到了楼下的接待处。火化结束后,他们穿过火葬场来到捡骨专用的三个房间。

一场火化完成后,火化机内通常会留有大量的骨头碎片(虽然是碎片,但能拼成一具完整的骷髅)。在西方,工作人员一般会把这些碎片研磨成粉末,但日本不是。死者家属会被带到捡骨室,与挚爱之人的遗骨会面。

火葬场给死者家属准备了两双筷子:一双竹筷,一双金属筷。丧主首先用竹筷捡起足骨放入骨灰瓮中,然后其他家属仿照丧主的做法,轮流处理剩下的遗骨。但头骨尺寸过大,没法直接放到骨灰瓮中。这时,工作人员就会用金属筷帮助死者家属把头骨夹成碎片。最后一块放入瓮里的是舌骨(下颌骨下方的马蹄形骨头)。

《吞食黑暗的人》是一部出色的非虚构小说,讲述了20世纪90年代两名女性在东京死于谋杀的事件。书中,作者理查德·劳埃德·帕里描述了澳大利亚女子卡莉塔·里奇韦的葬礼。她的父母乘飞机来到日本为她安葬,经历了平生第一次捡骨。

\begin{quotation}
……火葬场位于东京远郊,他们驾车行驶了很久才到达。卡莉塔安详地躺在撒满花瓣的棺材里。他们向她道别,目送她消失在火化机的不锈钢炉门后面。然而,接下来发生的事情出乎所有人的意料,他们被带到火葬场另一侧的一个房间,每个人都领到一副白手套和一双筷子。房间中有一张不锈钢操作台,上面摆放着刚刚从火化炉中取出的卡莉塔的遗骨,火化并没有将一切都化为乌有。虽然木头、衣服、头发、血肉已经全然不见,但腿骨、臂骨和头骨等尺寸较大的骨头都遗留了下来。这些遗骨大多带有裂痕,但形状仍清晰可辨。里奇韦一家没有想到,等待他们的不是已经装盒的骨灰,而是卡莉塔火化后的遗骸。按照日本火葬的传统,这家人需要用筷子把遗骨捡起并放进骨灰瓮里。“罗伯(卡莉塔的男友)根本下不去手。”奈杰尔(卡莉塔的父亲)说道,“当发现我们愿意试着去做时,他像看怪物一样看着我们。我想也许因为我们是她的父母,她是我们的女儿……这件事听起来可能很吓人,但那时我并不这么觉得。当时,我仿佛获得了情感上的支持,内心似乎都变得平静了。我觉得我们是在精心照顾卡莉塔。”
\end{quotation}

里奇韦一家没有捡骨的文化传统,但在他们生命中最艰难的时刻,捡骨是他们能为卡莉塔做的最有意义的事情。

并不是所有的骨头都需要放进骨灰瓮中,不同地区的习俗不同。在某些地方,死者家属可能需要把遗骨分开装在几个小袋子里带回家,也有人把遗骨留在了火葬场。这样的话,工作人员会把遗骨碾碎后放入袋子,然后放置在公众看不到的地方。当装有骨灰的袋子越积越多时,一组专业人员会把它们取走。这些人就是火葬场的骨灰捡收人员。他们把这些骨灰埋在山上的坟坑里,坟坑很大,长10英尺、宽8英尺、深度超过20英尺。铃木光指出,捡收人员会在坟上种植樱花树和针叶树,“这些樱花吸引了大量游客,但很少有人发现美景背后的秘密”。

与以前相比,这种樱花葬要巧妙得多。从前,人们把无人领取的骨灰直接埋在火葬场。随着瑞江殡仪馆这种豪华景观式火葬场的出现,“把骨灰倒在后院”的做法已经过时了。铃木光女士听闻捡收人员有时被称为“废物回收者”,直白地说就是“收垃圾的人”。她认为,这种称呼“把捡收骨灰的工作人员贬低为不用对死者负责的普通劳动力”。殡仪人员之所以是一群“专业人员”,正是因为他们要同死者和家属打交道。

把火化和骨灰捡收分为两个工种,在我看来有些微妙。我当火化工时,两种活儿我都干,没觉得有什么不一样。遗体在火化前是完整的,火化后就变成了骨灰。西方没有捡骨这个环节,因此西方国家的死者家属生怕拿错了骨灰,着魔似的反复问工作人员同一个问题:“骨灰盒里的真是我妈妈吗?”每当火化结束后,我都努力把遗留在炉内的遗骨和灰烬全部捡出来——有些坚硬的骨头碎片会卡在炉壁的裂缝里,然后把它们放进袋子保存。在加利福尼亚州,我们会把这些袋子撒入大海。我既是火化工,又是骨灰捡收人员;我既是一名“专业人员”,又是一个“收垃圾的人”。

\begin{center}
* * *
\end{center}

2010年,加藤宗现111岁的生日就要到了,他将成为东京最长寿的男性。当地政府人员来到他的府上为他庆生,然而加藤的女儿却拒绝让他们进入,声称父亲因试图修炼成为“即身佛”(佛教僧侣把自己变成木乃伊的仪式)而处在植物人的状态。

几次尝试之后,警方终于进入加藤家中,并找到了他的尸体。加藤至少已经死去了30年,早已变成了一具干尸(但身上还穿着内衣裤)。加藤先生的女儿不仅没有把父亲安葬,反而还一直把他锁在一楼的房间里。据加藤先生的孙女称:“妈妈说‘就让他待在那儿吧’,我们就没再管他。”这些年,加藤81岁的女儿总共从父亲的账户里取走了高达10万美元的养老金。

加藤一家的做法实在让人目瞪口呆。不仅因为他们一瞒就是30多年,还因为他们体现了日本人不断转变的“尸体观”。传统观念认为,尸体是不洁净的,家属应该积极地通过仪式将其净化,使其向善,不再具有威胁性。

对现在还活着的人来说,净化活人和净化死人的方法多到数不完。重点包括:喝清酒,与尸体接触之前和之后都要喝;点燃熏香和蜡烛,因为火光可以净化污物;守夜时不要睡觉,以防恶灵趁机钻入尸体;火化结束后,用盐搓手。

到了20世纪中叶,越来越多的人选择在医院临终,而不是在家里。随着丧葬事宜越发频繁地交给殡仪专业人员打理,“尸体是不洁净的”这一想法已经逐渐从日本人的脑海中消失了。火化率从25\%(20世纪初的数据)跃升至近100\%。人们认为经过火化的遗骨不存在让人受到感染的风险。美国也发生了类似的变化,但结果正好相反:美国殡葬业越来越专业的同时,公众对尸体的恐惧程度却达到了顶峰。这着实令人沮丧。再强调一次,这是透过镜子的凝视。

日本的人口第二大城市横滨有一家名为“后旅”的旅馆,由“最后”和“旅馆”两个词组合而成,意思是你这辈子待过的最后一家旅馆……因为入住的时候你已经死了。这是一家专门面向尸体的旅馆。与刻板印象不同,“后旅”的经理鹤雄先生并没有手持烛台带我们在层层蜘蛛网中穿梭。鹤雄先生是一个幽默、开朗的人,对这家旅馆和所提供的服务充满了热情。参观结束后,我悄悄地对着我的录音笔说:“我也要、我也要,我也想要一家尸体旅馆。”

鹤雄先生把我们领到电梯间内。“这个电梯不是给公众准备的,”他向我们致歉,“是让推着轮床的工人用的。”电梯里一尘不染,干净到可以直接在地板上用餐。我们来到六楼。这里有一个冷藏间,最多可以容纳20具遗体。

“我希望这里有其他机构没有的东西。”他告诉我们。这时,一架电动轮床沿着金属轨道向我们滑过来,上面放着一具白色的棺材。只见轮床自动把棺材抬起送到我们跟前。

四周的墙上布满了棺材大小的金属门。“这些门通到哪里?”我问道。

鹤雄先生做了一个“跟我来”的手势。我们跟着他来到一个小房间里,里面有熏香和几张沙发,还有几扇和外面一模一样但外观更加精致的金属门。有一扇门处于敞开的状态,白色的棺材直接滑了进去。

我们又参观了三间家属接待室。死者家属随时可以到这里来(遗体在这里平均储存四天),工作人员会帮你把遗体从冷库中调出。死者一般身穿僧侣式布衣或西装躺在棺材里,外表相对完好(没有防腐)。“也许你因为工作繁忙没能参加葬礼,”鹤雄先生说,“这时,你就可以在下班后来这里跟亲人道别。”

有一间接待室比其他房间都要宽敞,里面有一张舒服的大沙发、一台电视机和大把大把的鲜花。不同于美国的殡仪馆,在“后旅”,你可以舒舒服服地和死者待在一起,没有严格的时间限制。

“这种大房间的价格在10000日元(85美元)以上。”鹤雄先生说道。

“很划算!”我回答道。

没有时间限制,想来就来,不用预约,感觉比“付两个小时的费用,只能在瞻仰室里待两个小时”的西方殡仪馆高尚、开明得多。

“后旅”的九楼有一个干净整洁到闪闪发亮的沐浴间,死者在人世上的“最后一次沐浴”就发生在里面那个宽敞、典雅的操作台上。如今,传统的擦净身体的沐浴仪式已经变成面向死者亲密家人的商业服务,重新回归大众的视野。其中一家引入此仪式的公司的总裁告诉我们:“沐浴仪式能够(有助于)填补当代葬礼造成的心理空虚。”因为短时间内把遗体从死者家属身边带走的操作方法“导致死者家属没有充足的时间消化死亡这一事实”。

我从事殡仪工作时,确实发现清洁遗体并与遗体共处一室的做法能够有效地缓解死者家属的悲伤情绪。在这个过程中,哀悼者不再把遗体视作一件被诅咒的物体,而是曾经载有他们深爱之人的美丽载体。日本知名收纳大师近藤麻理惠在她的超级畅销书《怦然心动的人生整理魔法》中也提到了类似的看法。她的主张是,不要把想丢弃的杂物一股脑儿地扔进垃圾袋中,而是先“认真地感谢每一件物品所提供的服务”。有些人认为向一件破破烂烂的T恤表达感谢很愚蠢,但这种做法背后有着更深层的含义。每一次断舍离都宣告杂物的“死亡”,我们理应对其表达感恩之情,日本人和遗体的关系也是如此。你不只是目送母亲消失在火化机里,还要与她坐在一起,感谢她的遗体,感谢她作为母亲为你所做的一切。只有这样,你才能放手。

鹤雄先生继续带我们参观。我们走在鹅卵石铺成的小路上,但其实我们还在室内,这只是“后旅”的一条走廊,可我感觉自己正漫步在维多利亚时代圣诞节期间的商场里。走廊的尽头有一扇通往“房间”的门,鹤雄先生拿出几双鞋套让我们穿上。

“这里就是举行‘客厅式’家庭葬礼的场地。”他边说边开门。这是一个普通的日式套房(可惜不是走廊里那种维多利亚风格)。

“也就是说,这其实是一个普通的公寓,但没有人住?”我有些不理解。

“不,还是有人住的。守夜的整晚,你都可以和遗体一起待在这儿。”

这个套房设施完善,有微波炉、超大的浴室、沙发等,可以让一家人舒舒服服地度过一晚。房里的日式床垫最多可睡下15人。在横滨这样的大城市,普通人家公寓的空间有限,不足以让从外地赶来参加葬礼的客人过夜。而这里则为这样的家庭提供了一个集体守夜的机会。

这个套房让我百感交集。在美国,有一个让殡仪人员难以启齿的话题:对死者家属而言,瞻仰防腐后的遗体并不是愉快的经历。当然也有特例,但殡仪馆没有给直系亲属足够的时间陪在死者身边(大部分情况都是人刚一死,遗体就被迅速移走)。没等家属处理好自己的情绪,工作人员和远房亲戚就出现在葬礼现场了,家属只得把悲伤和谦逊展现在众人面前。

我时常在想:如果每个大型城市都有“后旅”,那会是怎样一番景象?在这样的空间里,家属只需与死去的亲人待在一起,没有死板、模式化的仪式流程,也不用在正式的瞻仰环节“做戏”。安全、舒适,宛如家中。
\begin{center}
* * *
\end{center}

历史中总有一些超越自身时代的发明创造。20世纪80年代的日本,相机公司的员工上田宏为了在旅行中给自己拍照,发明了历史上第一个“自拍杆”。这项发明于1983年取得专利权,但销量不佳。也许是因为这个功能太不起眼,自拍杆甚至被收录在“没用的发明大全”里(其他的发明还包括给猫穿的拖鞋、安装在筷子上用来给拉面降温的迷你电风扇等)。由于商业上的失败,上田宏持有的专利权于2003年失效。今天,人们一个个像自恋的绝地武士一样,挥舞着自拍杆疯狂自拍。上田宏对此毫不在意,平静地告诉英国广播公司(BBC)的记者:“我们把自拍杆称为‘凌晨3点的发明’,因为对于那个时代而言,它的出现还为时尚早。”

死亡和丧葬的历史中也不乏超越时代的奇思妙想——我管它们叫死神版“凌晨3点的发明”。比如,在19世纪20年代的伦敦,人们迫切地想解决城市墓地过度拥挤、臭气熏天的问题。土里的棺材层层叠叠,足有20英尺深,半腐烂的尸体暴露在众目睽睽之下,因为棺材板都被人偷走卖给穷人当柴火了。过于拥挤的公墓几乎影响到了每一个伦敦人。就像约翰·布莱克伯恩牧师说的:“许多纤弱的灵魂一定倍感不适,因为他们不得不目睹因吸收腐烂人体而变黑的泥土和散落在外的尸体残骸。”是时候尝试些新法子了。

改造伦敦殡葬系统的建议书如雪花般飞来。一位名叫托马斯·威尔逊的建筑师提议,如果土地有限,可以按照金字塔的外观把棺材一层一层地堆放在地上,而不是一味地埋入地下,原材料可采用石砖和花岗岩,最好建在山顶上——就是现在能够让你俯瞰伦敦市中心的樱草花山的位置。按照威尔逊的设计,这个金字塔共有94层,是圣保罗大教堂的四倍高,可以容纳500万具遗体。让我再重复一遍:500万具遗体。

金字塔占地仅18英亩,所容纳的遗体数量却等同于1000英亩土葬用地的容纳量。这座“巨型尸体金字塔”(原名是“大都会遗体安置所”,很酷对不对?)迎合了当时伦敦人对埃及手工艺品和建筑风格的热情。威尔逊甚至被邀请到议会陈述设计方案,然而公众却并不买账。《文学公报》称其为“愚蠢的巨怪”。人们喜欢位于郊外的花园似的墓地——远离市中心拥挤的教堂,扫墓的同时还能野餐。他们不想让一座象征着腐烂的巨型坟堆(金字塔的重量很可能把山压垮)霸占伦敦的“天际线”。

事件最后以遗憾告终。后来,威尔逊的设计思路被一名法国同行盗用。威尔逊控告其剽窃,而这名同行则指控他诽谤。“大都会遗体安置所”会不会就是殡葬业的“自拍杆”,对于当时的伦敦而言过于超前呢?我们在改革殡葬业的过程中实现的每一个飞跃,都有可能出现在“没用的发明大全”里。

从两国地铁站出来后步行5分钟,在国技馆的拐角处坐落着世界上最先进的殡仪机构之一。你趁午休时间跳上地铁,与身穿和服的相扑选手擦肩而过,接着便可抵达兼具寺庙和墓地双重功能的大德院两国陵苑。
大德院两国陵苑看起来一点儿都不像墓地,更像是一栋办公楼,无处不洋溢着大企业的派头。这一点从衣着光鲜的公关部女职员身上就能看得出来。此时,她正在大堂迎接我们。她所任职的觉王山陵苑是日本第三大殡葬公司,但其室内陵园业务在全国排名第一。“我们是室内陵园设施的开拓者,”她介绍道,“是在东京证券交易所上市的唯一的大型殡葬企业。”

我的DIY精神让我更倾向于特立独行的和尚手动点亮佛像这种操作,但我不得不承认觉王山陵苑的市场敏锐度很高。从20世纪80年代开始,东京的土地价格一路飙升。20世纪90年代,一块小型墓地的价格已然高达600万日元(约合53000美元)。这就让价格合理、位置便利(比如,就在地铁站附近)的城市墓地逐渐变得炙手可热。

当然,这里之所以被称为“高科技墓地”,不是因为与地铁挨得近。在机构负责人的带领下,我们首先来到一条长长的走廊。走廊的上方有一片亮白刺眼的射灯,下方铺着漆黑色的地板。地板高度反光,在灯光的照射下闪闪发光。每面墙上都有一些凹进去的独立空间,为了保护隐私,外面还配有透明的绿色玻璃门。整个场面像是复制了20世纪80年代电影里的未来场景,是我喜欢的设计风格。

独立空间中立着传统的花岗岩石碑,石碑下方有一个教科书大小的长方形浅槽,两边码放着插有鲜花的花瓶和熏香。负责人掏出一张智能卡,看起来与琉璃殿用的那种差不多。他仿照访客的样子把卡片贴在感应器上:“这张樱花卡能够识别出对应的骨灰瓮。”说罢,玻璃门随即闭合,把石碑关了起来。

接下来是奇迹发生的时刻。我听到机械手臂作业时发出的响声,说明它正在从4700个骨灰瓮中寻找对应的那罐。一分钟后,玻璃门打开了,石碑又出现在我们眼前。之前空空如也的长方形浅槽中,此时竟多出来一个镶有家纹和名字的骨灰瓮。“我们希望服务到尽可能多的人,这里的存储潜力非常可观。”负责人告诉我们。这个机构可储存7200罐骨灰,目前的存储量已经超过了一半。“如果你葬在自家的墓地,你的家人还得经常给你送花、上香,很辛苦。在这里,这些工作可以全部交给我们。”

那些忙得没时间扫墓的人,现在还可以选择线上虚拟扫墓。东京一家名叫I-Can的公司开发出一款虚拟产品:绿意盎然的电脑画面上是你祖先的坟墓,根据喜好,你可以选择上香、洒水、送花、送水果或送啤酒。

I-Can的总裁说:“当然,最好的方式莫过于亲自给祖先上坟,但我们的服务是面向那些相信在电脑前也能向祖先传达敬意的人。”

大德院两国陵苑的住持增田总是一副淡然、平静的样子,他和矢岛住持一样,全力支持佛教传统与时代创新相结合。大德院两国陵苑是增田住持的寺庙与觉王山陵苑的合作项目。经过多年的规划,双方最终打造出这栋楼宇式墓地,并于2013年正式开业。

“你刚刚在里面参观了一圈,感觉如何?”增田住持用略带自嘲的语气问我。

“这里比美国任何一家殡仪馆的技术都要先进,”我回答道,“而且特别干净。不管是墓地还是火化机,都比美国干净多了,而且没有那么工业化。”

“现在的殡仪服务比以前更加重视卫生。”住持说道,“人们曾经对遗体抱有恐惧,但在我们的努力下,不仅遗体变干净了,墓地也变得像公园一样整洁。”

我和增田住持聊了很多日本和美国当前的殡葬潮流,比如日本人已经不再亲自参与捡骨,而是让工作人员把遗骨碾碎并撒入深山。“从传统意义上来说,日本人很在意白骨,”他解释道,“所以日本有捡骨的风俗。日本人不想要骨灰,想要骨头。”

“那为什么现在人们不再这样做了?”

“当人们看到白骨时,会联想到灵魂和责任。骨头是看得见、摸得着的。”增田住持说道,“当他们把遗骨碾碎时,碾碎的还有他们想忘却的、不愿再思考的事情。”

“你不认为这是件好事吗?”

“我不这么觉得。你可以让殡葬变得更干净、卫生,但大地震和居高不下的自杀率让死亡离人们的生活越来越近。现在的自杀者中,还有不到10岁的儿童。你不得不承认,人们已经开始思考死亡了。”
\begin{center}
* * *
\end{center}

日本人一度认为遗体是肮脏不洁的。现在,他们克服了恐惧,不再把棺材里的遗体当作物品,而是当作一个人——躺在里面的不是受到诅咒的物件,而是自己深爱过的爷爷。日本人尽可能地让仪式围绕着遗体进行,确保家属有足够的时间陪伴在逝者的身边。与此同时,美国人却反其道而行。历史上,我们有过在家中悉心照料遗体的习俗,那时的我们与日本不同,我们并不害怕遗体,并且珍惜与逝者相伴的时刻。然而,当殡葬成为一个专业化的行业之后,我们被告知遗体是肮脏不洁的。近年来,越来越多的人开始害怕遗体,火化率也在逐年升高。

导致这种分歧的另一个原因在于,日本人勇于把科技创新融入葬礼和殡葬活动。琉璃殿中发光的佛像、大德院两国陵苑里的机械手臂,美国没有任何一家殡仪馆可以跟它们媲美。对于我们美国殡仪馆来说,能在线发布讣告,或者在葬礼上使用幻灯片播放照片,就算是引入高科技了。

总而言之,日本的殡葬业足以向西方世界证明,殡仪馆未必只能在技术和与遗体互动之间选择其一,你完全可以在不打破自己底线的情况下,把两种选择都提供给客户。...

\hr  


\section{玻利维亚的拉巴斯}
\hr  
保罗·库多那里斯头戴一顶郊狼皮制成的大帽子,帽子毛茸茸的,上面竖着一对尖尖的狼耳朵,一小串金色念珠悬挂在他黑色的山羊胡上。这身行头让他看起来像是赶去参加皮草展会的成吉思汗。

“我想艾利夫人应该会喜欢这顶郊狼皮帽。”他说道,“有一次,她把自己的猫打扮成了绝地武士。”在保罗看来,这两者之间存在着完美的联系。

艾利夫人的家距离拉巴斯城市公墓的后墙有三个街区。沿着鹅卵石小路一直走,就能看见她那栋有点儿不伦不类的小屋,门口只挂着一条破布单。这条街上的房屋大都一个模样:波纹状的屋顶、木头制成的墙、水泥铺成的地板。但只有艾利夫人的家中有67颗人类头骨。它们每一个都戴着无檐小便帽,随时准备帮助狂热的信徒达成心愿。

艾利夫人家里的这些骷髅头被称为“Ñatitas”,直译过来是“扁鼻头”或“狮子鼻”,对于骷髅而言可以说是非常可爱的昵称了。Ñatitas拥有联结活人和死人的能力。“只有人类的头骨才有资格成为Ñatitas,但并不是所有的人类头骨都能成为Ñatitas。”保罗总结道。

这些骷髅头并非来自艾利夫人的亲朋好友。它们以托梦的形式,在梦中告知艾利夫人自己的存在。艾利夫人走遍了坟场、市集、古迹和医学院,把梦中出现的骷髅头一一收集起来。艾利夫人就像一位守护者,会定期给它们献上供品。作为回报,这些骷髅头要事无巨细地保佑艾利夫人,不管是糖尿病还是债务。

艾利夫人立刻认出了保罗。在过去的11年里,保罗经常来拉巴斯给Ñatitas拍照。

“您的猫在哪儿?”保罗用西班牙语问道。

艾利夫人和保罗有两个超越了文化隔阂的共同爱好:一、他们都喜欢骷髅;二、他们都喜欢打扮自家的猫。保罗拿出手机,给艾利夫人展示他的爱猫芭芭的照片。照片中的芭芭嘴上贴着八字胡,脖子上挂着大金链,头上还有一顶波浪假发。而在另外一组照片中,芭芭换上了护士服,脖子上挂着听诊器,变身为喵界南丁格尔。

“啊啊啊啊!”艾利夫人开心地大叫着。那是只有在碰到了真正的知己时才会发出的声音。

说到骷髅,艾利夫人的Ñatitas都戴着一模一样的淡蓝色小帽子,每一顶都绣有它们各自的名字,就像婴儿室里的小宝宝似的:拉米罗、卡罗塔、何塞、沃尔多……不过,这些都不是它们的本名,而是当它们成为Ñatitas时,艾利夫人给它们起的名字。

每个Ñatitas的性格都不一样,拥有的能力也不一样。比如,祈求身体健康就要找卡利托斯,祈求大学的学业顺利就要找塞西莉亚。这些Ñatitas中有七个婴儿和儿童的头骨,因此如果你有与小孩相关的诉求,就可以找它们。Ñatitas的口中塞满了古柯叶和五颜六色的糖果,其他供品包括鲜花、汽水、完整的西瓜和菠萝,都是两三百名信徒送来的。

有几颗头骨的力量最为强大,可以说是Ñatitas中的重量级成员。头戴警察帽的奥斯卡待在架子最上方,它是艾利夫人18年来收集的第一个Ñatitas。“那时候,我们失去了自己的房子,没有钱,也没有工作。”艾利夫人讲述道,“是奥斯卡帮助我们渡过了难关。”艾利夫人坚信Ñatitas能够让奇迹发生,因为她自己亲身经历过。

另一个力量强大的Ñatitas是桑德拉。原因非常显而易见:在艾利夫人家中,至少有1/4的Ñatitas不是骷髅头,而是干尸状的木乃伊头,而桑德拉是其中最精致的一个。这是我见过的保存最完好的人头之一:双颊丰满,面带笑容。她的面部皮肤坚韧有力,包括嘴唇也是,这让她的嘴角上扬形成了一道微笑。两条花白的辫子垂在头部两侧,鼻子也完好无损(这很少见,应该把桑德拉从“扁鼻头”的类别中去掉)。

保罗走向桑德拉,给她拍照。“啊,请等一下。”艾利夫人看出保罗想拍摄特写,于是从架子上取下桑德拉并摘掉她的帽子,一颗完好无损的头颅呈现在我们面前。这时,艾利夫人让我帮她拿一会儿桑德拉,因为她想腾出手去找一些漂亮点儿的饰品。

“哦哦好,没问题。”我笨手笨脚地接过桑德拉。

我把桑德拉举到眼前,清楚地看到她眼睑上长长的淡色睫毛。如果她是美国医学博物馆或历史博物馆的展品,我和她之间得隔着一道玻璃。但在拉巴斯,没有玻璃,只有我和“唉,可怜的桑德拉”。

艾利夫人走过来,把一顶白色的高礼帽戴在桑德拉头上,保罗举起相机开始拍摄。“好的,让桑德拉更靠近你一些。就这样,别动。”他指挥着,“凯特琳,你能不能笑一笑,别耷拉个脸。”

“我手里的是一颗人头,我不需要一张抱着人头傻笑的照片。”我告诉他。

“桑德拉都比你笑得灿烂。麻烦你稍微努力一下,别那么严肃。”

我把桑德拉放回架子,准备和保罗离开。这时,我注意到门口有一摞崭新的无檐小便帽,颜色是青色的。一个正在等着咨询艾利夫人的女人告诉我们:“噢,它们每个月都要换不同颜色的新帽子。上个月是橙色的,我喜欢这次的青色,它们戴上一定很好看。”

\begin{center}
* * *
\end{center}

虽然艾利夫人拥有的Ñatitas数量惊人,但最知名的Ñatitas属于安娜夫人。事先声明,那天我没能见到安娜夫人。我们到访的时候,屋子里挤满了找她咨询的人,全围在一口大铁锅旁边等候。安娜夫人的Ñatitas在梦中和她交流,根据你的问题,她会告诉你应该去参拜哪个Ñatitas。

安娜夫人一共有24个Ñatitas,每一个都放在镶满亮片的垫子上,再装入玻璃箱中。它们头戴帽檐上装饰着花朵的旅行帽,眼窝里塞着棉花球,牙齿上还粘着一条一条的锡纸,看着跟金属做成的护牙托似的。

“那些锡纸是干什么用的?”我问保罗。

“保护它们的牙齿不受香烟的损害。”

“它们抽烟?”

“为什么不呢?”

总的说来,罗马天主教会非常介意拉巴斯的Ñatitas崇拜。从前,主持年度Ñatitas盛会的牧师曾公然向寻求祝福的教众宣布,“头骨必须掩埋”,并且“不应该被崇拜”。

保罗第一次来拉巴斯拍摄这场年度盛会时,跟着人们来到公墓前的教堂庆祝。结果发现教堂的大门紧闭,门口的布告写着教会拒绝为头骨祝福。愤怒的人们上街游行抗议,手上高举自家的Ñatitas并高喊“我们要祝福”,教堂不得不开了门。

拉巴斯的大主教艾德蒙多·阿巴斯托夫勒尤其对Ñatitas不满。“他当然不满意了,”保罗用嘲弄的口吻说,“Ñatitas把他搞得很难堪,让他看上去像是连自己的教区都管不好。”

在天主教会看来,安娜夫人和艾利夫人这样的女性是一种威胁。她们通过魔法、信仰和Ñatitas直接与冥冥之中的神秘力量联结在一起,完全不需要男性介入。这让我想到了墨西哥的Santa Muerte,即“死亡圣女”——一个毋庸置疑的女性死神。她手持镰刀,身披色彩鲜艳的长袍,长袍之下是已经化作白骨的躯体。

更让教会大为恼火的是,“死亡圣女”的追随者已经从拥有上百万信徒的墨西哥一路北上,扩散到了美国西南部。人们相信她是亡命之徒、穷人、性少数群体、罪犯,以及所有被天主教会所遗弃之人的庇护者。

...

\begin{center}
* * *
\end{center}

某一天吃晚饭时,我见到了安德烈兹·贝多亚。他是保罗在拉巴斯的朋友,一名艺术家。他提醒我:“不要认为玻利维亚是一个单一文化的国家。”他的最新作品是一系列使用皮革、铆钉和无数亮片制成的寿衣,全部是手工制作,每件耗时5个月。“玻利维亚的手工艺者总是被人瞧不起,仿佛他们的作品算不上‘真正’的艺术。在我看来,它们当然是真正的艺术品,我从中得到了很多启发。”

这些寿衣大多收藏在博物馆和画廊。在“灵魂霓裳”系列中,安德烈兹仪式化了自己和他人的悲伤。他不介意让真正的死人身穿自己的作品下葬,但他现在还没这么做。虽然玻利维亚不是单一文化的国家,但拉巴斯及周边地区仍按照统一的模式举办葬礼。人们通常会在家中或殡仪馆内正式举行一场持续一整天的守灵活动,死者家属会聘请当地的殡仪人员把棺材送至现场,并使用发出紫色光芒的十字架和花朵霓虹灯作为装饰(在玻利维亚,紫色象征死亡)。“有人觉得紫色的霓虹灯过于俗气,但我挺喜欢。”安德烈兹说道。遗体下葬会安排在第二天,棺材首先要挂在灵车后方行驶一个街区,再装回车里前往墓地。

安德烈兹的母亲在22年前离世,火化是她的遗愿。在拉巴斯,越来越多的人选择火化。然而,当地的火化效率直到最近才有所提高。拉巴斯海拔12000英尺,是世界上位置最高的首都。“这里的氧气含量不够,火化炉的温度一直上不去。”安德烈兹解释道。好在现在的火化机达到了所需温度,可以让遗体充分火化。

既然火化技术有所提高,安德烈兹就决定挖出母亲的遗体,实现她的愿望。但是,墓地需要安德烈兹在开棺后亲自辨认遗体。“要我辨认也可以,我还记得她下葬时穿的那身衣服。可我不想看到她成为白骨后的模样,我不想让这个画面伴随我一辈子。”安德烈兹说道。

对死亡的兴趣让安德烈兹开始探索Ñatitas文化。11月8日是“Ñatitas嘉年华日”。每年的这一天,Ñatitas的拥有者都会把自家的Ñatitas放在外面进行展示。嘉年华日的主角不是拥有者,而是这些头骨,因为节日的目的是感谢它们这一年来的工作。“一些浪漫主义者认为这个节日应该保持原样。但这样的话,你我就没有亲身体验的机会了。”安德烈兹指出。

“虽然‘Ñatitas嘉年华日’在全球范围内还没有什么名气,但在玻利维亚已经是一个颇为主流的节日了。”他告诉我。庆祝活动在城市公墓举行,这里曾经是有钱人专用的墓地,现在搬到了城市南部。目前,市政府正在对城市公墓进行改造,希望以此促进当地旅游业的发展。街头艺术家受邀在墓室外墙创作彩绘,剧团在圣灵节当晚进行现场演出,到场的观众多达几千人。

拉巴斯的Ñatitas文化源自玻利维亚第二大原住民族群艾玛拉人。艾玛拉人一直饱受歧视。20世纪90年代末,被蔑称为“Cholitas”的艾玛拉族女性不得进入某些政府办公场所和餐厅,也不能乘坐公交车。“让我直说吧,对女性而言,玻利维亚不是一个安全的国度,起码在某一时期是这样。”安德烈兹说道,“我们是南美洲最穷的国家。我们有一个词叫‘feminicidio’,专指对女人的谋杀。受害者之所以被杀,就因为生来是女性。凶手大多是死者的伴侣。”

不过在最近10年,少数民族的待遇得到了明显改善。这主要是因为玻利维亚总统埃沃·莫拉莱斯本身就是艾玛拉人,民族平等是他执政期间的重要纲领之一。现在,艾玛拉族女性夺回了自己的身份和地位,其中就包括自由穿戴传统民族服饰的权利——多层衬裙、羊毛披肩,还有高高的圆顶帽。她们也重返社会生活,从事记者、公务员等工作,不再像从前那样只能当女仆。每次在公墓举行的嘉年华日庆祝活动一结束,艾玛拉族女性就会在街上跳起民族舞蹈,边跳边赶去参加下一个派对。“去年,她们给原本象征‘屈服’的服装印上了迷彩花纹,就是军装上的那种,可把男人们气坏了。”安德烈兹大笑道,他当时给姑娘们拍了不少照片,“不要以为民族舞蹈已经成了老古董。在拉巴斯,它依然是一种现代的、不断进化的艺术形式。”

虽然越来越多的人开始接受艾玛拉人和Ñatitas,但被问及是否会在自己家中供奉Ñatitas,是否相信Ñatitas的力量时,许多玻利维亚人都会回答:“哦不不不,我才不会呢,那东西太吓人了!”这是因为他们不想被看成不安分的天主教徒。从这个角度来说,Ñatitas崇拜仍属于地下行为。因此,实际持有Ñatitas的玻利维亚人(其中甚至包括按摩治疗师、银行家等专业人才)的数量远比公开承认的人数要多。

“Ñatitas的拥有者貌似都信奉天主教。”保罗插话道,“我拍摄过的那些拥有Ñatitas的人,家里就没有不挂耶稣和圣母玛利亚像的。”

“这正是玻利维亚的古怪之处。”安德烈兹说,“最近,我和一个朋友谈起,我们根本没有把天主教和本土信仰‘结合’在一起——它俩只不过粘在一块儿了。”他边说边把两只手翻过来,手背贴着手背,做出一个古怪的手势,“我姐姐的公司至今都会请‘治愈者’来净化办公室。我父亲是一名地理学家,我小时候经常跟他一起去矿山。有一次,我看见矿工杀了一头美洲驼献祭,因为他们希望取得冥界统治者迪尤的欢心。这种魔法似的仪式在玻利维亚随处可见。”

\begin{center}
* * *
\end{center}

11月8日清晨,希梅娜把印着米老鼠和唐老鸭踢足球图案的提包放在城市公墓教堂门口的水泥地上,一个接一个地从包里拿出了四个Ñatitas在木凳上摆好。我请她向我介绍一下它们。她的第一个Ñatitas是卢卡斯舅舅——我之前提到过,Ñatitas通常都是陌生人的头骨,但有时也可以用家里人的。“因为有它在,我们家才没有被盗。”希梅娜解释道。

希梅娜给每一个Ñatitas都戴上了毛线帽和花环。她已经连续好几年带着它们参加“Ñatitas嘉年华日”了。“你把它们带到这里,是为了表示感谢吗?”我问道。

“可以这么说。但其实这是它们的节日,我们实际上是为它们庆祝。”她纠正我。

就在我们谈话期间,教堂的门打开了。手里举着Ñatitas的人群蜂拥而入,想方设法要挤到圣坛前面去。新人在这个时候会停下脚步,坐在后方的长椅上等待时机。而经验丰富的年长女性则毫不顾忌地推搡着穿过人群,协助亲友把Ñatitas堆在圣坛前方。

圣坛左边有一具装在玻璃罩内的真人大小的耶稣受难像,雕像的额头和脸颊上都是血,从紫色布料下伸出的双脚也是血淋淋的。一个把Ñatitas装在巧克力威化饼干纸盒里的女人走到雕像旁,用手在胸前画了一个十字,然后重新挤入圣坛前的人群中。

今天这名牧师的发言没有提及天主教会和Ñatitas信徒之间的争执,言语间多了一些和解的意味,着实令人意外。“当你拥有了信仰,”他说道,“你不必向任何人解释。我们每个人的人生故事都不尽相同。从某种角度来说,今天更像是一个生日聚会。很高兴我们能在今天相聚,共同分享这小小的幸福。”

一个挤在我身边的年轻女子向我解释了牧师接纳Ñatitas的原因:“现在,嘉年华日的影响力太大了,天主教会不得不服软。”

人们抱着Ñatitas来到教堂的两个侧门,圣水就盛放在门口旁的颜料罐里。塑料花做成的洒水器朝每个路过的Ñatitas喷洒圣水。有的Ñatitas戴着墨镜,有的戴着皇冠,有的只有一个纸盒子,一个女人把婴儿头骨做成的Ñatitas放在针织的饭盒保温袋里。每一个Ñatitas都得到了祝福。

\begin{center}
* * *
\end{center}

玻利维亚不是世界上唯一相信骷髅头能联结人和神的国家。虽然教会极度蔑视这种信仰,但讽刺的是,欧洲天主教用圣徒的遗物和遗骨作为与神沟通的媒介已经超过1000年。几年前,我在意大利那不勒斯旅行时,见到过与Ñatitas类似的习俗。

“您是英国人吗?”当地的出租车司机问我。

“接近了。”

“荷兰人?”

“我是美国人。”

“啊,美国人!您要去哪儿?”

“囟\footnote{xìn 囟门:婴儿头顶骨未合缝的地方。}门公墓……”我不得不掏出皱巴巴的行程单,“梅特德的囟门公墓。”

我从后视镜中看到司机眉头一皱。

“公墓?那个埋死人的地方?不不不,您还是别去了。”

“不去了?”我问道,“你是说他们今天不开门?”

“您是个漂亮姑娘。您是来度假的,对吧?漂亮姑娘不应该去墓地,那里不适合您。我带您去海边吧。那不勒斯有很多美丽的沙滩,您想去哪个?”

“说实话,我不是沙滩型的。”我试图向他解释。

“难道您真是墓地型的?”司机反问道。

既然说到这儿了,那我就承认了,只要墓地型指的不是死人就行。

“谢谢你这么说,伙计。不过,还是请你把我带到囟门公墓。”

司机耸了耸肩,带着我飞速行驶在那不勒斯蜿蜒崎岖的山间小道上。

把囟门公墓称作“公墓”其实不太恰当,因为这儿就是一个巨大的白色洞穴,或者确切来说,这里更像是个凝灰岩采石场(凝灰岩就是火山碎屑形成的岩石)。几个世纪以来,这个凝灰岩洞穴都是用来埋葬那不勒斯的穷人和无名氏的,大部分都是17世纪大瘟疫和18世纪中期霍乱肆虐时死亡的人。

1872年,加埃塔诺·巴尔巴蒂神父带领其教区的人对公墓进行了整理,把尸骨分门别类地规整好并贴上了标签。志愿者纷纷从那不勒斯赶来帮忙,像所有虔诚的天主教徒一样,他们一边把头骨和股骨一摞摞地靠墙堆起,一边为这些无名氏祈祷。然而问题来了,祷告并没有就此结束。

当地突然出现了对骷髅头的狂热崇拜。那不勒斯人涌入囟门公墓去看望那些“可怜的小家伙”。他们会“认领”某个头骨,给它洗澡、建圣坛、送贡品,向它许愿。人们还会根据梦中出现的名字给它们重新取名。

天主教会非常生气,于1969年关闭了公墓。当时的那不勒斯大主教把这种“死人崇拜”裁定为“任性、迷信”的行为。天主教允许你为炼狱中的灵魂(比如,那些无名尸)祈祷,但不承认它们有帮助活人实现愿望的超能力。那不勒斯的活人明显不这么认为。

...

2010年,囟门公墓重新开业。天主教会自此一直保持着高度警惕,但“死人崇拜”的现象并没有消失。在一片白骨的海洋中,几抹鲜艳的色彩颇为醒目。那是五彩斑斓的塑料念珠、红色的玻璃烛台、金色的硬币、祷告卡、塑料耶稣像等,有的遗骨周围还撒着彩票。看来新一代的信徒已经找到了属于自己的那个法力超群的“可怜的小家伙”。

\begin{center}
* * *
\end{center}

“Ñatitas嘉年华日”的庆祝活动在中午11点达到了高潮。“城市公墓”里摆满了受到祝福的Ñatitas,人们现在要给它们献上古柯叶和花瓣作为贡品。在公墓入口处巡逻的警察逐一检查包裹,看是否有人携带酒水。既然没有酒喝,Ñatitas只得沉迷于其他恶习,比如吸烟。有的Ñatitas的牙齿都被烟油染黄了。

“你觉得它们喜欢抽烟吗?”我问保罗。

“当然喜欢了,这多明显。”头戴郊狼皮帽的保罗不屑地说,随后消失在人群里。

现场有人用手风琴、吉他和木鼓伴奏。一个女人在这刺耳的音乐声中翩翩起舞,边扭屁股边把Ñatitas抛向空中。今天是这些骷髅头的节日,人们要为它们庆祝。

一个人和他父亲的头骨坐在一起,他的父亲原本埋在城市公墓里。这使我不得不陷入沉思:既然父亲已经下葬了,那他究竟是怎么拿到他老子的头骨的?为什么还要给老人家戴上金属框眼镜和用七朵鲜花做成的花环?

我在公墓里溜达时,看到一些空坟周围散落着碎玻璃和水泥块。这些墓碑上都贴着黄色的纸,上面写着:“最后警告!陵墓,1月4日。敬告死者家属某某某(此处输入姓名)……”

接下来的内容是,因为家属没有续交坟墓的租金,老爸的遗体只能被移走。不知道老爸的白骨是去了集体坟坑,还是回家当上了Ñatitas。

...

\hr  

\section{美国加利福尼亚州的约书亚树}
\hr 

看遍全世界的尸体之后,你会发现你最在意的还是自己后院里的那些。现在,我终于回到了自己在洛杉矶的殡仪馆,回到了已经等候多时的总监安珀身边。当我在玻利维亚祈求Ñatitas保佑我顺利筹到共同资金时,她正被火化和伤心欲绝的死者家属折腾得团团转。

根据日程安排,我们LA殡仪馆不久之后就要为谢泼德夫人举办无须防腐的自然土葬了。旅行中的所见所闻让我对这个葬礼产生了新的想法,我想象着她的家人心怀满满的爱意,亲手给她穿上绣有孔雀羽毛和棕榈叶图案的手工寿衣。等到黎明时分,我们就把她带到坟墓旁。一路上,参加葬礼的人们手持点亮的蜡烛,边撒花瓣边吟唱经文。

然而,这次的葬礼和我想象中的完全不一样。

等谢泼德夫人终于躺在我们遗体准备室的操作台上时,她已经死了六周。这段时间,她一直裹着塑料袋待在洛杉矶法医办公室的冷柜里。我和安珀分别站在她的两侧,一起拉开尸袋上的拉锁。她已经发了霉,霉菌从眼睛下方一直延伸至脖子和肩膀。胃部塌陷并呈现出深蓝色(红细胞腐烂后造成的),腿肚上的表层皮肤也已向外绽开。由于长期浸泡在血液和尸液中,她的尸袋几乎变成了一个沼泽。

我们把谢泼德夫人从塑料袋里解脱出来,开始进行清洗。肥皂水顺着金属操作台的边缘,消失在她脚旁的排水孔中。安珀负责洗头,谢泼德夫人原本的白发已经被血水染成了棕色。头皮也发了霉,必须下点儿功夫才能搞定。我俩一言不发地工作着,这具遗体的腐烂程度让我们没法像平常那样说个不停。我们把她擦干之后,发现仍有腐液从她的身体里流出。如果我们LA殡仪馆只是一家普通的殡仪馆,那我们的花招可多着呢(保鲜膜、尿布、化学粉末,甚至从头到脚的塑料连体衣),不愁对付不了这种恰如其名的“渗漏”现象。但是,自然土葬场拒绝接收经过上述方式处理的遗体。

我们只好直接给谢泼德夫人穿上寿衣。为了防止腐液渗透出来,我们给她裹了厚厚的好几层。寿衣是安珀用未漂白的棉布缝制而成的。这家人没什么钱,所以我们尽量帮助他们减少费用。前天,我收到安珀发来的信息,是一张乔安娜布料商店的收据照片。安珀在图片下方写道:“猜猜是谁用商店的积分帮客户省了40\%!”寿衣的成品很漂亮,领结和绑带一应俱全(但没有孔雀羽毛和棕榈树叶)。

我们把穿好寿衣的谢泼德夫人抬上货车,开始了两个半小时的公路之旅。我们从洛杉矶东部出发,穿过内陆帝国(不要被这托尔金式的名字骗了,这里就是一片郊区),最后来到莫哈韦沙漠。要想知道自己是否快接近沙漠区域,靠的不是地貌变化,而是赌场用来宣传现场演出(启用的都是过气明星,比如迈克尔·波顿和卢达·克里斯)的广告牌,之后你才真正进入沙漠。沙漠中,约书亚树(短叶丝兰)尖锐的树枝高高地伸向天空,奇特的模样如同苏斯博士笔下的角色。

约书亚树纪念公园原本不是自然土葬场所,和其他墓园一样,只有部分土地是用来进行自然土葬的。由于位置偏远,洛杉矶人一般不会选择把家人埋在这里,我们洛杉矶人喜欢让死去的亲人离家近一点儿。那么,我们有哪些选择呢?森林草坪公墓是洛杉矶最知名的墓地之一,很多明星都长眠于此。可这个公墓坚持使用水泥墓穴安置棺材,而且还不提供自然土葬服务。但对犹太人和穆斯林除外,因为这两个宗教的教义要求遗体进行自然土葬。基于此,森林草坪公墓会在水泥墓穴顶端凿出几个洞,让泥土象征性地落在遗体上。

圣莫妮卡的林地公墓最近刚开辟出一块自然土葬用地。不过,除了购买“坟坑”的费用,你还需要交纳几千美元的“绿色附加费”,哪怕自然土葬的流程比其他殡葬方式都简单。

约书亚树纪念公园的自然土葬场于2010年开业。场地周围有短木桩做成的围栏,里面可以容纳60具遗体,现在已经有40具了。在周边广袤沙漠的衬托下,这个场地显得尤为渺小,进一步印证了如今的殡葬政策是多么荒谬。曾几何时,整个世界都可以用来安葬遗体。农场、牧场、地方教堂——只要我们想,就没有不能下葬的地方。目前,有些州仍然允许私人土地埋葬遗体,但在加利福尼亚州不行。于是,我们的死者就只能待在沙漠中的坟坑里。

我曾经告诉增田住持,美国的火化率之所以不断上升,部分原因在于美国人担心土地不够用。增田住持无法理解这个动机:“在我这个日本人看来,美国是个面积庞大的国家,到处都是大片大片的土地。想修建大型墓地的话,应该不是件难事。”

有些人对“绿色殡葬”的理解还停留在字面上:青翠的山峦、茂密的树林,死者就安葬在嫩绿的柳树下。而约书亚树纪念公园实为一片荒凉之地,矮壮的仙人掌、木馏油小灌木、球形锦葵从沙砾中破土而出——完全算不上奇妙的极乐世界。

但是,沙漠不断孕育出叛逆的狂野之心。乡村歌手格兰·帕森斯死于吸毒过量时才26岁。当时,他在约书亚树区的旅馆房间内吸食了大量海洛因、吗啡、酒精的混合物。他那性格古怪(据说是)的继父误以为谁得到遗体谁就能掌控死者的遗产,因此强烈要求把帕森斯的遗体运回新奥尔良。

帕森斯的好友菲尔·考夫曼却另有计划。帕森斯生前和他约定,如果两个人中有一个人死了,另一个人要把遗体带到约书亚树纪念公园,敬上死者几杯,然后放火把遗体烧掉。

用酒精壮胆之后,考夫曼和另一个朋友成功地在洛杉矶国际机场拦下了帕森斯的棺材。眼看棺材就要被装上飞往新奥尔良的航班时,考夫曼告诉工作人员帕森斯一家改变了主意。于是,在机场警察和航空公司人员的帮助下,他们把棺材送上了考夫曼那辆权当是灵车的破车(没有牌照,窗户也碎了,车里都是酒)。他们急速行驶,棺材在后座上随着颠簸“啪啪”作响。

抵达约书亚树纪念公园后,二人把车停在自然形成的巨石“帽檐石”下方。他们抬出帕森斯的棺材,浇上了汽油。点火的一刹那,一团巨大的火球径直冲向夜空。

两个人仓皇而逃。只在表面浇上汽油是无法让遗体完全火化的。不久之后,人们就发现了帕森斯半焦的尸体。虽然做了这么多蠢事,考夫曼和他的同伙却只因盗窃棺材这种小罪被起诉(注意,不是因为盗窃尸体)。帕森斯剩下的遗体被送往新奥尔良埋葬——他的继父没有得到他的财产。

好在谢泼德夫人没留下“先喝上几杯,再把遗体烧掉”这种遗嘱。她生前是一名自由派人士,一生都致力于环保事业。因此,她的家人坚信给遗体防腐和使用金属棺材是对她的背叛。

浑身文身的托尼是约书亚树纪念公园的员工。他趁着毒辣的太阳还没升起,一大早就徒手挖好了一个4英尺深的坟坑。小山似的花岗岩沙土堆在坟坑旁,坑口处横着放了几块普通的长条木板。

我们亲手把谢泼德夫人的遗体抬过来,放在悬在坑口的木条板上。在寿衣的勾勒下,遗体的轮廓清晰可见。这真是一场朴实无华的葬礼,让我想到从前,当这里还是狂野的西部时,一把铁锹、几块木头、一条裹尸布和一个死去的男人或女人就是葬礼的全部。这时,三名墓地工作人员过来用绳子把遗体绑好。我蹲在旁边,帮他们把木条板从谢泼德夫人身下抽出。他们慢慢地把遗体放入坟坑,托尼站在一旁看着,确保谢泼德夫人安全着陆。

沉默片刻之后,三个工作人员拿起铁锹和耙子,把沙土盖在谢泼德夫人身上。为了防止郊狼损坏遗体,他们中途还在坑里铺上了一层石块(这个做法更像是迷信,目前没有证据证明自然土葬场所对食腐动物具有吸引力)。三个人花了10分钟把坟坑填好。在其他墓园,整个下葬过程不仅破坏了草坪,还打乱了原本对称的绿色景观——光秃秃的石碑在绿地中着实显眼。而当托尼和他的同事完工后,你根本看不出地上哪里有坟坑。谢泼德夫人就这样消失在无尽的沙漠中。

\begin{center}
* * *
\end{center}

这就是我理想中的死亡:消失。如果我足够幸运,我也想被大地吞噬,就像谢泼德夫人那样。但这不是我的首选。

\begin{quotation}
两分钟之后,他们带着空棺材和白布又回到了这里。他们刚关上门,十几只秃鹫就朝尸体俯冲下来。很快,其他秃鹫也纷纷加入它们的行列。只过了五分钟,我们就看到这群猛禽心满意足地飞回矮墙,懒洋洋地站在上面歇息。它们什么都没留下,除了一具白骨。
\end{quotation}

1876年,伦敦《时报》报道了发生在达克马的这一幕(达克马在西方有一个恶兆般的译名——沉默之塔)。那一天,成群的秃鹫大快朵颐,只用了几分钟就把一整具人类的尸体变成了白骨。这是帕西人(古代伊朗琐罗亚斯德教徒的后代)最渴望的处理遗体的方式。在他们的教义中,水、火、土等元素是神圣的,不能被肮脏的尸体玷污。因此,帕西人鲜少选择火葬和土葬。

帕西人在13世纪建造了第一座“沉默之塔”,目前有三座矗立在孟买富人区旁的山丘上。“沉默之塔”是一个圆柱形露天建筑,由石砖建造而成,内部设有多层环形天台。每年约有800具遗体安置在塔内,最外层的天台用来盛放男性,中间一层是女性,最里面的留给儿童。(秃鹫吃剩的)遗骨经收集后统一堆积在塔里的中心区域。在这里,遗骨将慢慢地在土壤里腐烂、分解。

帕西人的葬礼仪式较为繁复。遗体先在牛尿中浸泡,再由家人和塔里的侍从一同洗净。然后,是持续一整夜的诵经、燃烧圣火、守夜、祷告等活动,之后遗体才被带到塔中。

然而近些年,这个古老的仪式遇到了阻碍。印度秃鹫的数量曾高达4亿只,在1876年,没有比让秃鹫吃掉尸体再平常不过的了。“那时候,秃鹫都守在‘沉默之塔’上等着遗体到来。”研究琐罗亚斯德教的哈佛讲师尤涵·维瓦纳指出,“现在一只也没有了。”

没有火的火化很难,没有秃鹫的“鸟葬”更难。印度秃鹫的数量比从前下降了99\%。这是因为在20世纪90年代初,印度政府同意使用双氯芬酸治疗病牛。牛的病是治好了,但当它们死后,前来觅食的秃鹫因尸体中的双氯芬酸而胃衰竭。这种拥有“铁胃”、能够在烈日下吞食腐肉的猛禽竟然败给了类似于布洛芬的玩意儿,实在是不公平。

没有了秃鹫,遗体只能在“沉默之塔”上干躺着。整个街区都能闻到遗体腐烂的味道。2005年,丹·巴利亚把死去的母亲送上“沉默之塔”。一名工作人员告诉她,塔里的遗体几乎都呈半腐烂状态,因为一只秃鹫都没来过。巴利亚让一名摄影师悄悄潜入塔里偷拍,拍出的照片(照片上,遗体确实以半腐烂之姿躺在天台上)在帕西人之间引发了一场丑闻。

为了克服秃鹫稀少的问题,“沉默之塔”的工作人员想尽了办法。他们在天台上立起了大镜子,想通过镜面把太阳热能集中在遗体身上——跟9岁的小孩用放大镜烧虫子一个道理。但在多云的季风季节,这个办法就没用了。他们又试着往遗体上泼洒腐蚀性化学品,结果把现场弄得一团糟。丹·巴利亚和其他死者家属质问同族人为何不考虑改变传统,尝试使用土葬或火葬,让她的母亲和其他人不至于在冰冷的石头上烂掉。可惜牧师很顽固:不管有没有秃鹫,“沉默之塔”的操作是不会改变的。

这真是莫大的讽刺。在美国,有人非常希望把自己的遗体交给动物处理——事实上,我们有足够的秃鹫和其他类型的食腐动物来满足人们的这个愿望。但政府和宗教领袖等人永远不会允许这种“野蛮行径”在美国的国土上出现,他们只会告诉我们:“火葬和土葬,你只能二选一。”

越来越多的帕西人因家人的遗体没有被妥善对待而愤怒。他们将和丹·巴利亚一起探寻土葬的可能,但他们的族长只会告诉他们:“秃鹫是你唯一的选择。”

\begin{center}
* * *
\end{center}
我在人生的头30年里一直以动物为食。所以,为什么不让动物在我死后把我吃了呢?难道我不是动物吗?

但我的愿望很有可能实现不了。我不得不承认,考虑到社会变革,我也许永远也没有机会选择“鸟葬”了。更糟的是,我可能一辈子都没法去现场见证整个过程。

\hr


\section{编者按}
如何对待自己的尸体这个问题是荒诞的、无谓的。即使是古代贤明的君王也知道在墓穴上占用太多的土地,动用过多的人力和耗费过多的财物是一件可羞的事情,更何况是现代人乎。人死了世间的一切都和他无关了,这当然也包括自己的尸体。在这个问题上进行太多的思考是可笑的,但如何对待他人的遗体那就是另外一回事了。葬礼,就是活着的人的事,一个简洁的定义如下:葬礼是活着的人如何处理他人遗体的事。

任何社会问题都应当从三个角度来思考:一是个人的自由角度;二是社会整体运行的规范角度;三是弱势群体的救助角度。所谓兼听则明偏信则暗,不从多个角度思考问题,而只是从自我喜好的角度出发发表观点则总是偏颇的。

本文作者对于世界各地葬礼的客观观察,人情文字都是很有价值的。对于现有葬礼方式其中可能的该地区的土地、人口、经济、文化传统等客观现实的讨论也是可以参考借鉴的。其他观点文字则在个人自由,也就是葬礼自由和一些微小创新喜好角度上着墨甚多,我适当删去一部分了。作者整本书最大的一个诉求就是现代葬礼被商业化氛围包围之后,留给亲人悼念的时间和空间没有很好地被照顾,现代葬礼的相关商业机构是应该听取下这样的建议,看看有什么改进的地方。

从社会整体运行规范角度,作者更多的只是从自己熟悉的殡葬业出发做出讨论,而葬礼相关事宜实际上还牵涉到一个地区内刑事部门、医疗部门等等,作者则完全没有提及,这其中值得读者进行更多的思考。至于一些关于环保的讨论我则删除了,我认为当下任何葬礼方式都还不牵涉环保问题。

至于弱势群体的救助角度,我从日本东京的矢岛住持为没有亲人可以依靠的人诵经祈祷上看到了人性的光辉,但显然我们还应当从这个角度出发来对葬礼事宜进行更多的思考。



\part{附录}
\chapter{术语释义}
\section{世纪的早期中期和晚期}
表达时间不需要太过于精确的时候,会将一个世纪分为三个不同的时期来表达。一个世纪约一百年,前三十年称之为早期,中间四十年称之为中期,后三十年称之为晚期。

\section{GDP}
GDP是Gross Domestic Product(国内生产总值)的简称。国内生产总值是指一个地区(通常指一个国家)内的所有常住单位(所有企业和个人)在一定时期(通常指一年)内所生产和提供的\textbf{最终}产品和服务的价值总和。

国内生产总值由四个部分组成:GDP = 消费+个人投资+政府支出+净出口额


\section{CPI}
CPI是Consumer Price Index(消费者物价指数)的简称,是一个反映居民家庭一般购买的消费商品和服务价格水平变动情况的宏观经济指标。

美国经济学家戈登在1975年提出了核心CPI的概念,是指将受气候和季节因素影响较大的产品价格剔除之后的居民消费价格指数【具体就是剔除了食品和能源的影响】。

\section{PPI}
PPI是Producer Price Index(生产者价格指数)的简称,是用来衡量生产者在生产过程中所需采购品的物价状况的。

PPI反映的是生产环节的价格水平,CPI反映的是销售环节的价格水平,一般来说,CPI会跟随PPI波动,所以有PPI是CPI之先声的说法,但某些情况下,不尽然。

核心PPI是将食品和能源剔除之后的PPI指数。

\section{M0、M1和M2}
货币供应量分为三个层次:M0、M1和M2,各个国家对其定义略有不同,下面是中国人民银行公布的货币划分口径:

\begin{itemize}
\item M0:流通中现金
\item M1:M0+单位活期存款。 M1反应了社会商业经营活跃程度
\item M2:M1+单位定期存款+个人储蓄存款(包括活期和定期)+其他存款(比如股票的保证金等信托存款和委托存款)
\end{itemize}

从M1到M2货币的流动性\footnote{所谓流动性是指一种资产的变现速度和变现能力。}变弱,如果M2增速快于M1,市场给出的信息就是市面上货币过多,不需要这么多货币,所以货币流向了流动性弱的地方。 

\section{蓝筹股}
蓝筹一词源于西方赌场,在西方赌场中,有三种颜色的筹码,其中蓝色的筹码最值钱。证券市场上通常将那些经营业绩较好,具有稳定且较高现金股利支付的公司股票成为蓝筹股。


\section{非公制单位}
1码约0.914米

1磅约0.454千克

1加仑约4.546升

1立方英尺约0.028立方米

1英尺约0.3048米

1英寸约2.54厘米

1英亩约40.469公亩【1公亩也就是100平方米】

1公顷等于1万平方米

\chapter{百科知识}
\section{K线图}
K线图的技术分析流多少有点扯淡,不过K线图就作为一种图示方法,使用范围很广,简单了解下K线图是怎么一回事还是有必要的。

以下图片引用自参考资料9:

根据计算时间不同,有日K线,周K线等,下面以日K线为例来说明之。

如果当天收盘价高于开盘价,那么就绘制一个阳线实体。上下两个竖线指示的是当天的最低价和最高价。

\begin{figure}[H]
\centering
\includegraphics[width=\linewidth ,totalheight=0.95\textheight , keepaspectratio]{K线图阳线.jpg}
\caption{K线图阳线}
\end{figure}

如果当天收盘价低于开盘价,那么就绘制一个阴线实体。上下两条竖线指示的是当天的最低价和最高价。

\begin{figure}[H]
\centering
\includegraphics[width=\linewidth ,totalheight=0.95\textheight , keepaspectratio]{K线图阴线.jpg}
\caption{K线图阴线}
\end{figure}



\chapter{参考资料}
\begin{enumerate}
\item 《黄帝内经》.
\item 《好好告别:世界葬礼观察手记》 by Caitlin Doughty \& 崔倩倩[译] at 2022. 
\item 《金融学从入门到精通》 by 武永梅 at 2017.
\item 《逃不开的经济周期》 by 拉斯·特维德 \& 董裕平[译] at 2012.
\item 《千年金融史》by 威廉·戈兹曼  \& 张亚光[译] \& 熊金武[译] at 2017.
\item 《牛顿新传》by [英]罗布·艾利夫 \& 万兆元[译] at 2013.
\item 《牛顿与伪币制造者:科学巨匠鲜为人知的侦探生涯》by [美]托马斯·利文森 \& 周子平[译] at 2018. 
\item 《牛顿传》by [英]迈克尔·怀特 \& 陈可岗[译] at 2020.
\item 《从零开始学K线》by 唐韩 at 2012. 
\item 《聪明的投资者》by 本杰明·格雷厄姆 \& 贾森·兹威格[评] \& 王中华[译] \& 黄一义[译] at 2016.
\item 《美国四百年:冒险、创新与财富塑造的历史》 \& [美]布·斯里尼瓦桑 \& 扈喜林[译] at 2022.
\end{enumerate}

% 编者:Wander
\end{document}


