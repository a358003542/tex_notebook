% !Mode:: "TeX:UTF-8"

\documentclass[12pt,oneside]{book}

\newlength{\textpt}
\setlength{\textpt}{12pt}
    
\newcommand{\flypage}[1]{\begin{titlepage}\begin{center}\vspace*{\stretch{1}}#1\vspace*{\stretch{1}}\end{center}\end{titlepage}}
    
%========基本必备的宏包========%
\RequirePackage{calc,float,moresize}
%\RequirePackage[onehalfspacing]{setspace}
\linespread{1.5}
%1.3 onehalfspacing
%试卷或需要文字紧凑的
%1.6 doublespacing

%===========加入目录 某章或某节=====%
\makeatletter

\newcommand{\addchtoc}[1]{
        \cleardoublepage
        \phantomsection
        \addcontentsline{toc}{chapter}{#1}}

\newcommand{\addsectoc}[1]{
        \phantomsection
        \addcontentsline{toc}{section}{#1}}

%===========全文基本格式==========%
\setlength{\parskip}{1.6ex plus 0.2ex minus 0.2ex}   %段落間距
\setlength{\parindent}{\textpt * \real{2}}

%=========页面设置=========%
\RequirePackage[a4paper, %a4paper size 297:210 mm
  bindingoffset=10mm,%裝訂線
  top=35mm,  %上邊距 包括頁眉
  bottom=30mm,%下邊距 包括頁腳
  inner=10mm,  %左邊距or inner
  outer=10mm,  %右邊距or  outer
  headheight=10mm,%頁眉
  headsep=15mm,%
  footskip=15mm,%
  marginparsep=10pt, %旁註與正文間距
  marginparwidth=6em,includemp=true% 旁註寬度計入width%旁註寬度
  ]{geometry}

%color
\RequirePackage[table,svgnames]{xcolor}

%================字體================%
%设置数学字体
\RequirePackage{amssymb,amsmath}
\RequirePackage{stmaryrd}
\everymath{\displaystyle}

\RequirePackage{fontspec}
%設置英文字體
\setmainfont[Mapping=tex-text]{DejaVu Serif}
\setsansfont[Mapping=tex-text]{DejaVu Sans}
\setmonofont[Mapping=tex-text]{DejaVu Sans Mono}


%中文環境
\RequirePackage[]{xeCJK}
\xeCJKsetup{PunctStyle=plain}
\xeCJKDeclareSubCJKBlock{LIUSHISIGUA}{ "4DC0 -> "4DFF}
\setCJKmainfont[FallBack=DejaVu Serif, ItalicFont=KaiTi,LIUSHISIGUA=DejaVu Sans]{Source Han Serif CN}
\setCJKsansfont[FallBack=DejaVu Sans]{Source Han Sans CN}
\setCJKmonofont[FallBack=DejaVu Sans Mono]{KaiTi}


%%===============中文化=========%
\renewcommand\contentsname{目~录}
\renewcommand\listfigurename{插图目录}
\renewcommand\listtablename{表格目录}
\renewcommand\bibname{参~考~文~献}
\renewcommand\indexname{索~引}
\renewcommand\figurename{图}
\renewcommand\tablename{表}
\renewcommand\partname{部分}
\renewcommand\appendixname{附录}
\renewcommand{\today}{\number\year{}年\number\month{}月\number\day{}日}


%=======页眉页脚格式=========%
\RequirePackage{fancyhdr}   %頁眉頁腳
\RequirePackage{zhnumber}  %计数器中文化
\pagestyle{fancy}
\renewcommand{\sectionmark}[1]
{\markright{第\zhnumber{\arabic{section}}节~~#1}{}}

\fancypagestyle{plain}{%
    \fancyhf{}
    \renewcommand{\headrulewidth}{0pt}
    \renewcommand{\footrulewidth}{0pt}
    \fancyhf[HR]{\ttfamily \footnotesize \rightmark }
    \fancyhf[FR]{\thepage}}
\pagestyle{plain}


%=========章節標題設計=========%
\RequirePackage{titlesec}
%修改part
\titleformat{\part}{\huge\sffamily}{}{0em}{}
%修改chapter
\titleformat{\chapter}{\LARGE\sffamily}{}{0em}{}
%修改section
\titleformat{\section}{\Large\sffamily}{}{0em}{}
%修改subsection
\titleformat{\subsection}{\large\sffamily}{}{0em}{}
%修改subsubsection
\titleformat{\subsubsection}{\normalsize\sffamily}{}{0em}{}


%================目录===============%
%toc label to contents space   dynamic adjust
\RequirePackage{tocloft}%
\renewcommand{\numberline}[1]{%
  \@cftbsnum #1\@cftasnum~\@cftasnumb%
}

%==============超鏈接===============%
\RequirePackage[colorlinks=true,linkcolor=blue,citecolor=blue]{hyperref} %設置書簽和目錄鏈接等
\newcommand{\hlabel}[1]{\phantomsection \label{#1}}%某一小段的引用


%=================文字強調=========%
\RequirePackage{xeCJKfntef}

\let\oldemph\emph % Save emph in oldemph
\renewcommand{\emph}[1]{\textcolor{red}{#1}}  

%==================插入圖片=======%
\RequirePackage{wrapfig}
\RequirePackage{graphicx}
\graphicspath{{figures/}}
%change the caption style a little like 1-1
\renewcommand{\thefigure}{\arabic{chapter}-\arabic{figure}}


%==========其他宏包===========%
\RequirePackage{tikz} 
\usetikzlibrary{calc}

%========脚注=========%
\newcommand*\circled[1]{
\tikz[baseline=(char.base)]
\node[shape=circle,draw,inner sep=0.4pt,minimum size=4pt] (char) {#1};}

\newcommand*\circledarabic[1]{\circled{\arabic{#1}}}

\RequirePackage{perpage} %the perpage package
\MakePerPage{footnote} %the perpage package command

\renewcommand*{\thefootnote}{\protect\circledarabic{footnote}}


\renewcommand\@makefntext[1]{\vspace{5pt}\noindent
\makebox[20pt][c]{\fontsize{10pt}{12pt}\@thefnmark}
\fontsize{10pt}{12pt}\selectfont #1}

\setlength{\skip\footins}{20pt plus 10pt}
%main body 与脚注之间的距离

\RequirePackage{indentfirst} 

% hr
\newcommand\hr{\par\noindent\hrule}

%重新定义quotation
%\renewenvironment{quotation}
%{\list{}{\rightmargin\leftmargin %右间距等于左间距
%\itemindent 2em%item的缩进也就是第一段的缩进
%\listparindent \itemindent %第二段的缩进
%}%
%\item\relax
%\ttfamily}
%{\endlist}
\AtBeginEnvironment{quotation}{\ttfamily}

\makeatother

\title{冶文房}
\author{Wander}
\hypersetup{
  pdfkeywords={},
  pdfsubject={制作者邮箱:a358003542@outlook.com},
  pdfcreator={Wander}}
  
\newcommand{\bookcover}[1]{\tikz[remember picture,overlay]{\node[inner sep=0] at (current page.center)
{\includegraphics[width=\paperwidth,height=\paperheight]{#1}}}} 
 

  
\begin{document}
\frontmatter 

\thispagestyle{empty}

\bookcover{book_cover.png}

\cleardoublepage



\addchtoc{前言}
\chapter*{前言}
冶炼文字,以成此集。


\addchtoc{目录}
\setcounter{tocdepth}{2}    
\tableofcontents



\mainmatter

\part{十三世纪之前}
\chapter{黄帝内经关于养生之道的讨论}
\hr
[黄帝]乃问于天师曰:余闻上古之人,春秋皆度百岁,而动作不衰;今时之人,年半百而动作皆衰者,时世异耶,人将失之耶?

歧伯对曰:上古之人,其知道者,法于阴阳,和于术数,食饮有节,起居有常,不妄作劳,故能形与神俱,而尽终其天年,度百岁乃去。今时之人不然也,以酒为浆,以妄为常,醉以入房,以欲竭其精,以耗散其真,不知持满,不时御神,务快其心,逆于生乐,起居无节,故半百而衰也。夫上古圣人之教下也,皆谓之虚邪贼风,避之有时,恬惔虚无,真气从之,精神内守,病安从来。是以志闲而少欲,心安而不惧,形劳而不倦,气从以顺,各从其欲,皆得所愿。故美其食,任其服,乐其俗,高下不相慕,其民故曰朴。是以嗜欲不能劳其目,淫邪不能惑其心,愚智贤不肖不惧于物,故合于道。所以能年皆度百岁而动作不衰者,以其德全不危也。
\hr

 
黄帝曾向天师岐伯请教道:我听说上古时代的人们,都能够年过百岁,行动起来却不显衰老;而现在的人,年过半百,行动就已经显得很衰老了。这是由于时代不同了呢?还是由于人们违背了养生之道呢?

岐伯答道:上古时代的人,大都懂得养生之道,效法于天地阴阳变化,和谐于身心养生之术,饮食有节制,起居有规律,劳作适度,所以能够使身体与精神协调一致,从而尽享天年,度过百岁才离开人世。现在的人却不是这样,把酒当做果浆来喝,把任意妄为当作生活的常态,酒醉之后还强行房事,在放纵淫欲中使自己的精气枯竭,真元耗散,不懂得保持精气的盈满,不知适时御凝己神,只求一时快意开心,如此违逆了生活真正的乐趣,起居没有规律,故年半百就衰老了。上古时代圣人的教诲,都是这样说的:要适时回避四季不正的虚邪贼风,思想上恬淡虚无,体内真气随从其中,精神持守其内,疾病又从那里来呢。如此志闲淡而少欲,心安宁而无惧,形劳作而不疲倦,真气随从而调顺,每个人欲望和愿望都很容易得到满足。如此吃什么都很香甜,穿什么都很舒服,快乐地享受他们的习俗,彼此也不羡慕地位高低,民风可称纯朴。如此嗜好欲望颇深之物不能劳累其眼,淫乱邪恶之物不能惑乱其心,不论愚笨还是聪明,贤能或者不才之人都不惧怕物欲之物,而合于养生之道。所以他们能年过百岁而行动不衰,就是因为他们[如上]德行全备而无偏颇的缘故啊。


\part{十三世纪}

\part{十四世纪}

\part{十五世纪}


\part{十六世纪}

\part{十七世纪}

\part{十八世纪}

\part{十九世纪}

\part{二十世纪}

\part{二十一世纪}
\chapter{二十一世纪前期对世界各地葬礼的观察}
\begin{verbatim}
原文: 《From Here to Eternity》 by Caitlin Doughty 2017. 

译文: 《好好告别:世界葬礼观察手记》 by 崔倩倩 2022.
\end{verbatim}


\hr
早在2000多年前,古希腊历史学家希罗多德就写过两种文化对彼此的丧葬传统感到震惊的情形,可谓历史上对此现象最早的记录之一。在他的描述中,波斯帝国的统治者召集了一些希腊人到他面前。希腊人有火化遗体的传统,国王便问道:“什么样的奖赏能让你们吃掉父亲的遗体?”这个问题让希腊人惊慌不已,连连解释无论多么丰厚的赏赐都不足以把自己变成食人族。第二天,国王召集了因食用遗体而闻名的卡拉提亚人。国王问他们:“什么样的奖赏能让你们焚烧父亲的遗体?”卡拉提亚人表示这种行径“太可怕了”,请求国王不要再提及此事。
\hr

\section{美国科罗拉多州的克雷斯通镇}
\hr
8月的一个午后,我收到一封期待许久的来信:

凯特琳:

我们社区重要的一员劳拉今早离开了我们。她有心脏病史,刚刚过完75岁生日。不知道你现在身在何处,但欢迎你加入我们。

史蒂芬妮

劳拉的死完全出人意料。周日傍晚,她还在当地的音乐节上纵情起舞,周一一早就被发现倒在厨房的地板上没有了呼吸。火化仪式定在周四上午举行,届时她的家人都会出席,我也会在场。

...

劳拉的家人用布制的担架抬着劳拉穿过一片黑眼花,来到火葬用的火化台旁。一记嘹亮的锣声响彻天空。我沿着停车场旁的沙石小路向前走,一名笑容满面的志愿者递给我一捆新鲜的杜松枝。

劳拉平躺在一块金属炉箅\footnote{bì,蒸锅中的竹屉。}上,左侧和右侧各有一面光滑的白色水泥壁,上方则是一望无际的科罗拉多苍穹。我在这里见过两次空空如也的火化台,但这一次,遗体的出现令我理智、清晰地认识到这种仪式的目的。凭吊的人一个接一个地走上前来,把手中的杜松枝放到劳拉身上。作为唯一不认识劳拉的人,我有些犹豫——我管这个叫葬礼尴尬症。但我既不能一直把杜松枝拿在手里(太明显了),又不能放进背包(太没品了),于是我只好上前把枝条放在遗体上。

接下来,劳拉的家人(包括一个八九岁的小男孩)用矮松木的松枝和云杉原木把柴堆围起来。这两种木材越烧密度越高,所以通常是火化的首选。劳拉的伴侣和她已经成年的儿子手持火把站在角落里。随着信号出现,他们一起走向劳拉,将柴堆点燃。此时,太阳刚好从地平线上升起。

当火焰开始吞噬劳拉的身体时,白色的浓烟像小型旋风似的旋转上升,然后逐渐消失在清晨的天空中。

火化的味道让我想起爱德华·阿比作品中的一段描述:

\begin{quotation}
火焰。在我看来,杜松枝燃烧时发出的香味是世界上最甜美的,我怀疑但丁笔下的天堂都没有能与此媲美的熏香。就像雨后灌木丛的味道,只需闻一下,就能引发魔法般的通感:那是美妙的音乐,那是美国西部的广袤、光明、澄澈和神秘。希望火焰永远燃烧下去。
\end{quotation}

几分钟之后,螺旋状的浓烟不见了踪影,取而代之的是明亮的红色火焰。火焰攒足了能量,足足蹿到6英尺高。参加葬礼的一共有130人,他们全围绕在火化台旁,无声地看着眼前的一切。唯一的声响来自树枝断裂时的“噼啪”声,仿佛劳拉的每一段记忆都随着声响飘散在晨空中。

在科罗拉多州的克雷斯通进行的这种火葬仪式已有上万年的历史。古希腊人、古罗马人及印度教徒,都因使用最朴实的秘术——火——来消除肉体和净化灵魂而闻名于世。但火葬本身还可以追溯到更古老的时期。

20世纪60年代,一名年轻的地质学家在澳大利亚内陆发现了一具火化过的成年女性遗骸。根据他的预测,这具遗骸距今已有20000年。事实上,后续研究表明这名女性应该生活在42000年前,比原住民到达澳大利亚的推测时间还要早22000年。她应该是居住在布满植被的山坡上,和一群巨型生物(袋鼠、袋熊以及其他尺寸超常的啮齿类动物)为伴,以鱼肉、植物的种子和巨型鸸鹋\footnote{ér miáo,鸟纲,鸵鸟目,鸸鹋科。}的蛋为食。她死后,族人便把她(现在被称为“芒戈女士”)火化。没有焚烧完全的遗骨将被捣碎并进行二次火化。最后,族人用红色泥土把遗骸包裹起来埋在地下,一晃就是42000年。

...

美国其他地方也一样:同样规模的小镇,同样悲伤的人群,同样的露天火化台。但这显然不是事实。克雷斯通其实是美国,也是西方世界中唯一以社区为基础来实施露天火化的地方。

这种令人心潮澎湃的火化方式并非一直存在于克雷斯通。...“我们执着于用火化台进行露天火葬。”史蒂芬妮解释道,...

和史蒂芬妮一起工作的还有保罗·克鲁本伯格。他也很讨人喜欢,有着一口厚重的荷兰口音。他们二人带着火化台四处奔走,直接在别人家里进行火化。为了不让镇政府发现,两个人练就了一身速战速决的本领。就这样,他们用这套可移动设备完成了七场火葬。

“我们只要在你家院子的角落里把设备组装好就能开工。”保罗说道。

他们这套可移动火化台设备技术含量并不高,主要用煤渣砖制成,外加一个炉箅。每次火化时产生的高温都会让炉箅弯曲变形。“我们不得不开车碾过去,直接把它轧平。”史蒂芬妮说,“现在看来,我们当时确实很疯狂。”她边说边笑,并不觉得以前的做法有何不妥。

2006年,二人决定寻找一处固定场所来执行露天火葬。克雷斯通堪称完美之选:地理位置足够偏远,距离丹佛市中心约四个小时车程,全镇人口只有137人(周边地区人口1400人)。这种边缘性让克雷斯通具备了一种“老子的事儿政府最好别插手”的自由派气质,大麻和妓院在这里都是合法的(并不是说已经开了好几家妓院,只是说可以有)。

克雷斯通吸引了形形色色的“朝圣者”,人们从世界各地来到这里寻找精神家园。当地的天然食品店里贴着各种各样的广告宣传单:气功班、影子智慧班、面向儿童的“潜能开发”灵修班、北非民族舞蹈班,还有什么“魔法森林圣集会”。克雷斯通的本地居民不乏嬉皮士和信托自由儿\footnote{指来自富裕家庭的年轻人。},但大部分都是严肃的终身信徒。这里有佛教徒、苏菲主义者、卡迈尔修女等。刚刚过世的劳拉就是印度哲学家室利·阿罗频多的忠实追随者。

史蒂芬妮和保罗首次申请火葬固定用地时,就遭到了多个业主的反对。“他们是一帮烟民,就这德行。”保罗告诉我。这些业主义正词严地警告他们“别打我地盘的主意”。在史蒂芬妮看来,他们就是一群“守财奴”,根本不在乎二人提供的无林火风险、无异味、无水银或其他有毒物的证据。这些烟民集体给镇政府和环保局写信抗议。

为了抗争,史蒂芬妮和保罗把自己的业务进行了合法化。他们成立了一个名为“克雷斯通临终计划”的非营利机构,一刻不停地收集了400多个签名(相当于周边人口的1/3)。他们把法务文件、科学检测报告等材料通通收集起来,装订成一本厚厚的资料册。他们甚至走访了当地每一户居民,认真倾听大家的担心和顾虑。

一开始,他们遭到了人们的严重抵触。在一次反对露天火葬的集会上,有人叫他们“邻里互助火化二人组”。当史蒂芬妮和保罗提议(其实是个玩笑)在当地游行活动中使用卡通气模进行宣传时,有一家人走过来抗议说这种行为“简直是大不敬”,因为气模上带有火焰形状的纸质装饰品。

“镇上的居民甚至担心露天火葬会导致交通堵塞。”史蒂芬妮说道,“要知道在克雷斯通,一条街上同时出现6辆车都能被看作交通堵塞。”

保罗解释道:“这是因为人们充满了恐惧。‘这会不会造成空气污染?’‘你不觉得这很变态吗?’‘一想到与死有关的东西,我就害怕。’你必须保持耐心,仔细倾听他们的想法和需求。”

哪怕面临诸多法律困境,史蒂芬妮和保罗也没有退缩,因为他们坚信露天火葬能够启发整个社区(一些居民因为自己有机会被露天火葬而激动不已,要求史蒂芬妮和保罗把设备组装在自家的停车道上)。“如今,有多少人能提供真正让别人产生共鸣的服务?”史蒂芬妮说道,“我只做能唤起心灵共鸣的事,不然就不做。”

二人终于给自己的火化台找到了一个稳定的家。那是小镇外的一处空地,距离主路几百码远。这块土地是禅宗团体“龙山庙”捐赠给他们的。史蒂芬妮和保罗毫不遮掩地把火化台放在外面。当你沿着这条主路驶向克雷斯通时,你就会看到一个印有火焰图案的金属标志,上面写着“火化台”三个字。这个标志出自当地一位种土豆的农民之手(这个人同时兼任验尸官),可以说是一个明显的地标了。

火化台搭建在一片沙地之上,四周环绕着一片竹林。竹子的枝条形态各异,好似书法的笔画。目前,这里已经火化了50多人,包括把他们称为“邻里互助火化二人组”的那个人,他在临终前改变了自己的看法(反转无误)。

在为劳拉举行火葬的3天前,克雷斯通临终计划的志愿者来到她家帮忙。他们和劳拉的友人一起打理遗体,帮他们用低温毯裹好遗体以防腐化。他们还给劳拉穿上由天然面料制成的寿衣——化纤面料不适合用在火化台火葬中,比如涤纶。

克雷斯通临终计划会协助死者家属进行所有的葬礼后勤工作,但不为其提供资金支持。死者家属也可以选择露天火葬之外的火化方式:传统土葬(遗体经过防腐),自然土葬(没有墓穴或遗体不经过防腐),去几个镇子之外的火葬场进行火化。不管选择哪一种,志愿者都会协助家属进行相关的后勤工作。保罗把最后一种称为“商业火化”。

史蒂芬妮抗议道:“保罗,你应该叫它‘传统火化’。”

“不不,”我争辩道,“‘商业火化’这个称呼很贴切。”

克雷斯通让身为殡葬业专业人员的我备受鼓舞——这也是我多次来到这里的原因,但也让我感到些许(近乎嫉妒的)忧伤。这里的人们拥有一个庄严的火化台,就在浩瀚苍穹之下,我却只能把家人送到郊区厂房里那个又脏又吵的火葬场。如果我自己的殡仪馆能有一处如此壮观的火化场所,我发誓会请一个迪吉里杜管乐手在现场演奏。

以火化炉作为主要设备的工业火化首先出现在19世纪末期的欧洲。1869年,一群医学专家聚集在意大利佛罗伦萨,宣称土葬存在卫生问题,应该用火化取而代之。几乎是与此同时,一股提倡火化的风潮也席卷了美国,领头人是奥克塔维厄斯·B.弗洛汀汉姆牧师,他认为化作一堆“白色灰烬”的遗体比“高度腐烂”的遗体要好得多。

约瑟夫·亨利·路易斯·查尔斯·德·帕尔姆男爵是第一个在美国经历“科学的现代化”火葬的人。这名男爵是一位身无分文的奥地利落魄贵族,死于1876年5月。用《纽约论坛报》的话来说,“他主要是因为变成了一具尸体而出名”。

火化安排在当年12月,也就是他死去的六个月之后。在这半年里,他的遗体被注入砒霜以防腐烂。当砒霜逐渐失效,不足以抵抗大自然力量的时候,当地的一个殡葬人直接把器官从他体内拽出来,再用黏土和石炭酸的混合物涂满他的全身。在从纽约开往宾夕法尼亚州(火葬场所在地)的火车上,这具已然木乃伊化的尸体还在行李车厢中消失了一段时间。历史学家斯蒂芬·普罗特劳称这段插曲为“令人毛骨悚然的捉迷藏”。

这场具有划时代意义的火化在宾州的一处火葬场进行。这个火葬场由私人诊所改造而成,主要设备是一个火炉,炉箅下方装满了用作燃料的煤块。这样一来,火焰不用接触尸体,光凭高温就能让尸体“分崩离析”。尽管医学专家们一再强调这场火化是“一次严格遵循科学和卫生原理的实验”,德·帕尔姆男爵的遗体还是被撒上香料,安放在铺满了玫瑰、棕榈枝、报春花和松树枝的灵床上。当遗体刚被放入火炉时,观察人员称自己闻到了一股独特的烤人肉味,但很快这个味道就被花朵和香料的芬芳掩盖住了。一个小时之后,德·帕尔姆男爵的遗体上笼罩了一层闪闪发光的玫瑰色薄雾。接着,这层光芒逐渐转变为金色,最终变成了透明的亮红色。遗体就这样烧了两个半小时,直到从白骨化作灰烬。现场的记者和评论员后来在报道中写道,这次实验性火化“第一次实现了把人放在烤箱中进行安全、无味的烘烤”。

从那之后,火化设备变得越发庞大,运转速度越来越快,效率越来越高。大约150年之后,火葬的受欢迎程度达到了历史新高(美国2017年的火葬人数可能超过土葬人数,这在历史上还是头一遭)。但是,火葬的仪式和美学还是老样子。我们的火化设备还在使用19世纪70年代发明出来的款式——用钢铁、水泥和砖块建成的,重达24000磅的庞然大物。这些“怪物”每个月都要消耗上千美元的天然气,向大气中排放大量的二氧化碳、烟尘、二氧化硫,以及毒素含量极高的水银(主要来自遗体内的填充物)。

大城市里的火葬场大多存在于郊外工业区中不起眼的厂房里。...有的火葬场就位于墓地,...

有些火葬场把自己打造成“为生命喝彩”型或“火葬纪念堂”型机构。在那里,死者的亲朋好友聚集在空调屋里,透过玻璃窗目送遗体消失在火化炉的金属门后。火化机则躲藏在墙体后方,这样就不会被来宾看到,虽然它们和你在其他火葬场里见到的工业用火化炉没什么区别。这种伪装让还活着的人越发远离死亡这一事实,也越发远离火化方式笨重陈旧、污染环境这一真相。如果想让死去的母亲享受到“火葬纪念堂”型机构提供的优等待遇,你至少要花上5000美元。

我并没有说选择露天火葬就能解决这些问题。在印度、尼泊尔等将露天火葬作为主要殡葬方式的国家,每年的火化量高达百万场,耗费近5000万棵树,并释放出大量的黑碳气溶胶。黑碳气溶胶是继二氧化碳之后,造成气候变化的第二大人为因素。

但是,克雷斯通模式可以。克雷斯通临终计划接到了多个来自印度的殡葬业改革者的电话,想把他们的设备设计和操作方法引入印度——让火化台的位置远高于地面,这样就能减少木材用量和污染物排放量。既然我们能够改变古老的、与宗教和国家紧密相连的殡葬方式,那我们也有可能推动对现代工业化火葬的改革。

劳拉在克雷斯通生活了很多年,火化当天,全镇的人几乎都赶来参加她的葬礼。她的儿子杰森注视着火焰,首先打破了沉静。“妈妈,谢谢你爱我,”他用优美的嗓音说道,“不要再担心我们了,朝着自由飞翔吧!”

火焰继续燃烧。一位女士开始讲述她11年前一个人来到克雷斯通的情景,那时她已经被慢性疾病折磨了很多年。“我搬来克雷斯通是为了寻找快乐。起先,我认为是这里的云朵和天空治愈了我,但现在看来,应该是劳拉。”

...就在来宾一一发言时,火焰蹿上了她裸露的肌肤和软组织,将包含于其中的水分一层层抽干,留下越发干瘪、枯萎的肉体。她的内脏因此暴露无遗,眼看着就要被火焰吞噬。

对毫无经验的旁观者而言,这绝对是令人惊悚的一幕。好在机构的志愿者警惕性很高,把炉箅上的遗体挡得严严实实。他们的动作专业、优雅,确保来宾既闻不到异味,也看不到半熟的脑袋和焦黑的手臂。“我们并不是把遗体掩藏起来,”史蒂芬妮解释道,“但这里大多数的火葬都向全镇人开放。你不知道谁会来,也不知道他们会如何对待因目睹火葬而产生的心理动荡。有时,人们甚至把火化台上的遗体想象成自己。”

随着仪式的继续,志愿者悄然来到火化台旁边,往里面添加木材。整场火化使用了42.6立方英尺的木头,也就是1考得的1/3。

火焰终于碰到了劳拉的骨头。首先燃烧的是膝盖、脚踝和面部骨骼,烧了好一会儿才轮到骨盆和四肢。骨骼中的水分被蒸发掉,有机物也逐渐荡然无存。在这个过程中,她的遗骨先是经历了从白色变成灰色和黑色的过程,然后又从黑色变回白色。木材的重量让她破碎的遗体从炉箅穿过,径直落在下方的地面上。

...

离劳拉葬礼结束还有一个小时,此时的氛围已经不像之前那么哀伤了。最后一个人的发言要是放在一个半小时之前,肯定属于不合时宜。“你们刚才一直在说劳拉是个多么好的人,这我同意,她确实是个好人。但在我看来,劳拉永远都是一个狂野的疯婆子、一个派对女郎,我要为她喝彩。”

“嗷嗷嗷嗷嗷嗷!”她大声尖叫着。周围的人也随她一起号叫起来,我自己也小心翼翼地喊了一嗓子。要知道,我刚才连杜松枝都不敢往劳拉身上放。

到了早上9点30分,只有史蒂芬妮和我(以及劳拉的部分遗骨)还留在现场。我们坐在木头长凳上说话,此时火化台上只剩下三根木头,它们在灰烬中闪烁着轻柔的余光。消防队提供的红外线温度计显示,这些未尽的余火达到1250华氏度。

史蒂芬妮通常是第一个到达,也是最后一个离开火化现场的人。“我喜欢这份宁静。”她解释道。

史蒂芬妮沉默了一会儿,突然起身走到火化台旁,拾起一片金属炉箅仔细看了看。“这是保罗新设计的火花防护装置,能够把灰烬牢牢锁住,这样就不会让风吹走。烧剩下的木块也会被固定住。但上面要是有没熄灭的火星呢?”

不出几分钟,史蒂芬妮就给消防队打了电话,商量对火花防护装置进行检查和测试的事宜。精力充沛的史蒂芬妮不允许自己有片刻清闲。我非常好奇,究竟是什么样的恒心,能够让她十几年如一日地坚持工作,最终把露天火葬变为现实。“这是一个让人心力交瘁的过程,我们不得不耐心等待,直到人们接受我们。这太难了,我真想强行拉他们入伙。”

...


露天火葬受欢迎到什么程度呢?为了获得火葬资格,有人甚至在克雷斯通购买了地产。一名罹患宫颈癌的女士在42岁临终之前买下了一小块地皮。在她去世后,她12岁的女儿亲手给她的遗体做火化前的准备。
放眼全世界,人类对火焰的渴望不足为奇。印度人会把逝去的家人带到恒河边进行露天火葬,父亲的遗体由长子点燃。随着火焰的温度不断升高,死者的肉身也逐渐消失殆尽。当进行到一定时刻,负责火化的人会过来把头骨敲裂。印度人相信,头骨开裂后,人的灵魂才能从躯壳中释放出来。

放眼全世界,人类对火焰的渴望不足为奇。印度人会把逝去的家人带到恒河边进行露天火葬,父亲的遗体由长子点燃。随着火焰的温度不断升高,死者的肉身也逐渐消失殆尽。当进行到一定时刻,负责火化的人会过来把头骨敲裂。印度人相信,头骨开裂后,人的灵魂才能从躯壳中释放出来。

有人这样描述过自己双亲的火葬:“在此(敲碎头骨)之前,你惊慌得浑身发抖,眼前的这个人几个小时前明明还活着。但当骨头碎裂的那一刻,所有的痛苦都不见了,因为你意识到正在燃烧的不过是一具空壳。”灵魂获得了自由,就像现场播放的宗教歌曲中唱的那样:“死神,你以为征服了我们,但我们正在引吭高歌,赞美那熊熊燃烧的柴火。”

...

劳拉葬礼后的第二天清晨,我再次来到仪式场地。拴在火化台边上的两只大狗热情地迎接了我。麦格雷戈比我到得还早,此时正在用筛子仔细地清理劳拉的骨灰。他是劳拉的弟弟,自愿承担起清理骨灰的任务。骨灰和木头渣混在一起,大概有4.4加仑。他从灰烬中拣出最大的几块遗骸,分别是股骨、肋骨和头骨。有些家庭会把遗骨拿回家当作遗物保管。

与商业火化相比,露天火化产生的灰烬要多得多。前者留下的骨灰也就是一罐福尔杰牌咖啡那么多。加利福尼亚州要求我们用“骨灰研磨机”把残留的遗骨磨碎,达到“无法用肉眼识别”的状态,因为该州的法律不允许我们把形状明显的大块遗骨交给死者家属。

劳拉的几个朋友分走了一些骨灰,剩下的将撒在火化台附近的山丘或远处的山林里。“她肯定喜欢我们这样做,”杰森说道,“这样她就无处不在了。”

我问杰森,昨天的葬礼是否让他变得和以前不一样了。“我上次来克雷斯通时,妈妈带我参观了一下这个火化台。我当时就傻了,以为我得自己一个人坐在这里把她火化掉,这太吓人了。葬礼前三天,我已经害怕得不知该怎么办才好了。但是,她说过:‘这是我的选择,你来不来都可以。’”

杰森说,当他来到母亲家中参加守灵仪式的那一刻,他改变了之前的想法。特别是在火化仪式上,他意识到整个镇子的人都陪在他身边。在谈话和歌声中,他安心地接受了母亲的朋友们给予他的情感支持。“我非常感动,觉得一切都不一样了。”

...

\hr

\section{印度尼西亚的南苏拉威西岛}



\part{附录}
\chapter{术语释义}
\section{世纪的早期中期和晚期}
表达时间不需要太过于精确的时候,会将一个世纪分为三个不同的时期来表达。一个世纪约一百年,前三十年称之为早期,中间四十年称之为中期,后三十年称之为晚期。

\section{非公制单位}
1码约0.914米

1磅约0.454千克

1加仑约4.546升

1立方英尺约0.028立方米


\chapter{参考资料}
\begin{itemize}
\item 1
\end{itemize}

% 编者:Wander
\end{document}


